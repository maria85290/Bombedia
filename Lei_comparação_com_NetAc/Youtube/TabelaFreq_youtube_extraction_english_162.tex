\documentclass[11pt]{article}
\ExplSyntaxOn
\let\tl_length:n\tl_count:n
\ExplSyntaxOff
\usepackage{graphicx}
\usepackage{multirow}
\usepackage{colortbl}
\usepackage{longtable, array}
\usepackage{hyperref}
\usepackage[usenames,dvipsnames,svgnames,table]{xcolor}
\newlength\mylength
\usepackage[legalpaper, landscape, margin=0.8in]{geometry}
\newcommand{\MinNumber}{0}
\begin{document}

\textbf {\huge In youtube\_extraction\_english\_162.json :}\newline \par\Large\textbf {Title: \large  Age Discrimination in Hiring \& How to Overcome it - 3 Questions }\newline {\par\large --- Table  1: Summary of the results per comment; }

 {\par\large --- \hyperlink{Table 2}{\textcolor{blue}{\underline{Table 2}}}: Summary of the results per sociolinguistic variable;}\newline \normalsize\newline

\centering\textbf{\large Table  1: Summary of the results per comment 
}
\newcommand{\MaxNumber}{0}%
\newcommand{\ApplyGradient}[1]{%
\pgfmathsetmacro{\PercentColor}{100.0*(#1-\MinNumber)/(\MaxNumber-\MinNumber)}
\xdef\PercentColor{\PercentColor}%
\cellcolor{LightSpringGreen!\PercentColor!LightRed}{#1}
}
\newcolumntype{C}[2]{>{\centering\arraybackslash}p{#1}}
\begin{center}
\setlength\mylength{\dimexpr\textwidth - 1\arrayrulewidth - 50\tabcolsep}
\begin{longtable}{|C{.65\mylength}|C{.30\mylength}|C{.12\mylength}|C{.12\mylength}|C{.12\mylength}|}
\hline
\textbf{Comment} & \textbf{KeyWords} & \textbf{Sociolinguistic variables (Hiper - Hipo)}  & \textbf{Hate Speech Frequency} & \textbf{Hate Speech Frequency(\%)} \\
\hline\cellcolor{green!27}\small Age Discrimination in Hiring \& How to Overcome it  / Tell me, what ways will you use to improve your job search strategy?\normalsize   & \cellcolor{green!27}Age & \cellcolor{green!27}Age - General & \cellcolor{green!27}1/23 & \cellcolor{green!27}4.348 \\  \hline
  \cellcolor{green!5}\small The real trick is to show the following:1.  You're not just looking for a quiet pasture to bed down and spend your remaining time before retiring.2.  You're not just resting on your previous skills, you are continuously learning new ones.3.  Even health problems, if you have any (I'm 1 year 5 months cancer free, still recovering from several surgeries and whatnot), are not any obstacle to being amazingly effective at the job.4.  You can relate to people of any \textbf{age} or culture.5.  You won't be harassing, boring, depressing, negative, political, or offensive at work.etc.\normalsize   & \cellcolor{green!5}Age & \cellcolor{green!5}Age - General & \cellcolor{green!5}1/100 & \cellcolor{green!5}1.0 \\  \hline
  \cellcolor{green!27}\small in Thailand the employers like to hire the young \textbf{age} people (not over 35yrs). I am now 43yrs and also have gaps in employment 1.5yrs. super tough for me :(\normalsize   & \cellcolor{green!27}Age & \cellcolor{green!27}Age - General & \cellcolor{green!27}1/30 & \cellcolor{green!27}3.333 \\  \hline
  \cellcolor{green!5}\small \@Mike Morgan Pushing out \textbf{old} employees to recruit younger ones is an indicator a business is poorly governed. Well governed businesses adapt their interviewing styles to older employees because they know how to manage them. An advantage to working with top recruiters is that they only work with well governed businesses.\normalsize   & \cellcolor{green!5}Old & \cellcolor{green!5}Age - Over 65s & \cellcolor{green!5}1/51 & \cellcolor{green!5}1.961 \\  \hline
  \cellcolor{green!27}\small I don't want to take the name of the company that has almost hired me. It was a telephonic interview. All my details were given by sending the text messages. The company HR finally called me with the documents. I was traveling with the documents and in a way, I received a call from the same HR. The simple question asked me " What is your date of birth?  I answered him that I have given you my date of birth by sending you the text message long back. Please help me with your date of birth again. I gave him my date of birth. His immediate answer, Mohan, you are not qualified. I am sorry. I live in Mumbai, India. The country does not have "\textbf{Age} discrimination employment act" I am jobless for more than 3 years.  I do get job calls now and then but they are not in the line of my work experience. It is easy to make video and that you have good communication skill. The truth remains Truth. \textbf{Age} discrimination is worldwide I believe.\normalsize   & \cellcolor{green!27}Age & \cellcolor{green!27}Age - General & \cellcolor{green!27}2/180 & \cellcolor{green!27}1.111 \\  \hline
  \cellcolor{green!5}\small Does an employer asking your \textbf{age} job discrimination? I always looked older than my \textbf{age} since I was 13. I had a few employers asked me my \textbf{age}. At the time, I was 22 and 23 years \textbf{old}. I didn't know that question was illegal. I never got hired from those jobs. What do you think from my situation? Was that \textbf{age} discrimination?\normalsize   & \cellcolor{green!5}Age, Old & \cellcolor{green!5}Age - General, Age - Over 65s & \cellcolor{green!5}5/63 & \cellcolor{green!5}7.937 \\  \hline
  \cellcolor{green!27}\small Linda, the reality is that there are plenty of millenials hired who have \textbf{ZERO} ambition, \textbf{Zero} experience, plenty of schooling in unrelated fields but none in on-point topics, yet get hired because they fill a social quota, lower the group insurance \textbf{age} demographic, and look good in a company's website photos.  Seriously, would you say these same things to someone complaining about racism/sexism/genderism/etc?\normalsize   & \cellcolor{green!27}Age, Zero & \cellcolor{green!27}Age - General, Age - Youngsters & \cellcolor{green!27}3/63 & \cellcolor{green!27}4.762 \\  \hline
  \cellcolor{green!5}\small It's not "\textbf{ageism}".  Its called "\textbf{age} discrimination " and we should call it by its proper name.  However, I do agree with her - it does not have to be impossible.\normalsize   & \cellcolor{green!5}Age, Ageism & \cellcolor{green!5}Age - General & \cellcolor{green!5}2/31 & \cellcolor{green!5}6.452 \\  \hline
  \cellcolor{green!27}\small I have watched many of your videos and never commented. I am over 50, run 6 miles a day, and have worked in my field for 25 plus years. Do I feel funny being interviewed by hiring managers 20 years younger and with less experience than me who act like I am fresh out of college? It seems in your video you are stating that I should settle for ' consulting ' . Consulting offers no health benefits or the feeling of having steady employment. Since you look to be in your 30's, 40's tops, your probably have not encountered \textbf{ageism},yet. Well let me tell you, it is real and it stinks!!\normalsize   & \cellcolor{green!27}Ageism & \cellcolor{green!27}Age - General & \cellcolor{green!27}1/112 & \cellcolor{green!27}0.893 \\  \hline
  \cellcolor{green!5}\small I was expecting something that was actually helpful. No, it's not a make believe and all in someone's head; there's too much hard evidence to prove this. I wonder if she denies \textbf{racism} exist too, or if it's only problems she doesn't personally have to face.\normalsize   & \cellcolor{green!5}Racism & \cellcolor{green!5}Ethnicity - General & \cellcolor{green!5}1/46 & \cellcolor{green!5}2.174 \\  \hline
  \cellcolor{green!27}\small If there is \textbf{age} descrimination in the picture they won't even call you for an interview because they can do the math by glancing at your resume.\normalsize   & \cellcolor{green!27}Age & \cellcolor{green!27}Age - General & \cellcolor{green!27}1/27 & \cellcolor{green!27}3.704 \\  \hline
  \cellcolor{green!5}\small There are many positives to \textbf{ageism}. Older workers have advantages that younger workers do not: experience, maturity, loyalty, wisdom, industry knowledge, professionalism, poise and how to be an effective team member. They also have solid verbal/written communication skills because those invaluable attributes were ingrained in us from the time we were children.\normalsize   & \cellcolor{green!5}Ageism & \cellcolor{green!5}Age - General & \cellcolor{green!5}1/52 & \cellcolor{green!5}1.923 \\  \hline
  \cellcolor{green!27}\small I solved \textbf{age} discrimination by being a  contracting consultant. Now they beg me to go full time and I say No Thanks.\normalsize   & \cellcolor{green!27}Age & \cellcolor{green!27}Age - General & \cellcolor{green!27}1/22 & \cellcolor{green!27}4.545 \\  \hline
  \cellcolor{green!5}\small Yellow cab where I live makes people who fill  application's put down thier \textbf{age} \& Religion also don't like to hire Older people\normalsize   & \cellcolor{green!5}Age, Yellow & \cellcolor{green!5}Age - General, Ethnicity - Asian (East- China, Japan, Korea, Philippines, Vietnam), Nationality - Chinese, Nationality - Japanese & \cellcolor{green!5}4/23 & \cellcolor{green!5}17.391 \\  \hline
  \cellcolor{green!27}\small I never see these sorts of videos from folk in their 40s and 50s who have overcome covert \textbf{ageism}.....I wonder why?\normalsize   & \cellcolor{green!27}Ageism & \cellcolor{green!27}Age - General & \cellcolor{green!27}1/21 & \cellcolor{green!27}4.762 \\  \hline
  \cellcolor{green!5}\small I've looked for a position for over a year and had tons of phone interviews, then several in-person interviews.  I got called back twice in a few of them.  I look and present younger than my \textbf{age} (60) and employers LOVED my stellar resume and experience (and ME) but somehow I got passed over every time.  I have soul searched, looked within, revamped my appearance (which looks younger anyway) and even filmed myself in mock interviews and STILL did not get a job offer.  Mind you, when I was younger I committed every interview and resume mistake in the book and still got multiple job offers in no time.  I know just what you are advising here and how to prove to them that I am the ONE person they should hire, and yet it has not helped me.  I come across with enthusiasm, competence and energy and it still has not helped me.  I have come across like I want to contribute something valuable to an organization for many years to come.  So in my case, yes, \textbf{age} discrimination has to be THE only reason I have not gotten a job.  P.S.  I am anything but "average" - my mother edited resumes and I gave workshops in interview presentation when in college so I know a little more about this subject than most people, so if I'm not getting a job I can only conclude that it's \textbf{age} discrimination.  It's not enough to try to "fool" anyone anymore - they can look up your \textbf{age} online.  And mind you, these are companies that are often engaging in other forms of \textbf{age} discrimination in their practices, such as "retiring" older employees, passing them up for promotions, giving them false bad reviews, etc.  I've seen it before.  So I know it's not below them to engage in this in their hiring.\normalsize   & \cellcolor{green!5}Age & \cellcolor{green!5}Age - General & \cellcolor{green!5}5/310 & \cellcolor{green!5}1.613 \\  \hline
  \cellcolor{green!27}\small Now you are telling it how it really is and we need to expose how it really is - \textbf{age} discrimination is evil - it destroys income, it destroys families, and it destroy your health and  it can even kill you.   No one can live without an income indefinitely unless they are rich -  a weekly decent paycheck is vital living.\normalsize   & \cellcolor{green!27}Age & \cellcolor{green!27}Age - General & \cellcolor{green!27}1/61 & \cellcolor{green!27}1.639 \\  \hline
  \cellcolor{green!5}\small Peter Gosselin at ProPublica has been doing an excellent series on \textbf{age} discrimination.  Check it out.  Speaking up is the only way we'll see change.  Since we ALL get older, we ALL have an interest in fixing it.\normalsize   & \cellcolor{green!5}Age & \cellcolor{green!5}Age - General & \cellcolor{green!5}1/38 & \cellcolor{green!5}2.632 \\  \hline
  \cellcolor{green!27}\small Hi, Renee M.... I think \textbf{age} discrimination is rampant. After being laid off from a software company at 64, I did get a job, not a high paying one, but -- hey.  I made a conscious decision to avoid the I.T. field, which is my background, after seeing the look of shock on the faces of people interviewing me.  I listen to career coaches with some skepticism about beating \textbf{age} discrimination.  It all sounds good, but the reality is far different.\normalsize   & \cellcolor{green!27}Age & \cellcolor{green!27}Age - General & \cellcolor{green!27}2/81 & \cellcolor{green!27}2.469 \\  \hline
  \cellcolor{green!5}\small Ageism is a complicated issue, but Linda you're correct about being flexible when it comes to job hunting. My suggestion is for us to find out what other jobs are out there that would compliment our expertise and be creative when it comes to job titles. We don't have to do just one kind of job because that's what we know but rather, look for another job that still would require our abilities. Glassdoor actually has an article about jobs for those who are fifty and over, I suggest everybody to check it out.\normalsize   & \cellcolor{green!5}Ageism & \cellcolor{green!5}Age - General & \cellcolor{green!5}1/94 & \cellcolor{green!5}1.064 \\  \hline
  \cellcolor{green!27}\small Age is problem to me for find job\normalsize   & \cellcolor{green!27}Age & \cellcolor{green!27}Age - General & \cellcolor{green!27}1/8 & \cellcolor{green!27}12.5 \\  \hline
  \cellcolor{green!5}\small Age discrimination is a real problem in some professions. I work in software development, where we have \textbf{senior} developers in their mid-20s, have unspoken rules about ever hiring anyone over 30 and looking for ways to push people out when they get a family or turn 35. If a 30+ year-old perfectly fits a position we need filled, they will still look for someone in China or India. It's probably illegal, but impossible to prove, and when I spoke against it to management, I was pulled off some interviews I was supposed to conduct.\normalsize   & \cellcolor{green!5}Age, Senior & \cellcolor{green!5}Age - General, Age - Over 65s & \cellcolor{green!5}2/94 & \cellcolor{green!5}2.128 \\  \hline
  \cellcolor{green!27}\small Chances are the hiring company is going to be looking at healthcare for the \textbf{senior} worker as well. That's a huge disadvantage for older people, who, most likely, Are dealing with some sort of health issue related to \textbf{age}.No way I could compete with someone in their 20's health wise.\normalsize   & \cellcolor{green!27}Age, Senior & \cellcolor{green!27}Age - General, Age - Over 65s & \cellcolor{green!27}2/50 & \cellcolor{green!27}4.0 \\  \hline
  \cellcolor{green!5}\small Age discrimination is real, my husband lost 3 jobs in a short period of time, he is experienced but he gets  in the jobs but the young  managers let him go because they want young workers next to them, he has gone back to corporate to try and explain how he was pushed out, but they say they can't override the manager ( he has been hired by another person than the one that let him go, so very unfair) He is 62, yes it's up there, but he does not want to retire yet.  He has looked outside the box now for possible difference choices.  It's such a discriminatory world, against \textbf{age} and against our constitutional rights too,  young people are not good hearted like when we were young\normalsize   & \cellcolor{green!5}Age & \cellcolor{green!5}Age - General & \cellcolor{green!5}2/130 & \cellcolor{green!5}1.538 \\  \hline
  \cellcolor{green!27}\small I agree Kimberly. We (my husband and I) have experienced the same thing. I don't think it's just that we are thinking all the time, "Oh, it's my \textbf{age}" and that's our "reality" (per Lynda). I like her videos, but this one was offensive to me. She hasn't been there, so there is no way she can really know what it's like.\normalsize   & \cellcolor{green!27}Age & \cellcolor{green!27}Age - General & \cellcolor{green!27}1/62 & \cellcolor{green!27}1.613 \\  \hline
  \cellcolor{green!5}\small It is a federal crime that folk are getting away with not hiring more qualified and we'll seasoned older citizens on the grounds of their \textbf{age}!\normalsize   & \cellcolor{green!5}Age & \cellcolor{green!5}Age - General & \cellcolor{green!5}1/26 & \cellcolor{green!5}3.846 \\  \hline
  \cellcolor{green!27}\small Hi Linda.  I'm an avid watcher of your videos and they are filled with amazing and useful information.  I, too, have fallen victim to \textbf{age} discrimination, although honestly speaking, I did not ace every single interview, nor was every one a good fit.  However, due to personal experiences, hiring practices have most definitely changed throughout the decades.  For example, you must not go back any farther than 10 years if possible, create a gmail address which does not identify your \textbf{age}, eliminate any outdated skills and/or software from your resume.  Also, it is important not to admit to having "25 years of experience" in any software applications, skills or experience.  I learned this the hard way by being out of work for the past 8 months, and by volunteering information in every interview.  It actually did me a disservice.  Most of the job descriptions I've seen for an admin support position state the following:  "3-5 years experience with a Bachelors Degree".  This screams "Millennials straight out of college".   It gives me an unfair advantage because my resume can be systematically eliminated from the candidate pool based on the number of years experience I have.  At \textbf{age} 50, I am working as a temp for half the salary I used to make with no money left over to pay for healthcare.  This is a real crisis in today's day and \textbf{age}.   I still love your videos and your style, and look forward to watching more of them in the future.  Keep up the great work!\normalsize   & \cellcolor{green!27}Age & \cellcolor{green!27}Age - General & \cellcolor{green!27}4/254 & \cellcolor{green!27}1.575 \\  \hline
  \cellcolor{green!5}\small I am 60, and have deleted a significant portion of my past experience and it still does not help because they can look up your \textbf{age} online.  Even if you get the interview, once they meet you they "know" you're older than they thought you were and end up searching to find out how \textbf{old} you really are.  I am a high level executive assistant so I know what you mean about the admin. world.\normalsize   & \cellcolor{green!5}Age, Old & \cellcolor{green!5}Age - General, Age - Over 65s & \cellcolor{green!5}2/75 & \cellcolor{green!5}2.667 \\  \hline
  \cellcolor{green!27}\small This was so helpful! Real authentic perspective. The exact things that this 56 year \textbf{old} needed to hear. Thank you so much Linda!\normalsize   & \cellcolor{green!27}Old & \cellcolor{green!27}Age - Over 65s & \cellcolor{green!27}1/23 & \cellcolor{green!27}4.348 \\  \hline
  \cellcolor{green!5}\small So...why are they allowed to ask if you are over the \textbf{age} of 18 and under the \textbf{age} of 40 on your job application? Isn't that illegal? We have a lot more years until retirement.\normalsize   & \cellcolor{green!5}Age & \cellcolor{green!5}Age - General & \cellcolor{green!5}2/35 & \cellcolor{green!5}5.714 \\  \hline
  \cellcolor{green!27}\small Good point. Some jobs application ask that question. I understand that some Companies want to make sure they hire someone under the appropriate \textbf{age} to manage some equipments, this only applies if you are over 18\normalsize   & \cellcolor{green!27}Age & \cellcolor{green!27}Age - General & \cellcolor{green!27}1/36 & \cellcolor{green!27}2.778 \\  \hline
  \cellcolor{green!5}\small They should not ask any \textbf{age} related questions and not \textbf{race} or anything else.\normalsize   & \cellcolor{green!5}Age, Race & \cellcolor{green!5}Age - General, Ethnicity - General & \cellcolor{green!5}2/14 & \cellcolor{green!5}14.286 \\  \hline
  \cellcolor{green!27}\small Great Video Linda..Informative. But I would like to know from you, being a human resource expert , Is there any sphere of \textbf{age} limiting for a job  particularly involving physical works ..?? If so glad to have tips to counter that and if it doesn't than to counter the fear/trauma  of \textbf{age} discrimination  Thank you.\normalsize   & \cellcolor{green!27}Age & \cellcolor{green!27}Age - General & \cellcolor{green!27}2/55 & \cellcolor{green!27}3.636 \\  \hline
  \cellcolor{green!5}\small I'm a 51 year \textbf{old} guy and I think I I've had massive \textbf{age} disease. I am well qualified and very good in tech, but I don't put my \textbf{age} and all experience on resumes so people think I'm younger . I do good on the interviews but don't get offers. All the people interviewing me seem younger. Also I look \textbf{old} got for my \textbf{age} I'm 51 but look more like 60\normalsize   & \cellcolor{green!5}Age, Old & \cellcolor{green!5}Age - General, Age - Over 65s & \cellcolor{green!5}5/73 & \cellcolor{green!5}6.849 \\  \hline
  \cellcolor{green!27}\small \@\textbf{Apple} Bomb wow\normalsize   & \cellcolor{green!27}Apple & \cellcolor{green!27}Ethnicity - Native-American & \cellcolor{green!27}1/3 & \cellcolor{green!27}33.333 \\  \hline
  \cellcolor{green!5}\small Let's see where she is when she's a \textbf{senior}....\normalsize   & \cellcolor{green!5}Senior & \cellcolor{green!5}Age - Over 65s & \cellcolor{green!5}1/9 & \cellcolor{green!5}11.111 \\  \hline
  \cellcolor{green!27}\small Right?  She looks all of 32 years \textbf{old}.  She is really not in a position to advise on this.\normalsize   & \cellcolor{green!27}Old & \cellcolor{green!27}Age - Over 65s & \cellcolor{green!27}1/19 & \cellcolor{green!27}5.263 \\  \hline
  \cellcolor{green!5}\small Age discrimination is real.\normalsize   & \cellcolor{green!5}Age & \cellcolor{green!5}Age - General & \cellcolor{green!5}1/4 & \cellcolor{green!5}25.0 \\  \hline
  \cellcolor{green!27}\small That is the DUMBEST interviewing technique to come down the pike. Past behavior does NOT predict future behavior or we could ADVERT many of the disasters the population has seen. It is really for IQ challenged LAZY HR people who don't know how to properly evaluate a prospect. Read the literate out there and it all says. PASSE, but these HR depts are sold on it. Tragic because many can ace that garbage (I have a degree in Human Behavior-I have credentials) and that determines \textbf{NOTHING} but the ability to BS an inadequate HR person. Doesn't get quality candidates.\normalsize   & \cellcolor{green!27}Nothing & \cellcolor{green!27}Age - Youngsters & \cellcolor{green!27}1/99 & \cellcolor{green!27}1.01 \\  \hline
  \cellcolor{green!5}\small How about a video if you are 50+ and want to re-locate to another state to obtain a job?  If managers see your resume' you have too much job experience they are already getting an opinion on your \textbf{age}.  If you tell them you want to move closer to family, they are assuming you are trying to get your expenses paid to move there.  I just want a chance to just to let them know in an interview, try me, you might like me better than your local choices and are worth the risk to take hiring out of state.\normalsize   & \cellcolor{green!5}Age & \cellcolor{green!5}Age - General & \cellcolor{green!5}1/100 & \cellcolor{green!5}1.0 \\  \hline
  \cellcolor{green!27}\small Hi, Linda! Your videos really help me a lot! Could you do a "walk me through your resume" video, cuz that's one of the most \textbf{common} questions in an interview. Really need advice on it. Thx!!!\normalsize   & \cellcolor{green!27}Common & \cellcolor{green!27}Social Class - Working class & \cellcolor{green!27}1/36 & \cellcolor{green!27}2.778 \\  \hline
  \cellcolor{green!5}\small Some people do not know that wine get better with age\normalsize   & \cellcolor{green!5}Age & \cellcolor{green!5}Age - General & \cellcolor{green!5}1/11 & \cellcolor{green!5}9.091 \\  \hline
  \cellcolor{green!27}\small If  someone is living within a majority of \textbf{ethnic} group, most establishment located within or close to this majority prefer to have their personnel selected from within this community for many reasons, communication, mentality, \textbf{common} views on business issues etc...\normalsize   & \cellcolor{green!27}Common, Ethnic & \cellcolor{green!27}Ethnicity - General, Social Class - Working class & \cellcolor{green!27}2/40 & \cellcolor{green!27}5.0 \\  \hline
  \cellcolor{green!5}\small Hi, I am really facing the overqualified issue, and sometimes \textbf{age}, in addition is that the jobs opportunity turned to be socially communal rather then professional.\normalsize   & \cellcolor{green!5}Age & \cellcolor{green!5}Age - General & \cellcolor{green!5}1/26 & \cellcolor{green!5}3.846 \\  \hline
  \cellcolor{green!27}\small Throughout this video you are aiming towards older people who want a career change/part-time-retirement job. Do this video again and talk about people who are secondary school students wanting to save up for something they want but parents can't afford. People fresh outta secondary school. Fresh outta Sixth Form/Collage. Uni students who need extra cash to pay their way. People fresh outta Uni.Cause when it comes to \textbf{ageism} 13-25 year olds are hit the hardest. Especially the youngest in the age-group.\normalsize   & \cellcolor{green!27}Ageism & \cellcolor{green!27}Age - General & \cellcolor{green!27}1/82 & \cellcolor{green!27}1.22 \\  \hline
  \cellcolor{green!5}\small Shameful, aggravating, irritating mediocre video. I derived  through this tired speech that you are young and just because you are  happily employed, you are judging others because of there \textbf{age}. They are doing \textbf{nothing} wrong during these interviews that are merely set up to hire people 40  And younger.\normalsize   & \cellcolor{green!5}Age, Nothing & \cellcolor{green!5}Age - General, Age - Youngsters & \cellcolor{green!5}2/49 & \cellcolor{green!5}4.082 \\  \hline
  \cellcolor{green!27}\small IT IS TRUE, THE JOB MARKET IS PREJUDICE AGAINST OLDER HIGHLY QUALIFIED  CITIZENS. THE MINDSET IS NEGATIVE ON THE SIDE OF THE SO CALLED BIASED INTERVIEWERS , WHO KNOW WE \textbf{HO} THEY WANT ALREADY.\normalsize   & \cellcolor{green!27}Ho & \cellcolor{green!27}Gender - Female sexuality & \cellcolor{green!27}1/34 & \cellcolor{green!27}2.941 \\  \hline
  \cellcolor{green!5}\small Oh yea! It's so great to see a NEW video from you. Love that you're talking all about \textbf{age} discrimination in hiring. Such a great topic.\normalsize   & \cellcolor{green!5}Age & \cellcolor{green!5}Age - General & \cellcolor{green!5}1/26 & \cellcolor{green!5}3.846 \\  \hline
  \cellcolor{green!27}\small So if I'm over 40 and I'm making a career change, hiring managers are going to automatically decided that I don't have the energy to do the job.  You've decided that because I'm over a certain \textbf{age} I don't want to grow. It's crap to tell someone to stay put since their closer to retirement \textbf{age}. What if you hate your current job because you are working with toxic people and its just time to leave.  What if you're moving to a new city? What if you get laid off? This just sounds like go into business for yourself or jump off a cliff because you are too \textbf{old} and obsolete. I know you are giving it to us straight but this is dismal.\normalsize   & \cellcolor{green!27}Age, Old & \cellcolor{green!27}Age - General, Age - Over 65s & \cellcolor{green!27}3/124 & \cellcolor{green!27}2.419 \\  \hline
  \cellcolor{green!5}\small I didn't find this video helpful. \textbf{Ageism} does exist and they are looking for someone who will stay with the company for 30 years and have energy and pay less. The number one of business cuts is salary... they tend to hire cheap ones.\normalsize   & \cellcolor{green!5}Ageism & \cellcolor{green!5}Age - General & \cellcolor{green!5}1/44 & \cellcolor{green!5}2.273 \\  \hline
  \cellcolor{green!27}\small ivygailmcc I think she is trying to cover some basic info to help with \textbf{ageism}. I'm finding that being over 40, in a new state and rusty at interviewing is is definitely making it harder for me. That cliff seems more appealing everyday\normalsize   & \cellcolor{green!27}Ageism & \cellcolor{green!27}Age - General & \cellcolor{green!27}1/43 & \cellcolor{green!27}2.326 \\  \hline
  \cellcolor{green!5}\small petswriter1 minute ago (edited)Very good advice. However, coming from someone who has done freelance and contract work, it in no way provides the consistency or benefits of a traditional job.  Can you meet federal requirements for health insurance?  It's not inexpensive. Ask yourself if you'll be making enough money to hire a personal accountant to handle billing, taxes and other financials. Contracts for freelancers or consultants sometimes state that the amount of work available and the actual pay can vary. When cuts must be made, the easiest place to cut is the contractors. They probably won't get rid of you entirely if you're doing a fantastic job. But make no mistake, freelancing is a tough road. Depending on your situation, you will likely be working far more than 40 hours a week. And ultimately, it's still giving older people the shorter end of the stick.   It was very disheartening to me when I updated my skills by taking graphic design courses at a local college. One of the things they said to a class of people \textbf{age} 18 and up, is to avoid freelancing if at all possible. One of my instructors taught an entire class on this, backed by stats and research.  He and other professors had experience as high fashion retouchers, ect., and they have worked for major companies as contractors. Yet they all had another job outside of their freelance business. For them, it was teaching. At the college level, it pays very well part-time. That is also another possibility and an industry where "\textbf{age}" or experience tends to have value and not act as a hindrance. Great video, but I say all of that to say this. Be prepared to work very hard and long as a contractor, probably harder than you did at a traditional job.\normalsize   & \cellcolor{green!5}Age & \cellcolor{green!5}Age - General & \cellcolor{green!5}2/303 & \cellcolor{green!5}0.66 \\  \hline
  \cellcolor{green!27}\small I have experienced \textbf{age} discrimination. Your points, especially point 3, is on point (alliteration intended).\normalsize   & \cellcolor{green!27}Age & \cellcolor{green!27}Age - General & \cellcolor{green!27}1/15 & \cellcolor{green!27}6.667 \\  \hline
  \cellcolor{green!5}\small Welcome back Linda.  Hope you enjoyed your Youtube sabbatical :0). I couldn't agree more with your assessment.  This is largely a \textbf{common} mindset issue.  Many times adjusting the mindset can be difficult.  Us "\textbf{old}" folks are sometimes slow to change. My personal experience:  all the best jobs were ones that I was referred to or known to the hiring manager --my reputation was validated prior to the hiring process being initiated. Resume becomes less critical to a hiring process with an active network.\normalsize   & \cellcolor{green!5}Common, Old & \cellcolor{green!5}Age - Over 65s, Social Class - Working class & \cellcolor{green!5}2/83 & \cellcolor{green!5}2.41 \\  \hline
  \cellcolor{green!27}\small This is such a great topic. I have more than one friend who are limiting themselves in their job search cos they think they are now too \textbf{old}. Although they are all in late twenties early thirties lol. Gonna share this video with them. Awesome video!\normalsize   & \cellcolor{green!27}Old & \cellcolor{green!27}Age - Over 65s & \cellcolor{green!27}1/46 & \cellcolor{green!27}2.174 \\  \hline
  \cellcolor{green!5}\small Apple Chai She's Vietnamese (not Chinese) and a Canadian citizen.\normalsize   & \cellcolor{green!5}Apple & \cellcolor{green!5}Ethnicity - Native-American & \cellcolor{green!5}1/10 & \cellcolor{green!5}10.0 \\  \hline
  
\end{longtable}
\end{center}


\centering\textbf{\large \hypertarget{Table 2}{Table 2}: Summary of the results per sociolinguistic variable 
}
\newcolumntype{C}[2]{>{\centering\arraybackslash}p{#1}}
\begin{center}
\setlength\mylength{\dimexpr\textwidth - 1\arrayrulewidth - 40\tabcolsep}
\begin{longtable}{|C{.50\mylength}|C{.30\mylength}|C{.15\mylength}|C{.15\mylength}|C{.15\mylength}|}
\hline
\textbf{Sociolinguistic variables (Hiper - Hipo)} & \textbf{KeyWords} & \textbf{Number of occurrences} & \textbf{Frequency}  & \textbf{Frequency(\%)} \\
\hline\multirow{1}{*}{\cellcolor{red!27}Age - General}  & \cellcolor{red!27}Age, Ageism & \cellcolor{red!27}63 & \cellcolor{red!27}63/6579& \cellcolor{red!27}0.96 \\  \hline
  \multirow{1}{*}{\cellcolor{red!5}Age - Over 65s}  & \cellcolor{red!5}Old, Senior & \cellcolor{red!5}13 & \cellcolor{red!5}13/6579& \cellcolor{red!5}0.2 \\  \hline
  \multirow{1}{*}{\cellcolor{red!27}Age - Youngsters}  & \cellcolor{red!27}Zero, Nothing & \cellcolor{red!27}3 & \cellcolor{red!27}3/6579& \cellcolor{red!27}0.05 \\  \hline
  \multirow{1}{*}{\cellcolor{red!5}Ethnicity - General}  & \cellcolor{red!5}Racism, Race, Ethnic & \cellcolor{red!5}3 & \cellcolor{red!5}3/6579& \cellcolor{red!5}0.05 \\  \hline
  \multirow{1}{*}{\cellcolor{red!27}Ethnicity - Asian (East- China, Japan, Korea, Philippines, Vietnam)}  & \cellcolor{red!27}Yellow & \cellcolor{red!27}1 & \cellcolor{red!27}1/6579& \cellcolor{red!27}0.02 \\  \hline
  \multirow{1}{*}{\cellcolor{red!5}Nationality - Chinese}  & \cellcolor{red!5}Yellow & \cellcolor{red!5}1 & \cellcolor{red!5}1/6579& \cellcolor{red!5}0.02 \\  \hline
  \multirow{1}{*}{\cellcolor{red!27}Nationality - Japanese}  & \cellcolor{red!27}Yellow & \cellcolor{red!27}1 & \cellcolor{red!27}1/6579& \cellcolor{red!27}0.02 \\  \hline
  \multirow{1}{*}{\cellcolor{red!5}Ethnicity - Native-American}  & \cellcolor{red!5}Apple & \cellcolor{red!5}2 & \cellcolor{red!5}2/6579& \cellcolor{red!5}0.03 \\  \hline
  \multirow{1}{*}{\cellcolor{red!27}Social Class - Working class}  & \cellcolor{red!27}Common & \cellcolor{red!27}3 & \cellcolor{red!27}3/6579& \cellcolor{red!27}0.05 \\  \hline
  \multirow{1}{*}{\cellcolor{red!5}Gender - Female sexuality}  & \cellcolor{red!5}Ho & \cellcolor{red!5}1 & \cellcolor{red!5}1/6579& \cellcolor{red!5}0.02 \\  \hline
  
\end{longtable}
\end{center}


\textbf{\Large Result analysis:}

\begin{itemize}\item Taking into account the words that were detected, we can reach the conclusion these comments are associated with : : Age - General;Age - Over 65s;Age - Youngsters;Ethnicity - General;Ethnicity - Asian (East- China, Japan, Korea, Philippines, Vietnam);Nationality - Chinese;Nationality - Japanese;Ethnicity - Native-American;Social Class - Working class;Gender - Female sexuality;%.

\item The percentage of hate speech related words is 1.3832.

\item Considering that the variable \textbf{Age - General} has the most occurences in the post, we can interpret that this is the predominant hate speech.

\item Overall there were 92/154 occurences of hate speech related comments.\end{itemize}\end{document}