\documentclass[11pt]{article}
\ExplSyntaxOn
\let\tl_length:n\tl_count:n
\ExplSyntaxOff
\usepackage{graphicx}
\usepackage{multirow}
\usepackage{colortbl}
\usepackage{longtable, array}
\usepackage{hyperref}
\usepackage[usenames,dvipsnames,svgnames,table]{xcolor}
\newlength\mylength
\usepackage[legalpaper, landscape, margin=0.8in]{geometry}
\newcommand{\MinNumber}{0}
\begin{document}

\textbf {\huge In youtube\_extraction\_english\_127.json :}\newline \par\Large\textbf {Title: \large  How IBM quietly pushed out 20,000 older workers }\newline {\par\large --- Table  1: Summary of the results per comment; }

 {\par\large --- \hyperlink{Table 2}{\textcolor{blue}{\underline{Table 2}}}: Summary of the results per sociolinguistic variable;}\newline \normalsize\newline

\centering\textbf{\large Table  1: Summary of the results per comment 
}
\newcommand{\MaxNumber}{0}%
\newcommand{\ApplyGradient}[1]{%
\pgfmathsetmacro{\PercentColor}{100.0*(#1-\MinNumber)/(\MaxNumber-\MinNumber)}
\xdef\PercentColor{\PercentColor}%
\cellcolor{LightSpringGreen!\PercentColor!LightRed}{#1}
}
\newcolumntype{C}[2]{>{\centering\arraybackslash}p{#1}}
\begin{center}
\setlength\mylength{\dimexpr\textwidth - 1\arrayrulewidth - 50\tabcolsep}
\begin{longtable}{|C{.65\mylength}|C{.30\mylength}|C{.12\mylength}|C{.12\mylength}|C{.12\mylength}|}
\hline
\textbf{Comment} & \textbf{KeyWords} & \textbf{Sociolinguistic variables (Hiper - Hipo)}  & \textbf{Hate Speech Frequency} & \textbf{Hate Speech Frequency(\%)} \\
\hline\cellcolor{green!27}\small If all their employees are \textbf{old}, they will all retire in 15 years and leave them with \textbf{nothing}. Most tech companies have healthy high turnover to make this a non-issue.\normalsize   & \cellcolor{green!27}Nothing, Old & \cellcolor{green!27}Age - Over 65s, Age - Youngsters & \cellcolor{green!27}2/30 & \cellcolor{green!27}6.667 \\  \hline
  \cellcolor{green!5}\small Ageism the -ism that affects everybody.\normalsize   & \cellcolor{green!5}Ageism & \cellcolor{green!5}Age - General & \cellcolor{green!5}1/6 & \cellcolor{green!5}16.667 \\  \hline
  \cellcolor{green!27}\small For anyone who is an IT graduate and you want to get a job you will be able to keep forever:  Get a job with the government.  Sure, work for IBM for those 10 years from \textbf{age} 28 to \textbf{age} 38, but when you get fired for some made-up reason, go and apply for a government position managing computer resources, (or even working as a janitor somewhere), and your job will be 10-times more secure than for working for a private company.\normalsize   & \cellcolor{green!27}Age & \cellcolor{green!27}Age - General & \cellcolor{green!27}2/82 & \cellcolor{green!27}2.439 \\  \hline
  \cellcolor{green!5}\small GE does this too. They did this to my mother, said she missed to many days but hadn't used anymore than what they said she was allowed.   Too many \textbf{doctors} appointments and they let her go.  Turns out she had cancer, we learned this after she was let go.  After some talks, GE settled, and paid my mother a monthly retirement.\normalsize   & \cellcolor{green!5}doctors & \cellcolor{green!5}Nationality - General & \cellcolor{green!5}1/61 & \cellcolor{green!5}1.639 \\  \hline
  \cellcolor{green!27}\small Seeing \textbf{old} tech makes me wanna go outside and be active🤮\normalsize   & \cellcolor{green!27}Old & \cellcolor{green!27}Age - Over 65s & \cellcolor{green!27}1/11 & \cellcolor{green!27}9.091 \\  \hline
  \cellcolor{green!5}\small The future is now \textbf{old} man\normalsize   & \cellcolor{green!5}Old & \cellcolor{green!5}Age - Over 65s & \cellcolor{green!5}1/6 & \cellcolor{green!5}16.667 \\  \hline
  \cellcolor{green!27}\small Man these people are truly Dinosaurs, \textbf{nobody} stays 20 years in a company anymore. Go ask young people.\normalsize   & \cellcolor{green!27}Nobody & \cellcolor{green!27}Age - Youngsters & \cellcolor{green!27}1/18 & \cellcolor{green!27}5.556 \\  \hline
  \cellcolor{green!5}\small I wasn't even \textbf{old} when they pushed me out. I was given a choice to go live in banjo land or leave.\normalsize   & \cellcolor{green!5}Old & \cellcolor{green!5}Age - Over 65s & \cellcolor{green!5}1/22 & \cellcolor{green!5}4.545 \\  \hline
  \cellcolor{green!27}\small \@Majora Yes but these tectonic shifts dont happen every 5 years ! In IT it happens often. I am a python programmer. I was a \textbf{nobody} earlier. Now with the advent of big data, I am somebody :) Already Machine Learning is nipping at my heals :(\normalsize   & \cellcolor{green!27}Nobody & \cellcolor{green!27}Age - Youngsters & \cellcolor{green!27}1/47 & \cellcolor{green!27}2.128 \\  \hline
  \cellcolor{green!5}\small So are those younger people just supposed to waste their degrees on another job when they've worked hard to get the education to fill it? I mean it's only natural to move the \textbf{old} workers out and bring in new workers as the older ones have for many decades now, been working longer, meaning positions that would have normally invited in new fresh thinking to the company instead forcing a gap in the generations work force now creating a roadblock, with older workers  now being kept sometimes an entire decade longer. The workers that seem to be out of luck more than likely haven't saved their large salaries instead opting to buy nice stuff. What companies should do it help organize their workforce to save money for retirement and make it clear in the beginning that current methods are allowing workers to stay far to long she it's like keeping 10 boxes on a conveyer belt after it passes it's destination when it's time to put 100 new ones on it, and the belt only holds 100..\normalsize   & \cellcolor{green!5}Old & \cellcolor{green!5}Age - Over 65s & \cellcolor{green!5}1/177 & \cellcolor{green!5}0.565 \\  \hline
  \cellcolor{green!27}\small This makes me livid with anger. I thought 50 was the cut off, the disposal \textbf{age}, and now it appears to be 40. Older workers are deemed useless, but older executives are considered valuable and worth millions in compensation. What hypocrisy. Are they not subject to the same degradations of \textbf{age} - rotting neurons and slowed reflexes, collapsing health? The only solution is for workers of all ages to disavow loyalty and to always seek greener pastures, and to ensure their own retirement and security.\normalsize   & \cellcolor{green!27}Age & \cellcolor{green!27}Age - General & \cellcolor{green!27}2/85 & \cellcolor{green!27}2.353 \\  \hline
  \cellcolor{green!5}\small Randall Young  Apparently but it still doesn't make sense. Skill takes time to built up in any industry. And you could have a bunch of script kiddies and ‘just learned me a framework' hopping around for almost \textbf{nothing}, that's true. But none of them will have a real sense of what they're actually doing with the machines. Just trial and error.I'm a computer science student and.. I know how limited my knowledge is. If I'd have to choose between hiring me or one of my professors or docents, the choice would be so clear to me.\normalsize   & \cellcolor{green!5}Nothing & \cellcolor{green!5}Age - Youngsters & \cellcolor{green!5}1/97 & \cellcolor{green!5}1.031 \\  \hline
  \cellcolor{green!27}\small I'm 58 and I have almost always had multiple job offers, including many former employers.Had a layoff only once for two weeks when I was young due to business turn-down.The trick is to stay as current and knowledgeable in your field  as possible - Look for trends  in your field that may give you clues as to where it is headed.  But do not limit yourself to one field either. Be willing to work  sometimes unpleasant tasks without complaint - be flexible with work hours, assignments and tasks.   Get as much training as you possibly can, especially outside your comfort zone.  Your have to be willing to change jobs and employers often.   Don't get pigeonholed into repetitive tasks.  I change companies on average about every 5 years on my own initiative in order to maximize wages, because I get the highest wage increases when I move to a new employer.  Then try to save and invest   some percent of your extra income.  Don't waste your money buying ready made meals, make your own coffee and sandwiches at home.  Don't saddle yourself with \textbf{crazy} amounts of debt  if possible.Find your happiness outside work, work is to make money period ...Think of your self as a product that the employer needs, and make sure you are the best product he can buy by continuously self-improving.Be ready at all times for layoff!   Don't let your employer define who you are and have no fear of a layoff. I have purchased equipment that would help me start my own business if I had to, but I have not needed to yet so it will be re-sold upon my  "retirement."Do not be loyal to the company besides giving an honest days work, especially in an environment where it won't be recognized and rewarded.    You must look out for yourself and your family first.(edit) And don't let yourself get too stressed out...it can lead to bad health and death and then you won't be any good to anyone...if your job stress affects your health you need to bail...\normalsize   & \cellcolor{green!27}Crazy & \cellcolor{green!27}Physical Identity - Physical (and Mental) Impairments & \cellcolor{green!27}1/349 & \cellcolor{green!27}0.287 \\  \hline
  \cellcolor{green!5}\small well said Mike ! I am 51 years \textbf{old} and I agree 100\normalsize   & \cellcolor{green!5}Old & \cellcolor{green!5}Age - Over 65s & \cellcolor{green!5}1/13 & \cellcolor{green!5}7.692 \\  \hline
  \cellcolor{green!27}\small Mike Poulin, good points that I've known my entire adult life...it's why I went back to college at 30...I'm 59, soon to be 60.  Through a path, not of my choosing, I became \textbf{disabled}. I would say that not one of us should ever make the mistake of believing we'll work until retirement. Look ahead and save as much as you can just in case the unexpected happens. I was lucky in that I'd had steady jobs with good pay, so my disability payments are higher than lots of people I've talked to. It's not like having a work paycheck and I miss being in the workforce. So never forget that our lives can completely change very quickly... be prepared. Much luck to all.\normalsize   & \cellcolor{green!27}Disabled & \cellcolor{green!27}Physical Identity - Physical (and Mental) Impairments & \cellcolor{green!27}1/124 & \cellcolor{green!27}0.806 \\  \hline
  \cellcolor{green!5}\small Back before soc sec, if you were over 40, you could not get work - they ended up in horrible poor houses. Now they want to raise \textbf{age} for soc sec and not keep or hire older workers, how is that going to work out in long run? Great thank you for American workers who worked and brought in production profits, only to be thrown away when money can be saved. The so called invisible hand at work, only for supply side.\normalsize   & \cellcolor{green!5}Age & \cellcolor{green!5}Age - General & \cellcolor{green!5}1/82 & \cellcolor{green!5}1.22 \\  \hline
  \cellcolor{green!27}\small This is a problem with capitalism. It cannot be fixed by saying \textbf{age} discrimination is not good. Profit is the number one motive of any company and will always remain so. If they can fire 20k workers and the fines are less than the savings, they will do it.Capitalism is a inheranrly broken system, designed to create instability and siphon wealth out of the working class.\normalsize   & \cellcolor{green!27}Age & \cellcolor{green!27}Age - General & \cellcolor{green!27}1/67 & \cellcolor{green!27}1.493 \\  \hline
  \cellcolor{green!5}\small The Unions in Boeing allowed the same stuff, especially women. They'd get close to retirement like by months and get 'laid off' and never called back. They would appeal to the Union they had paid dues to for years and the Union would ignore them or laugh in their faces. Unions like to lie about what they really are about which is lining their own pockets. Wow, doing some reading about IBM. No layoffs for seven decades?! They used to be a cradle to grave corporation but have had major changes. Can you say \textbf{Apple}? Or Microsoft? Competition has tightened the \textbf{old} belt. Very interesting! Will read more, thanks!\normalsize   & \cellcolor{green!5}Apple, Old & \cellcolor{green!5}Age - Over 65s, Ethnicity - Native-American & \cellcolor{green!5}2/109 & \cellcolor{green!5}1.835 \\  \hline
  \cellcolor{green!27}\small Another \textbf{woman} CEO.\normalsize   & \cellcolor{green!27}Woman & \cellcolor{green!27}Gender - General & \cellcolor{green!27}1/3 & \cellcolor{green!27}33.333 \\  \hline
  \cellcolor{green!5}\small my dad who works in IBM told me that the workers that were layed off are the ones who have done the same thing for decades. He said that the company needs younger people with more energy and motivation to evolve the company into something different. The older generation may be what made IBM great, but they will not be the ones to make IBM great today.My dad even told me that they layed off a 50 year \textbf{old} guy. He was sad because the guy was a genius back in the day, but my dad said he was the kind of person who played mini golf all the time LOL.\normalsize   & \cellcolor{green!5}Old & \cellcolor{green!5}Age - Over 65s & \cellcolor{green!5}1/112 & \cellcolor{green!5}0.893 \\  \hline
  \cellcolor{green!27}\small IBM is now an \textbf{Indian} company head quartered in USA.\normalsize   & \cellcolor{green!27}Indian & \cellcolor{green!27}Ethnicity - Native-American & \cellcolor{green!27}1/10 & \cellcolor{green!27}10.0 \\  \hline
  \cellcolor{green!5}\small OMG... did these people really believe that IBM was going to save them or give them a job for life?  Ever since the mass layoffs of the \textbf{dot} com implosion of 2001, there is nomore loyalty in IT...    and this is new to IBM employees?   What rock were they hiding under?\normalsize   & \cellcolor{green!5}Dot & \cellcolor{green!5}Ethnicity - Asian (South- India, Pakistan, Bangladesh) & \cellcolor{green!5}1/51 & \cellcolor{green!5}1.961 \\  \hline
  \cellcolor{green!27}\small IBM has become the worse technology company in America. It is too scared to fund any product or idea organically due to constant pressure to cut cost. IBM buys new product companies from the market and takes out all the juice out of it like a sugarcane crushing machine until only wood is left. The day a company is bought by IBM, people should start looking for job elsewhere. I feel bad for \textbf{R\textbf{ed}} Hat people as they are now being prepared for the sugarcane crashing machine,\normalsize   & \cellcolor{green!27}Red & \cellcolor{green!27} Ideological and Political Identity - General, Ethnicity - Native-American & \cellcolor{green!27}2/87 & \cellcolor{green!27}2.299 \\  \hline
  \cellcolor{green!5}\small The last aerial shot secretly explained the reason of those firings. Thats the Budapest center. 1 \textbf{senior} US worker salary can be transformed to 2 or 3 in eastern European.\normalsize   & \cellcolor{green!5}Senior & \cellcolor{green!5}Age - Over 65s & \cellcolor{green!5}1/30 & \cellcolor{green!5}3.333 \\  \hline
  \cellcolor{green!27}\small Age discrimination probably the most undetectable thing on planet earth\normalsize   & \cellcolor{green!27}Age & \cellcolor{green!27}Age - General & \cellcolor{green!27}1/10 & \cellcolor{green!27}10.0 \\  \hline
  \cellcolor{green!5}\small I worked for IBM as a contractor from 2007 through 2014. During that period there were at least five layoffs that I was aware of. Nearly all of the non-contract employees affected were long-term employees with 20 or more years at IBM. To be fair, they were given fairly generous severance packages and most were pension eligible. However, most were not ready to retire and were essentially forced into it. Bottom line is IBM management was eliminating older pension-eligible workers in favor of younger workers with limited benefits and no pension. It's \textbf{age} discrimination for profit, plain and simple.I actually felt lucky that I was contract, I survived those layoffs mainly because I was a fixed, hourly-rate employee with no benefits. By all rights I should have been let go in favor of a direct employee, but it was all about the \$\$, employees be damned. When they talk about how their employees are their most valued resource, etc., they are flat-out lying.Towards the end was forced to accept a rate cut, then management decided they would arbitrarily furlough the contractors for weeks at a time (gotta make those year-end numbers look good). I left in the summer of 2014 and haven't looked back, better job now, more money and full benefits. IBM is not a company I would ever work for again. If you're just graduating with a STEM degree and thinking about applying there, don't walk but RUN elsewhere.\normalsize   & \cellcolor{green!5}Age & \cellcolor{green!5}Age - General & \cellcolor{green!5}1/243 & \cellcolor{green!5}0.412 \\  \hline
  \cellcolor{green!27}\small too bad technology is a field where the newest techniques and skills are the most important part of your relevance. A guy 5 yrs into his career is vastly more likely to know those skills right off the bat, is cheaper to employ, as well as not likely to be nearing the \textbf{age} where someone retires on their own\normalsize   & \cellcolor{green!27}Age & \cellcolor{green!27}Age - General & \cellcolor{green!27}1/59 & \cellcolor{green!27}1.695 \\  \hline
  \cellcolor{green!5}\small I work at a store that the register is an \textbf{old} IBM monitor bet you can't guess the store\normalsize   & \cellcolor{green!5}Old & \cellcolor{green!5}Age - Over 65s & \cellcolor{green!5}1/19 & \cellcolor{green!5}5.263 \\  \hline
  \cellcolor{green!27}\small they got rid of the the higher wage people and they didnt want to pay all the health care and retirement. it has \textbf{nothing} to do with soley their \textbf{age}. thats why i think this video is disinginuious\normalsize   & \cellcolor{green!27}Age, Nothing & \cellcolor{green!27}Age - General, Age - Youngsters & \cellcolor{green!27}2/38 & \cellcolor{green!27}5.263 \\  \hline
  \cellcolor{green!5}\small Google and microsoft didnt take thr 'lead'. Ibm just switched to company to company business. Look into it its \textbf{crazy} how much systems are based on ibm software. In addition I think it stupid to cry about the 20000 workers that 'disappeared'. I havent recognized any of this and I also work at IBM. Almost everywhere it is the trend to employ more younger workes because they depend to feel more comfortable with this technological era. Tbh I think this video just showed some 'facts' without any concept or real proof. Just think about what all those magazine etc. do these days. Headlining big companies or celebrities to get the most clicks and attention. Feel free to discuss. I really want to hear other opinions.\normalsize   & \cellcolor{green!5}Crazy & \cellcolor{green!5}Physical Identity - Physical (and Mental) Impairments & \cellcolor{green!5}1/125 & \cellcolor{green!5}0.8 \\  \hline
  \cellcolor{green!27}\small Its what happens when you let an unexperienced kid do the job of an \textbf{old} timer\normalsize   & \cellcolor{green!27}Old & \cellcolor{green!27}Age - Over 65s & \cellcolor{green!27}1/16 & \cellcolor{green!27}6.25 \\  \hline
  \cellcolor{green!5}\small Older worker doesn't mean smarter + more experienced. But More expensive? definitely. I worked with people twice my \textbf{age} but the skill difference in technology? not that much, some are even worse than me. But i do met people in their 35 - 45 that are scarily smart.\normalsize   & \cellcolor{green!5}Age & \cellcolor{green!5}Age - General & \cellcolor{green!5}1/48 & \cellcolor{green!5}2.083 \\  \hline
  \cellcolor{green!27}\small If you hate someone older because you fell they are outdated or slower....remember one day you'll be their \textbf{age}.\normalsize   & \cellcolor{green!27}Age & \cellcolor{green!27}Age - General & \cellcolor{green!27}1/19 & \cellcolor{green!27}5.263 \\  \hline
  \cellcolor{green!5}\small I promise, this video has 0 right information and only makes you feel bad about \textbf{old} people.\normalsize   & \cellcolor{green!5}Old & \cellcolor{green!5}Age - Over 65s & \cellcolor{green!5}1/17 & \cellcolor{green!5}5.882 \\  \hline
  \cellcolor{green!27}\small The thing with tech is that its constantly evolving and as the brain ages, it doesn't learn new tricks as easily as before. This means that older employees are dead weight in tech firms. This is actually better for younger employees, because older employees who just sit on their past accomplishments can slow down change, this is why governments are often slow and unwieldy, because \textbf{old} \textbf{fat} cats sit on their pensions and \textbf{nothing} gets done.\normalsize   & \cellcolor{green!27}Fat, Nothing, Old & \cellcolor{green!27}Age - Over 65s, Age - Youngsters, Physical Identity - Physical Features & \cellcolor{green!27}3/76 & \cellcolor{green!27}3.947 \\  \hline
  \cellcolor{green!5}\small You can lay people for financial reasons, order people are usually paid more so if you have layoffs for cost savings you get rid of more older workers.Suprenes Court has said that is not \textbf{age} discrimination.\normalsize   & \cellcolor{green!5}Age & \cellcolor{green!5}Age - General & \cellcolor{green!5}1/37 & \cellcolor{green!5}2.703 \\  \hline
  \cellcolor{green!27}\small I thought that company went out of business a long time ago. I guess they didn't. I don't really see them around but I'm also not looking for that kind of a computer. I remember back in the 80s we used to call \textbf{Old} Doss computers that ran Microsoft Doss IBM compatible.\normalsize   & \cellcolor{green!27}Old & \cellcolor{green!27}Age - Over 65s & \cellcolor{green!27}1/52 & \cellcolor{green!27}1.923 \\  \hline
  \cellcolor{green!5}\small So what?  No one wants to work with \textbf{old} ass geezers.  They refuse to adapt to changes in the workplace and are a burden to companies that are attempting to move their business forward.  Fire em all.  Walmart needs more receipt checkers anyway.\normalsize   & \cellcolor{green!5}Old & \cellcolor{green!5}Age - Over 65s & \cellcolor{green!5}1/43 & \cellcolor{green!5}2.326 \\  \hline
  \cellcolor{green!27}\small 50 years \textbf{old} in 3 weeks after 32 years at my Job. Since the \textbf{age} of 18 and now I'm too \textbf{old}? It happens everywhere. Now I just completed my CDL License so now I'm ready too do something else. 🤷‍♂️🤯🤬\normalsize   & \cellcolor{green!27}Age, Old & \cellcolor{green!27}Age - General, Age - Over 65s & \cellcolor{green!27}3/41 & \cellcolor{green!27}7.317 \\  \hline
  \cellcolor{green!5}\small Tbh IBM feels \textbf{old} to me, like it feels like it should be left in the past.\normalsize   & \cellcolor{green!5}Old & \cellcolor{green!5}Age - Over 65s & \cellcolor{green!5}1/17 & \cellcolor{green!5}5.882 \\  \hline
  \cellcolor{green!27}\small Young people work for cheap. I'm 35 I have to find something long term before 40. Your \textbf{race} and \textbf{gender} don't matter folks once you hit 40 and you make good pay......god help you it's so sad.\normalsize   & \cellcolor{green!27}Gender, Race & \cellcolor{green!27}Ethnicity - General, Gender - General & \cellcolor{green!27}2/37 & \cellcolor{green!27}5.405 \\  \hline
  \cellcolor{green!5}\small Node//Data - Stop being greedy and work for the minimum wage like everyone else. Just because you're \textbf{old}, doesn't mean you're entitled to anything.\normalsize   & \cellcolor{green!5}Old & \cellcolor{green!5}Age - Over 65s & \cellcolor{green!5}1/24 & \cellcolor{green!5}4.167 \\  \hline
  \cellcolor{green!27}\small Understand the sentiments; but, 30 years in one company and probably doing the same work (or version of it) for such long! That too with a technology company that would want to evolve and bring in new perspective all the time... IBM is not at complete fault. Loyalty means \textbf{nothing} in business world to employer or employee, unless it's a profitable marriage between the two. In my  opinion- Changing jobs \& roles is important to bring versatility in one's outlook as well.\normalsize   & \cellcolor{green!27}Nothing & \cellcolor{green!27}Age - Youngsters & \cellcolor{green!27}1/82 & \cellcolor{green!27}1.22 \\  \hline
  \cellcolor{green!5}\small Career IT guy here.   A lot of those older workers had strong IBM mainframe skills in an era when demand for such skills was decreasing.   I worked with some of them in various mainframe shops.   Some of them were very good, others were all talk and no deliverable or the deliverable they produced was something you didn't ask for. By the late eighties the "Service Reps" had become parasitic sales reps.   As the migration to Unix based solutions became more of a stampede the market for mainframe skills tanked.    You had to be very good, or you needed to learn new skills to stay in the game. Given the vast change in demand for skills;  The marketplace went from IBM Mainframe, Cobol, CICS IMS in the early eighties to \textbf{R\textbf{ed}} Hat Linux, web based apps and Oracle PL/SQL by 2010.   Why IBM didn't train the existing work force in new skills remains unanswered.   The most likely answer is because anyone who succeeds at the training and gains experience in the new skills is likely to be recruited by headhunters.   Another answer is the current crop of college grads already have academic experience in Unix, Linux, c, MariaDB and SQL so why invest in training when people with the needed experience can be hired at entry level salaries.    The H1B visa has provided a large pool of foreign workers. I'll note that I took steps to improve my skills several years ago and learned enough Unix/Linux to live in it every day.    Linux sysname 4.16.16-300.fc28.x86 64 1 SMP Sun Jun 17 03:02:42 UTC 2018 x86 64 x86 64 x86 64 GNU/Linux   It isn't always the case that the workers got \textbf{old}, it's that their skills became obsolete and they didn't keep up with the marketplace.\normalsize   & \cellcolor{green!5}Old, Red & \cellcolor{green!5} Ideological and Political Identity - General, Age - Over 65s, Ethnicity - Native-American & \cellcolor{green!5}3/290 & \cellcolor{green!5}1.034 \\  \hline
  \cellcolor{green!27}\small Don't bother telling young people how soon they will also be \textbf{old} - they won't listen.\normalsize   & \cellcolor{green!27}Old & \cellcolor{green!27}Age - Over 65s & \cellcolor{green!27}1/16 & \cellcolor{green!27}6.25 \\  \hline
  \cellcolor{green!5}\small 'Relocate or resign' is a ridiculous loophole and illegal in many advanced economies. E.g. I think in the UK you have to offer re-employment within a 20-mile radius of the \textbf{old} job, or a reasonable severance package if that's not available.\normalsize   & \cellcolor{green!5}Old & \cellcolor{green!5}Age - Over 65s & \cellcolor{green!5}1/41 & \cellcolor{green!5}2.439 \\  \hline
  \cellcolor{green!27}\small Unfortunately, forced retirement of older workers is standard practice in some countries, such as South Korea and China.  In those countries, it is harder for people even over 40 to get jobs.  Employers often have explicit policies that they only hire people (especially women) under 40 years old; that's middle \textbf{age}!  Doesn't make sense because older workers have more experience. \textbf{Gender} and \textbf{ethnic} discrimination is also \textbf{common} and often undisputed in those countries.\normalsize   & \cellcolor{green!27}Age, Common, Ethnic, Gender & \cellcolor{green!27}Age - General, Ethnicity - General, Gender - General, Social Class - Working class & \cellcolor{green!27}4/73 & \cellcolor{green!27}5.479 \\  \hline
  \cellcolor{green!5}\small And yet the republicans want to keep pushing the retirement \textbf{age} above \textbf{age} 70 because people are living longer..... Living longer doesn't mean working longer if \textbf{nobody} hires you or you are incapable of starting your own business. Not to mention there will be fewer and fewer jobs with the A.I. revolution happening around us.\normalsize   & \cellcolor{green!5}Age, Nobody & \cellcolor{green!5}Age - General, Age - Youngsters & \cellcolor{green!5}3/55 & \cellcolor{green!5}5.455 \\  \hline
  \cellcolor{green!27}\small And how does this surprise anyone? I remember when I was a kid in the 1980s my friend's father was an electronic engineer who never fully merged his analog education with the digital computer \textbf{age}. He had trouble keeping high paying jobs.\normalsize   & \cellcolor{green!27}Age & \cellcolor{green!27}Age - General & \cellcolor{green!27}1/42 & \cellcolor{green!27}2.381 \\  \hline
  \cellcolor{green!5}\small Henry TheGreatAmerican - Well he wasn't smart enough to switch and stay an active learner through education. It's no one else's fault but his. If you don't have the experience out you go. A company owes you \textbf{nothing}.\normalsize   & \cellcolor{green!5}Nothing & \cellcolor{green!5}Age - Youngsters & \cellcolor{green!5}1/38 & \cellcolor{green!5}2.632 \\  \hline
  \cellcolor{green!27}\small Yeah because \textbf{old} people are only allowed at the TOP of the company, not as part of the workforce\normalsize   & \cellcolor{green!27}Old & \cellcolor{green!27}Age - Over 65s & \cellcolor{green!27}1/19 & \cellcolor{green!27}5.263 \\  \hline
  \cellcolor{green!5}\small Google was never, is not and will never be a competitor of IBM. It would be nice if this video presented some information relevant to IBM's competitors. How about Oracle, Sun, Digital Equipment, Dell, \textbf{Apple}, Hewlett Packard. These companies took a bite out of IBM even though some of them are no longer around.\normalsize   & \cellcolor{green!5}Apple & \cellcolor{green!5}Ethnicity - Native-American & \cellcolor{green!5}1/54 & \cellcolor{green!5}1.852 \\  \hline
  \cellcolor{green!27}\small Older employees generally have higher salaries and compensation due to seniority of job titles, and are more likely to be in management positions. When conducting layoffs to restructure or cut costs, high paying and management positions are often the first eliminated. It's not that IBM targeted older workers, it's just that IBM targeted higher paying positions and management positions in its elimination of positions, which happened to impact older workers more.As for the "their performance review discriminatinated against older workers":No, it didn't. Generally, you would expect more of a \textbf{senior} employee than a younger one, as they've had more time to learn, adapt to the environment of the company, and they're getting paid more. So obviously you're not going to hold a fresh grad making \$45,000 a year to the same standards as a \textbf{senior} developer making \$120,000 a year. With a fresh grad that needs a lot of assistance, I'd be willing to work with them to get their performance up to invest in that employee. If someone that's been at the company 25 years is needed constant supervision and struggling to perform well, and making 6 figures, I'd be skeptical of that employee's value to the company.Lastly, they offered far better retirement and severence packages than is required by law. They could've fired their employees and provided \textbf{nothing}, but instead they provided many months of severance pay and early retirement packages, which were quite nice. If I was working at IBM and was offered an early retirement package, I probably would've taken it.\normalsize   & \cellcolor{green!27}Nothing, Senior & \cellcolor{green!27}Age - Over 65s, Age - Youngsters & \cellcolor{green!27}3/258 & \cellcolor{green!27}1.163 \\  \hline
  \cellcolor{green!5}\small Old people can't innovate social media apps. Technology millennials can't intuitively design adult diapers.\normalsize   & \cellcolor{green!5}Old & \cellcolor{green!5}Age - Over 65s & \cellcolor{green!5}1/14 & \cellcolor{green!5}7.143 \\  \hline
  \cellcolor{green!27}\small The thing is you don't want the tech to get old\normalsize   & \cellcolor{green!27}Old & \cellcolor{green!27}Age - Over 65s & \cellcolor{green!27}1/11 & \cellcolor{green!27}9.091 \\  \hline
  \cellcolor{green!5}\small Company is company, not government. Their only job is to be competitive. The promise in \textbf{old} time is like lover's promise. It is only true at the promising moment. There is no guarantee that it will be true forever! If there is a forever guarantee from IBM, then IBM is wrong. If not, I think there is no way to complain.\normalsize   & \cellcolor{green!5}Old & \cellcolor{green!5}Age - Over 65s & \cellcolor{green!5}1/61 & \cellcolor{green!5}1.639 \\  \hline
  \cellcolor{green!27}\small Corporations run the countries, and they are above the law. They do whatever they want, and \textbf{nothing} happens to them. Want to commit crimes? Be a CEO.\normalsize   & \cellcolor{green!27}Nothing & \cellcolor{green!27}Age - Youngsters & \cellcolor{green!27}1/27 & \cellcolor{green!27}3.704 \\  \hline
  \cellcolor{green!5}\small Comparing the performance of an individual with 40 years of experience and one with 5 and expecting them to perform at the same level isn't really reasonable, so the argument about the "job security" rating being different than the "are you ready to be promoted" metric being evidence of \textbf{ageism} is a very weak one.  If I've been working at a company 2 years and I'm already performing on par with someone who's been in the same position for 30 then that person is slacking.  It makes sense that the longer you work there the more effective you need to be to be considered a good investment because competence should increase over time.The problem fundamentally is that productivity doesn't increase forever.  People don't really want to acknowledge that the older we get the less productive we get as things outside of work take priority like expanding family responsibilities, and eventually, health concerns.  If my productivity goes down by 20 overnight I'm liable to be fired, but if that same drop happens over a 20 year period while my pay continues to rise is it \textbf{age} discrimination?We all get \textbf{old}, and people making these decisions skew older.  \textbf{Age} discrimination is complicated at best, and completely fictional at worst.  We've all experienced the person who's been in the department for 30 years and contributes half of what the trainees do but is kept around on pure inertia.  Using the same language and framing for this as you would for racial discrimination is intellectually dishonest.  Firing someone because you identify them as a poor financial investment is not the same as firing someone because you don't like the color of their skin.  The job of corporations is to make money, the job of government is to take care of their people.  Corporations are not your family, your government, or your feudal overlord.  They're machines made for generating profit.\normalsize   & \cellcolor{green!5}Age, Ageism, Old & \cellcolor{green!5}Age - General, Age - Over 65s & \cellcolor{green!5}4/317 & \cellcolor{green!5}1.262 \\  \hline
  \cellcolor{green!27}\small > In the past few decades, rulings in \textbf{age} discrimination cases have said former employers must prove that there were no factors other than \textbf{age} involved in their job changes.Yes, that's the crux of any discrimination case. Plus, if \textbf{age} were the sole factor, IBM would have no reason to fire the employees. IBM is a business, they don't just wake up in the morning thinking "oh, well wouldn't it be great to fire some of our older employees", they set policy based on what they think will give them better performance. It's entirely possible that there's a correlation between \textbf{age} and other factors which make it hard to justify keeping an employee.Not saying they're not discriminating, I can't know that, but you can't just put all this stuff out here and expect it to line up perfectly into a fair case against IBM's actions.\normalsize   & \cellcolor{green!27}Age & \cellcolor{green!27}Age - General & \cellcolor{green!27}4/147 & \cellcolor{green!27}2.721 \\  \hline
  \cellcolor{green!5}\small more like IBM disappeared for 12 years as HP, \textbf{apple}, and dell overpowered the corporate computer industry\normalsize   & \cellcolor{green!5}Apple & \cellcolor{green!5}Ethnicity - Native-American & \cellcolor{green!5}1/17 & \cellcolor{green!5}5.882 \\  \hline
  \cellcolor{green!27}\small It's scary, but \textbf{age} discrimination is rampant in IT. That computer science major doesn't look that lucrative anymore when you have to build experience and get a management position by 40.\normalsize   & \cellcolor{green!27}Age & \cellcolor{green!27}Age - General & \cellcolor{green!27}1/31 & \cellcolor{green!27}3.226 \\  \hline
  \cellcolor{green!5}\small According to this video 60 of their job cuts were to people aged 40 or older. The median \textbf{age} of workers in the US is 42. Taking out workers under 20 (who IBM isn't hiring), the median \textbf{age} goes up 43.6. So people over 40 don't make much less than 60 of the workforce. VOX is making a problem where one doesn't exist. Feel free to check the numbers -  [LINK] \normalsize   & \cellcolor{green!5}Age & \cellcolor{green!5}Age - General & \cellcolor{green!5}2/71 & \cellcolor{green!5}2.817 \\  \hline
  \cellcolor{green!27}\small Ok so a company desided to exercise their right to hire or fire anyone that they want. In this case these \textbf{old} people had less market value because the of less time to give to the company, boo hoo. Get over it it's a company trying to complete in a free market.\normalsize   & \cellcolor{green!27}Old & \cellcolor{green!27}Age - Over 65s & \cellcolor{green!27}1/52 & \cellcolor{green!27}1.923 \\  \hline
  \cellcolor{green!5}\small I'm kinda perplexed by this whole subject.  Who would have thought that in this day and \textbf{age} that people would actually think the company would treat their employees as anything other than a disposable piece of equipment?  Are there people that are that naive?   The second you're not making them money...they fire you.  Simple.  That's the way it's always been where ever I worked.\normalsize   & \cellcolor{green!5}Age & \cellcolor{green!5}Age - General & \cellcolor{green!5}1/64 & \cellcolor{green!5}1.562 \\  \hline
  \cellcolor{green!27}\small If you consider 45 years \textbf{old} "aging," then something is wrong with your industry.\normalsize   & \cellcolor{green!27}Old & \cellcolor{green!27}Age - Over 65s & \cellcolor{green!27}1/14 & \cellcolor{green!27}7.143 \\  \hline
  \cellcolor{green!5}\small Lol we've pushed the left into the stone \textbf{age} of socialist rhetoric: workers of the world uniteGreat job team\normalsize   & \cellcolor{green!5}Age & \cellcolor{green!5}Age - General & \cellcolor{green!5}1/20 & \cellcolor{green!5}5.0 \\  \hline
  \cellcolor{green!27}\small Have you found out why they fired the \textbf{old} people?\normalsize   & \cellcolor{green!27}Old & \cellcolor{green!27}Age - Over 65s & \cellcolor{green!27}1/10 & \cellcolor{green!27}10.0 \\  \hline
  \cellcolor{green!5}\small IBM was founded in 1911, so it's turning 107 years \textbf{old} in 2018. Its customers should start favoring IBMs much younger competitors...\normalsize   & \cellcolor{green!5}Old & \cellcolor{green!5}Age - Over 65s & \cellcolor{green!5}1/22 & \cellcolor{green!5}4.545 \\  \hline
  \cellcolor{green!27}\small Maybe those \textbf{old} people should have been as competent as the younger ones, then maybe they wouldn't be "pushed out".\normalsize   & \cellcolor{green!27}Old & \cellcolor{green!27}Age - Over 65s & \cellcolor{green!27}1/20 & \cellcolor{green!27}5.0 \\  \hline
  \cellcolor{green!5}\small I will never \textbf{age}, thank god!\normalsize   & \cellcolor{green!5}Age & \cellcolor{green!5}Age - General & \cellcolor{green!5}1/6 & \cellcolor{green!5}16.667 \\  \hline
  \cellcolor{green!27}\small repeated misuse of the term "aging" is annoying. ALL employees are aging at precisely the same rate. Every single employee is aging.  Figure out a way to say "\textbf{old} people" in a piece about...\textbf{old} people. I mean, if you consider 40-something \textbf{old}. but...aging? FFS.\normalsize   & \cellcolor{green!27}Old & \cellcolor{green!27}Age - Over 65s & \cellcolor{green!27}3/44 & \cellcolor{green!27}6.818 \\  \hline
  \cellcolor{green!5}\small My dad was laid off by CBRE at \textbf{age} 60, in the twilight of his career. It completely destroyed his retirement goals. He gave 35 years of his life to that company. He went thru hell and was not happy with his job, but he put up with it to raise my brother and I. 35 years for \textbf{nothing}.Executives of CBRE, I hope you're reading this. I'm going to lose my own job soon with no prospects of finding another. I will be desperate enough to have \textbf{nothing} to lose. I'm going to come for you. I'm going to spend my meager savings hunting you all down. I'm going to kill you all.\normalsize   & \cellcolor{green!5}Age, Nothing & \cellcolor{green!5}Age - General, Age - Youngsters & \cellcolor{green!5}3/114 & \cellcolor{green!5}2.632 \\  \hline
  \cellcolor{green!27}\small so sad a bunch of \textbf{old} \textbf{racist} white people lost their jobs . lol why do we care ?\normalsize   & \cellcolor{green!27}Old, Racist & \cellcolor{green!27}Age - Over 65s, Ethnicity - General & \cellcolor{green!27}2/19 & \cellcolor{green!27}10.526 \\  \hline
  \cellcolor{green!5}\small Rochester, MN, was ground \textbf{zero} for this.\normalsize   & \cellcolor{green!5}Zero & \cellcolor{green!5}Age - Youngsters & \cellcolor{green!5}1/7 & \cellcolor{green!5}14.286 \\  \hline
  \cellcolor{green!27}\small Now the younger generation has to take care of their needs, cause' they're \textbf{old} boomers that think they're entitled to a job. Sad.\normalsize   & \cellcolor{green!27}Old & \cellcolor{green!27}Age - Over 65s & \cellcolor{green!27}1/23 & \cellcolor{green!27}4.348 \\  \hline
  \cellcolor{green!5}\small The last shot taking a dig at IBM's own \textbf{age} is awesome.\normalsize   & \cellcolor{green!5}Age & \cellcolor{green!5}Age - General & \cellcolor{green!5}1/12 & \cellcolor{green!5}8.333 \\  \hline
  \cellcolor{green!27}\small IBM's business is changing, so their workforce needs to change. Sorry, but having "decades of experience" at one company is not necessarily a good thing. If you're knowledgeable in \textbf{ABC} and the company is pivoting to XYZ, of course they'll replace 1 older person with a college grad for 40 less.\normalsize   & \cellcolor{green!27}ABC & \cellcolor{green!27}Nationality - Chinese & \cellcolor{green!27}1/51 & \cellcolor{green!27}1.961 \\  \hline
  \cellcolor{green!5}\small I am only speculating, but I do not think IBM is intentionally discriminating against older workers. It is a business decision because younger workers may bring new skills and perspectives to the table. It is sad that they have to lay off workers, but times are tough and they are going to eliminate workers who are less productive, regardless of \textbf{age}. Also, as a statistician I would be wary of the "trend line" of the data at 4:10. To me, it appears there is the correlation between \textbf{age} and points is very weak and rather inconclusive based on this graph.\normalsize   & \cellcolor{green!5}Age & \cellcolor{green!5}Age - General & \cellcolor{green!5}2/100 & \cellcolor{green!5}2.0 \\  \hline
  \cellcolor{green!27}\small Take less money at the low levels and let the few at the top get even more. \textbf{Old} people need more money.\normalsize   & \cellcolor{green!27}Old & \cellcolor{green!27}Age - Over 65s & \cellcolor{green!27}1/22 & \cellcolor{green!27}4.545 \\  \hline
  \cellcolor{green!5}\small IBM absolutely kicks "older" (anyone above 30) employees to the curb today. Your desirability at IBM is all based on a ticking clock based on your \textbf{age}. Try applying for a job at IBM, it feels like you aren't desired immediately as you're thrown pointless math tests, vague and arbitrary other tests, a long waiting period before getting the results, and all without seeing another human who could see what you have to offer. With the job I have now I always am able to talk to people coming in from IBM and similar companies saying how they and their coworkers just got laid off for no real reason besides what they guess is their \textbf{age}. If that's true, it's pretty sad. People that have potential for new or interesting ideas or "passion" and a human element doesn't seem to matter at IBM, just numbers do, and I don't see them lasting much longer with an identity like that.\normalsize   & \cellcolor{green!5}Age & \cellcolor{green!5}Age - General & \cellcolor{green!5}2/159 & \cellcolor{green!5}1.258 \\  \hline
  \cellcolor{green!27}\small Here's the thing, IBM itself is an \textbf{ageing} monster! Full of bureaucracies and paper shuffling with very focus on actual delivery.I guess we'll be forcing it to resign soon.I worked with IBM for about 2 years a few years ago.I hated every minute of it. They kept selling to us how they cared for long-term workers!Am I glad I ran!\normalsize   & \cellcolor{green!27}Ageing & \cellcolor{green!27}Age - General & \cellcolor{green!27}1/64 & \cellcolor{green!27}1.562 \\  \hline
  \cellcolor{green!5}\small My father just found out that he's being forced to retire after working at IBM for nearly 40 years, after years of having to do the work of three people as they keep laying off experienced employees. And you can't blame technological advancement in his case, because he works in an administrative role that has \textbf{nothing} to do with research or design. They don't see a person when they look at him; they see dollar signs. He had become too valuable. Ironically.\normalsize   & \cellcolor{green!5}Nothing & \cellcolor{green!5}Age - Youngsters & \cellcolor{green!5}1/82 & \cellcolor{green!5}1.22 \\  \hline
  \cellcolor{green!27}\small It's austerity. Older workers will demand more money, so they puch them out in exchange for younger workers and foreigners, who will always work more hours for less pay. It happens everywhere in the tech world, the \textbf{old} programmer gets replaced by an h1b worker who earns less than half the \textbf{old} man.\normalsize   & \cellcolor{green!27}Old & \cellcolor{green!27}Age - Over 65s & \cellcolor{green!27}2/53 & \cellcolor{green!27}3.774 \\  \hline
  \cellcolor{green!5}\small Look, it sounded like people over 40 were slightly more affected by layoffs, but 60 isn't a huge number (since 40 were people under the \textbf{age} of 40). But all the shady stuff IBM does is still stupid, and I don't mean to downplay \textbf{age} discrimination. It's just that the intro makes it seem not that impressive...\normalsize   & \cellcolor{green!5}Age & \cellcolor{green!5}Age - General & \cellcolor{green!5}2/57 & \cellcolor{green!5}3.509 \\  \hline
  \cellcolor{green!27}\small well at least people are not jumping out of windows like at \textbf{apple} factories\normalsize   & \cellcolor{green!27}Apple & \cellcolor{green!27}Ethnicity - Native-American & \cellcolor{green!27}1/14 & \cellcolor{green!27}7.143 \\  \hline
  \cellcolor{green!5}\small \@Vox this is what is wrong with America. IT and Programming jobs are like living life of \textbf{race} horses. You get paid more, actually twice the average of any job however you have to keep yourself updated with latest technology or perish. No companies owes any employee long term job security, benefits , work from home, retirement plans and all the things in the world. It's sad that IBM has been shown In wrong light and given that it's one company which has worked tirelessly to keep Jobs in US and is now struggling to stay relevant.\normalsize   & \cellcolor{green!5}Race & \cellcolor{green!5}Ethnicity - General & \cellcolor{green!5}1/97 & \cellcolor{green!5}1.031 \\  \hline
  \cellcolor{green!27}\small okay, they fire \textbf{old} people and hire young and why is it bad exactly?\normalsize   & \cellcolor{green!27}Old & \cellcolor{green!27}Age - Over 65s & \cellcolor{green!27}1/14 & \cellcolor{green!27}7.143 \\  \hline
  \cellcolor{green!5}\small Im 21 years \textbf{old} studying in software engineering and this is already stressing me lol\normalsize   & \cellcolor{green!5}Old & \cellcolor{green!5}Age - Over 65s & \cellcolor{green!5}1/15 & \cellcolor{green!5}6.667 \\  \hline
  \cellcolor{green!27}\small Why would ibm fire employees that where exceptionaly good at their job, if you have to fire somone shouldnt it be the wost workers. Is vox shure that the \textbf{old} employees where high performers, maby their skills where obsolete\normalsize   & \cellcolor{green!27}Old & \cellcolor{green!27}Age - Over 65s & \cellcolor{green!27}1/39 & \cellcolor{green!27}2.564 \\  \hline
  \cellcolor{green!5}\small I'm a little bit suspicious of this video If there older, experienced workers with all the right stuff IBM would reassign them to a new project since they have knowledge that is difficult to obtain. But if there older workers that worked on a technology or piece of technology that is gradually becoming obsolete I can understand why IBM will want to get rid of them. People face the same problem as technology once your no longer any use your gone, but if they can find a new use for your golden. Ex. I recently installed ubuntu on my \textbf{old} laptop and found a new use otherwise I whould have thrown it out like a person with skills beyond there prime.\normalsize   & \cellcolor{green!5}Old & \cellcolor{green!5}Age - Over 65s & \cellcolor{green!5}1/121 & \cellcolor{green!5}0.826 \\  \hline
  \cellcolor{green!27}\small It could have been that there is a correlation between \textbf{age} and ability to cope with rapidly advancing technology\normalsize   & \cellcolor{green!27}Age & \cellcolor{green!27}Age - General & \cellcolor{green!27}1/19 & \cellcolor{green!27}5.263 \\  \hline
  \cellcolor{green!5}\small MOST places do this. I've applied for cashiers job since I became a \textbf{Senior} and I get same  \textbf{lame} excuse. Then when I go back into wherever I've tried to apply, I see new younger faces. I would like to find the company hiring decoys to catch these scumbags.\normalsize   & \cellcolor{green!5}Lame, Senior & \cellcolor{green!5}Age - Over 65s, Physical Identity - Physical (and Mental) Impairments & \cellcolor{green!5}2/49 & \cellcolor{green!5}4.082 \\  \hline
  \cellcolor{green!27}\small Aging is a \textbf{b*tch}...\normalsize   & \cellcolor{green!27}B*tch & \cellcolor{green!27}Gender - Female sexuality & \cellcolor{green!27}1/4 & \cellcolor{green!27}25.0 \\  \hline
  \cellcolor{green!5}\small Now, it appears \textbf{apple} wants it all\normalsize   & \cellcolor{green!5}Apple & \cellcolor{green!5}Ethnicity - Native-American & \cellcolor{green!5}1/7 & \cellcolor{green!5}14.286 \\  \hline
  \cellcolor{green!27}\small Now they know how millennials feel. We've been largely shut out of promotions and raises for the better part of our adulthood. I'm not saying IBM is right, but no one cares about \textbf{age} discrimination until happens to boomers, which is itself biased behavior.\normalsize   & \cellcolor{green!27}Age & \cellcolor{green!27}Age - General & \cellcolor{green!27}1/44 & \cellcolor{green!27}2.273 \\  \hline
  \cellcolor{green!5}\small Grandpa IBM wants to do away with all its aging \textbf{old} things associated with it and start fresh with its hip new competitors.\normalsize   & \cellcolor{green!5}Old & \cellcolor{green!5}Age - Over 65s & \cellcolor{green!5}1/23 & \cellcolor{green!5}4.348 \\  \hline
  \cellcolor{green!27}\small thats because those \textbf{old} people weren't up to date, and were paid the most\normalsize   & \cellcolor{green!27}Old & \cellcolor{green!27}Age - Over 65s & \cellcolor{green!27}1/14 & \cellcolor{green!27}7.143 \\  \hline
  \cellcolor{green!5}\small You know why \textbf{ageism} exist in the tech field and a lot of other different fields and careers its because of the high ira costs companies must have set up for employees and The longer those employees have been around the higher the retirement funds. Younger workers under 40 don't think about that because they grew up in with a media that didn't educate them about financial security enough. It's sad but true and so Young CEO s under 40 is being told by there much older financial advisors they cannot hire people much older than them even though it would be a great value in their company financially speaking it will cost a lot even though there's private business grants to cover this cost. You see the trickery is real on all levels. A college degree isn't worth much if you don't have any certifications first or along with it.😧😢😤😔this isn't right.\normalsize   & \cellcolor{green!5}Ageism & \cellcolor{green!5}Age - General & \cellcolor{green!5}1/153 & \cellcolor{green!5}0.654 \\  \hline
  \cellcolor{green!27}\small Getting \textbf{old} is something that happens.   The dream that a good stable high paying job is always going to be available is unrealistic.   Live a modest life, save money and have an exit strategy at any point.\normalsize   & \cellcolor{green!27}Old & \cellcolor{green!27}Age - Over 65s & \cellcolor{green!27}1/37 & \cellcolor{green!27}2.703 \\  \hline
  \cellcolor{green!5}\small I have yet to see a comment about the fact that nearly all tech companies are always destined to want younger people to work on new technologies when it comes to the \textbf{common} employee\normalsize   & \cellcolor{green!5}Common & \cellcolor{green!5}Social Class - Working class & \cellcolor{green!5}1/34 & \cellcolor{green!5}2.941 \\  \hline
  \cellcolor{green!27}\small Knowing that 60 of those let go were over 40 years \textbf{old} means \textbf{nothing} unless you know what percentage of IBM's workforce was over 40 at the time. It could be a more proportional cut than you're making it out to be.\normalsize   & \cellcolor{green!27}Nothing, Old & \cellcolor{green!27}Age - Over 65s, Age - Youngsters & \cellcolor{green!27}2/42 & \cellcolor{green!27}4.762 \\  \hline
  \cellcolor{green!5}\small Hey \@VOX, I agree that \textbf{ageism} is a thing... But I think you should do a video on Employment Law \& Severance Pay in Amerieca. Because that class action law  legal action for wrongful dismissal is kinda basic in most Canadian severance packages for companies earning more than 2.1 mil internationally. — I'm positive that your parent company would have the same in your severance package.\normalsize   & \cellcolor{green!5}Ageism & \cellcolor{green!5}Age - General & \cellcolor{green!5}1/67 & \cellcolor{green!5}1.493 \\  \hline
  \cellcolor{green!27}\small Old people suck at working, you can't blame a company for getting rid of them\normalsize   & \cellcolor{green!27}Old & \cellcolor{green!27}Age - Over 65s & \cellcolor{green!27}1/15 & \cellcolor{green!27}6.667 \\  \hline
  \cellcolor{green!5}\small After 10 years \textbf{apple} Google and Amazon will do the same. It's a  business. Don't expect it to be like govt to take care of you when you \textbf{age}.\normalsize   & \cellcolor{green!5}Age, Apple & \cellcolor{green!5}Age - General, Ethnicity - Native-American & \cellcolor{green!5}2/29 & \cellcolor{green!5}6.897 \\  \hline
  \cellcolor{green!27}\small This video begs a lot of questions:1. Is having 60 of your job cuts be people over 40 a significant stat when compared to job cuts industry wide? Country wide?2. Are those terms in the severance package agreement nonstandard?3. Is that point system weighting discriminatory or just a way of giving newly promoted people time to get used to new tasks? Is the fact that people with 15 years of experience are doing better pointswise than people with more experience a sign that the point system is bad or an indication of why they're getting fired in the first place?4. I've always heard the inverse situation is the true problem: unmotivated people populating the high level positions who aren't fired out of respect for their "company loyalty." Resulting in younger workers waiting for their death or retirement before being promoted. Is this a \textbf{common} misconception or does it exist and IBM is simply trying to curtail it?\normalsize   & \cellcolor{green!27}Common & \cellcolor{green!27}Social Class - Working class & \cellcolor{green!27}1/161 & \cellcolor{green!27}0.621 \\  \hline
  \cellcolor{green!5}\small This is sad news but shouldn't be limited to IBM only. The rapid development of software and paradigms specifically in the IT field, plus \textbf{old} well paying contracts make it pretty obvious that from a business perspective this is the way to go. But this is an issue that imo can only be tackled by government regulations (or being decent humans, but who are we kidding). It's a hate the game not the player type situation.\normalsize   & \cellcolor{green!5}Old & \cellcolor{green!5}Age - Over 65s & \cellcolor{green!5}1/76 & \cellcolor{green!5}1.316 \\  \hline
  \cellcolor{green!27}\small I feel for these guys, a lot of people who are above 40 grew up in an era where company loyalty was a real thing. Nowdays you have to constantly reinvent yourself to stay marketable. I've had to retrain due to automation in the Air Force. I guess thats the reality of living in the \textbf{age} of technology.\normalsize   & \cellcolor{green!27}Age & \cellcolor{green!27}Age - General & \cellcolor{green!27}1/58 & \cellcolor{green!27}1.724 \\  \hline
  \cellcolor{green!5}\small They let a bunch of people go to stay competative with companies like \textbf{Apple}... still it probably won't help them in the long run, IBM has been on the decline for years now.\normalsize   & \cellcolor{green!5}Apple & \cellcolor{green!5}Ethnicity - Native-American & \cellcolor{green!5}1/33 & \cellcolor{green!5}3.03 \\  \hline
  \cellcolor{green!27}\small So many people retroactively overestimate \textbf{Apple}'s influence in the '80s. Commodore was a far greater rival to IBM than \textbf{Apple} every were.\normalsize   & \cellcolor{green!27}Apple & \cellcolor{green!27}Ethnicity - Native-American & \cellcolor{green!27}1/22 & \cellcolor{green!27}4.545 \\  \hline
  \cellcolor{green!5}\small What IBM has done is morally wrong.  Loyalty is a two-way street.  When a company lays off its workforce it's a sign of poor leadership and/or poor prior planning.  Vision is key in leadership but unfortunately not \textbf{common} among those holding leadership positions.  Still, CEO's get paid way more than their worth.  Management tends to take care of themselves first, then throw bread crumbs to the worker bees and saying it's cake.\normalsize   & \cellcolor{green!5}Common & \cellcolor{green!5}Social Class - Working class & \cellcolor{green!5}1/72 & \cellcolor{green!5}1.389 \\  \hline
  \cellcolor{green!27}\small What you're finding is young people can't buy property and \textbf{old} people can't get jobs. So you'll be left with no job and no assets either. Well that's what things are pointing to anyway.\normalsize   & \cellcolor{green!27}Old & \cellcolor{green!27}Age - Over 65s & \cellcolor{green!27}1/34 & \cellcolor{green!27}2.941 \\  \hline
  \cellcolor{green!5}\small Adapt and change or get left behind. Every business needs to survive a decade or two or three the system easily can become obsolete specially in a competitive innovative market and if you don't adapt your out no company will drag its self down to oblivion because older crowd wants to stick to \textbf{old} ways. Comfort can lead to complacency and stagnation. Its sad with what happened but IBM thinks about its survival in the vicious capitalistic market.\normalsize   & \cellcolor{green!5}Old & \cellcolor{green!5}Age - Over 65s & \cellcolor{green!5}1/78 & \cellcolor{green!5}1.282 \\  \hline
  \cellcolor{green!27}\small If a person is a valuable asset to a company, they will not be laid off. No matter whether they are young or \textbf{old}. That's it.\normalsize   & \cellcolor{green!27}Old & \cellcolor{green!27}Age - Over 65s & \cellcolor{green!27}1/26 & \cellcolor{green!27}3.846 \\  \hline
  \cellcolor{green!5}\small What the hell is "\textbf{age} discrimination"? Older workers make more money because they've been working longer. The whole point of layoffs is to reduce costs because the profits aren't high enough to sustain so many of them.\normalsize   & \cellcolor{green!5}Age & \cellcolor{green!5}Age - General & \cellcolor{green!5}1/37 & \cellcolor{green!5}2.703 \\  \hline
  \cellcolor{green!27}\small Who in their right mind would fire legacy support or low level engineers/programmers in a company that still has an astounding number of active maintenance contracts for 286 machines?Just let them naturally phase out, the \textbf{old} businesses won't be replacing their \textbf{old} systems in that time period anyways.\normalsize   & \cellcolor{green!27}Old & \cellcolor{green!27}Age - Over 65s & \cellcolor{green!27}2/49 & \cellcolor{green!27}4.082 \\  \hline
  \cellcolor{green!5}\small Honestly I don't see what the problem is. Older people who have not adapted to the new changes in the field are bound to get laid off. I'm pretty sure there are \textbf{old} people in IBM who have managed to stay up to date with the ever changing trends in technology. As a firm IBM will obviously optimize their productivity as a whole. I mean why would the lay people off who are experienced and doing really well at their posts. Wouldn't that harm the company.\normalsize   & \cellcolor{green!5}Old & \cellcolor{green!5}Age - Over 65s & \cellcolor{green!5}1/86 & \cellcolor{green!5}1.163 \\  \hline
  \cellcolor{green!27}\small Off with the \textbf{old} in with new we all live by this u can apply this to your life therefore u shouldnt complain it ur own fault this is how our society works and u follow it and dont want to change cuz noone likes \textbf{old} things work wise new worker is gona do more then \textbf{old} and thats money money talks dont want to get recked by ur \textbf{age} save money\normalsize   & \cellcolor{green!27}Age, Old & \cellcolor{green!27}Age - General, Age - Over 65s & \cellcolor{green!27}4/72 & \cellcolor{green!27}5.556 \\  \hline
  \cellcolor{green!5}\small From a business stand point, \textbf{nothing} wrong with \textbf{Age} discrimination, you need new minds to keep the company afloat.  Besides, why would anyone want to work for one company for all their life anyway?  For the smarter older folks, management positions are always looking for people with years of experience.  For the rest of the lazy population who did \textbf{nothing} to advance their career or have a Plan B in place, plenty of jobs at walmart and security guards for older folks.\normalsize   & \cellcolor{green!5}Age, Nothing & \cellcolor{green!5}Age - General, Age - Youngsters & \cellcolor{green!5}3/82 & \cellcolor{green!5}3.659 \\  \hline
  \cellcolor{green!27}\small An IT worker in their 40s is hardly obsolete or over-the-hill.  People of that generation are the ones who built the Internet and created most of the IT infrastructure the world uses. The reason tech companies, and companies in general, try to shadily get rid of older workers is because people in their 40s are not as easy to exploit as new graduates who know \textbf{nothing} about the working world. That being said, a company that does this kind of thing usually ends up shooting itself in the foot. When was the last time something truly innovative came out of IBM anyways? Companies are only as good as they treat their staff.\normalsize   & \cellcolor{green!27}Nothing & \cellcolor{green!27}Age - Youngsters & \cellcolor{green!27}1/112 & \cellcolor{green!27}0.893 \\  \hline
  \cellcolor{green!5}\small Nothing matter other than "profits". Ethics , who cares when business plays within the borders of laws.\normalsize   & \cellcolor{green!5}Nothing & \cellcolor{green!5}Age - Youngsters & \cellcolor{green!5}1/17 & \cellcolor{green!5}5.882 \\  \hline
  \cellcolor{green!27}\small There are plenty of older programmers who just learned a framework and know \textbf{nothing} beyond how to use a long obsolete framework. Young programmers are like that too but at least their frameworks are still in use.\normalsize   & \cellcolor{green!27}Nothing & \cellcolor{green!27}Age - Youngsters & \cellcolor{green!27}1/37 & \cellcolor{green!27}2.703 \\  \hline
  \cellcolor{green!5}\small The difference between and older employees and a younger one is you have to pay older employees more. This is \textbf{nothing} more than a smart business move by IBM. Kick out the older, less efficient people and grab new kids right out of university.\normalsize   & \cellcolor{green!5}Nothing & \cellcolor{green!5}Age - Youngsters & \cellcolor{green!5}1/44 & \cellcolor{green!5}2.273 \\  \hline
  \cellcolor{green!27}\small I'm going to get trolled for this but discrimination is the practice of using irrelevant factors in hiring/promoting \textbf{age} is not irrelevant we are mortal  beings and \textbf{age} takes a toll on us\normalsize   & \cellcolor{green!27}Age & \cellcolor{green!27}Age - General & \cellcolor{green!27}2/33 & \cellcolor{green!27}6.061 \\  \hline
  \cellcolor{green!5}\small My mom worked there for 25 years and was forced to retire to keep her benefits. She kept telling me how no one felt safe in their positions and how they were being targeted and picked off because of \textbf{age}. I didn't realize how big it was...She was a workaholic and being forced into this decision wrecked her. Her life was her job. She went into a downward spiral and since then IBM, in a way, has been an antagonist in my life. Part of me will always hate them for what they did to her.\normalsize   & \cellcolor{green!5}Age & \cellcolor{green!5}Age - General & \cellcolor{green!5}1/96 & \cellcolor{green!5}1.042 \\  \hline
  \cellcolor{green!27}\small pushed out the \textbf{old} to keep these young generations? now that's funny. good luck.\normalsize   & \cellcolor{green!27}Old & \cellcolor{green!27}Age - Over 65s & \cellcolor{green!27}1/14 & \cellcolor{green!27}7.143 \\  \hline
  \cellcolor{green!5}\small I worked at IBM as a developer on an internship from 2014-2015.  If you look at October of 2014, you'll see a sharp \textbf{spike} downward in their stocks, that they've yet to recover from. I remember being there when that happened. Things changed dramatically after that. Much like the story and video say, a lot of very experienced developers never came back after Christmas vacation. Sweeping organizational changes were attempted to be put in place, trying to change to an "agile and fast approach," all of which felt incredibly rushed, desperate, and really didn't scale to the size of the organization they were trying to apply it to. They made a short attempt at trying to adopt a more startup-like young culture with office parties, lunches between upper management and us university interns, and event outings but it just resulted in more alienation of the older demographics.At the end of my internship, our team that had some ~80 student interns was replaced with 15. It felt like they had just given up. A few interns I worked with went back to IBM afterwards, but not many.It was bad. I've never seen a company so desperate to find it's place in a world that had seemingly left it behind, and then just give up.\normalsize   & \cellcolor{green!5}Spike & \cellcolor{green!5}Religious Identity - Protestant & \cellcolor{green!5}1/215 & \cellcolor{green!5}0.465 \\  \hline
  \cellcolor{green!27}\small If you want to get rid of older people from a sector, the political one is where it should be happening. Make anyone whose \textbf{age} is more than 50 of average life expectancy ineligible to run for office. Encourage a bit of long-term planning as anyone young enough to run for office will also be young enough to have to live with the consequences of their decisions made in office. No more of this "Why should I care? I'll be dead by then." mentality.\normalsize   & \cellcolor{green!27}Age & \cellcolor{green!27}Age - General & \cellcolor{green!27}1/84 & \cellcolor{green!27}1.19 \\  \hline
  \cellcolor{green!5}\small How stupid does Vox think we are, \textbf{age} discrimination is perfectly rational. When you get older you start to suck more at your job, everyone knows this, there are some people for whom this is still a high performance level but others on the margin will fall below an acceptable level.\normalsize   & \cellcolor{green!5}Age & \cellcolor{green!5}Age - General & \cellcolor{green!5}1/51 & \cellcolor{green!5}1.961 \\  \hline
  \cellcolor{green!27}\small My teacher had to leave midyear because his wife had to relocate. It's sickening and they should not be able to put the waiving of rights (of being able to sue for \textbf{age} discrimination) in the contract.\normalsize   & \cellcolor{green!27}Age & \cellcolor{green!27}Age - General & \cellcolor{green!27}1/37 & \cellcolor{green!27}2.703 \\  \hline
  \cellcolor{green!5}\small Like any company that gets big over time it will push out \textbf{old} values..and employees\normalsize   & \cellcolor{green!5}Old & \cellcolor{green!5}Age - Over 65s & \cellcolor{green!5}1/15 & \cellcolor{green!5}6.667 \\  \hline
  \cellcolor{green!27}\small How about "\textbf{old} directors" and promotors as well !\normalsize   & \cellcolor{green!27}Old & \cellcolor{green!27}Age - Over 65s & \cellcolor{green!27}1/9 & \cellcolor{green!27}11.111 \\  \hline
  \cellcolor{green!5}\small how can you be mad when you are fired based on \textbf{age} when you agree to possibly be fired based on your age\normalsize   & \cellcolor{green!5}Age & \cellcolor{green!5}Age - General & \cellcolor{green!5}2/23 & \cellcolor{green!5}8.696 \\  \hline
  \cellcolor{green!27}\small Wth 40 is \textbf{old}? Ok bye\normalsize   & \cellcolor{green!27}Old & \cellcolor{green!27}Age - Over 65s & \cellcolor{green!27}1/6 & \cellcolor{green!27}16.667 \\  \hline
  \cellcolor{green!5}\small Look hard enough and you will find \textbf{age} discrimination in almost every corporation in America. And, it's mostly being ignored.\normalsize   & \cellcolor{green!5}Age & \cellcolor{green!5}Age - General & \cellcolor{green!5}1/20 & \cellcolor{green!5}5.0 \\  \hline
  \cellcolor{green!27}\small So what? \textbf{Old} people are protected by an unjust law stating that only people above a certain \textbf{age} are protected against "\textbf{age} discrimination"...how is not in conflict with the 14th Amendment to the Constitution?\normalsize   & \cellcolor{green!27}Age, Old & \cellcolor{green!27}Age - General, Age - Over 65s & \cellcolor{green!27}3/34 & \cellcolor{green!27}8.824 \\  \hline
  \cellcolor{green!5}\small Old People Can't Keep up\normalsize   & \cellcolor{green!5}Old & \cellcolor{green!5}Age - Over 65s & \cellcolor{green!5}1/5 & \cellcolor{green!5}20.0 \\  \hline
  \cellcolor{green!27}\small It's not an aging problem for IBM every year they would post a list which showed U.S IBM employees decreasing and \textbf{Indian} IBM employees increasing to the point that \textbf{Indian} employees make up the majority of employees with over 100,000 once the media started shining light on it they stopped publishing the list, at this point they almost have a non-negligible amount of U.S employees and nearly all of their Workforce is based in India\normalsize   & \cellcolor{green!27}Indian & \cellcolor{green!27}Ethnicity - Native-American & \cellcolor{green!27}2/75 & \cellcolor{green!27}2.667 \\  \hline
  \cellcolor{green!5}\small Now please do similar material about \textbf{Apple}.\normalsize   & \cellcolor{green!5}Apple & \cellcolor{green!5}Ethnicity - Native-American & \cellcolor{green!5}1/7 & \cellcolor{green!5}14.286 \\  \hline
  \cellcolor{green!27}\small doing the same thing all over corporate workforce, move n relocate or resign y pay someone \@ 40 60k+ when we can pay 2 people half the \textbf{age} 30 basic but its counter intuitive because all they want is order taking drone...they rather hire someone w less credentials n experience for half the pay because all they are is a piece to the system every position is so singularly specialty rather sad n baby boomers refuse to retire or teach the next generation in fear of their own job when they at the point where they make the most amount n do the least amount of work in day to day operations - they staff for that specific thing now in low paying jobs making it impossible to get ahead unless you are in a specific niche field industry\normalsize   & \cellcolor{green!27}Age & \cellcolor{green!27}Age - General & \cellcolor{green!27}1/139 & \cellcolor{green!27}0.719 \\  \hline
  \cellcolor{green!5}\small Old dogs can't learn new tricks...\normalsize   & \cellcolor{green!5}Old & \cellcolor{green!5}Age - Over 65s & \cellcolor{green!5}1/6 & \cellcolor{green!5}16.667 \\  \hline
  \cellcolor{green!27}\small I love the fact that Vox, whose target demographic I assume is young people, are showing a video about older people facing \textbf{age} discrimination. It's a good lesson for younger viewers, myself included, to put in our back pockets.\normalsize   & \cellcolor{green!27}Age & \cellcolor{green!27}Age - General & \cellcolor{green!27}1/39 & \cellcolor{green!27}2.564 \\  \hline
  \cellcolor{green!5}\small Not once did this video mention why IBM might want to lay off older workers. Would have been nice to know why layoffs are disproportionately impacting the older workforce. Companies don't just fire people for the act of being \textbf{old}... Maybe a follow up video?\normalsize   & \cellcolor{green!5}Old & \cellcolor{green!5}Age - Over 65s & \cellcolor{green!5}1/45 & \cellcolor{green!5}2.222 \\  \hline
  \cellcolor{green!27}\small IBM isn't really relevant anymore. No one cares about it. It's synonymous with \textbf{old} people, and never was popular with the consumer market. It's really just known for research. Profits are low. I think Google and \textbf{Apple} will make them go bankrupt.\normalsize   & \cellcolor{green!27}Apple, Old & \cellcolor{green!27}Age - Over 65s, Ethnicity - Native-American & \cellcolor{green!27}2/42 & \cellcolor{green!27}4.762 \\  \hline
  \cellcolor{green!5}\small It's a business, not UNICEF.... if IBM is to stay competitive against \textbf{Apple} Google and Facebook it has to do what it has to do for its shareholders\normalsize   & \cellcolor{green!5}Apple & \cellcolor{green!5}Ethnicity - Native-American & \cellcolor{green!5}1/28 & \cellcolor{green!5}3.571 \\  \hline
  \cellcolor{green!27}\small Sorry but can you blame IBM? \textbf{Age} is valued because of experience, but the tech industry is advancing so rapidly that experience becomes redundant, where as an ability to adapt and learn new techniques becomes crucial. The older the employee, the less likely they are to be up to date with the bleeding edge of technology IBM is striving to attain. Therefore, I would argue that \textbf{age} discrimination is necessary, at least if we as the consumer are still worried about the rapid advancement of computers, smartphones, and almost every other form of technology we take for granted in 2018.\normalsize   & \cellcolor{green!27}Age & \cellcolor{green!27}Age - General & \cellcolor{green!27}2/100 & \cellcolor{green!27}2.0 \\  \hline
  \cellcolor{green!5}\small The nick name when I worked there "I've Been Moved"IBM normally overwork the newer younger employees, and hate the older ones they can't exploit and take advantage of. Another part of this story is their contracting they do for 10+ years for many employees. It was \textbf{crazy} seeing this article title, brings back nightmarish memories from working at IBM\normalsize   & \cellcolor{green!5}Crazy & \cellcolor{green!5}Physical Identity - Physical (and Mental) Impairments & \cellcolor{green!5}1/60 & \cellcolor{green!5}1.667 \\  \hline
  \cellcolor{green!27}\small they got too complacent on their job, this is an \textbf{age} where everyone have to watch their backs.\normalsize   & \cellcolor{green!27}Age & \cellcolor{green!27}Age - General & \cellcolor{green!27}1/18 & \cellcolor{green!27}5.556 \\  \hline
  \cellcolor{green!5}\small The idea is that younger workers are more creative and energetic and smarter than older workers. Older workers are being let go because it's believed their \textbf{age} means they're more set in their \textbf{old} ways and are less adaptable and will less be able to contribute to innovative strategies for a company going forward. It's just the culture of tech companies to favor youth over \textbf{age}. 35 is the new 60.\normalsize   & \cellcolor{green!5}Age, Old & \cellcolor{green!5}Age - General, Age - Over 65s & \cellcolor{green!5}3/71 & \cellcolor{green!5}4.225 \\  \hline
  \cellcolor{green!27}\small Age \textbf{old} saying "If you want loyalty, get a dog. I work for money" is still true to this day :)\normalsize   & \cellcolor{green!27}Age, Old & \cellcolor{green!27}Age - General, Age - Over 65s & \cellcolor{green!27}2/21 & \cellcolor{green!27}9.524 \\  \hline
  \cellcolor{green!5}\small But there's something in technology called a covenant not to compete. It can make it difficult to work for someone else. That way you are forced to be loyal to the company but they can exploit you. Those covenants are illegal in California. To get around that some of the big tech companies, such as \textbf{Apple} entered a pact with other tech giants agreeing not to hire each other's employees. They were found guilty of an antitrust violation and forced to pay damages to their \textbf{engineers}.\normalsize   & \cellcolor{green!5}Apple, Engineers & \cellcolor{green!5}Ethnicity - Native-American, Nationality - General & \cellcolor{green!5}2/86 & \cellcolor{green!5}2.326 \\  \hline
  \cellcolor{green!27}\small Nothing is coming here man! We are also struggling here.\normalsize   & \cellcolor{green!27}Nothing & \cellcolor{green!27}Age - Youngsters & \cellcolor{green!27}1/10 & \cellcolor{green!27}10.0 \\  \hline
  \cellcolor{green!5}\small They are smarter in China, India Japan and Korea ?  They Are !! They have better schools !! And the young people don't hang out  all the time, with \textbf{alcohol}, drugs and \textbf{sex} ! The USA has set back itself and will be a 3rd world country in 20 years from now.\normalsize   & \cellcolor{green!5}Sex, alcohol & \cellcolor{green!5}Behavioural Addiction - Alcohol, Gender - General & \cellcolor{green!5}2/52 & \cellcolor{green!5}3.846 \\  \hline
  \cellcolor{green!27}\small Willem van Ingen sure they do, that's why countries like Japan have one of the highest teenage suicide rates in the world. If u think US schools are bad then go look at some of these asian countries that force their children to compete and be number 1 from day 1 of 1st grade. There's no extracuriculars taken into consideration, your worth is defined by your scores which is alot of pressure for children.These countries only seem to have more intelligent people because the majority of \textbf{ethnicity} in india,china,japan,and Korea are the same which does not show an equal representation of intelligence as every single one of them us already adapted to the pressured of scores put on by their teachers and parents. Just know that not every Asian is smart, it's just a stereotype formulated by overexagreated information that only represents as minority of people in those countries.\normalsize   & \cellcolor{green!27}Ethnicity & \cellcolor{green!27}Ethnicity - General & \cellcolor{green!27}1/150 & \cellcolor{green!27}0.667 \\  \hline
  \cellcolor{green!5}\small ibm thinks an \textbf{old} dog can't learn new tricks\normalsize   & \cellcolor{green!5}Old & \cellcolor{green!5}Age - Over 65s & \cellcolor{green!5}1/9 & \cellcolor{green!5}11.111 \\  \hline
  \cellcolor{green!27}\small YES, at Intel for example, there are floors and floors of \textbf{Indian} H1B workers. Here in Silicon valley.\normalsize   & \cellcolor{green!27}Indian & \cellcolor{green!27}Ethnicity - Native-American & \cellcolor{green!27}1/18 & \cellcolor{green!27}5.556 \\  \hline
  \cellcolor{green!5}\small i think the hard truth is that people are just being nice when they say that \textbf{old} folks have the skills good enough to stay or be promoted. most older guys at tech companies aren't worth their salaries anymore, because they got complacent and haven't kept up with the newer developments in technology. About 1 in 4 maybe are truly skilled, flexible even in their \textbf{old} \textbf{age}, experienced and still hard working and able to work at the same pace as \textbf{engineers} below 40. Those few, are the key members that companies need to treasure. but... most people... just.. haven't been keeping up. maximum productivity comes from people between 30-40. experienced enough to do high level work, new enough to believe they still have more to learn.\normalsize   & \cellcolor{green!5}Age, Engineers, Old & \cellcolor{green!5}Age - General, Age - Over 65s, Nationality - General & \cellcolor{green!5}4/127 & \cellcolor{green!5}3.15 \\  \hline
  \cellcolor{green!27}\small age discrimination?really?so it couldn't be that technology is moving fast that older workers can't keep up?it's always discrimination by xzyit's never THEM\normalsize   & \cellcolor{green!27}Age & \cellcolor{green!27}Age - General & \cellcolor{green!27}1/26 & \cellcolor{green!27}3.846 \\  \hline
  \cellcolor{green!5}\small Watson, were you built the company that hates \textbf{old} people?\normalsize   & \cellcolor{green!5}Old & \cellcolor{green!5}Age - Over 65s & \cellcolor{green!5}1/10 & \cellcolor{green!5}10.0 \\  \hline
  \cellcolor{green!27}\small It's not just IBM, every tech industry in the U.S. is doing exactly what IBM has been doing.The median \textbf{age} of U.S. workers is 42.4 years \textbf{old}, yet for Google and most tech companies it is 29 years \textbf{old}.I wonder if anyone has done any studies showing a correlation between high rates of \textbf{age} discrimination and suicide. Currently the highest rate of both are between the ages if 40 and 55.\normalsize   & \cellcolor{green!27}Age, Old & \cellcolor{green!27}Age - General, Age - Over 65s & \cellcolor{green!27}4/73 & \cellcolor{green!27}5.479 \\  \hline
  \cellcolor{green!5}\small Go to China, where they value your \textbf{age} and experiences.\normalsize   & \cellcolor{green!5}Age & \cellcolor{green!5}Age - General & \cellcolor{green!5}1/10 & \cellcolor{green!5}10.0 \\  \hline
  \cellcolor{green!27}\small I don't \textbf{age}!IM BERNIE SANDERS!\normalsize   & \cellcolor{green!27}Age & \cellcolor{green!27}Age - General & \cellcolor{green!27}1/6 & \cellcolor{green!27}16.667 \\  \hline
  \cellcolor{green!5}\small It seems the modern trend is to favor corporations or organizations over workers rights. It's sad as I'm in my early 30's and I'm almost afraid of what the environment is going to be like when I hit retirement \textbf{age}.\normalsize   & \cellcolor{green!5}Age & \cellcolor{green!5}Age - General & \cellcolor{green!5}1/40 & \cellcolor{green!5}2.5 \\  \hline
  \cellcolor{green!27}\small Here's the thing that I don't think u r factoring (this is experience from talking to older IBM workers)- like tech in general, it can change absolutely rapidly very quickly. The people I knew that was fired or "retired" didn't know what even the newest version of Windows was out. I can see why IBM would fire older employees if they weren't brought up with how stuff is now, goes back to the saying "you can't teach an \textbf{old} dog new tricks"\normalsize   & \cellcolor{green!27}Old & \cellcolor{green!27}Age - Over 65s & \cellcolor{green!27}1/82 & \cellcolor{green!27}1.22 \\  \hline
  \cellcolor{green!5}\small You can bet your bottom dollar the \textbf{fruit} factory is watching this with interest...\normalsize   & \cellcolor{green!5}Fruit & \cellcolor{green!5}Sexual Identity - Male homosexuality & \cellcolor{green!5}1/14 & \cellcolor{green!5}7.143 \\  \hline
  \cellcolor{green!27}\small Correlation does not imply causation. In a rapidly-changing field like technology, it's only natural that older people with an older set of skill will be having an outdated set of experiences, skills and approaches that will be less useful to a company every passing day. And with \textbf{age} comes the resistance and inability to adapt to new technology. Being laid off probably meant that their outdated skill sets and experiences were not being useful to the company, and it's not necessarily true that they were laid off simply because they were "\textbf{old}". I am not speaking against protection from being laid off after decades of service to a company. But if you want to argue against this specific case of laying off, calling it "\textbf{age} discrimination" will probably be barking at the wrong tree and may not be helpful at all.\normalsize   & \cellcolor{green!27}Age, Old & \cellcolor{green!27}Age - General, Age - Over 65s & \cellcolor{green!27}3/141 & \cellcolor{green!27}2.128 \\  \hline
  \cellcolor{green!5}\small most tech companies hire younger workers and muscle off the older workers. they want the young new hip smart workers that they don't have to retrain who come to an interview as an expert instead of having 20 years of \textbf{old} obsolete programming they have to forget about and relearn.this sucks, but it also is probably an economic advantage that they don't have to spend the money to retrain people\normalsize   & \cellcolor{green!5}Old & \cellcolor{green!5}Age - Over 65s & \cellcolor{green!5}1/71 & \cellcolor{green!5}1.408 \\  \hline
  \cellcolor{green!27}\small The \textbf{ageism} in these comments is astounding. People in tech learn new skills constantly. Note that the video showed that older workers with excellent reviews for years, including recent years, were being let go. Did you actually watch it?\normalsize   & \cellcolor{green!27}Ageism & \cellcolor{green!27}Age - General & \cellcolor{green!27}1/39 & \cellcolor{green!27}2.564 \\  \hline
  \cellcolor{green!5}\small I am loosing my mind with the \textbf{ageist} comment. i am tapping my foot on the floor with anger. these kids will grow up by the middle of next decade and the next clueless batch will step in and tell them " er.. adapt or scram", but they are born with the fresh new technology on their hands, which will also \textbf{age} by the middle of next decade.  these are young imbeciles.\normalsize   & \cellcolor{green!5}Age, Ageist & \cellcolor{green!5}Age - General & \cellcolor{green!5}2/72 & \cellcolor{green!5}2.778 \\  \hline
  \cellcolor{green!27}\small Ageism is \textbf{common} in my field mechanical engineering. I'd advise anyone entering the field to get their professional license asap so when (not if) they get screwed over in their late 50s they can freelance and not  be forced into early retirement.\normalsize   & \cellcolor{green!27}Ageism, Common & \cellcolor{green!27}Age - General, Social Class - Working class & \cellcolor{green!27}2/42 & \cellcolor{green!27}4.762 \\  \hline
  \cellcolor{green!5}\small Nobody has a VCR from the 80s or 90s to use as a prop.\normalsize   & \cellcolor{green!5}Nobody & \cellcolor{green!5}Age - Youngsters & \cellcolor{green!5}1/14 & \cellcolor{green!5}7.143 \\  \hline
  \cellcolor{green!27}\small My dad was let go from NEC in the 90s with the same motive, to get the \textbf{old} members out. Intel's mass-layoff/retirement a couple years ago ended up doing something similar "unintentionally". This isn't new to the tech industry, it's an established practice.\normalsize   & \cellcolor{green!27}Old & \cellcolor{green!27}Age - Over 65s & \cellcolor{green!27}1/43 & \cellcolor{green!27}2.326 \\  \hline
  \cellcolor{green!5}\small Ageism in the Tech Industry is rampant. Most companies would rather higher clueless young kids fresh from college than seasoned industry veterans. The reason for this is two fold: - A younger work force is more willing and able pick up after management incompetence, which is the source of the industry standard practice of "every time is sprint time";- The industry obsession with the latest technological fad, and it's collective belief that "new is always better", which is hardly ever the case. I, for one, would rather work with a seasoned Java veteran that can tell me months in advance why something will not work, than a bunch of JavaScript kids that will eagerly jump down a cliff after me.\normalsize   & \cellcolor{green!5}Ageism & \cellcolor{green!5}Age - General & \cellcolor{green!5}1/121 & \cellcolor{green!5}0.826 \\  \hline
  \cellcolor{green!27}\small IBM helped carry out the holocaust by manufacturing machines that kept track of Jewish prisoners. So like why are you working for a company that helped carry out the holocaust in the first place? They helped carry out the holocaust and you expected them to not be evil? Although, the \textbf{age} discrimination is still bad.\normalsize   & \cellcolor{green!27}Age & \cellcolor{green!27}Age - General & \cellcolor{green!27}1/55 & \cellcolor{green!27}1.818 \\  \hline
  \cellcolor{green!5}\small Age discrimination is a disgusting tactic. However certain factors must be considered, such as the costs to keep the person employed and whether they are keeping up with ever changing work environments. If what they do can be done by someone they can pay less, it is fair for them to let go of the more expensive employee. There is no emotion in the business world when it comes to the bottom line. Also, not all, but many older generations do not stay up to date on changing technology. These individuals can not only be a drag on on productivity but they can be a security risk as proven by the increasing hacks going on around the world  This is of course not unique to our older population. But in general they are more likely to stick to what they know and be afraid of change. It's just human nature.\normalsize   & \cellcolor{green!5}Age & \cellcolor{green!5}Age - General & \cellcolor{green!5}1/150 & \cellcolor{green!5}0.667 \\  \hline
  \cellcolor{green!27}\small My mom is 60 yrs \textbf{old} and she's approaching her 30 year mark of working with them and she constantly has anxiety because of the constant layoffs that they are pushing out.\normalsize   & \cellcolor{green!27}Old & \cellcolor{green!27}Age - Over 65s & \cellcolor{green!27}1/32 & \cellcolor{green!27}3.125 \\  \hline
  \cellcolor{green!5}\small I feel bad for both IBM and the (older) employees. I used to work for them as contractor 13 years ago when they were much more well know than now. The things is : older "workers" have experience and are skilled but at the same time are slow and not very productive, comparing to the young counter part. They are more fitted as "manager", "adviser", etc.. being workers, the ones that implement/make things. Also , for the fields of computing ( IT, software, hardware, services,etc..) that IBM is in now , things changes so fast, like every 3 years or so, oldie are at disadvantage... I think best \textbf{age} are below 40-45 for "workers", after that the performance go down hill.  I hope the best for IBM and their staff ( current and fired )..  .\normalsize   & \cellcolor{green!5}Age & \cellcolor{green!5}Age - General & \cellcolor{green!5}1/136 & \cellcolor{green!5}0.735 \\  \hline
  \cellcolor{green!27}\small Wouldn't doubt these \textbf{imperialist} corporations use Computer Business Systems to manage their employees much like Amazon, and other large companies do.\normalsize   & \cellcolor{green!27}Imperialist & \cellcolor{green!27} Ideological and Political Identity - General & \cellcolor{green!27}1/21 & \cellcolor{green!27}4.762 \\  \hline
  \cellcolor{green!5}\small It can be argued that \textbf{age} in the tech industry does make a lot of difference (either negative or positive, depending on your position). You can't really expect to do the same job your whole career, like in most other "big employer" industries.\normalsize   & \cellcolor{green!5}Age & \cellcolor{green!5}Age - General & \cellcolor{green!5}1/43 & \cellcolor{green!5}2.326 \\  \hline
  \cellcolor{green!27}\small M RawashExcept the video stated that IBM considered the people they're firing to essentially be subject matter experts who are at the top of the totem pole. These aren't slackers, these are \textbf{old} techies who can run circles around anyone younger.\normalsize   & \cellcolor{green!27}Old & \cellcolor{green!27}Age - Over 65s & \cellcolor{green!27}1/42 & \cellcolor{green!27}2.381 \\  \hline
  \cellcolor{green!5}\small Gregory EversonEducation can marginally extend a technology expert's career, but it still wouldn't give them an advantage over a fresh graduate. More things change with \textbf{age} than just the field itself (e.g. a person's capacity to learn, innovate, work long hours, etc..), there are, of course, exceptions, but we're talking about the average here, not the outliers. In such a demanding and fast-moving industry, if you're not already the "boss" (i.e. in a managerial position) by a certain \textbf{age}, you've compromised the rest of your career, and you can't really blame the higher-ups for that, they still have to look out for the well-being of the company as a whole, or they risk putting EVERYONE out of a job.\normalsize   & \cellcolor{green!5}Age & \cellcolor{green!5}Age - General & \cellcolor{green!5}2/120 & \cellcolor{green!5}1.667 \\  \hline
  \cellcolor{green!27}\small As an inner-city public school teacher, I fully expect to be doing the same job until the day I die. \textbf{Nobody} wants my job and that's my life hack. Good luck being pushed out of the workforce as you get older. No computer can manage a classroom full of bad ass kids.\normalsize   & \cellcolor{green!27}Nobody & \cellcolor{green!27}Age - Youngsters & \cellcolor{green!27}1/52 & \cellcolor{green!27}1.923 \\  \hline
  \cellcolor{green!5}\small I love how there is \textbf{zero} mention of salary level in this video.  Where is the regression analysis that shows that the firings are not contingent on marginal product vs. salary level.  You are better off as a company firing expensive people (i.e. people with significant years experience) when you have younger people that cost 1/2 as much to do the same job.  If you were to cut the \textbf{old} people's wages they would complain about the wages, be dissatisfied and not do as good of a job.  Its about the bottom line, not that they are actually \textbf{old}.\normalsize   & \cellcolor{green!5}Old, Zero & \cellcolor{green!5}Age - Over 65s, Age - Youngsters & \cellcolor{green!5}3/99 & \cellcolor{green!5}3.03 \\  \hline
  \cellcolor{green!27}\small nothing wrong with capitalism fam\normalsize   & \cellcolor{green!27}Nothing & \cellcolor{green!27}Age - Youngsters & \cellcolor{green!27}1/5 & \cellcolor{green!27}20.0 \\  \hline
  \cellcolor{green!5}\small dirtywart sure, \textbf{nothing} wrong untill you don't go under a bridge\normalsize   & \cellcolor{green!5}Nothing & \cellcolor{green!5}Age - Youngsters & \cellcolor{green!5}1/11 & \cellcolor{green!5}9.091 \\  \hline
  \cellcolor{green!27}\small I was lucky and worked for a fantastically successful company that treated their employees well.IBM '68-'98.   Retiree since '98.  Starting about 1990 IBM changed because their business changed. They were no longer Intel, Google, Microsoft and \textbf{Apple} all rolled into one. They no longer dominated.  They had to change or die.They changed but can no longer afford be the best or most generous employer ever.  But in my 30 years they never lied to me.  They never promised what they failed to deliver.   I doubt they're lying now.So when you get that offer from Google or \textbf{Apple}.   I hope your new job works as well for you.But if they let you go and you sign a document restricting your options so you can to get your severance, don't whine about it after you've spent the money.\normalsize   & \cellcolor{green!27}Apple & \cellcolor{green!27}Ethnicity - Native-American & \cellcolor{green!27}2/140 & \cellcolor{green!27}1.429 \\  \hline
  \cellcolor{green!5}\small How to legally discriminate against workers: ‘We don't need you anymore'Do you really think employers are going to say: we are firing you because you're \textbf{black} or we are firing you because your \textbf{old}?Making it illegal to fire someone because they are \textbf{old} is just stupid. An employer can legally fire all the 60 year olds because they 'have to many employees' ever since they hired some new collage students.\normalsize   & \cellcolor{green!5}Black, Old & \cellcolor{green!5}Age - Over 65s, Ethnicity - Black & \cellcolor{green!5}3/72 & \cellcolor{green!5}4.167 \\  \hline
  \cellcolor{green!27}\small Notmy Realname True, a business should have every right to discriminate because of \textbf{age}. Maybe businesses can legally discriminate on \textbf{age}, I actually don't know, Vox kind of hinted at it but I'm not sure.\normalsize   & \cellcolor{green!27}Age & \cellcolor{green!27}Age - General & \cellcolor{green!27}2/35 & \cellcolor{green!27}5.714 \\  \hline
  \cellcolor{green!5}\small The hard truth that people don't understand is if you kept those \textbf{old} farts in the business new fresh Innovation would never occur.\normalsize   & \cellcolor{green!5}Old & \cellcolor{green!5}Age - Over 65s & \cellcolor{green!5}1/23 & \cellcolor{green!5}4.348 \\  \hline
  \cellcolor{green!27}\small Interesting that you would do such a deep dive into the process, but mention \textbf{nothing} of the 'why'. Is laying off older workers effective? What research has led to that conclusion? The company simply wouldn't take such drastic action without evidence.\normalsize   & \cellcolor{green!27}Nothing & \cellcolor{green!27}Age - Youngsters & \cellcolor{green!27}1/41 & \cellcolor{green!27}2.439 \\  \hline
  \cellcolor{green!5}\small IBM has also been \textbf{AGEING}, Time to force them to RETIREMENT! !!!\normalsize   & \cellcolor{green!5}Ageing & \cellcolor{green!5}Age - General & \cellcolor{green!5}1/12 & \cellcolor{green!5}8.333 \\  \hline
  \cellcolor{green!27}\small Doesn't surprise me. A lot of \textbf{senior} workers who were let go before they could collect retirement are now greeters at Walmart\normalsize   & \cellcolor{green!27}Senior & \cellcolor{green!27}Age - Over 65s & \cellcolor{green!27}1/22 & \cellcolor{green!27}4.545 \\  \hline
  \cellcolor{green!5}\small Phailox legal? The government is corrupt and works for there interests. I don't trust any of there regulations to 'help' the market. If it was better for ibm they would have kept them. Why stifle growth? So \textbf{old} people and the government are happy?\normalsize   & \cellcolor{green!5}Old & \cellcolor{green!5}Age - Over 65s & \cellcolor{green!5}1/44 & \cellcolor{green!5}2.273 \\  \hline
  \cellcolor{green!27}\small Phailox or you let two parties organize and act among themselves with out the visible hand of government forcing people into boxes and wasting corporate cash flow on highering \textbf{old} people\normalsize   & \cellcolor{green!27}Old & \cellcolor{green!27}Age - Over 65s & \cellcolor{green!27}1/31 & \cellcolor{green!27}3.226 \\  \hline
  \cellcolor{green!5}\small securing the financial well being of the future workers requires going through the parents. Their children have to go through extensive and long education in order to be competitve. But without financial stability, fewer kids will be able to get higher education, not to mention there'll be a decline in succesful adults wanting to have kids for the same reasons. This will result in a smaller work force which will have to support the rest. So while numbers might initially increase due to throwing out the \textbf{old}, and bringing in the new. Its simply not sustainable. so just to make \textbf{old} ppl and government happy? yeh, or our economy, u pick.either way im not interested in continuing this discussion. i mainly wanted to provoke ur initial statement. Not get into a debate on socioeconomics or whatever its called in english.\normalsize   & \cellcolor{green!5}Old & \cellcolor{green!5}Age - Over 65s & \cellcolor{green!5}2/141 & \cellcolor{green!5}1.418 \\  \hline
  \cellcolor{green!27}\small The Truth that's agist, saying young people are inherently stupider then old\normalsize   & \cellcolor{green!27}Old & \cellcolor{green!27}Age - Over 65s & \cellcolor{green!27}1/12 & \cellcolor{green!27}8.333 \\  \hline
  \cellcolor{green!5}\small So many \textbf{dumb} folks don't get it. They hide it with excuses like "not updated" etc. Its all money without morals.\normalsize   & \cellcolor{green!5}Dumb & \cellcolor{green!5}Physical Identity - Physical (and Mental) Impairments & \cellcolor{green!5}1/21 & \cellcolor{green!5}4.762 \\  \hline
  \cellcolor{green!27}\small The disregard the tech industry has for technical experience is alarmingly scary. Yes, technology moves on and progress happens, but throwing away valuable tribal knowledge and \textbf{senior} developers provides exactly what benefit, now? Or is it just about throwing away anyone who might think stupid, risky, and low-benefit technical decisions like "let's migrate everything to MongoDB" is a good idea?\normalsize   & \cellcolor{green!27}Senior & \cellcolor{green!27}Age - Over 65s & \cellcolor{green!27}1/60 & \cellcolor{green!27}1.667 \\  \hline
  \cellcolor{green!5}\small "Neuroplasticity" starts declining far younger than the average tech developer's \textbf{age}, though. The rest of the argument is literally "we don't want to pay for experience". Which strikes me as critically dangerous. We're basically an industry that rewrites the world every 20 years for the sake of being able to get a bargain on labor. Which, by the way, will cost you on the back end when your fancy new shiny tech breaks, and you have shittons of security vulnerabilities, or your database loses critical info.\normalsize   & \cellcolor{green!5}Age & \cellcolor{green!5}Age - General & \cellcolor{green!5}1/86 & \cellcolor{green!5}1.163 \\  \hline
  \cellcolor{green!27}\small a company's best interest doesn't necessarily line up with what society has deemed ethical and acceptable. for example, it is probably in the best interest of a company to enslave humans and pay them \textbf{nothing}, but we don't allow that any more through rule of law. i think the intent of the video was to show that society has also drawn a line in the sand regarding a person's biological age; there are laws regarding \textbf{ageism}, however it seems IBM and other companies get around it by arbitration. if the video is correct, and the trend of current laws is to erode protections for \textbf{ageism}, and companies are currently using arbitration to make existing laws unenforceable, then the question the video is posing is probably: "Is this acceptable?"\normalsize   & \cellcolor{green!27}Ageism, Nothing & \cellcolor{green!27}Age - General, Age - Youngsters & \cellcolor{green!27}3/128 & \cellcolor{green!27}2.344 \\  \hline
  \cellcolor{green!5}\small Markus are you implying that  this business was stuck in a time capsule wage wise ? they've adjusted all along and restructured their internal accounting systems with the times. These people do not have Ronald reagan's pictures in their board rooms. this isn't even what it is being discussed. Hear this, some of the devices that talk to each other from the early to might 80s are still being made precisely because \textbf{nobody} can crack those.How can you crack a tool remotely when there are no similar devices in the wild anyway. Lastly there is no such a thing as globalized wages. You made that up. Cheap labor existed then just like it is prevalent now in the filed.\normalsize   & \cellcolor{green!5}Nobody & \cellcolor{green!5}Age - Youngsters & \cellcolor{green!5}1/119 & \cellcolor{green!5}0.84 \\  \hline
  \cellcolor{green!27}\small I agree with Markus. Even high-tech jobs can be done to a similar standard in China/India for much cheaper making it very difficult to be as economically viable in a developed country (unless the gov props them up). Cheap labour did exist before, BUT the key factor is cheap, high-speed, high-throughput global transportation of goods which makes it so easy to relocate as a company. Thus the workers of one country don't just directly compete against each other, but now workers from other countries too. Supply goes up, wages fall.With regards to IBM's \textbf{old} tech, \textbf{nobody} can crack those remotely sure, but they're so out-dated hardware and software-wise, they're practically useless. For uncrackable storage, they'd rather use an isolated modern server (which would only take <30 people to develop and maintain, not in the 10000s) or just make the switch to accessible networks with much stronger encryption and security systems. The ease of access and usability drops their costs, speeds up business (thus more revenue), which would happily make up for the risk associated with a breached server. (NB by risk here, I mean the product of probability of breach and cost of breach, like a decision matrix)\normalsize   & \cellcolor{green!27}Nobody, Old & \cellcolor{green!27}Age - Over 65s, Age - Youngsters & \cellcolor{green!27}2/199 & \cellcolor{green!27}1.005 \\  \hline
  \cellcolor{green!5}\small The problem with the eternal search for cutting costs is that it works for individual companies, but if they are all doing it then as a \textbf{nation} we cut the throats of our own consumer buying power and the economy falls in on itself\normalsize   & \cellcolor{green!5}Nation & \cellcolor{green!5}Nationality - General & \cellcolor{green!5}1/44 & \cellcolor{green!5}2.273 \\  \hline
  \cellcolor{green!27}\small Maybe your right, we should form a tech workers union. We can prevent companies from laying off competent workers and replacing them with young people and foreigners. I'm surprised \textbf{nobody} has done it yet.\normalsize   & \cellcolor{green!27}Nobody & \cellcolor{green!27}Age - Youngsters & \cellcolor{green!27}1/34 & \cellcolor{green!27}2.941 \\  \hline
  \cellcolor{green!5}\small How would unions help someone stay in business?  Unions almost destroyed the American auto industry in the 70s in the face of Japanese imports.  They are great in theory, but if a company has to compete globally it cant keep people just because of seniority and such things.  I suspect these lay offs had more to do with wages than \textbf{age} itself.  Of course people with the longest service and highest wages are going to tend to be older.  I just don't understand how its better to lay off everyone because of lost market share than its is to pare down expensive workers to stay competitive.  It sucks, but no one has realistic alternatives.\normalsize   & \cellcolor{green!5}Age & \cellcolor{green!5}Age - General & \cellcolor{green!5}1/114 & \cellcolor{green!5}0.877 \\  \hline
  \cellcolor{green!27}\small This video implies that the older workers are just as skilled/useful as the younger workers. There is no way that is true.Exhibit A - the company WANTS TO GET RID OF OLDER WORKERS....why do you think they want to do that? Do you think  they just randomly hate \textbf{old} people?Or maybe they're looking out for the fate of the company....\normalsize   & \cellcolor{green!27}Old & \cellcolor{green!27}Age - Over 65s & \cellcolor{green!27}1/62 & \cellcolor{green!27}1.613 \\  \hline
  \cellcolor{green!5}\small Am I the only one who thinks this (the firing, not the signing rights away and arbitration stuff) is normal? I am a developer and I have been planning from the beginning that most likely I won't be able to stay up to date with technology once I get in my fifties (right now still in my twenties), so I have been planning a life after I lose my job sometimes around that \textbf{age}. As you grow more experienced you get better at management and architectural stuff, but the problem is that there is simply less people needed in those positions. My (lack of) skills will simply not be marketable when I reach that \textbf{age}, but the upside is that my pay is comparatively good right now... so end result is that I have to make sure I can survive for forty years without a job in tech.\normalsize   & \cellcolor{green!5}Age & \cellcolor{green!5}Age - General & \cellcolor{green!5}2/148 & \cellcolor{green!5}1.351 \\  \hline
  \cellcolor{green!27}\small It was an interesting article but I sense a strong omission of information to create a biased narrative. I don't see the issue with offering people severance and retirement packages when companies need to let people go. The number one factor is longevity in the company not so much \textbf{age}. Longevity creates a difference in rate of pay - a person being paid \$70/hour doing the same job as a person getting paid \$50/hour is going to affect decision making. Just because a company exists doesn't mean you're guaranteed to work in that spot for 40 years. A company is a business, and while people are important, so is the business itself.\normalsize   & \cellcolor{green!27}Age & \cellcolor{green!27}Age - General & \cellcolor{green!27}1/112 & \cellcolor{green!27}0.893 \\  \hline
  \cellcolor{green!5}\small If they lay off skilled workforce, how are they benefiting? If the "\textbf{old}" aren't bringing higher values to the company as young workers, but cost considerably more why would a company keep them if they do not have a problem to hire more people? \textbf{Nobody} in the right mind fires you because you are \textbf{old}. They fire you because you are not worth your salary.\normalsize   & \cellcolor{green!5}Nobody, Old & \cellcolor{green!5}Age - Over 65s, Age - Youngsters & \cellcolor{green!5}3/65 & \cellcolor{green!5}4.615 \\  \hline
  \cellcolor{green!27}\small Hang on, if the 20,000 job cuts to over 40s was 60 of job cuts in the last five years, isn't that roughly proportional? If 40 of the cuts in the last five years were to people under 40, there doesn't seem to be a lot of evidence of \textbf{ageism}. Furthermore, the points system referred to further in the video is based on length of time in a particular "job level", NOT the total amount of time in the company, meaning people were getting more points if they had been promoted recently, and less if they hadn't been promoted in a significant period of time (rather than because they had been with the company the longest). The legal stuff does indeed seem very fishy, but I think statistically the case in this video is a little unconvincing.\normalsize   & \cellcolor{green!27}Ageism & \cellcolor{green!27}Age - General & \cellcolor{green!27}1/137 & \cellcolor{green!27}0.73 \\  \hline
  \cellcolor{green!5}\small I don't think it was stated that all older workers were fit for their role. IBM has almost 400K workers. If half of them were over 40 then that 20K reduction (even assuming it occurred in a single year) means that only 10 of employees in that \textbf{age} group were let go. The relevant question is not whether the median 40+ employee is a good one, but rather, how effective are the worst 10 of employees in that \textbf{age} range performing.\normalsize   & \cellcolor{green!5}Age & \cellcolor{green!5}Age - General & \cellcolor{green!5}2/81 & \cellcolor{green!5}2.469 \\  \hline
  \cellcolor{green!27}\small no because the younger kids are expected to leave or be fired due to lack of skills/performance. with \textbf{age}, you are far less likely to leave or lose skills as you have decades of industry experience. therefore your reasoning is flawed and wrong. it is not balanced out.\normalsize   & \cellcolor{green!27}Age & \cellcolor{green!27}Age - General & \cellcolor{green!27}2/48 & \cellcolor{green!27}4.167 \\  \hline
  \cellcolor{green!5}\small Leftist millenials are the ones literally wishing that \textbf{old} people would "hurry up and die", so don't expect to get any compassion from them.\normalsize   & \cellcolor{green!5}Old & \cellcolor{green!5}Age - Over 65s & \cellcolor{green!5}1/24 & \cellcolor{green!5}4.167 \\  \hline
  \cellcolor{green!27}\small Millennials just want to get their foot into the door, so they will get as much experience as these older people this video was talking about, once they are the same \textbf{age} as these people who got "laid" off. How do you not understand that?!\normalsize   & \cellcolor{green!27}Age & \cellcolor{green!27}Age - General & \cellcolor{green!27}1/45 & \cellcolor{green!27}2.222 \\  \hline
  \cellcolor{green!5}\small Kel and that's the part that's missing in people's reasoning in the comment section. Kids cost peanuts to hire and when they start their first job they are still on their parent's insurance and sometimes even live at home. so everything is fine. and that is the naivete. people in this whole comment section are eating their avocado toast while laughing at us. we will all get \textbf{old} , unless you die.\normalsize   & \cellcolor{green!5}Old & \cellcolor{green!5}Age - Over 65s & \cellcolor{green!5}1/72 & \cellcolor{green!5}1.389 \\  \hline
  \cellcolor{green!27}\small You keep your job by providing a positive net value to a company. They aren't going to intentionally lay you off if you're doing that. When a company is doing poorly or not as well as it used to, it needs to figure out why and respond.The lowest hanging \textbf{fruit} is the salaries of the people it employs. When you have two employees that are under performing the same amount and you pay one less than the other, you lay off the one that you pay more first. Older employees tend to make more money. Hard to prove \textbf{ageism} indeed.I dislike those mandatory arbitration clauses. They seem like a way to deny people due process if they need it. I'd like to see that topic explored more.\normalsize   & \cellcolor{green!27}Ageism, Fruit & \cellcolor{green!27}Age - General, Sexual Identity - Male homosexuality & \cellcolor{green!27}2/129 & \cellcolor{green!27}1.55 \\  \hline
  \cellcolor{green!5}\small what a stupid news you always need a young blood in anything for innovation it s a fact you can not put a 50 years \textbf{old} person in a tech company it s not going to work the tech world is always changing\normalsize   & \cellcolor{green!5}Old & \cellcolor{green!5}Age - Over 65s & \cellcolor{green!5}1/43 & \cellcolor{green!5}2.326 \\  \hline
  \cellcolor{green!27}\small Amazing to see Vox looking into aspects that are ignored by most other media agencies.But in terms of story, the data in video above doesn't seem to display much coherence. 60 of firings involving people who are 40+ doesn't seem like a big deal, given that in a workforce with ages spread over 22-65, 40 would be the median number. However, the performance charts seem to indicate a different story given that there is a clear motive to bring down the points of \textbf{senior} employees through a biased grading system - looks like it's just the beginning of the apocalypse where the stage is being set for mass exodus of \textbf{senior} employees in near future and IBM wants to make sure its ass is covered. On another note, this is something that is happening in IT services firms in India on a massive scale (read hundreds of thousands of \textbf{senior} employees) to cut costs and has happened before in the \textbf{age} of manufacturing and automation  worldlwide - so \textbf{nothing} new about it.\normalsize   & \cellcolor{green!27}Age, Nothing, Senior & \cellcolor{green!27}Age - General, Age - Over 65s, Age - Youngsters & \cellcolor{green!27}5/173 & \cellcolor{green!27}2.89 \\  \hline
  \cellcolor{green!5}\small In my opinion, an individual or a private business can hire and fire as they please. I think the way to deal with this is to educate people that with \textbf{age}, you are less valuable for employers. People should be aware of that and get ready for situation that one day this may happen today. I am totally aware that when I get older, I will be less of an asset for employers. I am 31 and it's slowly coming. I have to get ready.\normalsize   & \cellcolor{green!5}Age & \cellcolor{green!5}Age - General & \cellcolor{green!5}1/85 & \cellcolor{green!5}1.176 \\  \hline
  \cellcolor{green!27}\small lump of labor fallacy?\normalsize   & \cellcolor{green!27}Lump & \cellcolor{green!27}Physical Identity - Physical Features & \cellcolor{green!27}1/4 & \cellcolor{green!27}25.0 \\  \hline
  \cellcolor{green!5}\small The video stated the firings had \textbf{nothing} to do with performance. Older workers with high performance ratings, older workers considered by their peers as subject matter experts are getting whacked.\normalsize   & \cellcolor{green!5}Nothing & \cellcolor{green!5}Age - Youngsters & \cellcolor{green!5}1/30 & \cellcolor{green!5}3.333 \\  \hline
  \cellcolor{green!27}\small They just dump \textbf{old} workers. And they dump US workers in favor of H-1Bs. Their performance has paralleled their use of H-1bs - they suck more as they get more H-1Bs.\normalsize   & \cellcolor{green!27}Old & \cellcolor{green!27}Age - Over 65s & \cellcolor{green!27}1/31 & \cellcolor{green!27}3.226 \\  \hline
  \cellcolor{green!5}\small I agree. People don't understand that tech changes every 5-10yrs. So you have to be aware of whats upcoming 5 - 10yrs and start learning the skills for whats upcoming. Problem is people can't make the jump or they have become to loyal to a particular language or technology and forgot to see its future obsolescence. The only chance older workers have is working on updating some \textbf{old} legacy or deprecated system thats being converted to the present.\normalsize   & \cellcolor{green!5}Old & \cellcolor{green!5}Age - Over 65s & \cellcolor{green!5}1/78 & \cellcolor{green!5}1.282 \\  \hline
  \cellcolor{green!27}\small Tech changes faster than a fashion model changes her shoes. The requirements for jobs are written with high specificity - 4.5 Years in AssSQL version 3.5. Then they can either select some young guy who has 5 years in that skill or hire an H1B because they can't find anyone else. If you can't go off on your own in tech, it is not a good field. My take? Work for an employer a few years, then develop your own product or service. If you are at the mercy of an employer, you are playing lotto. You are one upper management decision away from a Dear John letter. The CEOs get bonuses when deadweight drops. There is no pride anymore in having an employee have a lifetime living from your business.My last consulting job at GE opened my eyes. Hired at a crappy hourly rate, manager told us we like the Dell PCs. After 3 years the depreciation or expenses or whatever crosses a line and it just pays to get new ones that can be deducted. They are commodities anyway and \textbf{nothing} special. Same as the employees hired for skills. I know a company here that let guys go who were there since the 90s working on their VB6 product to early this year. They are completely unemployable, some tried in 100s of companies.\normalsize   & \cellcolor{green!27}Nothing & \cellcolor{green!27}Age - Youngsters & \cellcolor{green!27}1/226 & \cellcolor{green!27}0.442 \\  \hline
  \cellcolor{green!5}\small spelunkerd Shipped jobs tô inda. Simply don't hire non-\textbf{Indian} technicians. Hire ONE \textbf{Indian} and he will never hire a non-\textbf{Indian} for the rest of his time as manager.\normalsize   & \cellcolor{green!5}Indian & \cellcolor{green!5}Ethnicity - Native-American & \cellcolor{green!5}1/28 & \cellcolor{green!5}3.571 \\  \hline
  \cellcolor{green!27}\small People shouldn't have to squirrel away money for healthcare.  Healthcare should be provided like in every other developed \textbf{nation}.\normalsize   & \cellcolor{green!27}Nation & \cellcolor{green!27}Nationality - General & \cellcolor{green!27}1/19 & \cellcolor{green!27}5.263 \\  \hline
  \cellcolor{green!5}\small Building that cushion would be impossible. And then, the high-tech world isn't just one mega thing where everyone is doing the same thing. It really depends on what you do. If you work for certain companies with a lot of short term projects then, sure. But a lot of work is simple upkeep. Things need to be patched. \textbf{Old} work needs to be redone. Things slowly decay and fall apart. You could be doing this kind of work for a big system for a decade, no problem. It's just not as flashy as building a new AI mood reading coffee cup that connects to facebook that \textbf{nobody} needs.\normalsize   & \cellcolor{green!5}Nobody, Old & \cellcolor{green!5}Age - Over 65s, Age - Youngsters & \cellcolor{green!5}2/108 & \cellcolor{green!5}1.852 \\  \hline
  \cellcolor{green!27}\small Those \textbf{old} people voted for terrible worker protections for how many years?I'm against discrimination, but in this case you reap what you sow.\normalsize   & \cellcolor{green!27}Old & \cellcolor{green!27}Age - Over 65s & \cellcolor{green!27}1/24 & \cellcolor{green!27}4.167 \\  \hline
  \cellcolor{green!5}\small don't blame the company with a socialist mindset. 30-40 year \textbf{old} technical skills are of no use today, it is also upto the employees to keep up with the fast evolving nature of industry. the companies are not masters and employees are not servants, its a deal to get paid for the services  one provide, and both sides has the option to break it, and exit. when employees do that, switch for better pay, then i guess its fine from your point of view, .. but if company does that's wrong? if an employee understands his part of the partership, its way more easy to keep themselves updated with the latest technology/skills, new changes only come in incremental patches. if you really wanted to point out some wrong doings or bad practices by the company then you should have drilled down further as to what the younger employees have that the older generation don't.. that would have been better, maybe you would have discovered something relevant.\normalsize   & \cellcolor{green!5}Old & \cellcolor{green!5}Age - Over 65s & \cellcolor{green!5}1/166 & \cellcolor{green!5}0.602 \\  \hline
  \cellcolor{green!27}\small 4:35 when its own assessment states that they are good to stay and then rig the game with points to check out oldies. Its nasty behaviour . Bottom line is to check out "costly" employees and hire "cheap" labour and stress them out , before they turn wise and \textbf{old} and the cycle continues. Experience will teach you better.\normalsize   & \cellcolor{green!27}Old & \cellcolor{green!27}Age - Over 65s & \cellcolor{green!27}1/59 & \cellcolor{green!27}1.695 \\  \hline
  \cellcolor{green!5}\small Cut off \textbf{age} is only 40. We're getting older, yet we get rid of workers earlier and earlier. What could go wrong.\normalsize   & \cellcolor{green!5}Age & \cellcolor{green!5}Age - General & \cellcolor{green!5}1/22 & \cellcolor{green!5}4.545 \\  \hline
  \cellcolor{green!27}\small But why are they pushing older people out? Obviously there is a bias towards older people but that can't be the only reason. You just made a video telling us that an \textbf{age} group is being fired, then did not tell us why they're doing it\normalsize   & \cellcolor{green!27}Age & \cellcolor{green!27}Age - General & \cellcolor{green!27}1/46 & \cellcolor{green!27}2.174 \\  \hline
  \cellcolor{green!5}\small You mean your father ?  "my dad" , how \textbf{old} are you , 7 ? tsssk tsssk\normalsize   & \cellcolor{green!5}Old & \cellcolor{green!5}Age - Over 65s & \cellcolor{green!5}1/17 & \cellcolor{green!5}5.882 \\  \hline
  \cellcolor{green!27}\small Back in my days, people died at work when they became too \textbf{old} to clombe up the mine! you people got problems...\normalsize   & \cellcolor{green!27}Old & \cellcolor{green!27}Age - Over 65s & \cellcolor{green!27}1/22 & \cellcolor{green!27}4.545 \\  \hline
  \cellcolor{green!5}\small The future is now \textbf{old} man.\normalsize   & \cellcolor{green!5}Old & \cellcolor{green!5}Age - Over 65s & \cellcolor{green!5}1/6 & \cellcolor{green!5}16.667 \\  \hline
  \cellcolor{green!27}\small Not sure if this is a great argument. The reason older people may get laid off could also be similar to the reason people fresh out of high school don't get hired at large companies. It's not because of their \textbf{age} but because of their experience. While younger people may not be experienced enough older generations might have the experience but not the willingness to change and grow. You probably can't continue to grow your company if it just stays stagnant. But then again why do companies have to grow? Why can't they just be a good resource to their communities?\normalsize   & \cellcolor{green!27}Age & \cellcolor{green!27}Age - General & \cellcolor{green!27}1/101 & \cellcolor{green!27}0.99 \\  \hline
  \cellcolor{green!5}\small You gotta constantly update your skills in tech to survive. You can't blame companies for hiring fresh blood straight out of school with new knowledge of technology instead of some complacent \textbf{old} fart.\normalsize   & \cellcolor{green!5}Old & \cellcolor{green!5}Age - Over 65s & \cellcolor{green!5}1/33 & \cellcolor{green!5}3.03 \\  \hline
  \cellcolor{green!27}\small Is \textbf{age} discrimination even real? If it's real then why are car insurance, property insurance, and health insurance companies allowed to discriminate by \textbf{age}? And what about minors? De facto speaking, they are literally property of their guardians.\normalsize   & \cellcolor{green!27}Age & \cellcolor{green!27}Age - General & \cellcolor{green!27}2/38 & \cellcolor{green!27}5.263 \\  \hline
  \cellcolor{green!5}\small I hate working with \textbf{old} tech guys who don't see problems and pretend the \textbf{old} ways are better. They keep avoiding stressful new technology like Kubernetes while instead doing essentially the same thing over and over. For some reason our culture values comfort as you \textbf{age} more and more, your peers will view you as less of a person if you move around, change wives, and go out of town and make new friends often. We need to change the culture of aging in America.\normalsize   & \cellcolor{green!5}Age, Old & \cellcolor{green!5}Age - General, Age - Over 65s & \cellcolor{green!5}3/85 & \cellcolor{green!5}3.529 \\  \hline
  \cellcolor{green!27}\small In tech jobs, younger is cheaper.  It's actually really significant.  Companies often have to pay their \textbf{senior} employees close to double what they pay their junior employees.  Now the question becomes, do the older guys produce twice as much as the younger?  Some of them do that and more, others have outdated education and can't keep up with the younger guys in the first place.  If the company had to fire someone and two people were roughly equivalent in terms of ability, they're going to keep the younger one because they're cheaper.\normalsize   & \cellcolor{green!27}Senior & \cellcolor{green!27}Age - Over 65s & \cellcolor{green!27}1/92 & \cellcolor{green!27}1.087 \\  \hline
  \cellcolor{green!5}\small So the people IBM eliminated \textbf{age} 40+ made up 60 of their job cuts? That doesn't seem very biased... Also, your title says 20k older workers, but the video says 60 of 20k, which is only 12k. Which is it?\normalsize   & \cellcolor{green!5}Age & \cellcolor{green!5}Age - General & \cellcolor{green!5}1/40 & \cellcolor{green!5}2.5 \\  \hline
  \cellcolor{green!27}\small IBM is a business they cant just keep \textbf{old} people walking around doing \textbf{nothing}. Businesses are for profits!\normalsize   & \cellcolor{green!27}Nothing, Old & \cellcolor{green!27}Age - Over 65s, Age - Youngsters & \cellcolor{green!27}2/18 & \cellcolor{green!27}11.111 \\  \hline
  \cellcolor{green!5}\small Ageism is BS.\normalsize   & \cellcolor{green!5}Ageism & \cellcolor{green!5}Age - General & \cellcolor{green!5}1/3 & \cellcolor{green!5}33.333 \\  \hline
  \cellcolor{green!27}\small IBM did \textbf{nothing} wrong.\normalsize   & \cellcolor{green!27}Nothing & \cellcolor{green!27}Age - Youngsters & \cellcolor{green!27}1/4 & \cellcolor{green!27}25.0 \\  \hline
  \cellcolor{green!5}\small IBM is not even close to being the only company that does this. For a business writing class, I'm doing a report on unfair tech workplace practices that includes a section on \textbf{ageism}. In some places it's just the way people think and act, and others it's blatantly intentional. Regardless, most of the people getting fired are 40-50, meaning they're within their peak performance \textbf{age} and can't prove that they're competent workers when they got fired for something that's often assumed to be poor performance. It's hard to get rehired then, even when you had a really good recommendation from your \textbf{old} job.\normalsize   & \cellcolor{green!5}Age, Ageism, Old & \cellcolor{green!5}Age - General, Age - Over 65s & \cellcolor{green!5}3/103 & \cellcolor{green!5}2.913 \\  \hline
  \cellcolor{green!27}\small After  40  years  "How  \textbf{Apple}  quietly  pushed  out  1,00,000  older  workers."\normalsize   & \cellcolor{green!27}Apple & \cellcolor{green!27}Ethnicity - Native-American & \cellcolor{green!27}1/11 & \cellcolor{green!27}9.091 \\  \hline
  \cellcolor{green!5}\small kinda makes sense to let go \textbf{old} people from the company perspective\normalsize   & \cellcolor{green!5}Old & \cellcolor{green!5}Age - Over 65s & \cellcolor{green!5}1/12 & \cellcolor{green!5}8.333 \\  \hline
  \cellcolor{green!27}\small I don't blame IBM, many \textbf{old} peoples skills are just obsolete as technology has blown past them.\normalsize   & \cellcolor{green!27}Old & \cellcolor{green!27}Age - Over 65s & \cellcolor{green!27}1/17 & \cellcolor{green!27}5.882 \\  \hline
  \cellcolor{green!5}\small While this is a good point to bring to the public, the argument seems to miss/ignore the fact that IBM was close to dying and performed the single most successful pivot of any tech company into Cloud computing and services. A workforce experienced in hardware is unfortunately not very useful in that \textbf{transition}, so it makes sense to me that they had to \textbf{thin} the ranks to the best people to hire in expertise in cloud development. It's a shame, but that's the tech sector. It's a fast moving industry, and IBM was very close to being left behind. That wouldn't make as good of a story obviously and it's a lot easier to only consider one side and make IBM look like a corporate monster\normalsize   & \cellcolor{green!5}Thin, Transition & \cellcolor{green!5}Physical Identity - Physical Features, Sexual Identity - Transexuality & \cellcolor{green!5}2/126 & \cellcolor{green!5}1.587 \\  \hline
  \cellcolor{green!27}\small That performance to \textbf{age} chart commentary could be misleading. It's \textbf{common} to see older employees be 1. Very competent in what they know how to do 2. Incredibly complacent and unwilling to improve themselves.\normalsize   & \cellcolor{green!27}Age, Common & \cellcolor{green!27}Age - General, Social Class - Working class & \cellcolor{green!27}2/34 & \cellcolor{green!27}5.882 \\  \hline
  \cellcolor{green!5}\small I'm one minute in and your story has lost me. There's layoffs are across the board. 60 over the \textbf{age} of 40. These older people know significantly less than younger generations that come after them, are structureless less adaptive and working in segments not returning the great yields of yesteryear. End of story.\normalsize   & \cellcolor{green!5}Age & \cellcolor{green!5}Age - General & \cellcolor{green!5}1/53 & \cellcolor{green!5}1.887 \\  \hline
  \cellcolor{green!27}\small How do other tech Giants like \textbf{apple} get rid of their work force?\normalsize   & \cellcolor{green!27}Apple & \cellcolor{green!27}Ethnicity - Native-American & \cellcolor{green!27}1/13 & \cellcolor{green!27}7.692 \\  \hline
  \cellcolor{green!5}\small IBM is a private company and can do whatever tf they want. Of course a company wants younger employees so they can train them throughout their career and pay them less than older employees. And older employees are less needing of opportunities than younger employees. This video is dumb\normalsize   & \cellcolor{green!5}Dumb & \cellcolor{green!5}Physical Identity - Physical (and Mental) Impairments & \cellcolor{green!5}1/49 & \cellcolor{green!5}2.041 \\  \hline
  \cellcolor{green!27}\small EdwardERS pretty much proves that \textbf{age} discrimination in a company such as IBM is all about money.\normalsize   & \cellcolor{green!27}Age & \cellcolor{green!27}Age - General & \cellcolor{green!27}1/17 & \cellcolor{green!27}5.882 \\  \hline
  \cellcolor{green!5}\small 5:54 - Just like how \textbf{Apple} is very \textbf{old} now - much like its user-base...No real growth apart from sly tactics that I knew about in 2011 with the slowing down/weaker battery thing among MANY others yet to hit the spotlight.\normalsize   & \cellcolor{green!5}Apple, Old & \cellcolor{green!5}Age - Over 65s, Ethnicity - Native-American & \cellcolor{green!5}2/41 & \cellcolor{green!5}4.878 \\  \hline
  \cellcolor{green!27}\small Ieuan Hunt, that's stupid. It's a bunch of 1s and 0s. \textbf{Nobody} will forget how to use it anytime soon, we'll be using binary by the time we're dead.\normalsize   & \cellcolor{green!27}Nobody & \cellcolor{green!27}Age - Youngsters & \cellcolor{green!27}1/29 & \cellcolor{green!27}3.448 \\  \hline
  \cellcolor{green!5}\small Ultimately, it is not all just 1s and zeroes. Itr's case of knowing the assembly language for a given system or even an \textbf{old} language like COBOL or FORTRAN. The National Archives have run into this in a small way. They've learned that they need to retain and maintain a device that will play the media that they have,. Films have been "lost" until a device was reverse-engineered to play the media. I'm \textbf{old} enough to recall Hollerith (punch) cards and magnetic tapes for storing data. Is it reasonable to expect companies to transfer all of their data over onto other media every time the media changes. No. This is why backwards compatibility is such an important concern.\normalsize   & \cellcolor{green!5}Old & \cellcolor{green!5}Age - Over 65s & \cellcolor{green!5}2/118 & \cellcolor{green!5}1.695 \\  \hline
  \cellcolor{green!27}\small The Mad Librarian thank you. I am not too familiar with technology but I am an avid Gamer and archiving \textbf{old} games is a huge problem. There are many games from the Atari 2600 era that are impossible to find and there are few people out there who can program them. And gaming is a relatively young invention. Imagine trying to work with Technology from the Cold War. It would be a nightmare.\normalsize   & \cellcolor{green!27}Old & \cellcolor{green!27}Age - Over 65s & \cellcolor{green!27}1/73 & \cellcolor{green!27}1.37 \\  \hline
  \cellcolor{green!5}\small Sort of but MP4 and JPEG aren't as difficult to decode as tape. Pretty sure the problem with \textbf{age} discrimination will be even harder to prove in the future when all you need to do is read streams of pdf, a header with some basic encoding, or an audio file.\normalsize   & \cellcolor{green!5}Age & \cellcolor{green!5}Age - General & \cellcolor{green!5}1/50 & \cellcolor{green!5}2.0 \\  \hline
  \cellcolor{green!27}\small the reason why archiving \textbf{old} games is a problem is no one cares about the \textbf{old} junk. and retards talking about how people can't figure out how to use mp4 in the future. digital file format are not the same as programming language. all digital files are essentially ones and zeros. so in the future it'll be a lot easier to create new apps because you won't need to learn programming like you do today. infacct programming today is so much easier than it was 20 years ago. as AI gets better and the world gets more connected, it'll be almost impossible to lose data because of multiple backups\normalsize   & \cellcolor{green!27}Old & \cellcolor{green!27}Age - Over 65s & \cellcolor{green!27}2/109 & \cellcolor{green!27}1.835 \\  \hline
  \cellcolor{green!5}\small Even the \textbf{old} edison tubes are archived and sometimes reproduced. There will always be someone with the knowledge. Heck, the british museum even got experts who reversed engineered ancient cuneiform writing and language. The real issue is people must realise that all business are just that....only business and not a care for humanity unless it makes a good business to care....like legalized prostitution in europe.Those who will soon enter adult life should take this lesson at heart. Life is cruel.Terry Pratchett — 'THERE IS NO JUSTICE said Death JUST ME'\normalsize   & \cellcolor{green!5}Old & \cellcolor{green!5}Age - Over 65s & \cellcolor{green!5}1/92 & \cellcolor{green!5}1.087 \\  \hline
  \cellcolor{green!27}\small Till \textbf{old} \textbf{age} takes you back to the weeds.\normalsize   & \cellcolor{green!27}Age, Old & \cellcolor{green!27}Age - General, Age - Over 65s & \cellcolor{green!27}2/9 & \cellcolor{green!27}22.222 \\  \hline
  \cellcolor{green!5}\small Ibm did the right thing.. look what happened to japan when they keep their \textbf{old} workers, the company will stalling bcus lack of innovation.\normalsize   & \cellcolor{green!5}Old & \cellcolor{green!5}Age - Over 65s & \cellcolor{green!5}1/24 & \cellcolor{green!5}4.167 \\  \hline
  \cellcolor{green!27}\small My dad worked for them for 10 years and got laid off in 2003 at 41 years \textbf{old}. Little bit surreal to watch this\normalsize   & \cellcolor{green!27}Old & \cellcolor{green!27}Age - Over 65s & \cellcolor{green!27}1/24 & \cellcolor{green!27}4.167 \\  \hline
  \cellcolor{green!5}\small 40 is \textbf{old}? Jeez, I knew it wasn't young but I figured I'd at least have 10 more years before I'm forced out at work for being unmarketable. Btw when I was downsized 3 years ago the unemployment office told me to change my personal email so that employers couldn't see my birth year. He said, "it's not an issue now, bit in a few years, it will make you unmarketable." Ugh f this noise, it's time to start my own business.  Corporations suck\normalsize   & \cellcolor{green!5}Old & \cellcolor{green!5}Age - Over 65s & \cellcolor{green!5}1/84 & \cellcolor{green!5}1.19 \\  \hline
  \cellcolor{green!27}\small Until now, didn't know IBM was still in existence. \textbf{Old} dinosaur of a company.\normalsize   & \cellcolor{green!27}Old & \cellcolor{green!27}Age - Over 65s & \cellcolor{green!27}1/14 & \cellcolor{green!27}7.143 \\  \hline
  \cellcolor{green!5}\small I don't have enough information regarding the REASONS for the layoffs. It's easy to say it's \textbf{age} discrimination when there are statistically more older people in a business than younger, so when 20k people are laid off/fired the statistics are skewed. Basically you have 30 \textbf{r\textbf{ed}}, 30 orange, 20 \textbf{y\textbf{e\textbf{llow}}}, 10 green and 10 blue skittles in a bag and pull our 20 Skittles, odds are you'll have less Blue and Green skittles in that pool than \textbf{R\textbf{ed}} or Orange.I mean I don't know the details so I'm not going to say that this video does a poor job at investigating the story, but still... It makes you think.\normalsize   & \cellcolor{green!5}Age, Red, Yellow & \cellcolor{green!5} Ideological and Political Identity - General, Age - General, Ethnicity - Asian (East- China, Japan, Korea, Philippines, Vietnam), Ethnicity - Native-American, Nationality - Chinese, Nationality - Japanese & \cellcolor{green!5}8/110 & \cellcolor{green!5}7.273 \\  \hline
  \cellcolor{green!27}\small people get \textbf{old} and unnecessary, big deal.\normalsize   & \cellcolor{green!27}Old & \cellcolor{green!27}Age - Over 65s & \cellcolor{green!27}1/7 & \cellcolor{green!27}14.286 \\  \hline
  \cellcolor{green!5}\small You gunna be \textbf{old} and unnecessary one day.  I would love to see your face when you get fired because of it lol\normalsize   & \cellcolor{green!5}Old & \cellcolor{green!5}Age - Over 65s & \cellcolor{green!5}1/23 & \cellcolor{green!5}4.348 \\  \hline
  \cellcolor{green!27}\small My estimated IBM demographics: 25-40 years \textbf{old}, 40. 40+ years \textbf{old}, 60.\normalsize   & \cellcolor{green!27}Old & \cellcolor{green!27}Age - Over 65s & \cellcolor{green!27}2/12 & \cellcolor{green!27}16.667 \\  \hline
  \cellcolor{green!5}\small Meneer jeroen, 40 is \textbf{old} by current standards .\normalsize   & \cellcolor{green!5}Old & \cellcolor{green!5}Age - Over 65s & \cellcolor{green!5}1/9 & \cellcolor{green!5}11.111 \\  \hline
  \cellcolor{green!27}\small IBM had a recruitment in my school, and I just felt like it was a \textbf{communist} company, everything they said felt so fake.\normalsize   & \cellcolor{green!27}Communist & \cellcolor{green!27} Ideological and Political Identity - General & \cellcolor{green!27}1/23 & \cellcolor{green!27}4.348 \\  \hline
  \cellcolor{green!5}\small How many of those IBM \textbf{engineers} were voting for republicans all those years?\normalsize   & \cellcolor{green!5}Engineers & \cellcolor{green!5}Nationality - General & \cellcolor{green!5}1/13 & \cellcolor{green!5}7.692 \\  \hline
  \cellcolor{green!27}\small So, you are saying tech has changed, and IBM needed to reduce workforce. Older workers usually make more, but do the same work. That's wage discrimination, not age\normalsize   & \cellcolor{green!27}Age & \cellcolor{green!27}Age - General & \cellcolor{green!27}1/28 & \cellcolor{green!27}3.571 \\  \hline
  \cellcolor{green!5}\small In large \textbf{old} tech companies, you get raises and bonuses until you are too expensive compared to the work you do. Its not the workers fault - who is going to say no to more money. Its a \textbf{dumb} system.\normalsize   & \cellcolor{green!5}Dumb, Old & \cellcolor{green!5}Age - Over 65s, Physical Identity - Physical (and Mental) Impairments & \cellcolor{green!5}2/40 & \cellcolor{green!5}5.0 \\  \hline
  \cellcolor{green!27}\small Tristan Möller Signing away a person's rights should be illegal, and forcing the resignation of a totally competent but aging employee who you want to get rid of so you don't have to pay their retirement or whatever is legal, but its dishonest practice for a corporation that preaches about how valued their employees are. How would you feel being forced to resign at 42 after decades of hard work invested at the company for no other reason than your \textbf{age}?\normalsize   & \cellcolor{green!27}Age & \cellcolor{green!27}Age - General & \cellcolor{green!27}1/81 & \cellcolor{green!27}1.235 \\  \hline
  \cellcolor{green!5}\small I don't know where you got that information, because it is often more expensive to lay off the \textbf{elderly} due to the lost of expertise and brain drain that causes (Shah, B., \& Gregar, A, 2013) , besides that the \textbf{elder} workers are often found to be more engaged and " Engaged employees use less health care, take fewer sick days, are more productive, have longer tenure, and create stronger customer relationships" (Pitt‐Catsouphes, M., Matz‐ Kosta, C., \& Besen, E. ,2009).In terms of adaptability this is a \textbf{common} missinformation  that is often a source of the lack of employment for the \textbf{elderly} (Taskforce on the Aging of the American Workforce., 2008).With that being said this more of a lack of  corporate social responsibility and shared value and bad manegerial practices knowing the shrinking labor pool.aaaaaaaaaaaaaaaand you are right in terms that in certain jobs the \textbf{elderly} are less productive (I made this comment not to like annoy you or anything, I was just curious about if this was true, I don't mean any harm or insult).(this comment only applies in non physical demanding jobs)sources:-Pitt‐Catsouphes, M., Matz‐Kosta, C., \& Besen, E. (2009). Workplace flexibility: Findings from the \textbf{Age} and Generations Study. Boston: Sloan Center on Aging and Work, Boston College.-Taskforce on the Aging of the American Workforce. (2008). Report of the Taskforce on the Aging of the American Workforce. Retrieved from:  [LINK] -Shah, B., \& Gregar, A. (2013). Retention of Aging Employees and Organizational Performance: Comparative Study EU Countries and US. Proceedings Of The European Conference On Management, Leadership \& Governance, 449-454.\normalsize   & \cellcolor{green!5}Age, Common, Elder, Elderly & \cellcolor{green!5}Age - General, Age - Over 65s, Social Class - Working class & \cellcolor{green!5}6/268 & \cellcolor{green!5}2.239 \\  \hline
  \cellcolor{green!27}\small this should be illegal wtf, companies shouldn't be able to side step \textbf{age} discrimination\normalsize   & \cellcolor{green!27}Age & \cellcolor{green!27}Age - General & \cellcolor{green!27}1/14 & \cellcolor{green!27}7.143 \\  \hline
  \cellcolor{green!5}\small This is why china and south korea will lead the technology. If you don't perform, you're gone. IBM was in \textbf{r\textbf{ed}} for years. If these workers did great job, they would still be there\normalsize   & \cellcolor{green!5}Red & \cellcolor{green!5} Ideological and Political Identity - General, Ethnicity - Native-American & \cellcolor{green!5}2/34 & \cellcolor{green!5}5.882 \\  \hline
  \cellcolor{green!27}\small Yeah but why? Do older employees have higher salaries, and targeting them reduced costs? Does IBM need less skilled and experienced workers for their shifting business model? Why would this be happening? You said it's not because older employees were getting bad at their jobs.Also, 60 of terminations being 40+ does not seem disproportionate. If people start working at 18 and retire at 65, and if we assume IBM took on a good chunk of baby boomers in their formative years, 40+ is about the cutoff for 60 I'd expect.Did you just throw the \textbf{ageism} spin on a \textbf{nothing} story, like you did with your video about how poor sleep affects \textbf{black} Americans?\normalsize   & \cellcolor{green!27}Ageism, Black, Nothing & \cellcolor{green!27}Age - General, Age - Youngsters, Ethnicity - Black & \cellcolor{green!27}3/115 & \cellcolor{green!27}2.609 \\  \hline
  \cellcolor{green!5}\small IBM needed less skilled and experienced workers because it costs more to pay them in places where their job can be easily automated or handed out to a low skill worker. This is something we've seen time and time again, from the skilled workers in the Industrial Revolution losing their livelihood, to the American manufacturers losing their jobs to a small amount of machines and a huge population in third world countries, ready for exploitation by multinationals. While this layoff may not be directly tied to \textbf{age}, it is still the laying off of 20000 people. 20000 people who no longer have a livelihood, who can no longer own a home, all in the name of profit.\normalsize   & \cellcolor{green!5}Age & \cellcolor{green!5}Age - General & \cellcolor{green!5}1/117 & \cellcolor{green!5}0.855 \\  \hline
  \cellcolor{green!27}\small Ok but the question is still, why is this a story about \textbf{ageism} in IBM instead of the broader concern about job automation?\normalsize   & \cellcolor{green!27}Ageism & \cellcolor{green!27}Age - General & \cellcolor{green!27}1/23 & \cellcolor{green!27}4.348 \\  \hline
  \cellcolor{green!5}\small Older people are usually working on the older tech, because they were the ones who designed  them. Its full time to support the job (many times way over full time) and the company demands it because of contracts. If the older worker wants to learn new tech, it has to be done on their limited free time. And being older, what free time you get means more time with family demands, taking care of older parents, etc..Older workers had raises and more likely are in \textbf{old} fashion stock/bonus plans, which is rarely given to new hires. So yes, older workers are expensive. As the \textbf{old} tech is phased out, the new hires were working on next gen products. Only a few older workers are kept, the rest are given the boot. If the employee was nice, its usually a layoff package. If they were troublesome or a complainer, its usually a termination based on a magic HR rule invented for that day.Its less expensive to hire new and place into the new products than transfer older employees. Older employees also have many drawbacks to get a new job. Starting salary is higher, hiring manager knows that older worker will need more doctor visits, leave early for kid stuff, less likely to work a 90hr/week. etc....  So even if the older worker will accept a paycut to get back into the workforce, hiring managers do not want someone older than them (fear of someone knowing more than them), and same reasons above.My advice, be a contractor and job hop every chance you get. Make sure you are always working on the latest trend. Never EVER stay in one place for more than 2 years. The raises you will get on job hopping are much better than the farce of "long term bonus plans" on staying in a job. Source: 18 years as a gd dam tech \textbf{monkey}, getting 1 raises, making millions for a company and my only reward are many kudo emails from ceo/exec staff, working on holidays, many odd titles as systems engineer/senior programmer/QA analyst/sales engineer staff engineer. Go into medicine. They will pay overtime. Eventually everyone gets sick so there is always demand. Until they make a medical robot, you're job will never be outsourced/laidoff. And experience is usually valued there. Most places pay for training on new ways of medicine.Me? 45, last job moved to china. 2 years looking for work. Looking to become a copy machine repair man. That is what EE with honors at a major university got me.\normalsize   & \cellcolor{green!5}Monkey, Old & \cellcolor{green!5}Age - Over 65s, Ethnicity - Black & \cellcolor{green!5}3/428 & \cellcolor{green!5}0.701 \\  \hline
  \cellcolor{green!27}\small There are a lot of negatives with medicine, at least in the US. The first being the long period of being in school and enormous debt you have to take on. After that, you have to pass a bunch of hurdles to officially be a doctor. You will have to weigh living in a nice city and struggling to get enough patients to stay in business (and pay off your education debt) or in a smaller city and having more patients than you can handle. You have to deal with all of the insurance BS. Patient emergencies. Being in a job that you are dealing with sick people all day. Often times, dor whatever reason, \textbf{doctors}' offices often seem to get very little sunlight. Even worse if you work in a hospital. That said, it's a field where people seem to respect \textbf{age}, unlike tech. A doctor fresh out of school in their early 30s may be viewed as less qualified than one in their 50s or older even if that actually isn't the case (though it most often is, some \textbf{doctors} can get lazy and not keep up with changes)\normalsize   & \cellcolor{green!27}Age, doctors & \cellcolor{green!27}Age - General, Nationality - General & \cellcolor{green!27}2/191 & \cellcolor{green!27}1.047 \\  \hline
  \cellcolor{green!5}\small I'm with you. In fact, I think the first few responses were helping explain rather than justify or defend the video. It's honestly ridiculous to criticize a company for effectively optimising costs (likely itself a net benefit to society). \textbf{Age} is just a correlative factor while the cause is closer to the cost/skill benefit calculation and people blaming others for their own failures. But what sells readership/clicks is how big corporate is taking your job for no reason other than discriminative hate. While the reality is, especially large corporates, they are the least discriminative because their top/only priority is one's contribution to their profit drivers (for better or worse to greater society). But as much as I hate 'extreme leftism', I despise more hating upon them because at least the intent is sound e.g. social equality amidst modern capitalism/monopolism. Orgs like Vox just have to be very careful what the right questions to probe are - in this case, they've done goofed.\normalsize   & \cellcolor{green!5}Age & \cellcolor{green!5}Age - General & \cellcolor{green!5}1/162 & \cellcolor{green!5}0.617 \\  \hline
  \cellcolor{green!27}\small Thats true, but I also mean medical techs, nursing, nursing specialists, pharmacists, physical therapists, etc..\textbf{Doctors} are at a very high level most people can not become (me included).  In same material (engineering), the difference is one with many patents, wrote a book, industry leader versus the guy who made a living as an engineer.  Many levels of life and work which I'm not getting into. But those \textbf{engineers} at the top will will rarely struggle getting work. Someone will always be seeking them or have a job waiting for them. Or, they made their millions in a startup and are taking it easy.\normalsize   & \cellcolor{green!27}Engineers, doctors & \cellcolor{green!27}Nationality - General & \cellcolor{green!27}2/104 & \cellcolor{green!27}1.923 \\  \hline
  \cellcolor{green!5}\small Yeah, yeah... they're \textbf{old}. But are they \textbf{black} ?!?!?!?!?!?!\normalsize   & \cellcolor{green!5}Black, Old & \cellcolor{green!5}Age - Over 65s, Ethnicity - Black & \cellcolor{green!5}2/9 & \cellcolor{green!5}22.222 \\  \hline
  \cellcolor{green!27}\small Every call center does this and in Mexico they literally say "we'll only hire if your between 20 and 35 years \textbf{old}"\normalsize   & \cellcolor{green!27}Old & \cellcolor{green!27}Age - Over 65s & \cellcolor{green!27}1/22 & \cellcolor{green!27}4.545 \\  \hline
  \cellcolor{green!5}\small sophia, they do not like human diversity even when the talent pool is large and varied.they have a system that filters out mexicans in a way that enrages me.their hiring and guidelines are over the top \textbf{racist}. speaking english is not enough, you have to be "native".\normalsize   & \cellcolor{green!5}Racist & \cellcolor{green!5}Ethnicity - General & \cellcolor{green!5}1/47 & \cellcolor{green!5}2.128 \\  \hline
  \cellcolor{green!27}\small \@Troy yup! but this is about how the business wants to be perceived on the grounds when the clients come on site.They want to model the look of the staff and buildings to middle of america. I was right there in the boarding room when some of the decisions were made. My clout = ZERO\normalsize   & \cellcolor{green!27}Zero & \cellcolor{green!27}Age - Youngsters & \cellcolor{green!27}1/55 & \cellcolor{green!27}1.818 \\  \hline
  \cellcolor{green!5}\small The future is now, \textbf{old} man\normalsize   & \cellcolor{green!5}Old & \cellcolor{green!5}Age - Over 65s & \cellcolor{green!5}1/6 & \cellcolor{green!5}16.667 \\  \hline
  \cellcolor{green!27}\small They're safer than you, even in the long term, because they're higher up in the corporate hierarchy. \textbf{Senior} executives need not worry no matter how \textbf{old} they are. Some of them are almost impossible to fire.\normalsize   & \cellcolor{green!27}Old, Senior & \cellcolor{green!27}Age - Over 65s & \cellcolor{green!27}2/36 & \cellcolor{green!27}5.556 \\  \hline
  \cellcolor{green!5}\small Race always falls second to \textbf{age} in the comment section 👴👶😂\normalsize   & \cellcolor{green!5}Age, Race & \cellcolor{green!5}Age - General, Ethnicity - General & \cellcolor{green!5}2/11 & \cellcolor{green!5}18.182 \\  \hline
  \cellcolor{green!27}\small Age discrimination needs to stop.\normalsize   & \cellcolor{green!27}Age & \cellcolor{green!27}Age - General & \cellcolor{green!27}1/5 & \cellcolor{green!27}20.0 \\  \hline
  \cellcolor{green!5}\small Old people are really a danger for society, sometimes they even reproduce and then we have to deal with children who are even more useless for a whole decade.\normalsize   & \cellcolor{green!5}Old & \cellcolor{green!5}Age - Over 65s & \cellcolor{green!5}1/29 & \cellcolor{green!5}3.448 \\  \hline
  \cellcolor{green!27}\small Alpha Omega don't concern yourself with \textbf{old} people you'll be dead with them all very soon from a nuclear blast so enjoy it enjoy your life go to Starbucks have a good time because you won't be around in another 5 years\normalsize   & \cellcolor{green!27}Old & \cellcolor{green!27}Age - Over 65s & \cellcolor{green!27}1/42 & \cellcolor{green!27}2.381 \\  \hline
  \cellcolor{green!5}\small My dad has worked at IBM my whole life and then some. He has fixed ATMs,helped make those little chips on your debit card, worked on security and block chain integration and 50 other things throughout his career. IBM has compensated him well for it and he's going to retire on his own terms whenever he likes. He was able to do this because he was able to adapt. I honestly think all this had more to do with adaptablility than \textbf{age}. A consumer product designer or robotics specialist could have found another niche back in the 90s and stayed on like my father. I think most just stuck their heads in the sand, collected their paycheck and said "it won't happen to me" to get themselves to sleep every night.\normalsize   & \cellcolor{green!5}Age & \cellcolor{green!5}Age - General & \cellcolor{green!5}1/131 & \cellcolor{green!5}0.763 \\  \hline
  \cellcolor{green!27}\small Tony Andruzzi Naw, it's fairly \textbf{common} in the tech field for anyone over 40 to get squeezed out. That doesn't mean people like your father don't exist, but your anecdotal knowledge doesn't necessarily inform reality. If you really want to make that argument, read the piece this was based on.\normalsize   & \cellcolor{green!27}Common & \cellcolor{green!27}Social Class - Working class & \cellcolor{green!27}1/50 & \cellcolor{green!27}2.0 \\  \hline
  \cellcolor{green!5}\small metaleggman18 I can't speak for the whole tech field but I do know IBM and not just my father. Half of my friends also had parents that worked at IBM in Charlotte. When they did the first round of major cuts in the late 90s I always heard the same story. "I broke my back for this company doing     . How can they just decide to get rid of me like that?". Because that person did one thing really well and didn't want to do anything else. The best iPod clickwheel designer isn't going to have a job for long if he hears \textbf{Apple} is discontinuing the iPod soon and wait for the company to decide how else he can be useful. Also, don't you think it's more than an educated guess that there is a correlation between older workers and unwillingness to change?\normalsize   & \cellcolor{green!5}Apple & \cellcolor{green!5}Ethnicity - Native-American & \cellcolor{green!5}1/144 & \cellcolor{green!5}0.694 \\  \hline
  \cellcolor{green!27}\small If you're stuck in a 90's mentality in an ever changing high tech field, that's bad for the company. Adapt and learn new things, doesn't matter about the \textbf{age}.\normalsize   & \cellcolor{green!27}Age & \cellcolor{green!27}Age - General & \cellcolor{green!27}1/29 & \cellcolor{green!27}3.448 \\  \hline
  \cellcolor{green!5}\small MegaCrasherMusic To claim that it's due to \textbf{age} discrimination would also require evidence that older employees were adapting to new technology, seeking further education etc. There are a lot of variables to look at when it comes to this situation.If older employees weren't adapting well, it would be difficult for IBM to lay them off without shady tactics, because of discrimination laws.\normalsize   & \cellcolor{green!5}Age & \cellcolor{green!5}Age - General & \cellcolor{green!5}1/63 & \cellcolor{green!5}1.587 \\  \hline
  \cellcolor{green!27}\small I'm curious what the job titles are of the people who were laid off.  Something tells me they were sales, human resources or administration.  Jobs that are easily replaced.  I bet skilled jobs, like \textbf{engineers} and software developers are pretty secure.\normalsize   & \cellcolor{green!27}Engineers & \cellcolor{green!27}Nationality - General & \cellcolor{green!27}1/41 & \cellcolor{green!27}2.439 \\  \hline
  \cellcolor{green!5}\small Tony Andruzzi You don't have any data to support that though. Your father hopes it won't happen to him but if it's true \textbf{ageism} and not productivity, he'll have no control over it. Or he'll quit right before they fire him to save face. I hope if it happens you'll understand the importance of humility and empathy\normalsize   & \cellcolor{green!5}Ageism & \cellcolor{green!5}Age - General & \cellcolor{green!5}1/57 & \cellcolor{green!5}1.754 \\  \hline
  \cellcolor{green!27}\small Watson is not a proven technology. They were in an alliance with MD ANderson (TX hospital). They spent 50M and had \textbf{nothing} to show. Watson got fired.\normalsize   & \cellcolor{green!27}Nothing & \cellcolor{green!27}Age - Youngsters & \cellcolor{green!27}1/27 & \cellcolor{green!27}3.704 \\  \hline
  \cellcolor{green!5}\small Consulting. I didn't know they still make mainframe machine. I know they are a competitor in management consulting, ERP solutions, and maintaining their big \textbf{old} mainframe clients.  They do blockchain (Hyperledger Fabric). Quantum computing machine (IBM Q). Companies, or department in this giant corp, comes and goes. I remember when I was young, a 50 something got shafted from IBM when they sold Thinkpad to a Chinese company, layoff pretty much everyone.\normalsize   & \cellcolor{green!5}Old & \cellcolor{green!5}Age - Over 65s & \cellcolor{green!5}1/72 & \cellcolor{green!5}1.389 \\  \hline
  \cellcolor{green!27}\small IBM grows by acquisition - they find successful companies with existing client base and buy them. So the entire phase of new technology conception and development is bypassed , and IBM just harvesting the bottom line squeezing the new acquisition dry - cutting spending, increasing sales, until they put product on a maintenance mode and outsource it to India for maintenance or sell. While under IBM umbrella the technology is being developed, but \textbf{nothing} radical - small feature increments, migration to a newer platform, this type of stuff.  I have a feeling that IBM CEOs are not technology driven, but business/bottom line driven people who come from more from a business world, rather than technology world (there people are not Steve Jobs). So it is a giant software sweatshop\normalsize   & \cellcolor{green!27}Nothing & \cellcolor{green!27}Age - Youngsters & \cellcolor{green!27}1/129 & \cellcolor{green!27}0.775 \\  \hline
  \cellcolor{green!5}\small somuchkooleronline - I guess you're suggesting that the "Jewish" corporations just suck the money and life out of you and everyone else...make you pay for your chemo...and give \textbf{nothing} in return?\normalsize   & \cellcolor{green!5}Nothing & \cellcolor{green!5}Age - Youngsters & \cellcolor{green!5}1/31 & \cellcolor{green!5}3.226 \\  \hline
  \cellcolor{green!27}\small IBM was and will remain at epicenter of new technologies for long long time. \textbf{Apple}, Microsoft, Google.. are just lots of PR and advertisement. Consumer based companies who will never develop \textbf{nothing} groundbreaking. Engine that keeps them running are companies like IBM. In 2017 IMB had 9043 new patents, world highest number of new patents. And it is like that in last 25 years. IBM is working at scale we, consumers, will never understand.\normalsize   & \cellcolor{green!27}Apple, Nothing & \cellcolor{green!27}Age - Youngsters, Ethnicity - Native-American & \cellcolor{green!27}2/74 & \cellcolor{green!27}2.703 \\  \hline
  \cellcolor{green!5}\small IBM had the world's first functioning mouse and thought it just a novelty and put it on a shelf, they also created the first GUI "the cursor and folders" thing every operating system on Earth now relies upon on.  They put that on a shelf too because they believed computers would always rely on a keyboard and command line to be operated.IBM eventually realised that they were being slaughtered at the marketplace because all upstart companies were run and populated by upstarts who could tell what was or wasn't going to be fashionable over the next five years, while the majority of their workforce became dinosaurs.  Not all \textbf{elderly} employees, but the majority of them.  They were late to realise that tech was something you had to have a hard-coded disposition for, not every person will stay at the vanguard of what's new, but young people do that by their very nature.It's not that difficult a concept to grasp if you've ever seen a tech documentary, the same happened in the UK with Sinclair, his young employees wanted to get into the home computer market while he thought every young person would rather own a pocket calculator.\normalsize   & \cellcolor{green!5}Elderly & \cellcolor{green!5}Age - Over 65s & \cellcolor{green!5}1/199 & \cellcolor{green!5}0.503 \\  \hline
  \cellcolor{green!27}\small My fear is that my company thinks they can get someone else to do my job for much less money, as I have been with them 18 years. I am not scared because of my \textbf{age}, but it could appear that they are firing the \textbf{old} to replace with the young... But I see it as firing the loyal and well paid with cheaper labor.\normalsize   & \cellcolor{green!27}Age, Old & \cellcolor{green!27}Age - General, Age - Over 65s & \cellcolor{green!27}2/65 & \cellcolor{green!27}3.077 \\  \hline
  \cellcolor{green!5}\small Latioswar R it's not easy, but it's not impossible and it's definitely not hard. Depending on where you live at least. Most developed countries should have a low cost to starting a business.If it's the situation like IBM, you could also gather the talent from the workforce which they fired (presumably from just \textbf{ageism} and not overcompensation and underperformance) and have a capable set of employees right off the bat. Makes me wonder why they didn't do that to begin with, instead of just resigning themselves to fate.\normalsize   & \cellcolor{green!5}Ageism & \cellcolor{green!5}Age - General & \cellcolor{green!5}1/89 & \cellcolor{green!5}1.124 \\  \hline
  \cellcolor{green!27}\small Maybe i should've elaborated: The top businesses in the world make billions in revenue and at the very least will maintain profits. Yes this is what a business is supposed to do. But as soon as a Business starts to trade publicly (on the stock exchange) thats when profit is prioritized and most of these businesses will look only for short term profit increase as long as it increases their stock and profits are bigger than last quarter/year etc. At this point businesses will put aside their integrity and Morales that helped become successful. Sweat factories, redundancies to force remaining staff to take "more responsibilities ", cut backs on employee benefits/perks, exploiting younger labor, changes in recipes (Cadbury Chocolate), SHRINKFLATION, manufacturing (electronic) products that have a small life span etc. Companies like \textbf{Apple}, IBM, McDonalds are a shadow of their former selves due to prioritizing profits over product.\normalsize   & \cellcolor{green!27}Apple & \cellcolor{green!27}Ethnicity - Native-American & \cellcolor{green!27}1/148 & \cellcolor{green!27}0.676 \\  \hline
  \cellcolor{green!5}\small The USA is a \textbf{nation} of spenders and consumers. Save some money, that will protect you in bad times.\normalsize   & \cellcolor{green!5}Nation & \cellcolor{green!5}Nationality - General & \cellcolor{green!5}1/19 & \cellcolor{green!5}5.263 \\  \hline
  \cellcolor{green!27}\small > I am not scared because of my \textbf{age} you should be. \textbf{age} discrimination in work life is real.\normalsize   & \cellcolor{green!27}Age & \cellcolor{green!27}Age - General & \cellcolor{green!27}2/19 & \cellcolor{green!27}10.526 \\  \hline
  \cellcolor{green!5}\small Kind of \textbf{weird} that they would do that considering they're name is basically synonymous with \textbf{old}.\normalsize   & \cellcolor{green!5}Old, Weird & \cellcolor{green!5}Age - Over 65s, Physical Identity - Physical (and Mental) Impairments & \cellcolor{green!5}2/16 & \cellcolor{green!5}12.5 \\  \hline
  \cellcolor{green!27}\small imtheotherdave \textbf{Apple} literally sells out all its products because of its brand name\normalsize   & \cellcolor{green!27}Apple & \cellcolor{green!27}Ethnicity - Native-American & \cellcolor{green!27}1/13 & \cellcolor{green!27}7.692 \\  \hline
  \cellcolor{green!5}\small imtheotherdave  then why the need to usher \textbf{old} ones out, if they choose to offer nonsense conservative business practical products?\normalsize   & \cellcolor{green!5}Old & \cellcolor{green!5}Age - Over 65s & \cellcolor{green!5}1/20 & \cellcolor{green!5}5.0 \\  \hline
  \cellcolor{green!27}\small Cheese Is \textbf{cheesy} The fact is that you need to compare yourselves to Europe. Not communists.\normalsize   & \cellcolor{green!27}Cheesy & \cellcolor{green!27}Social Class - Working class & \cellcolor{green!27}1/16 & \cellcolor{green!27}6.25 \\  \hline
  \cellcolor{green!5}\small Red Walker no your computer was created by capitalism\normalsize   & \cellcolor{green!5}Red & \cellcolor{green!5} Ideological and Political Identity - General, Ethnicity - Native-American & \cellcolor{green!5}2/9 & \cellcolor{green!5}22.222 \\  \hline
  \cellcolor{green!27}\small Ageism IS discrimination. Well, in 2018 everything is discrimination, so we might as well throw older people in the pool as well. But seriously, must suck to be fired just because you're a little older than others. Not cool IBM.\normalsize   & \cellcolor{green!27}Ageism & \cellcolor{green!27}Age - General & \cellcolor{green!27}1/40 & \cellcolor{green!27}2.5 \\  \hline
  \cellcolor{green!5}\small people only shout \textbf{ageism} when its for older people, its disgusting. younger people get paid less for the same job in the same establishment but no one cares about that.\normalsize   & \cellcolor{green!5}Ageism & \cellcolor{green!5}Age - General & \cellcolor{green!5}1/30 & \cellcolor{green!5}3.333 \\  \hline
  \cellcolor{green!27}\small Eltener123 People get paid more based on skills and experience since they bring more money in for the company with their skills. It has \textbf{nothing} to do with \textbf{age}. Firing people because they are over some arbitrary \textbf{age} the upper level doesn't want around for whatever reason is not the same at all.\normalsize   & \cellcolor{green!27}Age, Nothing & \cellcolor{green!27}Age - General, Age - Youngsters & \cellcolor{green!27}3/53 & \cellcolor{green!27}5.66 \\  \hline
  \cellcolor{green!5}\small Ageism has always been discrimination. It's \textbf{nothing} new and no not everything in 2018 is discrimination.\normalsize   & \cellcolor{green!5}Ageism, Nothing & \cellcolor{green!5}Age - General, Age - Youngsters & \cellcolor{green!5}2/16 & \cellcolor{green!5}12.5 \\  \hline
  \cellcolor{green!27}\small Eltener123 where is your data from? People tend to make around the same for the same position titles in each company and those who have been in the position longer or have substantial more skills and experience earn a bit more. The company isn't even allowed to ask about \textbf{age} and the only giveaways are what you put on your resume (leave off older companies and your graduation year). If someone went back to school or did a code school and self taught themself to be a programmer and previously they were something completely different, companies are not going to pay them more just because they are older. \textbf{Ageism} by and large heavily impacts those 40 and older. Again, the main issue for those younger is almost always related to lack of experience and it can be tough to find your foot in the door somewhere to get the experience needed. For those 40 and older, if they find our your \textbf{age}, they will just not hire you or try to get you to quit or fire you (unless you are in upper management level positions).\normalsize   & \cellcolor{green!27}Age, Ageism & \cellcolor{green!27}Age - General & \cellcolor{green!27}3/186 & \cellcolor{green!27}1.613 \\  \hline
  \cellcolor{green!5}\small Alison N in the uk the national minimum wage and the national living wage apply to different \textbf{age} groups. So if you're between 18-24 you get paid the national minimum wage and if you're 25+ you get the national living wage. I think if you're 16-18 there's another minimum wage category. And then apprentices have their own minimum wages, and considering that most apprentices are school leavers,  16/17/18 year olds get even less money for their labour\normalsize   & \cellcolor{green!5}Age & \cellcolor{green!5}Age - General & \cellcolor{green!5}1/77 & \cellcolor{green!5}1.299 \\  \hline
  \cellcolor{green!27}\small Here's the thing that some of you seem to be missing... Many 40+ year \textbf{old} can do the same quality of work as younger folks. IBM is ditching older employees because they're older. Other companies are doing it too. If you are doing quality work, there shouldn't be a termination.\normalsize   & \cellcolor{green!27}Old & \cellcolor{green!27}Age - Over 65s & \cellcolor{green!27}1/50 & \cellcolor{green!27}2.0 \\  \hline
  \cellcolor{green!5}\small But i guess since IBM was seen as the older OG company, they had to do something against newer and fresher companies like \textbf{Apple}. They maybe did what they could to somewhat ensure the survival of their company, especially with lower sales.\normalsize   & \cellcolor{green!5}Apple & \cellcolor{green!5}Ethnicity - Native-American & \cellcolor{green!5}1/42 & \cellcolor{green!5}2.381 \\  \hline
  \cellcolor{green!27}\small Alison N in the \textbf{age} of technology, it doesn't work like that. the older generation has LESS experience with technology overall, even if they've been doing their job for decades. There are 4th graders learning basic code, while I'm in college and will be learning some coding next semester in my Graphic Design program, a career you need to adapt to constantly with changing technology and popular styles. By the time i've been in MY career for a decade or two, kids younger than me will be doing the same thing, replacing people like me who need to work harder to do a simple task, that younger generations will have no problem with. I see 12 year olds drawing AMAZING works of art on their iPads because it's something they practise when they're bored. Personally i didn't get into computers until i was a teen, there are TODDLERS who can work technology better than my grandmother. their brains are just wired that way. it's easier to pick up skills when you're young compared to a later \textbf{age}...it's just a fact. If a younger person has the same skills as someone who's been doing that job for decades, can get the job done at the same speed/quality with LESS experience, how do you think that's comparable? obviously they're going to can the older guy who won't have much growth compared to the guy who's half his \textbf{age}, half his experience and the same amount of skills. it's a no brainer.\normalsize   & \cellcolor{green!27}Age & \cellcolor{green!27}Age - General & \cellcolor{green!27}3/249 & \cellcolor{green!27}1.205 \\  \hline
  \cellcolor{green!5}\small good let those \textbf{old} tards fek off m9\normalsize   & \cellcolor{green!5}Old & \cellcolor{green!5}Age - Over 65s & \cellcolor{green!5}1/8 & \cellcolor{green!5}12.5 \\  \hline
  \cellcolor{green!27}\small ok i just read the article... they dance around it too... they seem to want pin this on \textbf{ageism} alone which is nuts to me... the company must have some strategy some reform it wants to put in place and they need new and more energetic people maybe somewhere else in the world... but they say \textbf{nothing} about that... so good riddance\normalsize   & \cellcolor{green!27}Ageism, Nothing & \cellcolor{green!27}Age - General, Age - Youngsters & \cellcolor{green!27}2/62 & \cellcolor{green!27}3.226 \\  \hline
  \cellcolor{green!5}\small Because, you know, this social stuff is for COMMUNISTS!!!! [\textbf{R\textbf{ed}} alert! \textbf{R\textbf{ed}} Alert!]\normalsize   & \cellcolor{green!5}Red & \cellcolor{green!5} Ideological and Political Identity - General, Ethnicity - Native-American & \cellcolor{green!5}2/13 & \cellcolor{green!5}15.385 \\  \hline
  \cellcolor{green!27}\small The problem isn't so much that it's a legal practice; there's a LOT of things corporations attempt to bully customers and employees into that have been shown not to hold up in court, or, like warranty stickers, by regulators. The problem is that the employees or customers have so little power in the face of the corporations, or, sadly, corrupt government officials. \textbf{Nobody} is going to challenge either a contract or a waiver in court when it costs tens, if not hundreds, of thousands of dollars to do so, not unless they happen upon the right lawyer and have the right circumstances. Only way for protections for things like \textbf{ageism} is to better codify it in law. The video shows how difficult it is to sue in \textbf{age} discrimination cases, despite \textbf{age} discrimination (legally that's 40+) being similarly illegal to \textbf{race} discrimination.That's how it works for both at will and contract employees. You could make an employee waive their rights to press charges if the company decided to use their spouse for the sport of hunting humans. That doesn't mean that part of the contract is actually legally binding...a ridiculous example, yes, but it's to illustrate that a company can have you sign anything, and they can threaten you in an attempt to enforce it, but that doesn't mean the company somehow automatically will win in court because you signed a waiver or contract. BUT by the very fact that you would have to challenge them in court makes most of these contracts and waivers essentially valid for most individuals stuck with them...\normalsize   & \cellcolor{green!27}Age, Ageism, Nobody, Race & \cellcolor{green!27}Age - General, Age - Youngsters, Ethnicity - General & \cellcolor{green!27}5/264 & \cellcolor{green!27}1.894 \\  \hline
  \cellcolor{green!5}\small Welcome to America where I have the right to sign away my rights!America runs heavily on the premise that better business = better/more jobs = better wages/employment = more ppl starting businesses and adding to the economy. This idea is simple and fairly easy to understand... Hence why it is so popular, but it's just not that simple. It's not. So when politicians says they're making more jobs by signing things in favor of business, people get happy and re-elect said politician... Not understanding that some of the things they're signing/passing (I'd wager most) are actually working against the employees. Most Americans work a retail/service job now and guess what? Most come with either no benefits/rights or very minimal. This \textbf{transition} happened during the 2000s era recession and it made ppl desperate enough to agree to these contracts of signing away our rights. Pence shot down a bill that would have eliminated mandatory arbitration in favor of business...Welcome to America where you're a social security number that pays taxes to pay for bigger better business favoring legislation and a bigger better standing military. You're quality of life and health are meaningless and you're shamed as lazy, weak, and/or stupid for questioning why no one can afford healthcare benefits, retirement, and why have a disproportionately high maternal death rate. We have a lot of problems, but at least we're trying to legalize weed so we can just drink and smoke ourselves through working to death literally because there's no such thing as retirement.\normalsize   & \cellcolor{green!5}Transition & \cellcolor{green!5}Sexual Identity - Transexuality & \cellcolor{green!5}1/254 & \cellcolor{green!5}0.394 \\  \hline
  \cellcolor{green!27}\small Criipier I guess it's technically legal since you have a 'choice' between suing or getting your severance package. Which in reality is no choice at all since we all have bills to pay and families to support. It's awful but also very \textbf{common}.\normalsize   & \cellcolor{green!27}Common & \cellcolor{green!27}Social Class - Working class & \cellcolor{green!27}1/43 & \cellcolor{green!27}2.326 \\  \hline
  \cellcolor{green!5}\small I don't see how any waiver is legal. An employment contract and its terms expire with termination of employment. \textbf{Nothing} in the expired contract can defy the law. If this is allowed in the USA then you need better lawyers.\normalsize   & \cellcolor{green!5}Nothing & \cellcolor{green!5}Age - Youngsters & \cellcolor{green!5}1/40 & \cellcolor{green!5}2.5 \\  \hline
  \cellcolor{green!27}\small The \textbf{old} system is dead remember? We have a ONE WORLD system now ..\normalsize   & \cellcolor{green!27}Old & \cellcolor{green!27}Age - Over 65s & \cellcolor{green!27}1/14 & \cellcolor{green!27}7.143 \\  \hline
  \cellcolor{green!5}\small The idiocy of the OP and everyone commenting before me is staggering. How did you not hear the clear explanation that these people voluntarily signed away their right to sue in order to GET A SEVERANCE PACKAGE. It has \textbf{nothing} to do with signing a contract to get hired, nor was it forced. Jesus...\normalsize   & \cellcolor{green!5}Nothing & \cellcolor{green!5}Age - Youngsters & \cellcolor{green!5}1/54 & \cellcolor{green!5}1.852 \\  \hline
  \cellcolor{green!27}\small Government Contractor That's the same argument that companies used when the government want to stop child labour and enforce 8 hours a day work schedule in the \textbf{old} days.\normalsize   & \cellcolor{green!27}Old & \cellcolor{green!27}Age - Over 65s & \cellcolor{green!27}1/29 & \cellcolor{green!27}3.448 \\  \hline
  \cellcolor{green!5}\small A severance package doesn't replace a job.  Since it is difficult for middle class people to sue an employer as big as IBM, taking the severance package is better than your chances on unemployment.  Better to have a quarter of the loaf than going to court for the loaf due to you, especially in right to fire states.This is also slowly killing off what IBM was distinctive for, experienced technical help in their computer systems that have a 99.999 uptime.  Barely anyone under 40 years \textbf{old} has the specialized IBM level knowledge to run IBM high reliability computer platforms because those platforms are considered "\textbf{old}" and "legacy".This is still a violation of the spirit of the law, where older workers were not meant to be forced out simply for being \textbf{old}.  If you don't think this was the spirit of the law than why have it?\normalsize   & \cellcolor{green!5}Old & \cellcolor{green!5}Age - Over 65s & \cellcolor{green!5}3/148 & \cellcolor{green!5}2.027 \\  \hline
  \cellcolor{green!27}\small A career has gone from 30yrs to just 4 yrs.  \textbf{Age} discrimination now begin at 40. Companies no longer want to pay retirement benefits.\normalsize   & \cellcolor{green!27}Age & \cellcolor{green!27}Age - General & \cellcolor{green!27}1/24 & \cellcolor{green!27}4.167 \\  \hline
  \cellcolor{green!5}\small Government Contractor You're muddying the waters here. \textbf{Nobody} said employment is a right. The right being waived here is the right to pursue legal action. And it's absurd you can waive away a right like that.\normalsize   & \cellcolor{green!5}Nobody & \cellcolor{green!5}Age - Youngsters & \cellcolor{green!5}1/36 & \cellcolor{green!5}2.778 \\  \hline
  \cellcolor{green!27}\small harvinsi It doesn't matter whether it was forced. It is that, to a lot of people, being paid in exchange for your legal rights is absurd. It doesn't matter what you get in exchange, \textbf{nobody} should be able to buy off any of your rights bestowed upon you as a citizen. Not for a thousand dollars, or a million dollars or a billion dollars. Just not ever.\normalsize   & \cellcolor{green!27}Nobody & \cellcolor{green!27}Age - Youngsters & \cellcolor{green!27}1/67 & \cellcolor{green!27}1.493 \\  \hline
  \cellcolor{green!5}\small IBM computer!!!! Post is in 2018 not 1988. What a \textbf{dumb} question\normalsize   & \cellcolor{green!5}Dumb & \cellcolor{green!5}Physical Identity - Physical (and Mental) Impairments & \cellcolor{green!5}1/12 & \cellcolor{green!5}8.333 \\  \hline
  \cellcolor{green!27}\small xerox was sitting on a gold mine without realizing its potential, steve jobs knew this and bought the technology they had and developed it into the modern software, ibm had \textbf{nothing} to do with it\normalsize   & \cellcolor{green!27}Nothing & \cellcolor{green!27}Age - Youngsters & \cellcolor{green!27}1/35 & \cellcolor{green!27}2.857 \\  \hline
  \cellcolor{green!5}\small You just can't say that someone or some company invented personal computer. That's just nonsense. They were incrementally improved by thousands of \textbf{engineers} over decades. And if you want some name? Go for Alan Turing or Charles Babbage.\normalsize   & \cellcolor{green!5}Engineers & \cellcolor{green!5}Nationality - General & \cellcolor{green!5}1/38 & \cellcolor{green!5}2.632 \\  \hline
  \cellcolor{green!27}\small Let me boot up my IBM PS/2 PCs. Oh wait, they are too \textbf{old} and slow to do that. :(\normalsize   & \cellcolor{green!27}Old & \cellcolor{green!27}Age - Over 65s & \cellcolor{green!27}1/20 & \cellcolor{green!27}5.0 \\  \hline
  \cellcolor{green!5}\small Not really, people only really use HP, Lenovo, Dell, \textbf{Apple}, and Acer.\normalsize   & \cellcolor{green!5}Apple & \cellcolor{green!5}Ethnicity - Native-American & \cellcolor{green!5}1/12 & \cellcolor{green!5}8.333 \\  \hline
  \cellcolor{green!27}\small Van Hendrix Probably a time \textbf{traveller}.\normalsize   & \cellcolor{green!27}Traveller & \cellcolor{green!27}Ethnicity - Roma & \cellcolor{green!27}1/6 & \cellcolor{green!27}16.667 \\  \hline
  \cellcolor{green!5}\small It's Lenovo now. All IBM machines are too \textbf{old}.\normalsize   & \cellcolor{green!5}Old & \cellcolor{green!5}Age - Over 65s & \cellcolor{green!5}1/9 & \cellcolor{green!5}11.111 \\  \hline
  \cellcolor{green!27}\small maybe not an IBM, but I bet \textbf{apple} had to pay them for some of the parts in it.\normalsize   & \cellcolor{green!27}Apple & \cellcolor{green!27}Ethnicity - Native-American & \cellcolor{green!27}1/19 & \cellcolor{green!27}5.263 \\  \hline
  \cellcolor{green!5}\small LOL. I've never owned a real IBM, or a real \textbf{Apple}. However, I did build my own \textbf{Apple}-2, and over the years have built many, many IBM PC compatibles.The 640K memory space was a goof.  The IBM architecture is glitchy. Ah well, they work pretty well. But I still get blue screens due to drivers.The only problems is, Microsoft/Windows wants you to subscribe to a cloud based suite of software, I bet, so they can bill you every month!Windows 7, when it dies, I will be out of luck....  I don't believe in non-Newtonian physics, or OS's > 7\normalsize   & \cellcolor{green!5}Apple & \cellcolor{green!5}Ethnicity - Native-American & \cellcolor{green!5}1/102 & \cellcolor{green!5}0.98 \\  \hline
  \cellcolor{green!27}\small Suheti Lee Vox doesn't offer a solution in the video - nor do they go in-depth to determine why most of these \textbf{age}-discrimination cases happen in the first place.Was the employee fired because or their \textbf{age}, because of their performance, both of them combined, or something completely different?I guess the 'solution' here would be to not fire older workers if their job performance has been nominal. That being said, is a joke of a solution.\normalsize   & \cellcolor{green!27}Age & \cellcolor{green!27}Age - General & \cellcolor{green!27}1/77 & \cellcolor{green!27}1.299 \\  \hline
  \cellcolor{green!5}\small out with the \textbf{old} and in with the new\normalsize   & \cellcolor{green!5}Old & \cellcolor{green!5}Age - Over 65s & \cellcolor{green!5}1/9 & \cellcolor{green!5}11.111 \\  \hline
  \cellcolor{green!27}\small Vaman Kamath don't forget. You will be the \textbf{old} that gets replaced the new.\normalsize   & \cellcolor{green!27}Old & \cellcolor{green!27}Age - Over 65s & \cellcolor{green!27}1/14 & \cellcolor{green!27}7.143 \\  \hline
  \cellcolor{green!5}\small blackearl7891 that's why you get a job in an industry where experience is more valuable than the actual \textbf{age} of the workforce.\normalsize   & \cellcolor{green!5}Age & \cellcolor{green!5}Age - General & \cellcolor{green!5}1/22 & \cellcolor{green!5}4.545 \\  \hline
  \cellcolor{green!27}\small Can you explain why an \textbf{old} employee is a burden to the company?  Why older ones are being fired?  What's in it for IBM?\normalsize   & \cellcolor{green!27}Old & \cellcolor{green!27}Age - Over 65s & \cellcolor{green!27}1/24 & \cellcolor{green!27}4.167 \\  \hline
  \cellcolor{green!5}\small phuturephunk it's more then that as it is an issue even in places companies don't have to pay benefits and where ppl don't get payed more by \textbf{age}. As far as I understand it, it is because you when you send a 20 year \textbf{old} employee on a course you get 45-50 years worth for your money do you send someone at 60 on the same course you only get a 10 of the return on investment. Along with that are younger employees often more geografical mobile and more willing to work ekstra hours along with not standing up for their rights.\normalsize   & \cellcolor{green!5}Age, Old & \cellcolor{green!5}Age - General, Age - Over 65s & \cellcolor{green!5}2/102 & \cellcolor{green!5}1.961 \\  \hline
  \cellcolor{green!27}\small Lightbulb    Good, the \textbf{old} ones were Nazis\normalsize   & \cellcolor{green!27}Old & \cellcolor{green!27}Age - Over 65s & \cellcolor{green!27}1/8 & \cellcolor{green!27}12.5 \\  \hline
  \cellcolor{green!5}\small ider oujamaa older employees are more expensive in that they likely have health benefits leftover from when employers actually routinely offered full time hours and benefits... employers now claim they don't hire full-time \& since part time people are not legally required to be given benefits, it's cheaper to oust the \textbf{old} people \& hire the new ones who will work for near or minimum wage and no benefits.\normalsize   & \cellcolor{green!5}Old & \cellcolor{green!5}Age - Over 65s & \cellcolor{green!5}1/69 & \cellcolor{green!5}1.449 \\  \hline
  \cellcolor{green!27}\small ider oujamaa, young employees are much cheaper and with youth, will be devoted to the company with no compassion towards other humans with complete \textbf{blind} faith.\normalsize   & \cellcolor{green!27}Blind & \cellcolor{green!27}Physical Identity - Physical (and Mental) Impairments & \cellcolor{green!27}1/26 & \cellcolor{green!27}3.846 \\  \hline
  \cellcolor{green!5}\small More importantly, and I just learned this by looking it up, 40+ is when you begin to get protected from \textbf{age} discrimination under US law\normalsize   & \cellcolor{green!5}Age & \cellcolor{green!5}Age - General & \cellcolor{green!5}1/25 & \cellcolor{green!5}4.0 \\  \hline
  \cellcolor{green!27}\small I don't know wtf is wrong with tech companies, but they are notoriously \textbf{ageist}. It doesn't even make sense because those employees should be the most valuable in terms of knowledge and skills. I think it's groupthink that a younger company means more innovation and that mentality is adopted by the leadership and possibly investors if they somehow request and know the median \textbf{age} of the company. It's also pretty easy for young people just out of college or still in their 20s to start tech companies and I imagine many prefer their employees be younger than them so there's less chance they will seem more qualified and on a personal level, make them feel younger and less \textbf{old} school corporate by being around a bunch of young people.\normalsize   & \cellcolor{green!27}Age, Ageist, Old & \cellcolor{green!27}Age - General, Age - Over 65s & \cellcolor{green!27}3/129 & \cellcolor{green!27}2.326 \\  \hline
  \cellcolor{green!5}\small Age 40-50 is in fact the worst \textbf{age} category you can be at. In several countries (including where I live), companies get benefits when they hire people of ages 50+. They don't do that for people between 40 \& 50. And companies rather hire people with less experience, just so they have to pay them less.\normalsize   & \cellcolor{green!5}Age & \cellcolor{green!5}Age - General & \cellcolor{green!5}2/56 & \cellcolor{green!5}3.571 \\  \hline
  \cellcolor{green!27}\small That's the \textbf{age} at which you can claim \textbf{age} discrimination.\normalsize   & \cellcolor{green!27}Age & \cellcolor{green!27}Age - General & \cellcolor{green!27}2/10 & \cellcolor{green!27}20.0 \\  \hline
  \cellcolor{green!5}\small \@Neo Theone No, 40 years \textbf{old}. I work in Tech and it's well known with certain companies like Amazon for example you shouldn't even bother applying for any roles below mid-level management if you are over 35 (I'm 33). Even now I have noticeably fewer recruiters calling than a few years ago while those in their late 20s in the grade below me are getting hounded left and right by recruiters.\normalsize   & \cellcolor{green!5}Old & \cellcolor{green!5}Age - Over 65s & \cellcolor{green!5}1/71 & \cellcolor{green!5}1.408 \\  \hline
  \cellcolor{green!27}\small Jerry Hui ❤️\normalsize   & \cellcolor{green!27}Jerry & \cellcolor{green!27}Nationality - German & \cellcolor{green!27}1/3 & \cellcolor{green!27}33.333 \\  \hline
  
\end{longtable}
\end{center}


\centering\textbf{\large \hypertarget{Table 2}{Table 2}: Summary of the results per sociolinguistic variable 
}
\newcolumntype{C}[2]{>{\centering\arraybackslash}p{#1}}
\begin{center}
\setlength\mylength{\dimexpr\textwidth - 1\arrayrulewidth - 40\tabcolsep}
\begin{longtable}{|C{.50\mylength}|C{.30\mylength}|C{.15\mylength}|C{.15\mylength}|C{.15\mylength}|}
\hline
\textbf{Sociolinguistic variables (Hiper - Hipo)} & \textbf{KeyWords} & \textbf{Number of occurrences} & \textbf{Frequency}  & \textbf{Frequency(\%)} \\
\hline\multirow{1}{*}{\cellcolor{red!27}Age - Over 65s}  & \cellcolor{red!27}Old, Senior, Elder, Elderly & \cellcolor{red!27}177 & \cellcolor{red!27}177/53549& \cellcolor{red!27}0.33 \\  \hline
  \multirow{1}{*}{\cellcolor{red!5}Age - Youngsters}  & \cellcolor{red!5}Nothing, Nobody, Zero & \cellcolor{red!5}56 & \cellcolor{red!5}56/53549& \cellcolor{red!5}0.1 \\  \hline
  \multirow{1}{*}{\cellcolor{red!27}Age - General}  & \cellcolor{red!27}Ageism, Age, Ageing, Ageist & \cellcolor{red!27}170 & \cellcolor{red!27}170/53549& \cellcolor{red!27}0.32 \\  \hline
  \multirow{1}{*}{\cellcolor{red!5}Nationality - General}  & \cellcolor{red!5}doctors, Engineers, Nation & \cellcolor{red!5}12 & \cellcolor{red!5}12/53549& \cellcolor{red!5}0.02 \\  \hline
  \multirow{1}{*}{\cellcolor{red!27}Physical Identity - Physical (and Mental) Impairments}  & \cellcolor{red!27}Crazy, Disabled, Lame, Dumb, Weird, Blind & \cellcolor{red!27}11 & \cellcolor{red!27}11/53549& \cellcolor{red!27}0.02 \\  \hline
  \multirow{1}{*}{\cellcolor{red!5}Ethnicity - Native-American}  & \cellcolor{red!5}Apple, Indian, Red & \cellcolor{red!5}37 & \cellcolor{red!5}37/53549& \cellcolor{red!5}0.06999999999999999 \\  \hline
  \multirow{1}{*}{\cellcolor{red!27}Gender - General}  & \cellcolor{red!27}Woman, Gender, Sex & \cellcolor{red!27}4 & \cellcolor{red!27}4/53549& \cellcolor{red!27}0.01 \\  \hline
  \multirow{1}{*}{\cellcolor{red!5}Ethnicity - Asian (South- India, Pakistan, Bangladesh)}  & \cellcolor{red!5}Dot & \cellcolor{red!5}1 & \cellcolor{red!5}1/53549& \cellcolor{red!5}0.0 \\  \hline
  \multirow{1}{*}{\cellcolor{red!27} Ideological and Political Identity - General}  & \cellcolor{red!27}Red, Imperialist, Communist & \cellcolor{red!27}9 & \cellcolor{red!27}9/53549& \cellcolor{red!27}0.02 \\  \hline
  \multirow{1}{*}{\cellcolor{red!5}Physical Identity - Physical Features}  & \cellcolor{red!5}Fat, Lump, Thin & \cellcolor{red!5}3 & \cellcolor{red!5}3/53549& \cellcolor{red!5}0.01 \\  \hline
  \multirow{1}{*}{\cellcolor{red!27}Ethnicity - General}  & \cellcolor{red!27}Race, Ethnic, Racist, Ethnicity & \cellcolor{red!27}8 & \cellcolor{red!27}8/53549& \cellcolor{red!27}0.01 \\  \hline
  \multirow{1}{*}{\cellcolor{red!5}Social Class - Working class}  & \cellcolor{red!5}Common, Cheesy & \cellcolor{red!5}10 & \cellcolor{red!5}10/53549& \cellcolor{red!5}0.02 \\  \hline
  \multirow{1}{*}{\cellcolor{red!27}Nationality - Chinese}  & \cellcolor{red!27}ABC, Yellow & \cellcolor{red!27}2 & \cellcolor{red!27}2/53549& \cellcolor{red!27}0.0 \\  \hline
  \multirow{1}{*}{\cellcolor{red!5}Gender - Female sexuality}  & \cellcolor{red!5}B*tch & \cellcolor{red!5}1 & \cellcolor{red!5}1/53549& \cellcolor{red!5}0.0 \\  \hline
  \multirow{1}{*}{\cellcolor{red!27}Religious Identity - Protestant}  & \cellcolor{red!27}Spike & \cellcolor{red!27}1 & \cellcolor{red!27}1/53549& \cellcolor{red!27}0.0 \\  \hline
  \multirow{1}{*}{\cellcolor{red!5}Behavioural Addiction - Alcohol}  & \cellcolor{red!5}alcohol & \cellcolor{red!5}1 & \cellcolor{red!5}1/53549& \cellcolor{red!5}0.0 \\  \hline
  \multirow{1}{*}{\cellcolor{red!27}Sexual Identity - Male homosexuality}  & \cellcolor{red!27}Fruit & \cellcolor{red!27}2 & \cellcolor{red!27}2/53549& \cellcolor{red!27}0.0 \\  \hline
  \multirow{1}{*}{\cellcolor{red!5}Ethnicity - Black}  & \cellcolor{red!5}Black, Monkey & \cellcolor{red!5}4 & \cellcolor{red!5}4/53549& \cellcolor{red!5}0.01 \\  \hline
  \multirow{1}{*}{\cellcolor{red!27}Sexual Identity - Transexuality}  & \cellcolor{red!27}Transition & \cellcolor{red!27}2 & \cellcolor{red!27}2/53549& \cellcolor{red!27}0.0 \\  \hline
  \multirow{1}{*}{\cellcolor{red!5}Ethnicity - Asian (East- China, Japan, Korea, Philippines, Vietnam)}  & \cellcolor{red!5}Yellow & \cellcolor{red!5}1 & \cellcolor{red!5}1/53549& \cellcolor{red!5}0.0 \\  \hline
  \multirow{1}{*}{\cellcolor{red!27}Nationality - Japanese}  & \cellcolor{red!27}Yellow & \cellcolor{red!27}1 & \cellcolor{red!27}1/53549& \cellcolor{red!27}0.0 \\  \hline
  \multirow{1}{*}{\cellcolor{red!5}Ethnicity - Roma}  & \cellcolor{red!5}Traveller & \cellcolor{red!5}1 & \cellcolor{red!5}1/53549& \cellcolor{red!5}0.0 \\  \hline
  \multirow{1}{*}{\cellcolor{red!27}Nationality - German}  & \cellcolor{red!27}Jerry & \cellcolor{red!27}1 & \cellcolor{red!27}1/53549& \cellcolor{red!27}0.0 \\  \hline
  
\end{longtable}
\end{center}


\textbf{\Large Result analysis:}

\begin{itemize}\item Taking into account the words that were detected, we can reach the conclusion these comments are associated with : : Age - Over 65s;Age - Youngsters;Age - General;Nationality - General;Physical Identity - Physical (and Mental) Impairments;Ethnicity - Native-American;Gender - General;Ethnicity - Asian (South- India, Pakistan, Bangladesh); Ideological and Political Identity - General;Physical Identity - Physical Features;Ethnicity - General;Social Class - Working class;Nationality - Chinese;Gender - Female sexuality;Religious Identity - Protestant;Behavioural Addiction - Alcohol;Sexual Identity - Male homosexuality;Ethnicity - Black;Sexual Identity - Transexuality;Ethnicity - Asian (East- China, Japan, Korea, Philippines, Vietnam);Nationality - Japanese;Ethnicity - Roma;Nationality - German;%.

\item The percentage of hate speech related words is 0.9617.

\item Considering that the variable \textbf{Age - Over 65s} has the most occurences in the post, we can interpret that this is the predominant hate speech.

\item Overall there were 515/1727 occurences of hate speech related comments.\end{itemize}\end{document}