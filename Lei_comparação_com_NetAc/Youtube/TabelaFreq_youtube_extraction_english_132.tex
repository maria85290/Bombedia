\documentclass[11pt]{article}
\ExplSyntaxOn
\let\tl_length:n\tl_count:n
\ExplSyntaxOff
\usepackage{graphicx}
\usepackage{multirow}
\usepackage{colortbl}
\usepackage{longtable, array}
\usepackage{hyperref}
\usepackage[usenames,dvipsnames,svgnames,table]{xcolor}
\newlength\mylength
\usepackage[legalpaper, landscape, margin=0.8in]{geometry}
\newcommand{\MinNumber}{0}
\begin{document}

\textbf {\huge In youtube\_extraction\_english\_132.json :}\newline \par\Large\textbf {Title: \large  Ageism Isn't Normal! How to Deal with Age-Related Bias }\newline {\par\large --- Table  1: Summary of the results per comment; }

 {\par\large --- \hyperlink{Table 2}{\textcolor{blue}{\underline{Table 2}}}: Summary of the results per sociolinguistic variable;}\newline \normalsize\newline

\centering\textbf{\large Table  1: Summary of the results per comment 
}
\newcommand{\MaxNumber}{0}%
\newcommand{\ApplyGradient}[1]{%
\pgfmathsetmacro{\PercentColor}{100.0*(#1-\MinNumber)/(\MaxNumber-\MinNumber)}
\xdef\PercentColor{\PercentColor}%
\cellcolor{LightSpringGreen!\PercentColor!LightRed}{#1}
}
\newcolumntype{C}[2]{>{\centering\arraybackslash}p{#1}}
\begin{center}
\setlength\mylength{\dimexpr\textwidth - 1\arrayrulewidth - 50\tabcolsep}
\begin{longtable}{|C{.65\mylength}|C{.30\mylength}|C{.12\mylength}|C{.12\mylength}|C{.12\mylength}|}
\hline
\textbf{Comment} & \textbf{KeyWords} & \textbf{Sociolinguistic variables (Hiper - Hipo)}  & \textbf{Hate Speech Frequency} & \textbf{Hate Speech Frequency(\%)} \\
\hline\cellcolor{green!27}\small We need to talk about \textbf{age} discrimination. One of the things I noticed was that the young people nver invited you to do things. I was even not included in the Christmas Dinner. It goes without saying that I was never invited out with them to the pub for lunch. Young people just stuck together in everything. No one questions it in the workplace\normalsize   & \cellcolor{green!27}Age & \cellcolor{green!27}Age - General & \cellcolor{green!27}1/64 & \cellcolor{green!27}1.562 \\  \hline
  \cellcolor{green!5}\small I am 67 and was offered a package  to leave a  hi tech company at \textbf{age} 62. The last five years I had been there, I experienced a lot of \textbf{ageism}. Additionally I wasn't given raises during that time frame. I had hoped i would last till I was 64 there, but it wasn't possible so I took the package. Certainly your friend's suggestion is to take the high road. I certainly tried to do that.\normalsize   & \cellcolor{green!5}Age, Ageism & \cellcolor{green!5}Age - General & \cellcolor{green!5}2/76 & \cellcolor{green!5}2.632 \\  \hline
  \cellcolor{green!27}\small Has anyone experienced wage discrimination due to being \textbf{age}?\normalsize   & \cellcolor{green!27}Age & \cellcolor{green!27}Age - General & \cellcolor{green!27}1/9 & \cellcolor{green!27}11.111 \\  \hline
  \cellcolor{green!5}\small Not sure as an \textbf{elderly} person, but definitely as a \textbf{woman}.\normalsize   & \cellcolor{green!5}Elderly, Woman & \cellcolor{green!5}Age - Over 65s, Gender - General & \cellcolor{green!5}2/11 & \cellcolor{green!5}18.182 \\  \hline
  \cellcolor{green!27}\small There is something to be said about a society and how they treat the young and the \textbf{elderly}. Uncaring and disrespectful, even in the family today.  I'm sure there is more \textbf{elder} abuse than we know about.\normalsize   & \cellcolor{green!27}Elder, Elderly & \cellcolor{green!27}Age - Over 65s & \cellcolor{green!27}2/37 & \cellcolor{green!27}5.405 \\  \hline
  \cellcolor{green!5}\small It's because our society is obsessed with youth. Becoming \textbf{old} is a privilege that many never had.\normalsize   & \cellcolor{green!5}Old & \cellcolor{green!5}Age - Over 65s & \cellcolor{green!5}1/17 & \cellcolor{green!5}5.882 \\  \hline
  \cellcolor{green!27}\small If i get \textbf{old}, im very afraid to get too much bias in my mindset. Too much protect my own truth, against the real truth. Hopefully i will keep on fighting my own bias. Because most people really have problems with cognitive bias and destroying our country. Why? Because \textbf{elderly} people are with more, that's why democracy is scewed by doing things more for \textbf{old} people rather for young. But because older people are way more biased according to research on this topic. This is dangerous for societies where the younger group is with few.\normalsize   & \cellcolor{green!27}Elderly, Old & \cellcolor{green!27}Age - Over 65s & \cellcolor{green!27}3/95 & \cellcolor{green!27}3.158 \\  \hline
  \cellcolor{green!5}\small \@\textbf{Herman} Willems I don't quite follow your sentence structure.  Are you trying to say "\textbf{old}" people are set in their ways; "young" people should have more attention/focus? If so,  your bias shows by separating people by \textbf{age}.  We are all equal and all individual! Life does not involve living in stages,  from young to \textbf{old}.  It should be lived as a total,  complete,  present experience in the Now,  by each individual creature. I hope you're not suggesting that younger people deserve more attention? If anything,  they need less!\normalsize   & \cellcolor{green!5}Age, Herman, Old & \cellcolor{green!5}Age - General, Age - Over 65s, Nationality - German & \cellcolor{green!5}4/88 & \cellcolor{green!5}4.545 \\  \hline
  \cellcolor{green!27}\small Yes, how do you deal with \textbf{age} bias - when it's in your own family?  I am 64, my sister is 55 and my daughter is 41.   My sister and I used to be very close and now I find she wants to be around my daughter more.  She has made comments about my being \textbf{old} and now seems to see herself as my daughter's peer.   We are both 'young' for our \textbf{age}, in good shape and active.    I have hurtful feelings over it.   Do I turn it back on her?  Any advice on how to handle such a situation?   It's getting harder to ignore it. I'm not sure whether I should say something or keep trying to ignore.\normalsize   & \cellcolor{green!27}Age, Old & \cellcolor{green!27}Age - General, Age - Over 65s & \cellcolor{green!27}3/119 & \cellcolor{green!27}2.521 \\  \hline
  \cellcolor{green!5}\small Very elegant look..\textbf{ageism} is rampant sadly  .she certainly took the high road.\normalsize   & \cellcolor{green!5}Ageism & \cellcolor{green!5}Age - General & \cellcolor{green!5}1/12 & \cellcolor{green!5}8.333 \\  \hline
  \cellcolor{green!27}\small I will engage the "young" as an individual,  equal person,  and to the extent that they will engage me.  Playing a role to be liked is, in this case, endorsing \textbf{ageism}.  My mother said that the only role she will have anything to do with is a jelly roll!\normalsize   & \cellcolor{green!27}Ageism & \cellcolor{green!27}Age - General & \cellcolor{green!27}1/49 & \cellcolor{green!27}2.041 \\  \hline
  \cellcolor{green!5}\small Ageism is real and rampant\normalsize   & \cellcolor{green!5}Ageism & \cellcolor{green!5}Age - General & \cellcolor{green!5}1/5 & \cellcolor{green!5}20.0 \\  \hline
  \cellcolor{green!27}\small My mother let \textbf{ageism} eat her up inside.\normalsize   & \cellcolor{green!27}Ageism & \cellcolor{green!27}Age - General & \cellcolor{green!27}1/8 & \cellcolor{green!27}12.5 \\  \hline
  \cellcolor{green!5}\small I'm sure I experience \textbf{ageism} in the doctor's office.  It is,  of course,  an extension of society's view.  I'm sure that information and treatment are,  sometimes,  withheld or not discussed,  too.  For instance,  the same condition/illness is not going to be treated,  necessarily,  in the same way,  and you won't be told! You are seen as a "\textbf{Senior}"; so you will be asked if you are lonely,  depressed,  forgetful..... if your diet includes vegetables and \textbf{fruit}.  In a way,  it reminds me of when your children reach preteen \textbf{age},  and start rolling their eyes and thinking they know more than you do. Actually,  this form of \textbf{ageism} is the reason I dread most doctor/dentist visits.  I am going all-out to practice Margaret's way of reacting to these yucky situations. Keep the faith and keep on rockin..\normalsize   & \cellcolor{green!5}Age, Ageism, Fruit, Senior & \cellcolor{green!5}Age - General, Age - Over 65s, Sexual Identity - Male homosexuality & \cellcolor{green!5}5/136 & \cellcolor{green!5}3.676 \\  \hline
  \cellcolor{green!27}\small I just started to experience \textbf{ageism} a few years ago. It's a very ugly thing. It's feels similar to \textbf{racism}. I've become almost invisible.\normalsize   & \cellcolor{green!27}Ageism, Racism & \cellcolor{green!27}Age - General, Ethnicity - General & \cellcolor{green!27}2/24 & \cellcolor{green!27}8.333 \\  \hline
  \cellcolor{green!5}\small When I was wakened by a nurse in the night, post-surgery, I heard her say that I was "older than dirt!" I laughed because I consider myself to be an attractive, young-thinking 74 yr. \textbf{old} \textbf{woman}! Nurses are under a great deal of stress and I realized her sense of humour kept her sane!\normalsize   & \cellcolor{green!5}Old, Woman & \cellcolor{green!5}Age - Over 65s, Gender - General & \cellcolor{green!5}2/54 & \cellcolor{green!5}3.704 \\  \hline
  \cellcolor{green!27}\small I started listening to this from the perspective of a person entering a stage of life where \textbf{ageism} may start to show up. But as I listened, I recalled a situation where I was working in my early 40s and a \textbf{woman} who had been retired for many years joined our team on a temporary assignment. I recall feeling frustrated with her on a few occasions but i believe my frustration was about suggestions which I felt were not realistic long term. I think people who have been out of the work force forget what it's like to be in the daily grind and the suggestions that they make live on after their assignment is done. There are some other factors that I think led to my frustration but it too long for this post. In my view when we're looking at Aging in a proud and strong way we should look at people like Maya Angelou, Jane Goodall etc... what these women had in \textbf{common} was a strong purpose to make the world a better place and I think we can all do that in our own way and therefore will spend less time thinking about what people think of us.\normalsize   & \cellcolor{green!27}Ageism, Common, Woman & \cellcolor{green!27}Age - General, Gender - General, Social Class - Working class & \cellcolor{green!27}3/202 & \cellcolor{green!27}1.485 \\  \hline
  \cellcolor{green!5}\small Sometimes it's cultural. A group of  students from India were quite shocked, and told me, that they never see a \textbf{woman} my \textbf{age} (49 at the time) studying in college. Another person told me recently that it's unusual to see someone my \textbf{age} (now 64) with all their personality still intact lol! I have learned to just ignore their ignorance....\normalsize   & \cellcolor{green!5}Age, Woman & \cellcolor{green!5}Age - General, Gender - General & \cellcolor{green!5}3/60 & \cellcolor{green!5}5.0 \\  \hline
  \cellcolor{green!27}\small Thank you Margaret, I've experienced \textbf{age} bias at the workplace when I was younger and older. My last job I was terribly abused emotionally by younger workers and those my \textbf{age} that after four years, I left, that is when I retired. It's very long to go into, but from day one I showed them up by being professional. That's how I work, from how I treat people  on the phone or in person by my demeanor and speech. They were the opposite and didn't like it, so they would make remarks, faces behind my back, the environment was very toxic. The nicer I would be, the worse it would get. For the first time in my life I called them into the manager. \textbf{Nothing} changed. In fact, one person in particular was called into the office several times by others, but they never let her go. The stress was way too much, I needed to have left years earlier. But I had to work. So when I was at the youngest retirement \textbf{age}, I left. The bottom line was they were jealous of me and I couldn't do anything about that and I tried with all my might. Very painful memories. I don't have to deal with that abuse anymore. It's nice being retired and free. Thank you for listening. I'm sure there are many stories out there. In fact, I don't know of anyone in the work place who hasn't had experienced similar situations. It's the human factor and how individually we decide to deal with it. Stay or leave. I moved on. Thank you again for your subject matter.\normalsize   & \cellcolor{green!27}Age, Nothing & \cellcolor{green!27}Age - General, Age - Youngsters & \cellcolor{green!27}4/272 & \cellcolor{green!27}1.471 \\  \hline
  \cellcolor{green!5}\small Maria Rooney   I am so pleased to read your sense of relief. However, no one deserves to be driven out of the workplace especially someone who is professional and has lots to offer and contribute. I am particularly annoyed when I see the process for bringing a complaint of discrimination and harassment breaks down because the employer refuses to take appropriate action. In most countries, there are employment standards that overlap and integrate to address physical, \textbf{mental}, and emotional health in the workplace. Most standards are specific around 'zero tolerance' for discrimination and harassment. However, so much is still imbedded and accepted as normal. My one hope for 'younger' women in the workforce is that they will shake up the norm and protest the bad behaviour in the workplace. We, young and \textbf{old}, need to make employers accountable for the environment THEY create and perpetuate. Off my soap box now.\normalsize   & \cellcolor{green!5}Mental, Old & \cellcolor{green!5}Age - Over 65s, Physical Identity - Physical (and Mental) Impairments & \cellcolor{green!5}2/150 & \cellcolor{green!5}1.333 \\  \hline
  \cellcolor{green!27}\small I, too, experienced this.  I never thought of it as '\textbf{ageism}' at the time, probably because I was focused on how disintegration in philosophy had affected their stubborn resistance to facts.  \textbf{Ageism} was merely one of their prejudges.  I was home-schooled by the company and they all went to a university.  It was also that.\normalsize   & \cellcolor{green!27}Ageism & \cellcolor{green!27}Age - General & \cellcolor{green!27}1/55 & \cellcolor{green!27}1.818 \\  \hline
  
\end{longtable}
\end{center}


\centering\textbf{\large \hypertarget{Table 2}{Table 2}: Summary of the results per sociolinguistic variable 
}
\newcolumntype{C}[2]{>{\centering\arraybackslash}p{#1}}
\begin{center}
\setlength\mylength{\dimexpr\textwidth - 1\arrayrulewidth - 40\tabcolsep}
\begin{longtable}{|C{.50\mylength}|C{.30\mylength}|C{.15\mylength}|C{.15\mylength}|C{.15\mylength}|}
\hline
\textbf{Sociolinguistic variables (Hiper - Hipo)} & \textbf{KeyWords} & \textbf{Number of occurrences} & \textbf{Frequency}  & \textbf{Frequency(\%)} \\
\hline\multirow{1}{*}{\cellcolor{red!27}Age - General}  & \cellcolor{red!27}Age, Ageism & \cellcolor{red!27}22 & \cellcolor{red!27}22/2041& \cellcolor{red!27}1.08 \\  \hline
  \multirow{1}{*}{\cellcolor{red!5}Gender - General}  & \cellcolor{red!5}Woman & \cellcolor{red!5}4 & \cellcolor{red!5}4/2041& \cellcolor{red!5}0.2 \\  \hline
  \multirow{1}{*}{\cellcolor{red!27}Age - Over 65s}  & \cellcolor{red!27}Elderly, Elder, Old, Senior & \cellcolor{red!27}13 & \cellcolor{red!27}13/2041& \cellcolor{red!27}0.64 \\  \hline
  \multirow{1}{*}{\cellcolor{red!5}Nationality - German}  & \cellcolor{red!5}Herman & \cellcolor{red!5}1 & \cellcolor{red!5}1/2041& \cellcolor{red!5}0.05 \\  \hline
  \multirow{1}{*}{\cellcolor{red!27}Sexual Identity - Male homosexuality}  & \cellcolor{red!27}Fruit & \cellcolor{red!27}1 & \cellcolor{red!27}1/2041& \cellcolor{red!27}0.05 \\  \hline
  \multirow{1}{*}{\cellcolor{red!5}Ethnicity - General}  & \cellcolor{red!5}Racism & \cellcolor{red!5}1 & \cellcolor{red!5}1/2041& \cellcolor{red!5}0.05 \\  \hline
  \multirow{1}{*}{\cellcolor{red!27}Social Class - Working class}  & \cellcolor{red!27}Common & \cellcolor{red!27}1 & \cellcolor{red!27}1/2041& \cellcolor{red!27}0.05 \\  \hline
  \multirow{1}{*}{\cellcolor{red!5}Age - Youngsters}  & \cellcolor{red!5}Nothing & \cellcolor{red!5}1 & \cellcolor{red!5}1/2041& \cellcolor{red!5}0.05 \\  \hline
  \multirow{1}{*}{\cellcolor{red!27}Physical Identity - Physical (and Mental) Impairments}  & \cellcolor{red!27}Mental & \cellcolor{red!27}1 & \cellcolor{red!27}1/2041& \cellcolor{red!27}0.05 \\  \hline
  
\end{longtable}
\end{center}


\textbf{\Large Result analysis:}

\begin{itemize}\item Taking into account the words that were detected, we can reach the conclusion these comments are associated with : : Age - General;Gender - General;Age - Over 65s;Nationality - German;Sexual Identity - Male homosexuality;Ethnicity - General;Social Class - Working class;Age - Youngsters;Physical Identity - Physical (and Mental) Impairments;%.

\item The percentage of hate speech related words is 2.2048.

\item Considering that the variable \textbf{Age - General} has the most occurences in the post, we can interpret that this is the predominant hate speech.

\item Overall there were 45/42 occurences of hate speech related comments.\end{itemize}\end{document}