\documentclass[11pt]{article}
\ExplSyntaxOn
\let\tl_length:n\tl_count:n
\ExplSyntaxOff
\usepackage{graphicx}
\usepackage{multirow}
\usepackage{colortbl}
\usepackage{longtable, array}
\usepackage{hyperref}
\usepackage[usenames,dvipsnames,svgnames,table]{xcolor}
\newlength\mylength
\usepackage[legalpaper, landscape, margin=0.8in]{geometry}
\newcommand{\MinNumber}{0}
\begin{document}

\textbf {\huge In youtube\_extraction\_english\_140.json :}\newline \par\Large\textbf {Title: \large  Ageism in The Design Industry: Where Are All the Older Creative Professionals? }\newline {\par\large --- Table  1: Summary of the results per comment; }

 {\par\large --- \hyperlink{Table 2}{\textcolor{blue}{\underline{Table 2}}}: Summary of the results per sociolinguistic variable;}\newline \normalsize\newline

\centering\textbf{\large Table  1: Summary of the results per comment 
}
\newcommand{\MaxNumber}{0}%
\newcommand{\ApplyGradient}[1]{%
\pgfmathsetmacro{\PercentColor}{100.0*(#1-\MinNumber)/(\MaxNumber-\MinNumber)}
\xdef\PercentColor{\PercentColor}%
\cellcolor{LightSpringGreen!\PercentColor!LightRed}{#1}
}
\newcolumntype{C}[2]{>{\centering\arraybackslash}p{#1}}
\begin{center}
\setlength\mylength{\dimexpr\textwidth - 1\arrayrulewidth - 50\tabcolsep}
\begin{longtable}{|C{.65\mylength}|C{.30\mylength}|C{.12\mylength}|C{.12\mylength}|C{.12\mylength}|}
\hline
\textbf{Comment} & \textbf{KeyWords} & \textbf{Sociolinguistic variables (Hiper - Hipo)}  & \textbf{Hate Speech Frequency} & \textbf{Hate Speech Frequency(\%)} \\
\hline\cellcolor{green!27}\small I find being open, flexible and not so worried about hierarchy works well. Some \textbf{senior} people in the design industry guard a sense of importance. This old-school stiff authoritative figure just makes everyone feel uncomfortable especially if your young and not so bothered about hierarchy anymore. Dropping the ego and focusing on being open, encouraging and positivity will help allot. I'm 46 and love working with young designers, I'm freelance and not worried about my position I just enjoy great ideas and being around good people.\normalsize   & \cellcolor{green!27}Senior & \cellcolor{green!27}Age - Over 65s & \cellcolor{green!27}1/86 & \cellcolor{green!27}1.163 \\  \hline
  \cellcolor{green!5}\small Is 59 too \textbf{old} to continue in web design and Elearning?\normalsize   & \cellcolor{green!5}Old & \cellcolor{green!5}Age - Over 65s & \cellcolor{green!5}1/11 & \cellcolor{green!5}9.091 \\  \hline
  \cellcolor{green!27}\small Thank you for this. I was particularly happy when you said don't let your \textbf{age} stop you and that it is never too late. I know you were talking to more like the '30s or '40s, but this 60-year-old is taking it to heart and going full steam ahead. Thank you again for your content, it has been very helpful.\normalsize   & \cellcolor{green!27}Age & \cellcolor{green!27}Age - General & \cellcolor{green!27}1/60 & \cellcolor{green!27}1.667 \\  \hline
  \cellcolor{green!5}\small That is so funny Picasso was very creative when he was very \textbf{old} - this focus on young fools only is actually what creates shallow brands and very cheap tricks - I would mix teams with \textbf{old} brand veterans and young tough cats\normalsize   & \cellcolor{green!5}Old & \cellcolor{green!5}Age - Over 65s & \cellcolor{green!5}2/43 & \cellcolor{green!5}4.651 \\  \hline
  \cellcolor{green!27}\small The same with technology... There was an article where they were asking why there wasn't many women in the friend. One of the reasons was that the youngers had not an \textbf{old} role models in the field.\normalsize   & \cellcolor{green!27}Old & \cellcolor{green!27}Age - Over 65s & \cellcolor{green!27}1/37 & \cellcolor{green!27}2.703 \\  \hline
  \cellcolor{green!5}\small Kola, That's true and it is a self-perpetuating scenario. It starts with sexist or \textbf{ageist} or \textbf{racist} hiring practices and morphs into a long-term deficit in diversity in the workplace.\normalsize   & \cellcolor{green!5}Ageist, Racist & \cellcolor{green!5}Age - General, Ethnicity - General & \cellcolor{green!5}2/30 & \cellcolor{green!5}6.667 \\  \hline
  \cellcolor{green!27}\small YES! and sexist as well groovy middle aged men in \textbf{black} tees will be ranked higher than a 50 something female sad and true\normalsize   & \cellcolor{green!27}Black & \cellcolor{green!27}Ethnicity - Black & \cellcolor{green!27}1/24 & \cellcolor{green!27}4.167 \\  \hline
  \cellcolor{green!5}\small Ageism is so real! I went through a phase of applying to full time jobs, and I'm being interviewed by 30 year olds and the whole team is under 35. 35 is not \textbf{old}. Because of this, and many other observations, I was confirmed the corporate world is not for me. But, I didn't let it bother me, and others over 40 shouldn't be concerned either. There are other sectors to design for with an older demographic. And, there are 4 decades (people in their 40s, 50s, 60s and even 70s) of business people who have different focuses than the youth working at Google, Facebook, Amazon, etc... and they need design too. If you are good and stay current, there is a tribe for you who needs your design. Make a client base who is your \textbf{age} and older and you will not feel the immature prejudice of \textbf{ageism}. I realize my comments apply more to people who are running their own business, and I guess that is where most older designer go.\normalsize   & \cellcolor{green!5}Age, Ageism, Old & \cellcolor{green!5}Age - General, Age - Over 65s & \cellcolor{green!5}4/173 & \cellcolor{green!5}2.312 \\  \hline
  \cellcolor{green!27}\small Very true. Dude we have a LOT of perspectives in \textbf{common}. Be sure to connect with me on LinkedIn.\normalsize   & \cellcolor{green!27}Common & \cellcolor{green!27}Social Class - Working class & \cellcolor{green!27}1/19 & \cellcolor{green!27}5.263 \\  \hline
  \cellcolor{green!5}\small I 37... And I guess that is not so \textbf{old}. Nut I feel like the older you ate the more experience you have...  \textbf{Age} comes value.\normalsize   & \cellcolor{green!5}Age, Old & \cellcolor{green!5}Age - General, Age - Over 65s & \cellcolor{green!5}2/26 & \cellcolor{green!5}7.692 \\  \hline
  \cellcolor{green!27}\small Thanks jarmelo2006! Always good to see you here my friend. No, 37 is not that \textbf{old} - and experience does count for a lot in the design field, even though it can be \textbf{ageist}. There are a lot of companies, agencies and clients who value deeper experience in the field to help them with their challenges.\normalsize   & \cellcolor{green!27}Ageist, Old & \cellcolor{green!27}Age - General, Age - Over 65s & \cellcolor{green!27}2/56 & \cellcolor{green!27}3.571 \\  \hline
  \cellcolor{green!5}\small Philip, I love your videos and mailing! Keep bringing them! Thanks a lot for sharing! That's a topic (\textbf{ageism}) I have been thinking a lot! I work exclusively as freelancer since 2013 and it seems to be almost the only way to keep surviving in the creative industry (and for book design, it really seems to be the only way)\normalsize   & \cellcolor{green!5}Ageism & \cellcolor{green!5}Age - General & \cellcolor{green!5}1/60 & \cellcolor{green!5}1.667 \\  \hline
  \cellcolor{green!27}\small Why does \textbf{ageism} exist? Because employers can get away with paying junior wages. I got into digital design 30 years ago then transferred into motion graphics then video production. I've never been more skilled and accomplished in my life. However, If I worked for someone else I'd get less than half of what I'm worth, not enough to pay bills. I agree, the answer is to build your own brand AND work with someone who excels in marketing before you go it alone. Design is very accessible now, the market is saturated and there are too many players. Employers know this too and prey of it.\normalsize   & \cellcolor{green!27}Ageism & \cellcolor{green!27}Age - General & \cellcolor{green!27}1/106 & \cellcolor{green!27}0.943 \\  \hline
  \cellcolor{green!5}\small Yani- I don't really recommend sites like UpWork, 99 Designs etc. They are a \textbf{race} to the bottom - favor the client and devalue the designer - and I completely am against spec work - which is what these sites promote. I've done a bunch of videos on how to find clients. Look through my past videos in my channel catalog and you'l find some great ideas.\normalsize   & \cellcolor{green!5}Race & \cellcolor{green!5}Ethnicity - General & \cellcolor{green!5}1/67 & \cellcolor{green!5}1.493 \\  \hline
  \cellcolor{green!27}\small I started working at the office, but then moved to freelancing when I was 22. I never regretted this decision. No one even knows my \textbf{age}, and I never even seen my clients. I earn approximately the same amount of money. The drawback I have to constantly learn something new to stay relevant. I am more flexible, and can go to the gym when I want. After 35, the issue of \textbf{age} brings up on interviews. I'm not sure there's anything we can do about it. Everybody wants fresh new minds. My girlfriend works as a recruiter for manual labor workers, and they take workers only up to 45. That's the upper limit. In CG, they expect you to be a solid professional by the \textbf{age} of 30-35.\normalsize   & \cellcolor{green!27}Age & \cellcolor{green!27}Age - General & \cellcolor{green!27}3/128 & \cellcolor{green!27}2.344 \\  \hline
  \cellcolor{green!5}\small This was a powerful message and I believe we all can benefit from this. The take away is young designers need guidance and older designers contextually should be building brands not just freelancing. By staying 20 steps ahead not 10 and creating content along with amazing portfolios based off our experience will allow for timeless work that can not be defined by your age\normalsize   & \cellcolor{green!5}Age & \cellcolor{green!5}Age - General & \cellcolor{green!5}1/64 & \cellcolor{green!5}1.562 \\  \hline
  \cellcolor{green!27}\small Without questions. \textbf{Age} discrimination is real in the world of graphic design and marketing. I'm very methodical and critical in my approach of creating logos or corporate branding. To me, that's very important when it comes to designing something that will meet ALL requirements of the purpose in which it was designed and created for.\normalsize   & \cellcolor{green!27}Age & \cellcolor{green!27}Age - General & \cellcolor{green!27}1/55 & \cellcolor{green!27}1.818 \\  \hline
  \cellcolor{green!5}\small Nice article Phil'. Here in indonesia, clients (my clients) have  recently gone through a process of figuring out that the quality of (youngsters in-house) work they now get is... well,.. poor and limited, to say the least. This has led to a 'swing around', literally a rebound effect that has been a boost to our agency. However, the knack, and always will be, is to do great work.. whatever your \textbf{age}.\normalsize   & \cellcolor{green!5}Age & \cellcolor{green!5}Age - General & \cellcolor{green!5}1/71 & \cellcolor{green!5}1.408 \\  \hline
  \cellcolor{green!27}\small I'm 25 and working as a creative in the adverting industry. I have worked at 2 medium to large sized agencies, and I have never come across another creative that is over the \textbf{age} 50. I often think about this, and what it entails for my future and career.\normalsize   & \cellcolor{green!27}Age & \cellcolor{green!27}Age - General & \cellcolor{green!27}1/49 & \cellcolor{green!27}2.041 \\  \hline
  \cellcolor{green!5}\small Thank you for this video. \textbf{Age} discrimination in creative and tech industries are a real problem in today's culture. As the adage would say "\textbf{Old} dogs can learn new tricks."\normalsize   & \cellcolor{green!5}Age, Old & \cellcolor{green!5}Age - General, Age - Over 65s & \cellcolor{green!5}2/30 & \cellcolor{green!5}6.667 \\  \hline
  \cellcolor{green!27}\small Malik - and more importantly - they already know all the \textbf{old} tricks, which the young pups do not - and who fall victim to them all-the-freakin-time.\normalsize   & \cellcolor{green!27}Old & \cellcolor{green!27}Age - Over 65s & \cellcolor{green!27}1/27 & \cellcolor{green!27}3.704 \\  \hline
  \cellcolor{green!5}\small A major factor may be money. For the price of an older, \textbf{senior} professional, the agency may be able to get 2 fresh and young hopefuls who will be more malleable.\normalsize   & \cellcolor{green!5}Senior & \cellcolor{green!5}Age - Over 65s & \cellcolor{green!5}1/31 & \cellcolor{green!5}3.226 \\  \hline
  \cellcolor{green!27}\small \@Beerhunter68 Beer, you're very right. I have "traded" on my successful 25 year in the agency and corporate world - establishing professional credibility by virtue of the clients and projects I have worked on as well as the huge departments and budgets I have run over the  years. Not to mention the massive network of clients and partners I have amassed. All that made what I am doing now possible. But how early the age-discrimination came shocked me - and the more I looked at it across people I knew in the industry the more disturbing it became...Also, how unprepared I was to make the \textbf{transition} to an independent position - is why I'm always ringing the bell of personal branding to whoever will listen.\normalsize   & \cellcolor{green!27}Transition & \cellcolor{green!27}Sexual Identity - Transexuality & \cellcolor{green!27}1/125 & \cellcolor{green!27}0.8 \\  \hline
  \cellcolor{green!5}\small Just some observations...a) Wisdom is seen as relativeb) Culture is relativec) Wisdom to know who fits in their culture is relatived) Stereotypes are relativee) Energy to force a fit in vs a Quick fitf) Success views are relativeg) Quick success vs long term success is relativeh) Talent is relativei) etc is relative lolIt seems anyone can fit anywhere given enough time, effort, acceptance, skill level etc. A theatrical example may be the Google movie (The Internship) with the older gent forcing his fit into the younger culture as an "older Google employee.As always generally speaking younger folks like to be around younger folks for "perceived" \textbf{common} denominators such as energy levels, political views, religious views, us vs them thought processes, definition of wisdom etc. The paradigms go on forever but like you suggest it's sometimes more wise to move on...amen.Love your videos and demeanor very much!\normalsize   & \cellcolor{green!5}Common & \cellcolor{green!5}Social Class - Working class & \cellcolor{green!5}1/158 & \cellcolor{green!5}0.633 \\  \hline
  \cellcolor{green!27}\small Hey James, Very well articulated. Thanks for watching and taking the time to comment. I appreciate your sharing your POV. Basically "birds of a feather flock together" to feel understood and safe. But it is in forcing opposites or opposing "categories" (\textbf{age} groups) together that innovative thinking and transformative ideas germinate. That's just IMHO. - Glad you are liking my channel - I'd stoked to have you here. I like people who truly engage.\normalsize   & \cellcolor{green!27}Age & \cellcolor{green!27}Age - General & \cellcolor{green!27}1/74 & \cellcolor{green!27}1.351 \\  \hline
  \cellcolor{green!5}\small Excellent piece! I work in IT and work specifically in communications (web design, graphic design and content creation). I started in my 20s and am now 45, but because I work in the public service, I've continued to consult externally over the years since it's the only way to learn new skills and remain current. In my case, I got sidelined for any advancement but the work keeps coming. \textbf{Senior} leadership rave about summer students coming in to work in my unit, not because their work is great, but because they're young. There's this belief that everything they put out is "gonna be amazing!" Meanwhile, due to their youth they haven't fully developed the skills to do the job as well. Not saying they're bad, but years of experience (along with constant learning and growth) means as an older creative, my work is usually not only better, but delivered with a faster turn around. Add to that not all of them are open to suggestions on how to do things, because they already believe they're better than me just by being younger. The sidelining is like an Atlas Shrugged moment, where I'm expected do my job and pick up their slack, but am never recognized for how much that truly involves.  If they do, it's usually begrudgingly. For the most part they act like they're entitled to all this experience and we're supposed to act like we don't know how good we are. It's not just the workplace either, but academic institutions as well. Back in my 20s you couldn't get a university degree in creative design, but I loved it so I'm self-taught. Today, one of our best universities for such programs won't offer it part-time. Full-time only, means people like us with years of practical experience but lacking that piece of paper, can't get in unless we quit our day jobs. Not one image on their website of all the happy students in that program, look a day over 22...it speaks volumes.Freelancing keeps me going at least. It's my "Who is John Galt?" stance to the day job...save the real creativity for myself. LOL! So until my eyes go or arthritis kicks in, they won't \textbf{age} me out.    \normalsize   & \cellcolor{green!5}Age, Senior & \cellcolor{green!5}Age - General, Age - Over 65s & \cellcolor{green!5}2/372 & \cellcolor{green!5}0.538 \\  \hline
  \cellcolor{green!27}\small Ive always thought the industry is very \textbf{ageist}.. ive experienced it soo many times\normalsize   & \cellcolor{green!27}Ageist & \cellcolor{green!27}Age - General & \cellcolor{green!27}1/14 & \cellcolor{green!27}7.143 \\  \hline
  \cellcolor{green!5}\small Ageism is terrible and it's getting worse, in every field/industry but particularly those in the creative/design areas. Employers and agencies seem to think that those above 30 (35 best case scenario) have \textbf{nothing} to contribute. Some of the best designers and creative minds I know are well into their 40s or even older. Sometimes I think it's just an excuse to pay the employees less as a young person might be willing to accept a lower pay in order to get the gig or because they need less money to get by. One time, at an agency I worked at, we got an incredible portfolio and resume, we called the person in for an interview/meeting and when he got to the offices he was an older guy, possibly in his late 40s early 50s, the look on the faces of some of my colleagues was such a let down. It was like his \textbf{age} had already disqualified him and some didn't pay attention to what he had do said during the meeting, that was so rude and I even had an argument with some of them because of that. Of course, they didn't hire him and soon afterwords I started looking for another place to work, I just didn't feel that was the place for me anymore. I couldn't believe they were so elitist and snobs when it came to \textbf{age} when all that mattered was his incredible work, skills, and abilities.\normalsize   & \cellcolor{green!5}Age, Ageism, Nothing & \cellcolor{green!5}Age - General, Age - Youngsters & \cellcolor{green!5}4/242 & \cellcolor{green!5}1.653 \\  \hline
  \cellcolor{green!27}\small I'm really enjoying your content. I'm 50 and just getting into UX  UI Design as a student. I've experienced \textbf{ageism} in my last career (Account Executive/sales), so I'm familiar. I'm nervous about this topic, but (finally) excited about what I'm doing.\normalsize   & \cellcolor{green!27}Ageism & \cellcolor{green!27}Age - General & \cellcolor{green!27}1/42 & \cellcolor{green!27}2.381 \\  \hline
  \cellcolor{green!5}\small \@Philip VanDusen This is exactly what I've been doing. Building my runway in the background and when the gravy train ends, I'll be able to \textbf{transition} smoothly.\normalsize   & \cellcolor{green!5}Transition & \cellcolor{green!5}Sexual Identity - Transexuality & \cellcolor{green!5}1/27 & \cellcolor{green!5}3.704 \\  \hline
  \cellcolor{green!27}\small ABC, damn straight.\normalsize   & \cellcolor{green!27}ABC & \cellcolor{green!27}Nationality - Chinese & \cellcolor{green!27}1/3 & \cellcolor{green!27}33.333 \\  \hline
  \cellcolor{green!5}\small Schlomo - the EU might be better at it. But here in the US, \textbf{age} is less valued as a rule in the creative fields.\normalsize   & \cellcolor{green!5}Age & \cellcolor{green!5}Age - General & \cellcolor{green!5}1/25 & \cellcolor{green!5}4.0 \\  \hline
  \cellcolor{green!27}\small I'm wondering if its even something brought up during the interview. I ended up in this group of people unfortunately due to life circumstances, but I haven't given up my strive or creative arts. I think I'm on the final stretch, and this year will probably be the year that makes or breaks me. I'm at 40 now, and I've learned a lot, and gained experience, but I've never worked in this field professionally. Should I stop, or go forward full force? I have two degrees in design, and some adobe certs. I feel as if I've put a lot of time before I just throw in the towel due to this \textbf{age} stereotype.\normalsize   & \cellcolor{green!27}Age & \cellcolor{green!27}Age - General & \cellcolor{green!27}1/114 & \cellcolor{green!27}0.877 \\  \hline
  \cellcolor{green!5}\small I imagine that there exists a power-house pool of highly valuable aged out talent that, should it choose to, could coalesce and rise as a substantial competitor to younger firms. Tech isn't an Olympic event where you can tear a muscle - it's a tool. Adopting it and learning it isn't like taking up title fighting at 40. I think that some of the cuts come down to higher levels of pay at a certain level - not just \textbf{age}. Young can also mean cheap to a degree. My industry is a lot less design than image making - illustration and art - but I started out as a designer. Another great video - much appreciated! Solidly in your corner!\normalsize   & \cellcolor{green!5}Age & \cellcolor{green!5}Age - General & \cellcolor{green!5}1/120 & \cellcolor{green!5}0.833 \\  \hline
  \cellcolor{green!27}\small I got my MFA (in illustration at \textbf{age} 58!) so I could teach in my later creative life and have come up against the same problems!  Lots of people in higher education, specially in tech and the creative professions, are still hesitant to hire someone older.\normalsize   & \cellcolor{green!27}Age & \cellcolor{green!27}Age - General & \cellcolor{green!27}1/46 & \cellcolor{green!27}2.174 \\  \hline
  \cellcolor{green!5}\small Hire yourself! Do good work and shout about it! \textbf{Age} is just a number!!!! (unless, of course, you're like a-hundred-and-eight...)\normalsize   & \cellcolor{green!5}Age & \cellcolor{green!5}Age - General & \cellcolor{green!5}1/20 & \cellcolor{green!5}5.0 \\  \hline
  
\end{longtable}
\end{center}


\centering\textbf{\large \hypertarget{Table 2}{Table 2}: Summary of the results per sociolinguistic variable 
}
\newcolumntype{C}[2]{>{\centering\arraybackslash}p{#1}}
\begin{center}
\setlength\mylength{\dimexpr\textwidth - 1\arrayrulewidth - 40\tabcolsep}
\begin{longtable}{|C{.50\mylength}|C{.30\mylength}|C{.15\mylength}|C{.15\mylength}|C{.15\mylength}|}
\hline
\textbf{Sociolinguistic variables (Hiper - Hipo)} & \textbf{KeyWords} & \textbf{Number of occurrences} & \textbf{Frequency}  & \textbf{Frequency(\%)} \\
\hline\multirow{1}{*}{\cellcolor{red!27}Age - Over 65s}  & \cellcolor{red!27}Senior, Old & \cellcolor{red!27}12 & \cellcolor{red!27}12/6497& \cellcolor{red!27}0.18 \\  \hline
  \multirow{1}{*}{\cellcolor{red!5}Age - General}  & \cellcolor{red!5}Age, Ageist, Ageism & \cellcolor{red!5}29 & \cellcolor{red!5}29/6497& \cellcolor{red!5}0.44999999999999996 \\  \hline
  \multirow{1}{*}{\cellcolor{red!27}Ethnicity - General}  & \cellcolor{red!27}Racist, Race & \cellcolor{red!27}2 & \cellcolor{red!27}2/6497& \cellcolor{red!27}0.03 \\  \hline
  \multirow{1}{*}{\cellcolor{red!5}Ethnicity - Black}  & \cellcolor{red!5}Black & \cellcolor{red!5}1 & \cellcolor{red!5}1/6497& \cellcolor{red!5}0.02 \\  \hline
  \multirow{1}{*}{\cellcolor{red!27}Social Class - Working class}  & \cellcolor{red!27}Common & \cellcolor{red!27}2 & \cellcolor{red!27}2/6497& \cellcolor{red!27}0.03 \\  \hline
  \multirow{1}{*}{\cellcolor{red!5}Sexual Identity - Transexuality}  & \cellcolor{red!5}Transition & \cellcolor{red!5}2 & \cellcolor{red!5}2/6497& \cellcolor{red!5}0.03 \\  \hline
  \multirow{1}{*}{\cellcolor{red!27}Age - Youngsters}  & \cellcolor{red!27}Nothing & \cellcolor{red!27}1 & \cellcolor{red!27}1/6497& \cellcolor{red!27}0.02 \\  \hline
  \multirow{1}{*}{\cellcolor{red!5}Nationality - Chinese}  & \cellcolor{red!5}ABC & \cellcolor{red!5}1 & \cellcolor{red!5}1/6497& \cellcolor{red!5}0.02 \\  \hline
  
\end{longtable}
\end{center}


\textbf{\Large Result analysis:}

\begin{itemize}\item Taking into account the words that were detected, we can reach the conclusion these comments are associated with : : Age - Over 65s;Age - General;Ethnicity - General;Ethnicity - Black;Social Class - Working class;Sexual Identity - Transexuality;Age - Youngsters;Nationality - Chinese;%.

\item The percentage of hate speech related words is 0.7696.

\item Considering that the variable \textbf{Age - General} has the most occurences in the post, we can interpret that this is the predominant hate speech.

\item Overall there were 50/142 occurences of hate speech related comments.\end{itemize}\end{document}