\documentclass[11pt]{article}
\ExplSyntaxOn
\let\tl_length:n\tl_count:n
\ExplSyntaxOff
\usepackage{graphicx}
\usepackage{multirow}
\usepackage{colortbl}
\usepackage{longtable, array}
\usepackage{hyperref}
\usepackage[usenames,dvipsnames,svgnames,table]{xcolor}
\newlength\mylength
\usepackage[legalpaper, landscape, margin=0.8in]{geometry}
\newcommand{\MinNumber}{0}
\begin{document}

\textbf {\huge In Youtube\_extraction\_portuguese\_63.json :}\newline \par\Large\textbf {Title: \large  DENTISTA DROGADA, BÊBADA, RACISTA E HOMOFÓBICA }\newline {\par\large --- Table  1: Summary of the results per comment; }

 {\par\large --- \hyperlink{Table 2}{\textcolor{blue}{\underline{Table 2}}}: Summary of the results per sociolinguistic variable;}\newline \normalsize\newline

\centering\textbf{\large Table  1: Summary of the results per comment 
}
\newcommand{\MaxNumber}{0}%
\newcommand{\ApplyGradient}[1]{%
\pgfmathsetmacro{\PercentColor}{100.0*(#1-\MinNumber)/(\MaxNumber-\MinNumber)}
\xdef\PercentColor{\PercentColor}%
\cellcolor{LightSpringGreen!\PercentColor!LightRed}{#1}
}
\newcolumntype{C}[2]{>{\centering\arraybackslash}p{#1}}
\begin{center}
\setlength\mylength{\dimexpr\textwidth - 1\arrayrulewidth - 50\tabcolsep}
\begin{longtable}{|C{.65\mylength}|C{.30\mylength}|C{.12\mylength}|C{.12\mylength}|C{.12\mylength}|}
\hline
\textbf{Comment} & \textbf{KeyWords} & \textbf{Sociolinguistic variables (Hiper - Hipo)}  & \textbf{Hate Speech Frequency} & \textbf{Hate Speech Frequency(\%)} \\
\hline\cellcolor{green!27}\small Kkk o \textbf{baleia} me larga Aki rs\normalsize   & \cellcolor{green!27}Baleia & \cellcolor{green!27}Physical Identity - Physical Features & \cellcolor{green!27}1/7 & \cellcolor{green!27}14.286 \\  \hline
  \cellcolor{green!5}\small Eu nao sirvo p ser policial nao, pq se eu sou uma \textbf{mulher} policial nessa cidade e tivesse vendo uma coisa dessas eu podia ta em outra ocorrencia q eu ia p la e desmaiava essa doida no tapa, spray d pimenta e arma d choque, os pms ai estao praticamente impotentes pq alem d nao poder agredir bandido ou em ninguem q foi abordado ela e \textbf{mulher}, ai e capaz ate deles serem presos por bater numa \textbf{mulher}, mais uma pm feminina podeia bater nela nem q eu fosse processada mais eu nao ia aceitar um absurdo desses nao, ela podia rolar na terra pelada, mais chamar policial d \textbf{preto}, \textbf{gordo}, \textbf{veado}, e o maximo do desrespeito\normalsize   & \cellcolor{green!5}Gordo, Mulher, Preto, Veado & \cellcolor{green!5}Ethnicity - Black, Gender - General, Physical Identity - Physical Features, Sexual Identity - Male homosexuality & \cellcolor{green!5}6/118 & \cellcolor{green!5}5.085 \\  \hline
  \cellcolor{green!27}\small Vei, alem de bebada é \textbf{racista} e nogenta\normalsize   & \cellcolor{green!27}Racista & \cellcolor{green!27}Ethnicity - General & \cellcolor{green!27}1/8 & \cellcolor{green!27}12.5 \\  \hline
  \cellcolor{green!5}\small Aff, que \textbf{mulher} nojenta\normalsize   & \cellcolor{green!5}Mulher & \cellcolor{green!5}Gender - General & \cellcolor{green!5}1/4 & \cellcolor{green!5}25.0 \\  \hline
  \cellcolor{green!27}\small Bebada e racista\normalsize   & \cellcolor{green!27}Racista & \cellcolor{green!27}Ethnicity - General & \cellcolor{green!27}1/3 & \cellcolor{green!27}33.333 \\  \hline
  
\end{longtable}
\end{center}


\centering\textbf{\large \hypertarget{Table 2}{Table 2}: Summary of the results per sociolinguistic variable 
}
\newcolumntype{C}[2]{>{\centering\arraybackslash}p{#1}}
\begin{center}
\setlength\mylength{\dimexpr\textwidth - 1\arrayrulewidth - 40\tabcolsep}
\begin{longtable}{|C{.50\mylength}|C{.30\mylength}|C{.15\mylength}|C{.15\mylength}|C{.15\mylength}|}
\hline
\textbf{Sociolinguistic variables (Hiper - Hipo)} & \textbf{KeyWords} & \textbf{Number of occurrences} & \textbf{Frequency}  & \textbf{Frequency(\%)} \\
\hline\multirow{1}{*}{\cellcolor{red!27}Physical Identity - Physical Features}  & \cellcolor{red!27}Baleia, Gordo & \cellcolor{red!27}2 & \cellcolor{red!27}2/239& \cellcolor{red!27}0.84 \\  \hline
  \multirow{1}{*}{\cellcolor{red!5}Gender - General}  & \cellcolor{red!5}Mulher & \cellcolor{red!5}2 & \cellcolor{red!5}2/239& \cellcolor{red!5}0.84 \\  \hline
  \multirow{1}{*}{\cellcolor{red!27}Ethnicity - Black}  & \cellcolor{red!27}Preto & \cellcolor{red!27}1 & \cellcolor{red!27}1/239& \cellcolor{red!27}0.42 \\  \hline
  \multirow{1}{*}{\cellcolor{red!5}Sexual Identity - Male homosexuality}  & \cellcolor{red!5}Veado & \cellcolor{red!5}1 & \cellcolor{red!5}1/239& \cellcolor{red!5}0.42 \\  \hline
  \multirow{1}{*}{\cellcolor{red!27}Ethnicity - General}  & \cellcolor{red!27}Racista & \cellcolor{red!27}2 & \cellcolor{red!27}2/239& \cellcolor{red!27}0.84 \\  \hline
  
\end{longtable}
\end{center}


\textbf{\Large Result analysis:}

\begin{itemize}\item Taking into account the words that were detected, we can reach the conclusion these comments are associated with : : Physical Identity - Physical Features;Gender - General;Ethnicity - Black;Sexual Identity - Male homosexuality;Ethnicity - General;%.

\item The percentage of hate speech related words is 3.3473.

\item Considering that the variable \textbf{Physical Identity - Physical Features} has the most occurences in the post, we can interpret that this is the predominant hate speech.

\item Overall there were 10/12 occurences of hate speech related comments.\end{itemize}\end{document}