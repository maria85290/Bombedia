\documentclass[11pt]{article}
\ExplSyntaxOn
\let\tl_length:n\tl_count:n
\ExplSyntaxOff
\usepackage{graphicx}
\usepackage{multirow}
\usepackage{colortbl}
\usepackage{longtable, array}
\usepackage{hyperref}
\usepackage[usenames,dvipsnames,svgnames,table]{xcolor}
\newlength\mylength
\usepackage[legalpaper, landscape, margin=0.8in]{geometry}
\newcommand{\MinNumber}{0}
\begin{document}

\textbf {\huge In youtube\_extraction\_english\_139.json :}\newline \par\Large\textbf {Title: \large  The Roots and Consequences of Ageism in America }\newline {\par\large --- Table  1: Summary of the results per comment; }

 {\par\large --- \hyperlink{Table 2}{\textcolor{blue}{\underline{Table 2}}}: Summary of the results per sociolinguistic variable;}\newline \normalsize\newline

\centering\textbf{\large Table  1: Summary of the results per comment 
}
\newcommand{\MaxNumber}{0}%
\newcommand{\ApplyGradient}[1]{%
\pgfmathsetmacro{\PercentColor}{100.0*(#1-\MinNumber)/(\MaxNumber-\MinNumber)}
\xdef\PercentColor{\PercentColor}%
\cellcolor{LightSpringGreen!\PercentColor!LightRed}{#1}
}
\newcolumntype{C}[2]{>{\centering\arraybackslash}p{#1}}
\begin{center}
\setlength\mylength{\dimexpr\textwidth - 1\arrayrulewidth - 50\tabcolsep}
\begin{longtable}{|C{.65\mylength}|C{.30\mylength}|C{.12\mylength}|C{.12\mylength}|C{.12\mylength}|}
\hline
\textbf{Comment} & \textbf{KeyWords} & \textbf{Sociolinguistic variables (Hiper - Hipo)}  & \textbf{Hate Speech Frequency} & \textbf{Hate Speech Frequency(\%)} \\
\hline\cellcolor{green!27}\small World-wide, we live in societies that mainly perpetuates the idea of young male strength through the military and sports which dominates all cultures. From an early \textbf{age}, females are socialized into weak and passive roles. When they are adults, the majority are invisible to power jobs or governmental participation although a few become tokens, most often, if they inherit wealth. \textbf{Ageism} is practiced more severely against women because men are more prepared financially in their \textbf{senior} years. Also, the media and social norms generate the notion that males maintain their \textbf{sex} appeal and virility in that they can produce sperm to produce offspring with younger females while women of the same \textbf{age} are unproductive. US President Trump married and conceived a son with a \textbf{woman} that was 24 years younger than he was and maintains the most important job in government even though he is 71 years \textbf{old}. Through history and presently, wars and crimes are commonly committed by males, who unyielding dominant \textbf{woman} and the world, at any \textbf{age}. It's ironic that the females, who are restrained and devalued, outlive the males.\normalsize   & \cellcolor{green!27}Age, Ageism, Old, Senior, Sex, Woman & \cellcolor{green!27}Age - General, Age - Over 65s, Gender - General & \cellcolor{green!27}9/183 & \cellcolor{green!27}4.918 \\  \hline
  \cellcolor{green!5}\small The \textbf{old} want to hold the young back and the young want the \textbf{old} out of the way.\normalsize   & \cellcolor{green!5}Old & \cellcolor{green!5}Age - Over 65s & \cellcolor{green!5}2/18 & \cellcolor{green!5}11.111 \\  \hline
  \cellcolor{green!27}\small It's interesting how White \textbf{elderly} folks complain about experiencing \textbf{ageism}. Now they understand that prejudice isn't a myth. Now they understand what we \textbf{ethnic} minority groups go through all our lives, even when we were younger.\normalsize   & \cellcolor{green!27}Ageism, Elderly, Ethnic & \cellcolor{green!27}Age - General, Age - Over 65s, Ethnicity - General & \cellcolor{green!27}3/36 & \cellcolor{green!27}8.333 \\  \hline
  \cellcolor{green!5}\small Affects just \textbf{old} people ?? NO I am afraid it affects even younger people, for example in my case young people who are 5 or more years younger than me avoid having anything to do with me.\normalsize   & \cellcolor{green!5}Old & \cellcolor{green!5}Age - Over 65s & \cellcolor{green!5}1/37 & \cellcolor{green!5}2.703 \\  \hline
  \cellcolor{green!27}\small Ageism doesn't just apply to physiologically \textbf{old} people it also applies to young people as well i would say more so even than older people, many young people are far more intelligent than their older counterparts but are dismissed because of \textbf{age} without anyone ever having considered their arguments.\normalsize   & \cellcolor{green!27}Age, Ageism, Old & \cellcolor{green!27}Age - General, Age - Over 65s & \cellcolor{green!27}3/49 & \cellcolor{green!27}6.122 \\  \hline
  \cellcolor{green!5}\small Ageism is the only widely accepted prejudice in the world. It's wrong. It short changes everyone. I believe I can learn something from everyone from infant to centarians. All people have value.\normalsize   & \cellcolor{green!5}Ageism & \cellcolor{green!5}Age - General & \cellcolor{green!5}1/32 & \cellcolor{green!5}3.125 \\  \hline
  \cellcolor{green!27}\small Yes it really does effect young \& \textbf{old} alike. I made an objective video about \textbf{Ageism} on my page, I think you'd enjoy it.\normalsize   & \cellcolor{green!27}Ageism, Old & \cellcolor{green!27}Age - General, Age - Over 65s & \cellcolor{green!27}2/24 & \cellcolor{green!27}8.333 \\  \hline
  \cellcolor{green!5}\small Ageism its so ridiculous and irrational that it serious piss me off badly. I wonder, Does people of twenty´s know that some day they will be the \textbf{old} ones?. Im 34 by the way.\normalsize   & \cellcolor{green!5}Ageism, Old & \cellcolor{green!5}Age - General, Age - Over 65s & \cellcolor{green!5}2/34 & \cellcolor{green!5}5.882 \\  \hline
  \cellcolor{green!27}\small Ageism happens to younger people too. Particularly people \textbf{age} 18-20. They're quite demonized by laws and regulations. (Harsh DMV punishments on youth, can't drink, can't own handguns, can't adopt, can't rent a car, can't rent a hotel room, can't go to a casino, can't be public officials or police officers, BUT they can die for their country).\normalsize   & \cellcolor{green!27}Age, Ageism & \cellcolor{green!27}Age - General & \cellcolor{green!27}2/57 & \cellcolor{green!27}3.509 \\  \hline
  \cellcolor{green!5}\small I'm 27 and when I told a few teens my \textbf{age}, they said I was \textbf{old}, which made me feel bad, because I don't feel \textbf{old}. I think that's probably a \textbf{common} experience throughout life, when people just label you that way. It's a terrible mentality. I never felt that way about anyone when I was younger. I hope as we go forward people realize the narrow-mindedness of the way they treat people, and change their ways.\normalsize   & \cellcolor{green!5}Age, Common, Old & \cellcolor{green!5}Age - General, Age - Over 65s, Social Class - Working class & \cellcolor{green!5}4/77 & \cellcolor{green!5}5.195 \\  \hline
  \cellcolor{green!27}\small This is good stuff!  I intend to share this with the paramedic class I teach.  Over 30 of our patients in EMS are over the \textbf{age} of 65 and I believe we make far too many assumptions when we deal with different \textbf{age} groups.\normalsize   & \cellcolor{green!27}Age & \cellcolor{green!27}Age - General & \cellcolor{green!27}2/44 & \cellcolor{green!27}4.545 \\  \hline
  
\end{longtable}
\end{center}


\centering\textbf{\large \hypertarget{Table 2}{Table 2}: Summary of the results per sociolinguistic variable 
}
\newcolumntype{C}[2]{>{\centering\arraybackslash}p{#1}}
\begin{center}
\setlength\mylength{\dimexpr\textwidth - 1\arrayrulewidth - 40\tabcolsep}
\begin{longtable}{|C{.50\mylength}|C{.30\mylength}|C{.15\mylength}|C{.15\mylength}|C{.15\mylength}|}
\hline
\textbf{Sociolinguistic variables (Hiper - Hipo)} & \textbf{KeyWords} & \textbf{Number of occurrences} & \textbf{Frequency}  & \textbf{Frequency(\%)} \\
\hline\multirow{1}{*}{\cellcolor{red!27}Gender - General}  & \cellcolor{red!27}Sex, Woman & \cellcolor{red!27}3 & \cellcolor{red!27}3/783& \cellcolor{red!27}0.38 \\  \hline
  \multirow{1}{*}{\cellcolor{red!5}Age - General}  & \cellcolor{red!5}Age, Ageism & \cellcolor{red!5}13 & \cellcolor{red!5}13/783& \cellcolor{red!5}1.66 \\  \hline
  \multirow{1}{*}{\cellcolor{red!27}Age - Over 65s}  & \cellcolor{red!27}Old, Senior, Elderly & \cellcolor{red!27}11 & \cellcolor{red!27}11/783& \cellcolor{red!27}1.4000000000000001 \\  \hline
  \multirow{1}{*}{\cellcolor{red!5}Ethnicity - General}  & \cellcolor{red!5}Ethnic & \cellcolor{red!5}1 & \cellcolor{red!5}1/783& \cellcolor{red!5}0.13 \\  \hline
  \multirow{1}{*}{\cellcolor{red!27}Social Class - Working class}  & \cellcolor{red!27}Common & \cellcolor{red!27}1 & \cellcolor{red!27}1/783& \cellcolor{red!27}0.13 \\  \hline
  
\end{longtable}
\end{center}


\textbf{\Large Result analysis:}

\begin{itemize}\item Taking into account the words that were detected, we can reach the conclusion these comments are associated with : : Gender - General;Age - General;Age - Over 65s;Ethnicity - General;Social Class - Working class;%.

\item The percentage of hate speech related words is 3.7037.

\item Considering that the variable \textbf{Age - General} has the most occurences in the post, we can interpret that this is the predominant hate speech.

\item Overall there were 31/18 occurences of hate speech related comments.\end{itemize}\end{document}