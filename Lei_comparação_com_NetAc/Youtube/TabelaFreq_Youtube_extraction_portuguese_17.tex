\documentclass[11pt]{article}
\ExplSyntaxOn
\let\tl_length:n\tl_count:n
\ExplSyntaxOff
\usepackage{graphicx}
\usepackage{multirow}
\usepackage{colortbl}
\usepackage{longtable, array}
\usepackage{hyperref}
\usepackage[usenames,dvipsnames,svgnames,table]{xcolor}
\newlength\mylength
\usepackage[legalpaper, landscape, margin=0.8in]{geometry}
\newcommand{\MinNumber}{0}
\begin{document}

\textbf {\huge In Youtube\_extraction\_portuguese\_17.json :}\newline \par\Large\textbf {Title: \large SOU TRANS, MAS A CIRURGIA NÃO ME FEZ MAIS MULHER | PAPO KABELO COM KAROL PINHEIRO | Salon Line}\newline {\par\large --- Table  1: Summary of the results per comment; }

 {\par\large --- \hyperlink{Table 2}{\textcolor{blue}{\underline{Table 2}}}: Summary of the results per sociolinguistic variable;}\newline \normalsize\newline

\centering\textbf{\large Table  1: Summary of the results per comment 
}
\newcommand{\MaxNumber}{0}%
\newcommand{\ApplyGradient}[1]{%
\pgfmathsetmacro{\PercentColor}{100.0*(#1-\MinNumber)/(\MaxNumber-\MinNumber)}
\xdef\PercentColor{\PercentColor}%
\cellcolor{LightSpringGreen!\PercentColor!LightRed}{#1}
}
\newcolumntype{C}[2]{>{\centering\arraybackslash}p{#1}}
\begin{center}
\setlength\mylength{\dimexpr\textwidth - 1\arrayrulewidth - 50\tabcolsep}
\begin{longtable}{|C{.65\mylength}|C{.30\mylength}|C{.12\mylength}|C{.12\mylength}|C{.12\mylength}|}
\hline
\textbf{Comment} & \textbf{KeyWords} & \textbf{Sociolinguistic variables (Hiper - Hipo)}  & \textbf{Hate Speech Frequency} & \textbf{Hate Speech Frequency(\%)} \\
\hline\cellcolor{green!27}\small Novo por aqui...Mas gostaria de colocar uns pontos, apresentadora foi admirável em assumir que não entende no assunto.Música de fundo meio perdida né, o mesmo toque repetido fica bem cansativo.Cenário tira bem o foco viu, não sei quem já assiste e tá acostumado tudo bem mas eu em particular daria uma organizada a parte visual e quem sabe  mesclar as cores, as canecas tbm ficam bem perdidas ali né, tudo tem duas cores menos ela, tudo bem que dentro é \textbf{a\textbf{marelo}} e fora é azul, mas seria legal uma amarela e uma azul.E sobre esse tema acredito que foi mais que necessário, pois precisamos mostrar para as pessoas que Xs Trans estão ai e pra ficar.\normalsize   & \cellcolor{green!27}Amarelo & \cellcolor{green!27}Nationality - Chinese, Nationality - Japanese & \cellcolor{green!27}2/120 & \cellcolor{green!27}1.667 \\  \hline
  \cellcolor{green!5}\small Deixando claro de primeira não tenho preconceito com nada . Porém tenho uma dúvida , se a pessoa "trans pro lado feminino" já nasce se  sentindo \textbf{mulher} ; mais ela não tem útero como assim  ? Se ela nasce \textbf{mulher} no corpo teria que ser igual ao de uma \textbf{mulher} por dentro ou não ? Isso é confuso pra mim.\normalsize   & \cellcolor{green!5}Mulher & \cellcolor{green!5}Gender - General & \cellcolor{green!5}3/60 & \cellcolor{green!5}5.0 \\  \hline
  \cellcolor{green!27}\small Sou Homem, sisgênero, Criado em um ambiente \textbf{machista}, sou cristão protestante e sou fã da Thiessa, porque a palavra antes de tudo diz que eu não devo julgar!\normalsize   & \cellcolor{green!27}Machista & \cellcolor{green!27}Gender - General & \cellcolor{green!27}1/28 & \cellcolor{green!27}3.571 \\  \hline
  \cellcolor{green!5}\small Que \textbf{mulher} maravilhosa ❤️❤️❤️❤️OBS: Tá tudo bem a entrevista ficar sem música de fundo pq o que importa é a conversa <3\normalsize   & \cellcolor{green!5}Mulher & \cellcolor{green!5}Gender - General & \cellcolor{green!5}1/23 & \cellcolor{green!5}4.348 \\  \hline
  \cellcolor{green!27}\small Uma \textbf{transexual} jamais tomará lugar de uma \textbf{mulher} em um lar, não tem sistema reprodutor nem as características de uma \textbf{mulher}...\normalsize   & \cellcolor{green!27}Mulher, Transexual & \cellcolor{green!27}Gender - General, Sexual Identity - Transexuality & \cellcolor{green!27}3/21 & \cellcolor{green!27}14.286 \\  \hline
  \cellcolor{green!5}\small Continua sendo homem, xy.O que Deus fez tá feito.Ninguém \textbf{muda}.Que nossa senhora nos proteja das coisas do mundo. Amém.\normalsize   & \cellcolor{green!5}Muda & \cellcolor{green!5}Physical Identity - Physical (and Mental) Impairments & \cellcolor{green!5}1/22 & \cellcolor{green!5}4.545 \\  \hline
  \cellcolor{green!27}\small ela e uma \textbf{mulher},  e liiinda!!!!!\normalsize   & \cellcolor{green!27}Mulher & \cellcolor{green!27}Gender - General & \cellcolor{green!27}1/6 & \cellcolor{green!27}16.667 \\  \hline
  \cellcolor{green!5}\small A biologia diz q homem e homem e \textbf{mulher} e mulhet.  XX E XY. Oh, desculpa, eu fico com a biologia.\normalsize   & \cellcolor{green!5}Mulher & \cellcolor{green!5}Gender - General & \cellcolor{green!5}1/21 & \cellcolor{green!5}4.762 \\  \hline
  \cellcolor{green!27}\small o pessoal \textbf{LGBT} como a Thiesita acham que vão mudar a ciência. se vc nasce com penis vc é sim homem, e se nasce com vagina vc é \textbf{mulher} sim. não sou eu que falo é a ciência.\normalsize   & \cellcolor{green!27}LGBT, Mulher & \cellcolor{green!27}Gender - General, Sexual Identity - General & \cellcolor{green!27}2/38 & \cellcolor{green!27}5.263 \\  \hline
  \cellcolor{green!5}\small chocada com a \textbf{idade} dessa meninaaaaa\normalsize   & \cellcolor{green!5}Idade & \cellcolor{green!5}Age - General & \cellcolor{green!5}1/6 & \cellcolor{green!5}16.667 \\  \hline
  \cellcolor{green!27}\small Gente que \textbf{mulher} linda, toda felicidade do mundo pra voce!!! Eu sobro preconceito pq sou divorciada, e realmente so sinto pena dessas pessoas, na verdade elas gostariam de ter a nossa coragem pra ser que a gente quiser ser....amei a entrevista\normalsize   & \cellcolor{green!27}Mulher & \cellcolor{green!27}Gender - General & \cellcolor{green!27}1/41 & \cellcolor{green!27}2.439 \\  \hline
  \cellcolor{green!5}\small vim deixar um oi junto com uma reflexão:O mundo está nas mãos daqueles que têm a coragem de sonhar e de correr o risco de viver seus sonhos. seja feliz!ps por favor deixa um oi no meu canal e  Escute a música Meu amor do cantor e compositor gato amarelo\normalsize   & \cellcolor{green!5}Amarelo & \cellcolor{green!5}Nationality - Chinese, Nationality - Japanese & \cellcolor{green!5}2/50 & \cellcolor{green!5}4.0 \\  \hline
  \cellcolor{green!27}\small Mesmo com o pênis mutilado continua sendo um homem. E como deve ser triste o pai desse rapaz por ter tido um filho que recusou seu órgão \textbf{sexual} dado por Deus.\normalsize   & \cellcolor{green!27}Sexual & \cellcolor{green!27}Gender - General & \cellcolor{green!27}1/31 & \cellcolor{green!27}3.226 \\  \hline
  \cellcolor{green!5}\small vídeo lindo e esclarecedor! Força a todas as meninas que nesse momento estão passando por essa \textbf{transição} exterior e interior.  Admiro essa luta\normalsize   & \cellcolor{green!5}Transição & \cellcolor{green!5}Sexual Identity - Transexuality & \cellcolor{green!5}1/23 & \cellcolor{green!5}4.348 \\  \hline
  \cellcolor{green!27}\small Ela e linda pego na hora que ela quiser, agora ser feminista e coisa \textbf{idiota} não faz sentido\normalsize   & \cellcolor{green!27}Idiota & \cellcolor{green!27}Physical Identity - Physical (and Mental) Impairments & \cellcolor{green!27}1/18 & \cellcolor{green!27}5.556 \\  \hline
  \cellcolor{green!5}\small Fico perdida com esse assunto,buguei total.Se ela fez a cirurgia tornou se \textbf{mulher}.Pq tem que falar que entes era homem 🤷‍♀️.No caso ela não tem documentos com a nova identidade?\normalsize   & \cellcolor{green!5}Mulher & \cellcolor{green!5}Gender - General & \cellcolor{green!5}1/33 & \cellcolor{green!5}3.03 \\  \hline
  \cellcolor{green!27}\small Já fui muito preconceituoso com \textbf{gay} trans ,hoje ainda percebo que algumas atitudes minhas são \textbf{machista} e preconceituosa ,mas estou tentando trabalhar isso porque e como ela disse Jesus ensinou a gente a amar o próximo não julgar.\normalsize   & \cellcolor{green!27}Gay, Machista & \cellcolor{green!27}Gender - General, Sexual Identity - General & \cellcolor{green!27}2/38 & \cellcolor{green!27}5.263 \\  \hline
  \cellcolor{green!5}\small Que hino de vídeo. Thiessa é uma \textbf{mulher} maravilhosa 👌🔝\normalsize   & \cellcolor{green!5}Mulher & \cellcolor{green!5}Gender - General & \cellcolor{green!5}1/10 & \cellcolor{green!5}10.0 \\  \hline
  \cellcolor{green!27}\small Gentem, a entrevistadora tenta esconder, mas que \textbf{puta} preconceito ela tem hein afff só a as caras q ela faz já mostra\normalsize   & \cellcolor{green!27}Puta & \cellcolor{green!27}Gender - Female sexuality & \cellcolor{green!27}1/22 & \cellcolor{green!27}4.545 \\  \hline
  \cellcolor{green!5}\small pessoal, sexo é diferente de gênero. a identidade de gênero é como você se vê na sociedade, independente do seu órgão \textbf{sexual}, então PAREM de falar sobre cromossomos e sobre órgãos, isso não tem relação nenhuma e não é um argumento válido.\normalsize   & \cellcolor{green!5}Sexual & \cellcolor{green!5}Gender - General & \cellcolor{green!5}1/42 & \cellcolor{green!5}2.381 \\  \hline
  \cellcolor{green!27}\small O que me chocou foi a \textbf{idade} desse ser! Menina essa cara é de 18, morri\normalsize   & \cellcolor{green!27}Idade & \cellcolor{green!27}Age - General & \cellcolor{green!27}1/16 & \cellcolor{green!27}6.25 \\  \hline
  \cellcolor{green!5}\small karol pinheiro é muito viagem errada, no inicio dá entrevista já leva as cortadas da mana trans. \textbf{Mulher} por favor estude antes de fazeer essa entrevista, olha esse pink money as custas das trans. meio ridículo vari os comentários\normalsize   & \cellcolor{green!5}Mulher & \cellcolor{green!5}Gender - General & \cellcolor{green!5}1/39 & \cellcolor{green!5}2.564 \\  \hline
  \cellcolor{green!27}\small Então ser \textbf{mulher} é um sentimento? Ata\normalsize   & \cellcolor{green!27}Mulher & \cellcolor{green!27}Gender - General & \cellcolor{green!27}1/7 & \cellcolor{green!27}14.286 \\  \hline
  \cellcolor{green!5}\small Clara Muniz \textbf{misoginia}, pode? Porque o movimento trans só tem isso.\normalsize   & \cellcolor{green!5}Misoginia & \cellcolor{green!5}Gender - General & \cellcolor{green!5}1/11 & \cellcolor{green!5}9.091 \\  \hline
  \cellcolor{green!27}\small Acusar o movimento trans de ser totalmente \textbf{misógino} é extramemente mentiroso. É o mesmo que dizer que todo \textbf{gay} é promíscuo,  que toda \textbf{mulher} é falsa, que toda feminista é peluda suja, mal comida e que todo religioso é mente fechada.Pra mim tudo é sentimento,  a vida é sentimento.  Pra você pode não ser, mas é sua OBRIGAÇÃO respeitar. Até porque sua língua não vai cair se chama alguém pelo nome que ela quer.\normalsize   & \cellcolor{green!27}Gay, Misógino, Mulher & \cellcolor{green!27}Gender - General, Sexual Identity - General & \cellcolor{green!27}3/75 & \cellcolor{green!27}4.0 \\  \hline
  \cellcolor{green!5}\small Agora eu sei o que é uma pessoa sem informação, mdss, hoje que eu descobri o que realmente é Trans e orientação \textbf{sexual}.\normalsize   & \cellcolor{green!5}Sexual & \cellcolor{green!5}Gender - General & \cellcolor{green!5}1/23 & \cellcolor{green!5}4.348 \\  \hline
  \cellcolor{green!27}\small EU amei a entrevista muito mesmo admiro muito a Thiessa pelo character,pela força de seguir sendo a \textbf{mulher} Linda que es e Nunca baixar a cabeça PRA NADA e ninguem,um DIA EU fui assim mas infelizmente a Vida me fez trilhar outro caminho mas um DIA EU volto a ser a pessoal feliz,auto astral,bem humorada e communicative que EU era 😞😞😞😥😥😥 sinto saudades....Beijos 😘😘😘❤️❤️❤️❤️😥😥😞\normalsize   & \cellcolor{green!27}Mulher & \cellcolor{green!27}Gender - General & \cellcolor{green!27}1/65 & \cellcolor{green!27}1.538 \\  \hline
  \cellcolor{green!5}\small ​\@Matheus Rodrigues KKKKKKKKKKKKKKKKKKKKKKKKKKKKKKKKKKKKKKKK A estupidez não encontra limites, \textbf{puta} que pariu. Por acaso você já buscou ler, estudar e entender sobre o assunto ou sua fonte de informações é a bíblia e os tweets de pastores e políticos de extrema-direita? Por que você acha que existe o tratamento de \textbf{transição}, ingestão ou aplicação de hormônios, cirurgias de redesignação \textbf{sexual}, avaliações que levam ANOS para concluir antes de começar qualquer procedimento? Ou você continua acreditando que qualquer um pode aparecer ali, falar que é trans e já ganha autorização para iniciar o tratamento? Por que você acha que existem diversos estudos e grande parte apontam para causas genéticas? Estudos esses que já apontaram que o cérebro de mulheres trans são muito parecidos com os cérebros de mulheres cis, e o mesmo se aplica aos homens. Você acha que esses médicos, psiquiatras, cirurgiões, geneticistas, pesquisadores, etc., são comunistas marxistas anti-cristianismo que querem destruir os valores morais e os bons costumes? Transexualidade não surgiu agora, sempre existiu e houve DÉCADAS de estudos e terapias de reversão, há um motivo porque a medicina tem essa posição sobre tal questão. Por isso a luta dos direitos de transgêneros é legítima, porque existe embasamento, se fosse só uma simples ideologia não teria essa proporção que tem hoje. Mas do jeito que você é \textbf{imbecil}, você vai ler tudo e simplesmente ignorar, preferindo continuar dentro da bolha ignorante em que você se encontra. Boa sorte na vida, porque a história não vai ser gentil com você.\normalsize   & \cellcolor{green!5}Imbecil, Puta, Sexual, Transição & \cellcolor{green!5}Gender - Female sexuality, Gender - General, Physical Identity - Physical (and Mental) Impairments, Sexual Identity - Transexuality & \cellcolor{green!5}4/249 & \cellcolor{green!5}1.606 \\  \hline
  \cellcolor{green!27}\small \@Israel Oliveira você me chama de \textbf{imbecil} mas supõe que meus argumentos são baseados em bíblia e pior, deduz que eu sou cristão. Não sou. 1)Sim, já li. Na verdade, faço medicina justamente por causa de assuntos como esses, e sei bem que "aplicação de hormônios" contrários ao seu sexo são péssimos à saúde, principalmente em mulheres, e por isso não devem ser incentivados, como no vídeo. 2) jamais disse que deve-se proibir quem deseja realizar uma "mudança de sexo"(que é impossível). A não ser que sejam em crianças. Mas você veio com tanta prepotência que presumiu algo que não escrevi, mostrando seu claro preconceito contra os cristãos. Deixando claro, sou agnóstico. 3) na verdade os cérebros são todos iguais kk, Não sei o que você quis dizer com isso. 4)Te contradizendo.Sim, é apenas uma questão ideológica, não têm embasamento científico algum. 5) diferença de sexo e gênero: um é um fato científico, já o outro... bom, baseados em estudos sociológicos que contrariam ridiculamente a ciência. PIADA.\normalsize   & \cellcolor{green!27}Imbecil & \cellcolor{green!27}Physical Identity - Physical (and Mental) Impairments & \cellcolor{green!27}1/167 & \cellcolor{green!27}0.599 \\  \hline
  \cellcolor{green!5}\small Matheus Rodrigues pois eu me sinto muito representada por suas palavras, Matheus. Resumir ser \textbf{mulher} a uma mera questão de sentimento é uma ofensa, ignora toda a nossa materialidade genética e biológica e nos resume a uma matéria fluida. Esse povo entra em tamanha contradição que, segundo esse pensamento, onde entraria, então, a questão de local de fala e vivência? Que vivência uma \textbf{mulher} trans tem do que é ser \textbf{mulher} pra dizer que é uma? Feministas radicais estão denunciando isso há anos e em vários lugares do mundo, esse povo só finge demência e ignora. Ridículos, acham que falam em nome da ciência e das mulheres na integralidade. Não falam!\normalsize   & \cellcolor{green!5}Mulher & \cellcolor{green!5}Gender - General & \cellcolor{green!5}3/111 & \cellcolor{green!5}2.703 \\  \hline
  \cellcolor{green!27}\small \@Carol você assistiu o vídeo? Pelo que lembro tem uma parte que ela fala que sobre ser tratada diferente enquanto vista como "homem" ou como "\textbf{mulher}", e acredito que a vivência dela não elimina a sua. PROS DEMAIS, QUE FALAM QUE NINGUÉM TÁ SE INTROMETENDO NA VIDA DE NINGUÉM: imaginem uma situação bem simples, uma pessoa magra, que já é ciente sobre sua situação, saúde e tals. Agora imagina você chegando nela e falando sobre ela ser magra, citando inúmeras caralhadas de médico e bla bla bla. Meu com isso ponto é, ninguém pediu a informação, acredito que aqui não seja o caso realmente, já que é um vídeo público e aberto a isso, mas em uma situação do cotidiano, que merda vai adiantar tu falar pra pessoa sobre cromossomos? Pra mim, isso é uma conversa que apenas gera desconforto pras duas partes, é tão difícil, respeitar a moça, e chamar ela pelo pronome certo? Pq eu tenho certeza que ela já sabe sobre isso, e ela com certeza não te perguntou sobre.\normalsize   & \cellcolor{green!27}Mulher & \cellcolor{green!27}Gender - General & \cellcolor{green!27}1/173 & \cellcolor{green!27}0.578 \\  \hline
  \cellcolor{green!5}\small Desculpe mais ele nunca será uma \textbf{mulher}.\normalsize   & \cellcolor{green!5}Mulher & \cellcolor{green!5}Gender - General & \cellcolor{green!5}1/7 & \cellcolor{green!5}14.286 \\  \hline
  \cellcolor{green!27}\small Porque eu sou obrigada a aceitar ela como \textbf{mulher} na sociedade ela pode se sentir \textbf{mulher} porém nunca será, é a minha opinião não tem porque se exaltar VC fica nervosinho assim porque sabe que o q eu digo é verdade ele pode fazer 20.000 procedimentos continuará um homem.\normalsize   & \cellcolor{green!27}Mulher & \cellcolor{green!27}Gender - General & \cellcolor{green!27}2/49 & \cellcolor{green!27}4.082 \\  \hline
  \cellcolor{green!5}\small Sun flower eu respeito porém não sou obrigada a dizer que é uma \textbf{mulher} pois não é ,VC só nasce com um gênero feminino ou masculino ele nasceu homem mais quer se vestir como uma \textbf{mulher},porém seu DNA é masculino e sempre será,não importa o quanto tente,agora o engraçado é que os LGBTs querem q aceitemos eles e a opção \textbf{sexual} deles mais se outras pessoas tem uma opinião diferente da deles ficam de mimimi, porque só o jeito deles é certo ? E porque eu que biologicamente e fisicamente sou uma \textbf{mulher} tenho q aceitar q um homem se apresente como uma \textbf{mulher} quando na verdade não é.ele pode dizer o quanto quiser q é uma \textbf{mulher} mais tanto eu quanto ele sabemos que nunca será uma \textbf{mulher}. Eu estaria mentindo se eu dissesse q ele é uma \textbf{mulher} é a natureza quem fez ele homem não adianta vir aqui nos comentários dispejar sobre mim a sua insatisfação por eu não concordar q ele é uma \textbf{mulher},essa é minha opinião e não adianta VC pedir respeito se VC não sabe respeitar.\normalsize   & \cellcolor{green!5}Mulher, Sexual & \cellcolor{green!5}Gender - General & \cellcolor{green!5}9/182 & \cellcolor{green!5}4.945 \\  \hline
  \cellcolor{green!27}\small \@roberta taciano A partir do momento que você respeita, você se preocupa a não fazer aquela pessoa sentir-se humilhada com o seu discurso.Pra você ela não é \textbf{mulher}, mas e ai?Vai cair a língua por chamar ela do nome que ela preferir?\normalsize   & \cellcolor{green!27}Mulher & \cellcolor{green!27}Gender - General & \cellcolor{green!27}1/44 & \cellcolor{green!27}2.273 \\  \hline
  \cellcolor{green!5}\small Adorooo essa \textbf{mulher} , e agora eu sou inscrita nesse canal  , gostei desse canal ❤️😍👏👏👏\normalsize   & \cellcolor{green!5}Mulher & \cellcolor{green!5}Gender - General & \cellcolor{green!5}1/16 & \cellcolor{green!5}6.25 \\  \hline
  \cellcolor{green!27}\small A Thiessa é sem base de tão maravilhosa! QUE \textbf{MULHER} ❤️\normalsize   & \cellcolor{green!27}Mulher & \cellcolor{green!27}Gender - General & \cellcolor{green!27}1/11 & \cellcolor{green!27}9.091 \\  \hline
  \cellcolor{green!5}\small Pra você a única coisa que define um homem ou uma \textbf{mulher} são seus órgãos genitais? HAHAHAHAHAHAHAHA entendi então, argumento mais... pobre hahahahahahahaha\normalsize   & \cellcolor{green!5}Mulher & \cellcolor{green!5}Gender - General & \cellcolor{green!5}1/23 & \cellcolor{green!5}4.348 \\  \hline
  \cellcolor{green!27}\small Bruno Fernandes  Nem tudo que sentimos ser, é o que Realmente somos.Foi você que só citou os órgãos genitais. O homem não engravida, e a \textbf{mulher} não produz espermatozóide\normalsize   & \cellcolor{green!27}Mulher & \cellcolor{green!27}Gender - General & \cellcolor{green!27}1/30 & \cellcolor{green!27}3.333 \\  \hline
  \cellcolor{green!5}\small \@THEO AW9000 a questão é rotular esse ou aquele, ou mesmo colocar um padrão em sociedade. A famosa sociedade alinhada, homem \textbf{mulher} e filhos. Ou quando as pessoas dizem que o que Deus criou não \textbf{muda}, dentre outros argumentos meio pobres na minha opinião. Acredito que o que falte no ser humano é enxergar além da caixinha e ver o que está além.\normalsize   & \cellcolor{green!5}Muda, Mulher & \cellcolor{green!5}Gender - General, Physical Identity - Physical (and Mental) Impairments & \cellcolor{green!5}2/63 & \cellcolor{green!5}3.175 \\  \hline
  \cellcolor{green!27}\small Eu tbm sou trans nasci homen mais me sinto \textbf{mulher} esse vídeo foi muito útil pra mim que vai fazer a \textbf{transição} quando eu atingir a maioridade\normalsize   & \cellcolor{green!27}Mulher, Transição & \cellcolor{green!27}Gender - General, Sexual Identity - Transexuality & \cellcolor{green!27}2/27 & \cellcolor{green!27}7.407 \\  \hline
  \cellcolor{green!5}\small Diferença de transsexual e transgênero: a palavra transsexual era mais utilizada antigamente. A palavra transgênero é melhor porque ela especifica que a questão é em relação à identidade de gênero (se a pessoa se vê como \textbf{mulher} ou homem, independente do órgão que tem). A palavra \textbf{transexual} pode parecer ultrapassada pq tem o final \textbf{SEXUAL} nela, e a questão da pessoa não tem nada a ver com sexualidade, e sim com gÊnero. Assim como ela disse, ela pode ter nascido com pênis, e ainda assim se ver como \textbf{mulher}, mas ser lésbica (uma \textbf{mulher} trans lésbica), e gostar de \textbf{mulher}. São coisas separadas, o jeito como ela se vê e quem ela gosta de transar. Por isso TRANSGÊNERO é mais adequado.\normalsize   & \cellcolor{green!5}Mulher, Sexual, Transexual & \cellcolor{green!5}Gender - General, Sexual Identity - Transexuality & \cellcolor{green!5}6/121 & \cellcolor{green!5}4.959 \\  \hline
  \cellcolor{green!27}\small Também tenho questões com a edição, com o volume do BG e com a entrevistadora interrompendo, mas tudo isso é tão menor diante do depoimento da Thiessa que assisti tudo e nem lembrei mais destas "falhas" e fiquei emocionada tb. Na minha ignorância, tenho uma duvida que so quem entende poderá me ajudar. A pessoa trans que conquista (me ajudem com os termos!) sua passabilidade deve se apresentar em uma roda como "sou trans"?  Ou apenas pessoa que está no processo de  transcendência que pode falar? Porque entendo como a Thiessa, ninguém se apresenta falando, "olá, sou \textbf{mulher} cis!".\normalsize   & \cellcolor{green!27}Mulher & \cellcolor{green!27}Gender - General & \cellcolor{green!27}1/99 & \cellcolor{green!27}1.01 \\  \hline
  \cellcolor{green!5}\small A coisa mais chocante do vídeo é a \textbf{idade} da Thiessa. COMO ASSIM ESSA \textbf{MULHER} NÃO TEM 16 ANOS????\normalsize   & \cellcolor{green!5}Idade, Mulher & \cellcolor{green!5}Age - General, Gender - General & \cellcolor{green!5}2/19 & \cellcolor{green!5}10.526 \\  \hline
  \cellcolor{green!27}\small Geneticamente se analisarem você estará lá macho ou fêmea? Macho claro. Você \textbf{muda} sua aparência externa, mas continua sem ovários e útero, toma artificialmente hormônios continuamente pra sua voz não engrossar e sua barba, gogó não aparecerem.. Então concluindo você é um indivíduo do gênero masculino que se tornou um trânsito, porém ainda não inventaram uma mudança genética tal que transforme um homem em \textbf{mulher} ou uma \textbf{mulher} em homem. Ainda é necessário a mutilação do órgão \textbf{sexual}, e cirurgias. Portanto não é um pênis ou vagina apenas que define o gênero, o gênero está na célula, no sangue, no interior de um ser. A não aceitação da condição seja \textbf{sexual},  é outra coisa. Não vamos deixar de sermos racionais e querer impor uma meia verdade. Vamos continuar raciocinando e vendo tudo na mira da verdade\normalsize   & \cellcolor{green!27}Muda, Mulher, Sexual & \cellcolor{green!27}Gender - General, Physical Identity - Physical (and Mental) Impairments & \cellcolor{green!27}5/136 & \cellcolor{green!27}3.676 \\  \hline
  \cellcolor{green!5}\small Aliás era NÃO! Continua sendo homem porque esse é o DNA dele! O que diferencia é que sua aparência é de \textbf{mulher}! 👍\normalsize   & \cellcolor{green!5}Mulher & \cellcolor{green!5}Gender - General & \cellcolor{green!5}1/23 & \cellcolor{green!5}4.348 \\  \hline
  \cellcolor{green!27}\small Ser um homem no entender de muitos é apenas ter um pinto no meio das pernas? Pra mim... ela foi um homem, não da pra falar que isso daí é um homem amigo... aqueles velhos conceitos ultrapassados de que "E Deus fez o homem e a \textbf{mulher} --'"\normalsize   & \cellcolor{green!27}Mulher & \cellcolor{green!27}Gender - General & \cellcolor{green!27}1/48 & \cellcolor{green!27}2.083 \\  \hline
  \cellcolor{green!5}\small \@THEO AW9000 o que define um homem e uma \textbf{mulher} de verdade são apenas seus órgãos sexuais?Se a resposta for sim você precisa rever urgentemente seus conceitos amigo.\normalsize   & \cellcolor{green!5}Mulher & \cellcolor{green!5}Gender - General & \cellcolor{green!5}1/29 & \cellcolor{green!5}3.448 \\  \hline
  \cellcolor{green!27}\small Excelente vídeo,acabou de falar oque todo \textbf{gay},trans gostaria de saber transformando seu pênis em vagina,não vai deixar sua origem,parabéns pela coragem de expor sua experiência !!! Você merece muito respeito.\normalsize   & \cellcolor{green!27}Gay & \cellcolor{green!27}Sexual Identity - General & \cellcolor{green!27}1/30 & \cellcolor{green!27}3.333 \\  \hline
  \cellcolor{green!5}\small \@THEO AW9000  Pode ser, mas nesse ponto simboliza a mudança. A não aceitação do corpo que a sociedade julga como ideal, pra isso tem o acompanhamento psicológico pra entender essa complexidade. Apenas algo que não encaixa, sua mente diz uma coisa mas seu corpo diz outra diferente. O que não pode a meu ver é rotular dizendo que ainda é um homem ou é uma \textbf{mulher}. Mesma coisa Roberta Close, pra mim sempre foi uma \textbf{mulher}.\normalsize   & \cellcolor{green!5}Mulher & \cellcolor{green!5}Gender - General & \cellcolor{green!5}2/76 & \cellcolor{green!5}2.632 \\  \hline
  \cellcolor{green!27}\small Como assim ELA não sabe a diferença de transgenero e \textbf{transexual}?Transgenero: é uma pessoa que se identifica como sendo do sexo oposto, mas ainda tem o corpo biológico que não se identifica.\textbf{Transexual}: É a FASE onde um (a) transgenero já transicionou cirurgicamente e até optou ou não pela mudança do seu órgão \textbf{sexual} também, ex: Ariadna do BBB!\normalsize   & \cellcolor{green!27}Sexual, Transexual & \cellcolor{green!27}Gender - General, Sexual Identity - Transexuality & \cellcolor{green!27}2/60 & \cellcolor{green!27}3.333 \\  \hline
  \cellcolor{green!5}\small Não se nasce \textbf{mulher}, torna-se \textbf{mulher}, euzinha nasci com meu sexo biologico que me indentifico mas eu me TORNEI mulher\normalsize   & \cellcolor{green!5}Mulher & \cellcolor{green!5}Gender - General & \cellcolor{green!5}3/20 & \cellcolor{green!5}15.0 \\  \hline
  \cellcolor{green!27}\small tira esse reboco da tua cara viado nojento que todos vão ver que tu é um cara \textbf{baitola}, vai se entregar a jesus !! ganhou um visualização, mas ganhou um deslike !!\normalsize   & \cellcolor{green!27}Baitola & \cellcolor{green!27}Sexual Identity - Male homosexuality & \cellcolor{green!27}1/32 & \cellcolor{green!27}3.125 \\  \hline
  \cellcolor{green!5}\small Você frequenta a igreja ofendendo as pessoas desse jeito? ainda tem a audácia de pronunciar o nome de Jesus? garanto que ele seria o primeiro a te expulsar da casa dele. Jesus jamais teria essa atitude podre que você está tendo. É por isso que pessoas de bem de verdade cada vez menos se aproximam da SUA igreja! Só pra você saber... O que acha que os reis nos castelos faziam por manter os eunucos de guarda? tinham relações só com mulheres HAHAHAHAHAHAHAHAHA, vou mais longe, na bíblia mesmo basta procurar, casal \textbf{gay} Jonathas e Davi uma linda história de amor :)\normalsize   & \cellcolor{green!5}Gay & \cellcolor{green!5}Sexual Identity - General & \cellcolor{green!5}1/102 & \cellcolor{green!5}0.98 \\  \hline
  \cellcolor{green!27}\small Sou evangélica, \textbf{mulher} hétero e amo meu próximo. Não importa o pecado do outro, eu não vim a esse mundo pra julgar. Eu procuro transmitir muito amor aos homossexuais que eu conheço pois sei que essas pessoas são muito feridas diariamente. A minha missão na Terra não é julgar mas sim amar. Beijos Thiessa! Deus te abençoe grandemente 💖\normalsize   & \cellcolor{green!27}Mulher & \cellcolor{green!27}Gender - General & \cellcolor{green!27}1/59 & \cellcolor{green!27}1.695 \\  \hline
  \cellcolor{green!5}\small Caraca q \textbf{mulher} maravilhosa 😍😍!! Ameeeeeei!!!\normalsize   & \cellcolor{green!5}Mulher & \cellcolor{green!5}Gender - General & \cellcolor{green!5}1/6 & \cellcolor{green!5}16.667 \\  \hline
  \cellcolor{green!27}\small Vc nasceu com útero? Se nasceu vc é \textbf{mulher}!! E ponto. Isso é genética, biologia, tenho certeza que todos aqui frequentaram aulas de biologia e sabem ao menos o mínimo sobre isso. XX e XY. Agora se nasceu com pênis mas se identifica com menina, tudo bem, cada faz o que quer da vida! Nada contra, o que importa é se te faz feliz. Quer se parecer com \textbf{mulher}!? Ok. Mas se não nasceu com útero, não nasceu \textbf{mulher}. Essa mudança é uma opção...Vem cá, 16 anos de PT e ainda vcs são assassinados!!??? Jura?! Como pode isso? PT e a esquerda não fez nada concreto por vcs??? Fala sério!!Se vc fica com um cara ai o cara fica todo apaixonado e depois vc conta que é trans, isso não é preconceito por parte dele. Vcs precisam respeitar o direito das pessoas que terem suas opções sexuais! Vcs não podem obrigar um cara hetero a se relacionar com um cara só pq ele se parece \textbf{mulher}. É opção da pessoa. Vcs não podem impor que a pessoa aceite e queira continuar um relacionamento. Pra serem respeitados precisam respeitar as opções das pessoas de não quererem se relacionar com vcs! A opção \textbf{sexual} sua é sua!! E a dele é dele!! Não queira dizer que é preconceito. Pense... sua mudança de sexo pode ser vista como um preconceito com o sexo que nasceu e não aceita. Já pensou nisso??? Fica a reflexão... e mais respeito pela opção \textbf{sexual} dos heteros! Não tenho absolutamente nada contra trans, tenho amigos gays... e os adoro! Só não imponham ou rotulem de preconceito, rejeição e a opção \textbf{sexual} de um hetero. Sejam felizes.\normalsize   & \cellcolor{green!27}Mulher, Sexual & \cellcolor{green!27}Gender - General & \cellcolor{green!27}7/280 & \cellcolor{green!27}2.5 \\  \hline
  \cellcolor{green!5}\small Pra mim ela é uma \textbf{mulher}, esse conceito de que "não nasceu com o órgão então é como está" é um argumento bem ultrapassado pelo menos pra mim. Muita gente acha que ser homem é apenas ter um pinto no meio das pernas, e ser \textbf{mulher} é ter uma vagina no meio das pernas e ter nascido com útero. Acredito que tanto um homem quanto \textbf{mulher} de verdade para serem denominados assim tem que ter caráter e saber respeitar as escolhas de cada um independente de quais sejam. Aquele \textbf{velho} preconceito disfarçado de opinião: "não tenho preconceitos, tenho até amigos assim, mas não concordo, não acho certo e é contra a natureza". Miga... hahahahahaha algo errado não está certo em sua afirmação.e apenas uma correção, não se chama OPÇÃO \textbf{sexual}, se chama ORIENTAÇÃO \textbf{sexual}, pois eu sou \textbf{gay} e não escolhi sofrer preconceito por aí, não escolhi ser quem sou, tentei mudar várias vezes, até na igreja já mandaram eu ir. Tentei até suicídio uma época por não me aceitar. Acredito que ela não teria OPTADO por esse caminho sabendo que: a sociedade viraria as costas, pessoas a rejeitariam, família, amigos se afastariam e a julgariam. Tudo isso que ela contou é triste! e mais triste ainda é saber que mesmo após tudo isso, as pessoas não enxergam que o preconceito parece não ter fim para a comunidade \textbf{LGBT}. Meu...\normalsize   & \cellcolor{green!5}Gay, LGBT, Mulher, Sexual, Velho & \cellcolor{green!5}Age - Over 65s, Gender - General, Sexual Identity - General & \cellcolor{green!5}8/231 & \cellcolor{green!5}3.463 \\  \hline
  \cellcolor{green!27}\small Eu acho que tem que avisar, \textbf{mulher} cis não avisa pq nasceu \textbf{mulher}, vc não nasceu, então eu acho que vc tem q avisar sim, pq talvez o cara não queira ou não goste, claro te respeitando, mas eu contaria se fosse. É melhor vc se liberta de uns lixos, se gosta de vc te aceita.\normalsize   & \cellcolor{green!27}Mulher & \cellcolor{green!27}Gender - General & \cellcolor{green!27}2/56 & \cellcolor{green!27}3.571 \\  \hline
  \cellcolor{green!5}\small Eu amo demais essa \textbf{mulher} e me inspiro muito nela. Eu tenho 32 anos e vivo o dilema de tentar descobrir quem eu sou de fato, pois tenho pensamentos característicos de uma pessoa trans. E na  questão da auto cobrança, eu iniciei 2 transições e parei logo no início por achar que eu não era "perfeita" o bastante para me assumir trans. Estou em cima do muro e não sei quando esse sofrimento vai acabar!😢😢😢\normalsize   & \cellcolor{green!5}Mulher & \cellcolor{green!5}Gender - General & \cellcolor{green!5}1/75 & \cellcolor{green!5}1.333 \\  \hline
  \cellcolor{green!27}\small 'Brasil o país que mais mata \textbf{lgbt} do mundo !!'  [LINK]  Estranho... cadê o Brasil nessa lista? Tem muito \textbf{gay} sim que se faz se vítima falando de perseguição e afins, aí vai ver a pessoa vive num ambiente saudável, com pais que entendem a pessoa e dão suporte, amigos bons e afins... preconceito SEMPRE vai ter uma coisa que muita gente não entende... o jeito é saber lidar com isso e continuar seguindo em frente\normalsize   & \cellcolor{green!27}Gay, LGBT & \cellcolor{green!27}Sexual Identity - General & \cellcolor{green!27}2/76 & \cellcolor{green!27}2.632 \\  \hline
  \cellcolor{green!5}\small \@Rienzi Bartolomeu Ser \textbf{Gay}, não é ser Trans. Começa por esse fato !\normalsize   & \cellcolor{green!5}Gay & \cellcolor{green!5}Sexual Identity - General & \cellcolor{green!5}1/13 & \cellcolor{green!5}7.692 \\  \hline
  \cellcolor{green!27}\small Lucas Gilli Por isso eu disse \textbf{GAY} e não trans\normalsize   & \cellcolor{green!27}Gay & \cellcolor{green!27}Sexual Identity - General & \cellcolor{green!27}1/10 & \cellcolor{green!27}10.0 \\  \hline
  \cellcolor{green!5}\small Cara, cai na realidade, OBVIO que a grande maioria das Trans, são assassinadas pelo simples fatos de serem trans. Imagina se ser \textbf{gay} já te coloca em risco de sair na rua e alguém querer te espancar, imagina um transsexual ? Isso é fato , o brasileiro ainda é Transfóbico e muito !\normalsize   & \cellcolor{green!5}Gay & \cellcolor{green!5}Sexual Identity - General & \cellcolor{green!5}1/53 & \cellcolor{green!5}1.887 \\  \hline
  \cellcolor{green!27}\small Lucas Gilli Ser \textbf{gay} te coloca em perigo dependendo do lugar onde você reside, não dá pra generalizar, não posso dizer o mesmo pra quem é trans.\normalsize   & \cellcolor{green!27}Gay & \cellcolor{green!27}Sexual Identity - General & \cellcolor{green!27}1/27 & \cellcolor{green!27}3.704 \\  \hline
  \cellcolor{green!5}\small \@Rienzi Bartolomeu Mas exatamente sobre isso que falei, se ser \textbf{gay} já é dificil, imagina ser trans, tipo, andar na rua sendo trans já pode ser motivo de agressão, mas enfim, acredito que estamos caminhando para um mundo mais respeitoso e também sei que o Brasil não é pais que mais mata trans no mundo .\normalsize   & \cellcolor{green!5}Gay & \cellcolor{green!5}Sexual Identity - General & \cellcolor{green!5}1/56 & \cellcolor{green!5}1.786 \\  \hline
  \cellcolor{green!27}\small O Brasil é o país mais transfóbico do mundo. Além de ser o país que mais mata \textbf{LGBT}+ no mundo. A expectativa é que a cada 27 horas, um \textbf{LGBT} é assassinado. Você não é do movimento, não venha problematizar isso. Se informe antes. O próprio presidente do país é \textbf{homofóbico}, acho difícil esse problema ser resolvido... ou qualquer outro, já que aquela ameba não faz merda nenhuma.\normalsize   & \cellcolor{green!27}LGBT, homofóbico & \cellcolor{green!27}Sexual Identity - General & \cellcolor{green!27}3/68 & \cellcolor{green!27}4.412 \\  \hline
  \cellcolor{green!5}\small \@Rienzi Bartolomeu Nego, não entendi esta sua exaltação sobre algo que é um fato. Se você quer acreditar nisso... o problema não é meu. Você é \textbf{gay}? Faz parte do movimento LGBTQA? Pesquise sobre essas estatísticas mais aprofundado, pois vejo que você vive numa bolha. Eu estou no meu lugar de fala, faço parte do movimento, esse tema é algo que faz parte da minha vida, e não da sua! Por esse fator, eu falo com total convicção. Não perca seu tempo tentando idealizar algo que não é concreto.\normalsize   & \cellcolor{green!5}Gay & \cellcolor{green!5}Sexual Identity - General & \cellcolor{green!5}1/89 & \cellcolor{green!5}1.124 \\  \hline
  \cellcolor{green!27}\small Tonny Santos Eu sou \textbf{gay} e mesmo se eu não fosse eu posso falar sobre sim, eu hein. Eu pesquisei já e só achei dados antigos, quem vive numa bolha é você aparentemente por achar que todo \textbf{LGBT} do Brasil é perseguido e afins\normalsize   & \cellcolor{green!27}Gay, LGBT & \cellcolor{green!27}Sexual Identity - General & \cellcolor{green!27}2/44 & \cellcolor{green!27}4.545 \\  \hline
  \cellcolor{green!5}\small \@Rienzi Bartolomeu "queria ver se vocês aguentariam uma semana..." 😂 garoto, tu é muito alienado. Eu sou \textbf{Gay} Não-Binário, eu resisto há ANOS. Essa é a minha realidade, não a sua. Para, porfavor, que tá feio.\normalsize   & \cellcolor{green!5}Gay & \cellcolor{green!5}Sexual Identity - General & \cellcolor{green!5}1/36 & \cellcolor{green!5}2.778 \\  \hline
  \cellcolor{green!27}\small \@Rienzi Bartolomeu Você é um \textbf{gay} de taubaté, pelo visto, ou ainda não saiu do armário, pois pra tá dizendo essas coisas... desculpe mas não acredito em você.\normalsize   & \cellcolor{green!27}Gay & \cellcolor{green!27}Sexual Identity - General & \cellcolor{green!27}1/28 & \cellcolor{green!27}3.571 \\  \hline
  \cellcolor{green!5}\small Tonny Santos \textbf{Gay} de Taubaté? Você nem me conhece cara, e sim sou assumido sim e meus pais me amam e me apoiam e é disso que eu preciso. Sinto muito se você não tem o mesmo.\normalsize   & \cellcolor{green!5}Gay & \cellcolor{green!5}Sexual Identity - General & \cellcolor{green!5}1/37 & \cellcolor{green!5}2.703 \\  \hline
  \cellcolor{green!27}\small liiparks Bom, alguma coisa mudou, em dois anos coisas mudam sim se pra você não \textbf{muda} ai o problema tá com você mesmo\normalsize   & \cellcolor{green!27}Muda & \cellcolor{green!27}Physical Identity - Physical (and Mental) Impairments & \cellcolor{green!27}1/23 & \cellcolor{green!27}4.348 \\  \hline
  \cellcolor{green!5}\small Lindaaaa Thiessaa😍 que você tenha todo o sucesso e toda boa sorte do mundo🍀 que \textbf{mulher}!!!!🌹\normalsize   & \cellcolor{green!5}Mulher & \cellcolor{green!5}Gender - General & \cellcolor{green!5}1/16 & \cellcolor{green!5}6.25 \\  \hline
  \cellcolor{green!27}\small uma pequena observação sobre a diferença (existente, diga-se de passagem) entre transgênero e \textbf{transexual}\textbf{transexual} trata-se de alguém com disforia em relação à genitália e ao gênero que lhe foi designado, e transgênero não necessariamente será alguém com disforia genital.(por sinal, se eu estiver errada, podem me corrigir)\normalsize   & \cellcolor{green!27}Transexual & \cellcolor{green!27}Sexual Identity - Transexuality & \cellcolor{green!27}1/49 & \cellcolor{green!27}2.041 \\  \hline
  \cellcolor{green!5}\small Meu.. vi vídeos dessa menina novinhaaaa. Tô chocada na \textbf{idade} que é minha \textbf{idade}.. kkkkk\normalsize   & \cellcolor{green!5}Idade & \cellcolor{green!5}Age - General & \cellcolor{green!5}2/15 & \cellcolor{green!5}13.333 \\  \hline
  \cellcolor{green!27}\small Que menina forte! Te desejo tempos mais suaves, querida!Vc é uma \textbf{mulher} linda!\normalsize   & \cellcolor{green!27}Mulher & \cellcolor{green!27}Gender - General & \cellcolor{green!27}1/14 & \cellcolor{green!27}7.143 \\  \hline
  \cellcolor{green!5}\small Na real: Enquanto você tiver um cromossomo Y nas células do corpo, você é homem e não poderá fazer nada para mudar isso, você pode mudar seu corpo, pode brincar de faz de conta, pode criar leis que obriguem as pessoas a fazerem de conta que você é \textbf{mulher}, mas você ainda será homem.\normalsize   & \cellcolor{green!5}Mulher & \cellcolor{green!5}Gender - General & \cellcolor{green!5}1/54 & \cellcolor{green!5}1.852 \\  \hline
  \cellcolor{green!27}\small \@Lucas Gilli teu dircurso de ódio contra héteros não fará um homem virar mulher\normalsize   & \cellcolor{green!27}Mulher & \cellcolor{green!27}Gender - General & \cellcolor{green!27}1/14 & \cellcolor{green!27}7.143 \\  \hline
  \cellcolor{green!5}\small Te fazer de doido não vai transformar um homem em \textbf{mulher}, nem anular teu ódio por héteros!\normalsize   & \cellcolor{green!5}Mulher & \cellcolor{green!5}Gender - General & \cellcolor{green!5}1/17 & \cellcolor{green!5}5.882 \\  \hline
  \cellcolor{green!27}\small Desconfiei que tu era chegado em pau mesmo hehehehe, vou mandar o kid bengala no meu lugar porque meu negocio é com mulher\normalsize   & \cellcolor{green!27}Mulher & \cellcolor{green!27}Gender - General & \cellcolor{green!27}1/23 & \cellcolor{green!27}4.348 \\  \hline
  \cellcolor{green!5}\small Não adianta fazer cirurgia. Você não tem uma vagina .Você é apenas um homem mutilado. Nunca será \textbf{mulher}. Não tem útero. Tem próstata. Aceita que dói menos\normalsize   & \cellcolor{green!5}Mulher & \cellcolor{green!5}Gender - General & \cellcolor{green!5}1/27 & \cellcolor{green!5}3.704 \\  \hline
  \cellcolor{green!27}\small \@Kelli Martins pq vc nao vira presidente da republica pra falar isso, linda? Você é Deus pra decidir quem é homem e quem é \textbf{mulher}? Eu hein... Quero saber onde foi que disseram que Kelli Martins era especialista em alguma coisa, e levaram a opinião dela em conta.Mais uma coisa: IDENTIDADE DE GÊNERO NÃO TEM RELAÇÃO COM ORGÃO \textbf{SEXUAL}. PAZ\normalsize   & \cellcolor{green!27}Mulher, Sexual & \cellcolor{green!27}Gender - General & \cellcolor{green!27}2/61 & \cellcolor{green!27}3.279 \\  \hline
  \cellcolor{green!5}\small Laerte Júnior. Deus fez homem e mulher\normalsize   & \cellcolor{green!5}Mulher & \cellcolor{green!5}Gender - General & \cellcolor{green!5}1/7 & \cellcolor{green!5}14.286 \\  \hline
  \cellcolor{green!27}\small Laerte Júnior. Essa é a sua opinião que não corresponde a da maioria. A sociedade brasileira é a favor dos valores cristãos e da família. Deus fez homem e \textbf{mulher}.\normalsize   & \cellcolor{green!27}Mulher & \cellcolor{green!27}Gender - General & \cellcolor{green!27}1/30 & \cellcolor{green!27}3.333 \\  \hline
  \cellcolor{green!5}\small Katiuscia Trindade ninguém escolhe ser \textbf{LGBT}+. A gente nasce assim e vai se descobrindo com o tempo... Fora isso seu textinho tá uma gracinha😌💕\normalsize   & \cellcolor{green!5}LGBT & \cellcolor{green!5}Sexual Identity - General & \cellcolor{green!5}1/24 & \cellcolor{green!5}4.167 \\  \hline
  \cellcolor{green!27}\small Fala pra que é \textbf{mulher} o dia q vc menstruar e gerar uma vida dentro do útero. Viado eunuco!\normalsize   & \cellcolor{green!27}Mulher & \cellcolor{green!27}Gender - General & \cellcolor{green!27}1/19 & \cellcolor{green!27}5.263 \\  \hline
  \cellcolor{green!5}\small \@Sad Doug é comigo? O cidadão em questão arrancou a pica , nao tem útero,  nao tem ciclo mestrual e usa de métodos cientificos pra parecer \textbf{mulher}. Mas é homem.to errado?\normalsize   & \cellcolor{green!5}Mulher & \cellcolor{green!5}Gender - General & \cellcolor{green!5}1/31 & \cellcolor{green!5}3.226 \\  \hline
  \cellcolor{green!27}\small \@Iuri Barros esse mimimi atingiu a todos nós no brasil, nao percebeu ou vc é \textbf{idiota} msm?\normalsize   & \cellcolor{green!27}Idiota & \cellcolor{green!27}Physical Identity - Physical (and Mental) Impairments & \cellcolor{green!27}1/17 & \cellcolor{green!27}5.882 \\  \hline
  \cellcolor{green!5}\small \@Aramat Bertran Atingiu de que forma \textbf{idiota}? Me explica aí então, se é que você tem algo coerente pra falar e tal né?Mas sou todo à ouvidos.. ;)\normalsize   & \cellcolor{green!5}Idiota & \cellcolor{green!5}Physical Identity - Physical (and Mental) Impairments & \cellcolor{green!5}1/29 & \cellcolor{green!5}3.448 \\  \hline
  \cellcolor{green!27}\small \@Iuri Barros desde de que foram criados essas ONGs lgbts e tal , iniciou-se esse vitimismo , a prepotência e a falta de respeito da grande maioria de adeptos ao homossexualismo. isso atinge a população dado o fato das informacoes hj dia serem muito mais acessíveis. homem é homem, \textbf{mulher} é \textbf{mulher} e \textbf{gay} é \textbf{gay}. e acabou amigo. Trans é o caralho"Nasci com alma feminina"  hahaha faça me o favor né? É viado e acabou irmão, tem preferência \textbf{sexual} pelo msm sexo. Legal, respeito. mas mas parem de vitimizar e inventar nomenclaturas para uma coisa que toda a humanidade conhece como homossexualismo apenas. Veja o relato deste rapaz ai e veja o quanto foi frustante e decepcionante pro pai dele. antes de mais nada, gays sao doentes mentais. Minha opinião, me desculpe\normalsize   & \cellcolor{green!27}Gay, Mulher, Sexual & \cellcolor{green!27}Gender - General, Sexual Identity - General & \cellcolor{green!27}5/134 & \cellcolor{green!27}3.731 \\  \hline
  \cellcolor{green!5}\small Aramat Bertran Aramat Bertran Primeiramente, gays são homens pas\textbf{Mulher} não é definida por ciclo menstrual ou ter filhos, porque se fosse desse jeito uma \textbf{mulher} estéril seria o que? \textbf{Puta} pensamento de merda.Meu amigo pelo o que eu vejo quem é doente mental é você, mas né rs cada um com seus ideais... mas pelo o amor quem caga em paz com um pensamento desse? O cara não conta azulejo conta quantos pais de gays, trans, a comunidade \textbf{Lgbt} em si, se frustraram com a sexualidade da tua cria. Pqp né se fosse um pai que amasse seu filho de verdade estaria pouco se fodendo pra isso e não estaria perdendo tempo falando merda e sim abraçando seu filho. Vai tomar um achocolatado assistindo Hora de aventura que tu aprende mais :v\normalsize   & \cellcolor{green!5}LGBT, Mulher, Puta & \cellcolor{green!5}Gender - Female sexuality, Gender - General, Sexual Identity - General & \cellcolor{green!5}4/134 & \cellcolor{green!5}2.985 \\  \hline
  \cellcolor{green!27}\small Chorei litros, que aula a Thiessa deu, quanto orgulho sinto dessa \textbf{mulher}.\normalsize   & \cellcolor{green!27}Mulher & \cellcolor{green!27}Gender - General & \cellcolor{green!27}1/12 & \cellcolor{green!27}8.333 \\  \hline
  \cellcolor{green!5}\small Thiessa, que \textbf{mulher} maravilhosa!!!!\normalsize   & \cellcolor{green!5}Mulher & \cellcolor{green!5}Gender - General & \cellcolor{green!5}1/4 & \cellcolor{green!5}25.0 \\  \hline
  \cellcolor{green!27}\small Thiessa , vc pode se aceitar como \textbf{mulher}, mas a sociedade não tem a obrigação de aceitar  Vc como tal. A sociedade tem a obrigação de te respeitar. Sou \textbf{gay} , mas nasceu com pipi é homem , nasceu com xoxota é  \textbf{mulher} fato .(minha opinião)\normalsize   & \cellcolor{green!27}Gay, Mulher & \cellcolor{green!27}Gender - General, Sexual Identity - General & \cellcolor{green!27}3/46 & \cellcolor{green!27}6.522 \\  \hline
  \cellcolor{green!5}\small Respeito sua opinião como sendo \textbf{gay} também mas acho um argumento muito pobre definir as pessoas apenas por seus órgãos sexuais, tanto homem e \textbf{mulher} de verdade tem que ter caráter, isso é o que define seus gêneros não apenas seus órgãos (minha opinião)\normalsize   & \cellcolor{green!5}Gay, Mulher & \cellcolor{green!5}Gender - General, Sexual Identity - General & \cellcolor{green!5}2/44 & \cellcolor{green!5}4.545 \\  \hline
  \cellcolor{green!27}\small Que pena e pesar para comunidade \textbf{LGBT} ter pessoas que pensam assim como vc, mas da tempo de estudar ainda amigo, desconstruir a ideologia de que um pênis ou uma vagina podem definir seus sentimentos.\normalsize   & \cellcolor{green!27}LGBT & \cellcolor{green!27}Sexual Identity - General & \cellcolor{green!27}1/35 & \cellcolor{green!27}2.857 \\  \hline
  \cellcolor{green!5}\small Mesma linha de raciocínio de quem te condena por ser \textbf{gay}. Nasceu homem tem que gostar de  \textbf{mulher} e vice-versa. Sugiro refletir e pesquisar sobre os debates de gênero.\normalsize   & \cellcolor{green!5}Gay, Mulher & \cellcolor{green!5}Gender - General, Sexual Identity - General & \cellcolor{green!5}2/29 & \cellcolor{green!5}6.897 \\  \hline
  \cellcolor{green!27}\small ignorante apoiar o mesmo tipo de raciocínio que crucifica os gays. tenha mais humanidade. se você fosse 'normal', gostaria de mulher\normalsize   & \cellcolor{green!27}Mulher & \cellcolor{green!27}Gender - General & \cellcolor{green!27}1/21 & \cellcolor{green!27}4.762 \\  \hline
  \cellcolor{green!5}\small \@Sun Flower sim sofro preconceito, mas por incrível que pareça o preconceito vem do próprio meio \textbf{LGBT} kyz.\normalsize   & \cellcolor{green!5}LGBT & \cellcolor{green!5}Sexual Identity - General & \cellcolor{green!5}1/18 & \cellcolor{green!5}5.556 \\  \hline
  \cellcolor{green!27}\small \@Bruno Fernandes claro com caráter fato , mas só existe 2 sexos masculino e feminino. O 3 é \textbf{aberração}.\normalsize   & \cellcolor{green!27}Aberração & \cellcolor{green!27}Sexual Identity - Male homosexuality & \cellcolor{green!27}1/19 & \cellcolor{green!27}5.263 \\  \hline
  \cellcolor{green!5}\small \@Lucas Gilli só conheço 2 sexos o masculino e o feminino, o 3 é \textbf{aberração}.\normalsize   & \cellcolor{green!5}Aberração & \cellcolor{green!5}Sexual Identity - Male homosexuality & \cellcolor{green!5}1/15 & \cellcolor{green!5}6.667 \\  \hline
  \cellcolor{green!27}\small \@Gabriel Cailleaux ser \textbf{gay} não vejo nada de "anormal" , mas querer ser uma \textbf{mulher} sendo que nasceu homem , aí já é um absurdo. Aceitar isso jamais, porem temos que ter a paciência de respeitar né ? Fazer o que.\normalsize   & \cellcolor{green!27}Gay, Mulher & \cellcolor{green!27}Gender - General, Sexual Identity - General & \cellcolor{green!27}2/41 & \cellcolor{green!27}4.878 \\  \hline
  \cellcolor{green!5}\small \@Diego Berry acho que você precisa rever urgentemente seus conceitos. \textbf{Aberração}?Em pleno século XXI a gente é obrigado a ouvir tamanha asneira, os anos passam mas as opiniões são tão antigas ainda...Mas ok. Sua opinião não vai mudar. E menos ainda a minha. Apenas pense um pouco, tenha empatia e leia mais livros a respeito do assunto. Você pela foto é um rapaz bonito, inteligente creio. Seria interessante você se aprofundar no assunto e entender um pouco mais. Conversar com alguém do meio... enfim...Empatia, amor, caráter e menos julgamentos é o que precisamos nesses tempos sombrios\normalsize   & \cellcolor{green!5}Aberração & \cellcolor{green!5}Sexual Identity - Male homosexuality & \cellcolor{green!5}1/99 & \cellcolor{green!5}1.01 \\  \hline
  \cellcolor{green!27}\small \@Diego Berry E sua forma de pensar é uma \textbf{aberração} ! Parabéns , \textbf{aberração} !\normalsize   & \cellcolor{green!27}Aberração & \cellcolor{green!27}Sexual Identity - Male homosexuality & \cellcolor{green!27}2/15 & \cellcolor{green!27}13.333 \\  \hline
  \cellcolor{green!5}\small A sociedade não também aceita \textbf{gay} kkk dois homens faz nada .\normalsize   & \cellcolor{green!5}Gay & \cellcolor{green!5}Sexual Identity - General & \cellcolor{green!5}1/12 & \cellcolor{green!5}8.333 \\  \hline
  \cellcolor{green!27}\small Vamos lá aos meus comentários sobre ela, descobrir a pouco tempo, um vídeo dela caiu de paraquedas no meu canal. Vi muitos vídeos e muitas fotos, na real não conseguia ficar que é era um homem. Muito \textbf{mulher} eu nunca iria desconfiar ... To passada até agora! E 28 anos, quando tu estiver nos 40 meu amorrrrr vai tá lá em cima!\normalsize   & \cellcolor{green!27}Mulher & \cellcolor{green!27}Gender - General & \cellcolor{green!27}1/62 & \cellcolor{green!27}1.613 \\  \hline
  \cellcolor{green!5}\small coitado dos caras, eles não estão errados, ele tem o direito de não gostar, ela não tem utero, tem diferensa sim, não é preconceito, o cara quer uma \textbf{mulher} pra forma própria família.\normalsize   & \cellcolor{green!5}Mulher & \cellcolor{green!5}Gender - General & \cellcolor{green!5}1/33 & \cellcolor{green!5}3.03 \\  \hline
  \cellcolor{green!27}\small Ta mas quando a \textbf{mulher} que nasce \textbf{mulher} e se indentifica como \textbf{mulher} mas nao pode ter filho fica o questionamento não precisa responder so pensa nisso\normalsize   & \cellcolor{green!27}Mulher & \cellcolor{green!27}Gender - General & \cellcolor{green!27}3/27 & \cellcolor{green!27}11.111 \\  \hline
  \cellcolor{green!5}\small mas \textbf{mulher} trans não é uma \textbf{mulher}, é uma \textbf{mulher} trans, uma categoria a parte, o meu corpo de \textbf{mulher} não é igual ao corpo de uma \textbf{mulher} trans, porem somos tipo de mulher\normalsize   & \cellcolor{green!5}Mulher & \cellcolor{green!5}Gender - General & \cellcolor{green!5}6/34 & \cellcolor{green!5}17.647 \\  \hline
  \cellcolor{green!27}\small É pq fora é Verde e dentro \textbf{a\textbf{marelo}}, aí dá para ver um pouco da cor dentro.\normalsize   & \cellcolor{green!27}Amarelo & \cellcolor{green!27}Nationality - Chinese, Nationality - Japanese & \cellcolor{green!27}2/17 & \cellcolor{green!27}11.765 \\  \hline
  \cellcolor{green!5}\small Pra mim é \textbf{A\textbf{marelo}} × Azul Piscina\normalsize   & \cellcolor{green!5}Amarelo & \cellcolor{green!5}Nationality - Chinese, Nationality - Japanese & \cellcolor{green!5}2/7 & \cellcolor{green!5}28.571 \\  \hline
  \cellcolor{green!27}\small Com certeza a cirurgia não faz ninguém mais \textbf{mulher},até pq SER \textbf{MULHER} é uma questão biológica e não sentimental.Cada um sente  o que quiser dentro de si mesmo.\normalsize   & \cellcolor{green!27}Mulher & \cellcolor{green!27}Gender - General & \cellcolor{green!27}2/28 & \cellcolor{green!27}7.143 \\  \hline
  \cellcolor{green!5}\small Ideologia de gênero é a concepção de que a identidade de gênero de um indivíduo é uma construção social. Essa teoria afirma que o nosso gênero é um produto sociocultural. Exemplo: se uma criança for criada dentro do espectro da 'feminilidade', esta será \textbf{mulher}, independente da genital com que tenha nascido. Em nenhum momento, esse pensamento foi pontuado no vídeo, justamente pq ideologia de gênero n tem nada a ver com transexualidade. N sabe nem oq significa o termo e quer usar pra tentar validar sua transfobia e ignorância. É uma acéfala mesmo.\normalsize   & \cellcolor{green!5}Mulher & \cellcolor{green!5}Gender - General & \cellcolor{green!5}1/93 & \cellcolor{green!5}1.075 \\  \hline
  \cellcolor{green!27}\small Fico imaginando se a Natalia aqui sofre com o preconceito na pele, recebe ameaças de mortes constantemente (medo de morrer em assalto não é ameaça de morte), ser espancada na rua a noite, ser negada acesso em qualquer ambiente fora da bolha \textbf{LGBT}, não conseguir emprego, ser tratada como \textbf{aberração} pelos outros. Imagino que seja uma sofredora mas que não se "vitimiza", uma inspiração... Ah, a próxima vez que puder decidir entre falar merda e ser retardada ou pesquisar sobre o assunto para pelo menos ter noção de alguma coisa, escolha a segunda opção! :)\normalsize   & \cellcolor{green!27}Aberração, LGBT & \cellcolor{green!27}Sexual Identity - General, Sexual Identity - Male homosexuality & \cellcolor{green!27}2/95 & \cellcolor{green!27}2.105 \\  \hline
  \cellcolor{green!5}\small 11:45 expor um cara que prefere \textbf{mulher} cis? É obrigado a aceitar uma \textbf{mulher} que não vai poder dar filhos biológicos pra ele?Escrota.E dentro do feminismo tem \textbf{mulher} transfobica.\normalsize   & \cellcolor{green!5}Mulher & \cellcolor{green!5}Gender - General & \cellcolor{green!5}3/31 & \cellcolor{green!5}9.677 \\  \hline
  \cellcolor{green!27}\small Bom, vc sai à noite nos barzinhos ou restaurantes vc não encontra cadeirantes, mas encontra gays , trans . Vc vai à praia vc não encontra um grande queimado ( queimadura por acidente), mas vc encontra gays e trans. Vc vai ao shopping vc não encontra um número notório de amputados , mas vc encontra gays e trans . A dor é para todos os que são ' diferentes' , a dor de vcs não é maior do que essas , tenho certeza absoluta. Não estou sendo preconceituosa, pelo Amor de Deus.. só estou mostrando que têm dor muuuuuito maior. Parem de chorar! Chega! Foi uma entrevista dramática. E olha que eu gosto dessa menina. Ela é linda e perfeita. Que sortuda! É assim que uma \textbf{mulher} que têm o rosto deformado por queimadura ( e existem muitas) adoraria ser. Já trabalhei com grandes queimados por isso a minha indignação. Por quê ninguém fala deles?\normalsize   & \cellcolor{green!27}Mulher & \cellcolor{green!27}Gender - General & \cellcolor{green!27}1/155 & \cellcolor{green!27}0.645 \\  \hline
  \cellcolor{green!5}\small THIESSA É MUITO O AMOR DA MINHA VIDA \textbf{PUTA} QUE ME PARIU, af eu amo uma \textbf{mulher} ✊🏽❤️\normalsize   & \cellcolor{green!5}Mulher, Puta & \cellcolor{green!5}Gender - Female sexuality, Gender - General & \cellcolor{green!5}2/18 & \cellcolor{green!5}11.111 \\  \hline
  \cellcolor{green!27}\small Thyessa é tão linda e tão sensível. Te acompanho desde o começo do canal e sei o quanto é importante falar tudo isso. Fiz um trabalho na faculdade sobre inclusão. Enquanto a maioria pensava nos alunos com dificuldade de aprendizagem, síndromes e transtornos, me sensibilizei com a sua história, e de muitos outros canais que abriram minha mente em busca do conhecimento. Vamos pesquisar antes de julgar ou afirmar qualquer coisa. Sucesso em tudo que fizer. Saiba que você nunca estará sozinha. É \textbf{mulher}, feminista e uma vitoriosa! Valeu, mana! Tamo junto! Ngm solta! 😘💚\normalsize   & \cellcolor{green!27}Mulher & \cellcolor{green!27}Gender - General & \cellcolor{green!27}1/95 & \cellcolor{green!27}1.053 \\  \hline
  \cellcolor{green!5}\small Thi aaa que hino de mulher\normalsize   & \cellcolor{green!5}Mulher & \cellcolor{green!5}Gender - General & \cellcolor{green!5}1/6 & \cellcolor{green!5}16.667 \\  \hline
  \cellcolor{green!27}\small Não entendi nada ...isso é \textbf{mulher} ou homem\normalsize   & \cellcolor{green!27}Mulher & \cellcolor{green!27}Gender - General & \cellcolor{green!27}1/8 & \cellcolor{green!27}12.5 \\  \hline
  \cellcolor{green!5}\small É um ser humano! Uma \textbf{mulher} trans! Bjs\normalsize   & \cellcolor{green!5}Mulher & \cellcolor{green!5}Gender - General & \cellcolor{green!5}1/8 & \cellcolor{green!5}12.5 \\  \hline
  \cellcolor{green!27}\small Thiessa é uma \textbf{mulher} muito forte supera todas as barreiras é um exemplo de superação e tem uma maturidade tão grande ela se emociona de ver o tanto de coisas que ela já superou e de ter enfrentado  tanto  preconceito infelizmente tem muitas pessoas que são preconceituoso pessoas assim temos que ponhar na mão de Deus thiessa te desejo tudo de bom na sua vida ser Feliz é o que importa.que Deus te ilumine todos os dias\normalsize   & \cellcolor{green!27}Mulher & \cellcolor{green!27}Gender - General & \cellcolor{green!27}1/77 & \cellcolor{green!27}1.299 \\  \hline
  \cellcolor{green!5}\small damallex Ela não quer SER uma garota, ela É uma garota!! Não é questão de escolha, é a orientação \textbf{sexual} dela, desde sempre!! como ela mesma ressaltou no vídeo.\normalsize   & \cellcolor{green!5}Sexual & \cellcolor{green!5}Gender - General & \cellcolor{green!5}1/29 & \cellcolor{green!5}3.448 \\  \hline
  \cellcolor{green!27}\small A maioria do pessoal mesmo apoiando buga. Acha que quem sente atração ou namora uma \textbf{transexual} tem que ser \textbf{homossexual}. Se for assim vai ficar com homens de pênis. Gente sem noção...\normalsize   & \cellcolor{green!27}Homossexual, Transexual & \cellcolor{green!27}Sexual Identity - General, Sexual Identity - Transexuality & \cellcolor{green!27}2/32 & \cellcolor{green!27}6.25 \\  \hline
  \cellcolor{green!5}\small A entrevistadora boiava todo tempo. Coisas tão simples e ela totalmente leiga .... não leu ou viu vídeos sobre o assunto? Tem um monte de amigo \textbf{gay} e lésbica vive no mundo YouTube que é cheio de diversidade e não consegui fazer a entrevista direto!\normalsize   & \cellcolor{green!5}Gay & \cellcolor{green!5}Sexual Identity - General & \cellcolor{green!5}1/45 & \cellcolor{green!5}2.222 \\  \hline
  \cellcolor{green!27}\small Ja acompanho a thiessita no canal dela, e foi por ela que entendi muita coisa, também n tenho amigos(as) trans mais foi pelas belas palavras dela que hoje vejo o mundo completamente diferente. Obs: tenho muitas coisas ainda para aprender, gostaria de agradecer e dizer que te admiro muito sou sua fã, parabéns e continue sendo essa \textbf{mulher} maravilhosa. (sigo te acompanhando).\normalsize   & \cellcolor{green!27}Mulher & \cellcolor{green!27}Gender - General & \cellcolor{green!27}1/62 & \cellcolor{green!27}1.613 \\  \hline
  \cellcolor{green!5}\small Eu sou \textbf{mulher} por que nasci \textbf{mulher}, isso é biológico.  Uma "\textbf{Mulher} trans" sempre foi e sempre vai ser um homem, não adianta tomar hormônios e/ou fazer uma cirurgia, isso nunca vai mudar o fato de ter nascido um homem! Mulheres trans nunca vão saber o que é ser \textbf{mulher} de verdade. Por que tenho que aceitar um homem que resolveu ser \textbf{mulher}  ter os mesmos direitos de uma \textbf{mulher}?  E eu não odeio os homens por acharem que podem ser mulheres,  eu odeio a ideia de que nós mulheres temos que aceitar que ser "\textbf{mulher}" é um sentimento,  é ter cabelo longo,  usar vestido e batom \textbf{v\textbf{ermelho}}. Eu nao odeio ninguém,  mas posso falar isso milhoes de vezes que ainda vou ser taxada como feminista, transfóbica ou o que vocês quiserem falar,  mas eu sinto muito,  \textbf{mulher} ele nunca vai ser.\normalsize   & \cellcolor{green!5}Mulher, Vermelho & \cellcolor{green!5}Ethnicity - Native American, Gender - General, Ideological and Political Identity - General & \cellcolor{green!5}10/142 & \cellcolor{green!5}7.042 \\  \hline
  \cellcolor{green!27}\small Mulher ela é sim, só fêmea não...\normalsize   & \cellcolor{green!27}Mulher & \cellcolor{green!27}Gender - General & \cellcolor{green!27}1/7 & \cellcolor{green!27}14.286 \\  \hline
  \cellcolor{green!5}\small Se você não aceita e não entende, ok, mas não \textbf{muda} o fato dela ser \textbf{mulher}.  A biologia não é esse \textbf{preto} no \textbf{branco} que vocês gostariam que fosse pra legitimizar discriminação. Apenas respeite e busque entender que diferenças existem, quer você goste ou não. Além disso, o mundo não gira em torno de você, só porque você  pertence a uma maioria.\normalsize   & \cellcolor{green!5}Branco, Muda, Mulher, Preto & \cellcolor{green!5}Ethnicity - Black, Ethnicity - White, Gender - General, Physical Identity - Physical (and Mental) Impairments & \cellcolor{green!5}4/62 & \cellcolor{green!5}6.452 \\  \hline
  \cellcolor{green!27}\small Pois é menina, nunca irá ser uma \textbf{mulher}!\normalsize   & \cellcolor{green!27}Mulher & \cellcolor{green!27}Gender - General & \cellcolor{green!27}1/8 & \cellcolor{green!27}12.5 \\  \hline
  \cellcolor{green!5}\small E você é \textbf{mulher} de verdade por conta do que? Pq tem vagina e menstrua? E mulheres "biológicas" como vc que perderam ou nasceram sem útero? Você é \textbf{mulher} pq nasceu XX? Então me explica as outras múltiplas variações que existem, a questão NUNCA vai ser a sua biologia. Ninguém "resolve" virar \textbf{mulher} pra ler tanta merda e conviver com tanta gente ignorante e conservadora igual você, "homem" nenhum resolve virar \textbf{mulher} pra sofrer tanto numa sociedade cruel e preconceituosa dessas, você assistiu a porra do vídeo? Ninguém escolhe morrer ou apanhar nas ruas apenas por SER inevitavelmente quem é. Próximo.\normalsize   & \cellcolor{green!5}Mulher & \cellcolor{green!5}Gender - General & \cellcolor{green!5}4/101 & \cellcolor{green!5}3.96 \\  \hline
  \cellcolor{green!27}\small Ser \textbf{mulher} não é um sentimento ? Oi ??? Viver é um sentimento, independente da identidade de gêneroou vc tem sentimento na sua vagina ? ou eu como homem deveria ter sentimentos no meu pênis ? Bem transfobica sua forma de pensar ! EVOLUA !\normalsize   & \cellcolor{green!27}Mulher & \cellcolor{green!27}Gender - General & \cellcolor{green!27}1/46 & \cellcolor{green!27}2.174 \\  \hline
  \cellcolor{green!5}\small \@Laiza Santos Poxa, então minha irmã que não pode gerar uma criança deveria virar homem ? EVOLUA ! Ela é \textbf{mulher} tanto quanto você, pq ser \textbf{mulher} é muito mais que gerar filhos ou ter uma vagina, nossa identidade de gênero não esta nos órgãos sexuais e sim no nosso cérebro !\normalsize   & \cellcolor{green!5}Mulher & \cellcolor{green!5}Gender - General & \cellcolor{green!5}2/52 & \cellcolor{green!5}3.846 \\  \hline
  \cellcolor{green!27}\small \@purrella Mulheres sem útero continuam sendo mulheres pois são do sexo masculino. Bem diferente de um homem achar que é \textbf{mulher} porque usa vestidos.\normalsize   & \cellcolor{green!27}Mulher & \cellcolor{green!27}Gender - General & \cellcolor{green!27}1/24 & \cellcolor{green!27}4.167 \\  \hline
  \cellcolor{green!5}\small \@Lucas Gilli Ser \textbf{mulher} é muito mais do que usar maquiagem e ter cabelo comprido. Ser \textbf{mulher} não é um sentimento e sim uma REALIDADE BIOLÓGICA.\normalsize   & \cellcolor{green!5}Mulher & \cellcolor{green!5}Gender - General & \cellcolor{green!5}2/26 & \cellcolor{green!5}7.692 \\  \hline
  \cellcolor{green!27}\small Samantha D. Poxa é mesmo? Que pena, mulheres trans vão continuar sendo mulheres na LEI, mudando até nome é certidão, agora eu te desafio chegar pra uma \textbf{mulher} trans pessoalmente a tratando como homem, pra vc levar um processinho gostoso pra deixar de ser uma escrota burra e conservadora\normalsize   & \cellcolor{green!27}Mulher & \cellcolor{green!27}Gender - General & \cellcolor{green!27}1/49 & \cellcolor{green!27}2.041 \\  \hline
  \cellcolor{green!5}\small \@Samantha D. Sim te entendo, mas ela é uma \textbf{mulher}, tem cérebro e sentimentos de \textbf{mulher}, porém não a biologia, logo ela é uma \textbf{mulher} TRANSGÊNERO, mas isso teria que pesquisar e estudar pra tentar começar a entender. Enfim, respeito sua opinião !\normalsize   & \cellcolor{green!5}Mulher & \cellcolor{green!5}Gender - General & \cellcolor{green!5}3/43 & \cellcolor{green!5}6.977 \\  \hline
  \cellcolor{green!27}\small \@purrella Quando um homem entrar no banheiro feminino alegando ser \textbf{mulher} quero ver você reclamar. Você acha que todas transsexuais parecem mulheres? Tem muito macho por aí que acha que é só por uma maquiagem que já é \textbf{mulher}.Ninguém é obrigado a ver transsexuais da maneira que eles se veem. Homens e mulheres não são do mesmo sexo.\normalsize   & \cellcolor{green!27}Mulher & \cellcolor{green!27}Gender - General & \cellcolor{green!27}2/59 & \cellcolor{green!27}3.39 \\  \hline
  \cellcolor{green!5}\small \@Lucas Gilli o que seria sentimentos de \textbf{mulher}? Só quem nasce \textbf{mulher} sabe o que é ser \textbf{mulher}. E não existe "cérebro de \textbf{mulher} e de homem".\normalsize   & \cellcolor{green!5}Mulher & \cellcolor{green!5}Gender - General & \cellcolor{green!5}4/27 & \cellcolor{green!5}14.815 \\  \hline
  \cellcolor{green!27}\small Samantha D. Seu argumento é tão podre e inválido, tem tanta \textbf{mulher} que parece homem, lésbicas, mulheres com traços masculinos, e continuam usando a merda do banheiro feminino, e vc quer usar esse argumento lixo de \textbf{mulher} trans que parece 'macho'??? Kkkk o número de pessoas trans que assediaram alguém num banheiro é ZERO, elas só querem poder fazer xixi ou retocar uma maquiagem em paz, o único problema num banheiro feminino é pra elas encontrarem alguém doente que as ameacem igual vc\normalsize   & \cellcolor{green!27}Mulher & \cellcolor{green!27}Gender - General & \cellcolor{green!27}2/83 & \cellcolor{green!27}2.41 \\  \hline
  \cellcolor{green!5}\small Ester Mattivi, realmente, você é transfóbica e você sabe disso. A questão da transexualidade não é 'sentir-se' \textbf{mulher} e sim ter nascido em um corpo que não coincide com o que ela realmente se reconhece. A sua separação (e de muitas outras) é chula, muito transfóbica e é de tamanha falta de caráter e respeito o fato de você não concordar com a transexualidade e, por isso, não a chamá-la no gênero que ela se reconhece. MUITO FÁCIL PEDIR RESPEITO, MAIS É PARA POUCOS RESPEITAR AS DIFERENÇAS.  E antes de me julgar falando que sou menino e não devo me meter nesses assuntos, uma palavra que vejo muito o movimento feminista usar é 'Sororidade' e desculpa por parecer que estou te julgando, mas, embora o feminismo se divida em varias camadas (como feminismo trans, feminismo \textbf{negro} e entre outros, me corrija se eu estiver falando algo de errado) a sororidade, não é só entre as mulheres que vivem na sua camada feminista e sim entre todas, independentemente, de suas diferenças.E me desculpe, se aparentei ser agressivo com você, em algum momento.Um beijo no coração e por favor, crie empatia e compaixão pelas pessoas.\normalsize   & \cellcolor{green!5}Mulher, Negro & \cellcolor{green!5}Ethnicity - Black, Gender - General & \cellcolor{green!5}2/195 & \cellcolor{green!5}1.026 \\  \hline
  \cellcolor{green!27}\small \@Laiza Santos Nem toda muher quer gerar criança. Eu não quero e não sou menos \textbf{mulher}  por isso, muito menos ela, que pode adotar como é o meu desejo.Simples\normalsize   & \cellcolor{green!27}Mulher & \cellcolor{green!27}Gender - General & \cellcolor{green!27}1/30 & \cellcolor{green!27}3.333 \\  \hline
  \cellcolor{green!5}\small \@Samantha D. Quem é você para dizer o que existe ou não existe? Você é medica?  Psiquiatra?  Você passouQuem é você para dizer o que existe ou não existe? Você é medica?  Psiquiatra?  Você passou quantos anos  estudando sobre transgêneros? Me poupe, nenhum homem ou \textbf{mulher} trans precisa da sua aprovação, é tão ridículo achar que SUA definição limitada sobre a vida e as pessoas representam a realidade do mundo e outros precisam se adaptar a ela, putz se melhore quantos anos  estudando sobre transgêneros? Me poupe, nenhum homem ou \textbf{mulher} trans precisa da sua aprovação, é tão ridículo achar que SUA definição limitada sobre a vida e as pessoas representam a realidade do mundo e outros precisam se adaptar a ela, putz se melhore.\normalsize   & \cellcolor{green!5}Mulher & \cellcolor{green!5}Gender - General & \cellcolor{green!5}2/125 & \cellcolor{green!5}1.6 \\  \hline
  \cellcolor{green!27}\small Olha,não sou preconceituosa e bem longe disso,mas agora vc falar que a sociedade está errada em dizer que homem tem que ter penis e \textbf{mulher} tem que ter vagina é demais,fale isso pra mãe natureza e pra biologia humana.É normal ter essa transgressão,não gostar doque vc é ,mas as pessoas tem que entender que vc pode se considerar homem ou \textbf{mulher} ,mas pros seus códigos genéticos vc nunca vai ser ,entao o certo nao é falar que a sociedade está errada,pq assim vc estaria falando que a biologia está errada incluindo a mãe natureza,o errado ai é o preconceito!\normalsize   & \cellcolor{green!27}Mulher & \cellcolor{green!27}Gender - General & \cellcolor{green!27}2/99 & \cellcolor{green!27}2.02 \\  \hline
  \cellcolor{green!5}\small No Inicio a entrevistada diz: te disseram que vc é \textbf{mulher}...OI?\normalsize   & \cellcolor{green!5}Mulher & \cellcolor{green!5}Gender - General & \cellcolor{green!5}1/11 & \cellcolor{green!5}9.091 \\  \hline
  \cellcolor{green!27}\small ​\@Joseli Farias Lopes existe uma diferença muio grande entre SEXO masculino/feminino e GÊNERO feminino/masculino, ninguem está negando a biologia e que vagina pertence ao sexo feminino e pênis ao sexo masculino, isso é fato, porém, ela esta falando sobre o papel de genero, ex: se você é \textbf{mulher} voce tem que ser isso, isso e isso, e vice-versa\normalsize   & \cellcolor{green!27}Mulher & \cellcolor{green!27}Gender - General & \cellcolor{green!27}1/58 & \cellcolor{green!27}1.724 \\  \hline
  \cellcolor{green!5}\small Mai O.c E em algum momento eu disse que eles,nós ,nao somos preconceituosos ?saiba interpretar e depois venha querer dar lição de moral ,cada um tem sua opinião e se vc acha certo ela falar que a sociedade está errada pq  designa que homem tem penis e \textbf{mulher} vagina ,a opinião é toda sua! So que depois ache uma explicação nos genes humanos que corrobore com esse seu pensamento,contra fatos não há argumentos e eu provo a vc que homem pode tirar o pinto mas sempre continuará sendo homem !\normalsize   & \cellcolor{green!5}Mulher & \cellcolor{green!5}Gender - General & \cellcolor{green!5}1/90 & \cellcolor{green!5}1.111 \\  \hline
  \cellcolor{green!27}\small Lucas Gilli Meu Deus,oq acontece? Foi oq eu falei sua anta ,que essa transgressão é normal,olha oq vc falou,que a mãe natureza pode estar errada?oi? o preconceito vem da sociedade sim ,mas oq eu coloquei como errada foi a coloção que a moça usou!Pra vc ter uma noção,desde a Grécia antiga isso é evidente,amor msm so era aceito entre homem e homem ,e \textbf{mulher} so tinha o papel de procriação,agora deixa eu explicar oq nenhum de vcs entendeu até agora,deve ser falta de estudo duranto o ensino médio,é normal ocorrer essas divergências,é algo NORMAL que querendo ou não todo mundo tem,o ser humano é algo incrível,mas oq eu quis dizer é que por mais que a pessoa se considere homem ou \textbf{mulher} ,biologicamente e quimicamente,dentro do Dna,ela sempre vai ser homem,e não tem oq mude isso,tirando o caso de uma mutação gênica,então vc acha certo ver os outros como errados?O homem que é hetero nao é homem de vdd, a sociedade que ensinou????o cromossomo xy responde isso pra vc !\normalsize   & \cellcolor{green!27}Mulher & \cellcolor{green!27}Gender - General & \cellcolor{green!27}2/170 & \cellcolor{green!27}1.176 \\  \hline
  \cellcolor{green!5}\small Tá ligada que mutação acontece em todos os indivíduos né?! Mutação é algo natural, portanto tbm é natural. Falar que é mutação no intuito de dizer que n vale é o argumento mais \textbf{idiota} possível haha\normalsize   & \cellcolor{green!5}Idiota & \cellcolor{green!5}Physical Identity - Physical (and Mental) Impairments & \cellcolor{green!5}1/36 & \cellcolor{green!5}2.778 \\  \hline
  \cellcolor{green!27}\small ​\@isa m Ai minha filha, primeiramente lhe faltou estudo pelo visto primário, ANTAS são extremamente inteligentes, segundo, SIM ela esta certa em dizer que a sociedade esta errada e de forma preconceituosa lhe afirmar que ela não é \textbf{mulher}, pois segundo  a psicologia e disso eu entendo, não se trata de mutação física somente , é muito além, mas para isso o preconceito deve ser deixado de lado, mas nem todos conseguem né ????existe alguns artigos e sites que podem ajudar a entender a psicologicamente como funciona a mutação no cérebro de um transgênero, e olha que massa e novidade para quem nunca estudou : O Cérebro também é um órgão do corpo, inclusive o mais importante. Ela é uma \textbf{MULHER} e sempre será e a sociedade esta equivocada em achar que simplesmente os cromossomos determinam o gênero de uma pessoa , lembrando que Sexo é biológico e gênero é social, sendo assim socialmente erramos em dizer que ela nunca será uma \textbf{mulher}, pois biologicamente suas percepções e alterações cerebrais são femininas e após tratamento hormonal isso se torna mais característico ainda.\normalsize   & \cellcolor{green!27}Mulher & \cellcolor{green!27}Gender - General & \cellcolor{green!27}3/183 & \cellcolor{green!27}1.639 \\  \hline
  \cellcolor{green!5}\small a thiessa é muito rainha socorrooo, eu aprendo tanto com essa \textbf{mulher}!!! e olhava pra karol e só via a lucy hale, q semelhança jesus amado <3\normalsize   & \cellcolor{green!5}Mulher & \cellcolor{green!5}Gender - General & \cellcolor{green!5}1/27 & \cellcolor{green!5}3.704 \\  \hline
  \cellcolor{green!27}\small Thiessa \textbf{mulher} linda 🥰🥰😍😍❤️ sou inscrita no canal dela adoro seus vídeos 🥰🥰\normalsize   & \cellcolor{green!27}Mulher & \cellcolor{green!27}Gender - General & \cellcolor{green!27}1/13 & \cellcolor{green!27}7.692 \\  \hline
  \cellcolor{green!5}\small O que mais me chocou nesse vídeo foi a \textbf{idade} dela\normalsize   & \cellcolor{green!5}Idade & \cellcolor{green!5}Age - General & \cellcolor{green!5}1/11 & \cellcolor{green!5}9.091 \\  \hline
  \cellcolor{green!27}\small Acho que independente de gênero não devemos cobrar nada de uma criança eu mesma sou \textbf{mulher} nasci assim mas nunca gostei de "coisas de menina " principalmente roupas e me obrigavam a usar saía vestido e quando comecei a me vestir sozinha minha mãe brigava comigo me chamava de "sapatão"Enfim ainda quero viver em um mundo em que as pessoas tenham poder de ser quem é só isso não temos que querer pessoas ideias afinal o ideal só existe na nossa imaginação  😉\normalsize   & \cellcolor{green!27}Mulher & \cellcolor{green!27}Gender - General & \cellcolor{green!27}1/84 & \cellcolor{green!27}1.19 \\  \hline
  \cellcolor{green!5}\small Não diria que é uma \textbf{mulher} transgenêro. Maravilhosa 👏👏👏👏\normalsize   & \cellcolor{green!5}Mulher & \cellcolor{green!5}Gender - General & \cellcolor{green!5}1/9 & \cellcolor{green!5}11.111 \\  \hline
  \cellcolor{green!27}\small Onde essa \textbf{mulher} tem 28 anos, jamais. Parece menos achava que tinha máximo 22\normalsize   & \cellcolor{green!27}Mulher & \cellcolor{green!27}Gender - General & \cellcolor{green!27}1/14 & \cellcolor{green!27}7.143 \\  \hline
  \cellcolor{green!5}\small a unica coisa que me incomoda muito nesse tipo de conversa envolvendo alguém lgbtq+ é como as pessoas vêem como um album aberto para fazer qualquer tipo de pergunta, das mais pessoais até as mais inconvenientes. entendemos sua curiosidade e compreendemos seu anseio para mata-la, porém é evitável certas dúvidas, como perguntar sobre reações familiares, o "quão difícil foi uma reação", porque essas coisas são TÃO íntimas, elas podem desencadear TANTAS memórias e sentimentos-bons ou ruins. é um vídeo importantíssimo, trazer informação para o público em todas as plataformas de comunicação é super necessário, porém não é porque é uma entrevista cara a cara que você tem o direito de perguntar qualquer coisa antes de se fazer certeza que o entrevistado está confortável com isso e de acordo em responder. isso é cuidado, carinho e compreensãonão sei se a entrevistadora se preocupou com isso e se sim, ótimo, perfeito, continue assim, porém me incomodou muito o fato de que uma das primeiras perguntas foi sobre a resposta do pai sobre sua \textbf{transição} e as histórias por trás disso. a thiessa falando que ia chorar e a karol falando sobre a maquiagem que estava por trás...isso é tudo tão delicado, eu, como \textbf{mulher} \textbf{homossexual}, me sinto TÃO incomodada quando qualquer pessoa faz qualquer pergunta se achando íntimo e no direito, sem ao menos ter a noção se isso vai ser ou não inconveniente para mim.\normalsize   & \cellcolor{green!5}Homossexual, Mulher, Transição & \cellcolor{green!5}Gender - General, Sexual Identity - General, Sexual Identity - Transexuality & \cellcolor{green!5}3/236 & \cellcolor{green!5}1.271 \\  \hline
  \cellcolor{green!27}\small Só Existe Macho e Fêmea, Não Existe Hôrmônio \textbf{Gay} e Nem Hormônio Trangênero,  Não Existe Celula \textbf{Gay},  Não Existe Neurônio \textbf{Gay},  Por Um \textbf{Esqueleto}  de Centenas de anos Pelo D.N.A  sabe-se Se é Homem Ou \textbf{Mulher}. E A Ditadura \textbf{gay} Está Oprimindo a Ciência.\normalsize   & \cellcolor{green!27}Esqueleto, Gay, Mulher & \cellcolor{green!27}Gender - General, Physical Identity - Physical Features, Sexual Identity - General & \cellcolor{green!27}6/44 & \cellcolor{green!27}13.636 \\  \hline
  \cellcolor{green!5}\small Ser \textbf{mulher} não é um sentimento. 'Essência feminina' é um conceito retrógrado, sexista.\normalsize   & \cellcolor{green!5}Mulher & \cellcolor{green!5}Gender - General & \cellcolor{green!5}1/13 & \cellcolor{green!5}7.692 \\  \hline
  \cellcolor{green!27}\small Ser é \textbf{mulher} é subjetivo para cada uma, o gênero em si é uma construção social ou seja pode se sentir/identificar com o mesmo.\normalsize   & \cellcolor{green!27}Mulher & \cellcolor{green!27}Gender - General & \cellcolor{green!27}1/24 & \cellcolor{green!27}4.167 \\  \hline
  \cellcolor{green!5}\small Deus é perfeito Ele nunca errou Deus fez o homem e vendo que estava só fez a \textbf{mulher}! Deus destruiu Sodoma e Gomorra devido ao pecado!Deus ama o pecador e não o pecado!Ter que provar para alguém alguma coisa? Imaturidade!Falar que tudo é preconceito? Não!Se sigamos a Jesus não amamos o pecado e sim buscamos nos santificar nos todos os dias!O ser humano ele se revela contra o criador pq por trás existe um único inimigo que quer distorcer a imagem de Deus pois somos semelhantes a Deus! Seja plástica exageradas,seja corpos malhados parecidos com homens!Quem veio para roubar,matar e destruir? Amo o meu semelhante e tenho amizades com vários LGBTs e sigo a Cristo e não ao modernismo cristão!Assistam Flávio Amaral o ex \textbf{travesti} e vejam sua história e sua libertação,quem melhor que ele pra contar!\normalsize   & \cellcolor{green!5}Mulher, Travesti & \cellcolor{green!5}Gender - General, Sexual Identity - Transexuality & \cellcolor{green!5}2/137 & \cellcolor{green!5}1.46 \\  \hline
  \cellcolor{green!27}\small "Deus é perfeito Ele nunca errara!" então a comunidade \textbf{LGBT} não é um erro, próximo.\normalsize   & \cellcolor{green!27}LGBT & \cellcolor{green!27}Sexual Identity - General & \cellcolor{green!27}1/15 & \cellcolor{green!27}6.667 \\  \hline
  \cellcolor{green!5}\small a cara da karol de chocada com as vivências de uma \textbf{mulher} trans.\normalsize   & \cellcolor{green!5}Mulher & \cellcolor{green!5}Gender - General & \cellcolor{green!5}1/13 & \cellcolor{green!5}7.692 \\  \hline
  \cellcolor{green!27}\small Claro que não é a mesma dor mas, tive uma boa dose de aprendizados me assumindo lésbica.Me abriu a mente para as muitas mulheres, vós, trans, negras, brancas, amarelas, indígenas, para tudo o que passamos sendo \textbf{mulher}, nessa sociedade, nesse país, nesse mundo.Da pânico ser \textbf{mulher}, é minha vivência perto de outras não é tão negativa quanto já li, já assisti em canais de pessoas lgbtq.Haja tratamento, tenho mesmo crises em pensar o que será de cada um de nós, conscientes daquilo que estamos vivendo.Parabéns pelo vídeo, parabéns por trazer o tema à tona e por favor, não parem.\normalsize   & \cellcolor{green!27}Mulher & \cellcolor{green!27}Gender - General & \cellcolor{green!27}2/103 & \cellcolor{green!27}1.942 \\  \hline
  \cellcolor{green!5}\small Ser passável é estar dentro do padrão heteronormativo, mas pera aí, a comunidade \textbf{lgbt} é contra esse padrão?\normalsize   & \cellcolor{green!5}LGBT & \cellcolor{green!5}Sexual Identity - General & \cellcolor{green!5}1/18 & \cellcolor{green!5}5.556 \\  \hline
  \cellcolor{green!27}\small Vídeo incrível, Karol e uma \textbf{mulher} maravilhosa, super aberta a conhecer pessoas diferentes do seu convívio. Se o mundo fosse assim seríamos um mundo melhor 💜\normalsize   & \cellcolor{green!27}Mulher & \cellcolor{green!27}Gender - General & \cellcolor{green!27}1/26 & \cellcolor{green!27}3.846 \\  \hline
  \cellcolor{green!5}\small Pergunta: o que é ser \textbf{mulher}? Usar maquiagem te torna \textbf{mulher}? Tomar hormônio feminino te torna \textbf{mulher}? Se "sentir" \textbf{mulher} te torna \textbf{mulher}? O que é se sentir \textbf{mulher}? Fico de cara. Então se eu me sentir \textbf{negra} e fazer bronzeamento artificial, colocar ácido no lábio e fazer permanente eu me torno \textbf{negra}, mesmo que meus antepassados (e eu, indiretamente) tenham oprimido e escravizado pessoas negras? Por que, ao invés de dizer que é \textbf{mulher}, não diz que é um homem que gosta de maquiagem, vestidos e atividades que caracterizam o que é uma \textbf{mulher} na nossa sociedade, já que essas não são e nem devem ser coisas exclusivamente (e obrigatoriamente) femininas? Tu não acha que fato de tu se sentir \textbf{mulher} por querer performar feminilidade (que é gostar de maquiagem e tudo o que dizem que é uma \textbf{mulher} hoje) reforça esse esterótipo e obrigatoriedade sobre mulheres? Vocês tão aí pra quê, pra acabar de vez com o \textbf{patriarcado} e com a opressão feminina ou pra, com esse discurso liberal de "faço o que eu quero", continuar sendo oprimidas e fazendo o que QUEREM que vocês façam? Qual é o objetivo de vocês?\normalsize   & \cellcolor{green!5}Mulher, Negra, Patriarcado & \cellcolor{green!5}Ethnicity - Black, Gender - General & \cellcolor{green!5}13/194 & \cellcolor{green!5}6.701 \\  \hline
  \cellcolor{green!27}\small qual objetivo de vcs- ser feliz do nosso jeito e sendo respeitado por todos\textbf{lgbt}, negros, pobres, mulheres\normalsize   & \cellcolor{green!27}LGBT & \cellcolor{green!27}Sexual Identity - General & \cellcolor{green!27}1/19 & \cellcolor{green!27}5.263 \\  \hline
  \cellcolor{green!5}\small Lógico que ele nunca vai ser \textbf{mulher}, Deus não errou ao criar elael, tudo o Deus faz é bom e é perfeitlo o ser humano que não sabe agradecer somente pelo fato de estar respirando,  se ele fosse uma \textbf{mulher} ele tinha menstruação e engravidava\normalsize   & \cellcolor{green!5}Mulher & \cellcolor{green!5}Gender - General & \cellcolor{green!5}2/45 & \cellcolor{green!5}4.444 \\  \hline
  \cellcolor{green!27}\small Amei ❤️❤️ Precisamos também de um vídeo com uma mina \textbf{travesti} e um homem trans, vamos aumentar nossa visibilidade nesse mês que além de ser mês da saúde mental também se fala da visibilidade trans ❤️\normalsize   & \cellcolor{green!27}Travesti & \cellcolor{green!27}Sexual Identity - Transexuality & \cellcolor{green!27}1/36 & \cellcolor{green!27}2.778 \\  \hline
  \cellcolor{green!5}\small Dona Salon Line segue a sugestão e \textbf{muda} o cenário pra ser mais acolhedor :) Bota um sofazão com almofadas pra as gurias ficarem mais próximas! Fora isso, esse quadro é maravilhoso e muito necessário. Ótima iniciativa <3\normalsize   & \cellcolor{green!5}Muda & \cellcolor{green!5}Physical Identity - Physical (and Mental) Impairments & \cellcolor{green!5}1/38 & \cellcolor{green!5}2.632 \\  \hline
  \cellcolor{green!27}\small Só digo uma coisa: ser \textbf{mulher} não é sentimentos. É socialização.\normalsize   & \cellcolor{green!27}Mulher & \cellcolor{green!27}Gender - General & \cellcolor{green!27}1/11 & \cellcolor{green!27}9.091 \\  \hline
  \cellcolor{green!5}\small Ser \textbf{mulher} é subjetivo pra cada pessoa, mas tomando do seu ponto de vista \textbf{mulher} é uma construção social ou seja pessoas podem ou não se identificar com essa construção.\normalsize   & \cellcolor{green!5}Mulher & \cellcolor{green!5}Gender - General & \cellcolor{green!5}2/30 & \cellcolor{green!5}6.667 \\  \hline
  \cellcolor{green!27}\small \@emily \textbf{Mulher} é alguém do sexo feminino, apenas isso.\normalsize   & \cellcolor{green!27}Mulher & \cellcolor{green!27}Gender - General & \cellcolor{green!27}1/9 & \cellcolor{green!27}11.111 \\  \hline
  \cellcolor{green!5}\small \@Samantha D. Fêmea é alguém do sexo feminino meu anjo, \textbf{Mulher} é uma construção social\normalsize   & \cellcolor{green!5}Mulher & \cellcolor{green!5}Gender - General & \cellcolor{green!5}1/15 & \cellcolor{green!5}6.667 \\  \hline
  \cellcolor{green!27}\small \@Joanna Augusta Exato, e para transformar em \textbf{mulher} de verdade seria necessário trocar o cromossomo Y de cada celula do corpo por um cromossomo X, enquanto não fizer isso, ainda será um homem.Tipo, nada contra quem quer se operar porque se sente melhor tendo um corpo de \textbf{mulher} e uma genitália de \textbf{mulher}, mas fica por ai, pra mim é um homem com corpo de \textbf{mulher} e genitália de \textbf{mulher}, mas não é \textbf{mulher}.\normalsize   & \cellcolor{green!27}Mulher & \cellcolor{green!27}Gender - General & \cellcolor{green!27}6/75 & \cellcolor{green!27}8.0 \\  \hline
  \cellcolor{green!5}\small \@SalonLineBrasil sem dúvidas ela são muito incríveis, a equipe salon line toda e incrível por dar visibilidade a essa \textbf{mulher} incrível que é a thiessa nesse mundo de transfobia, beijos\normalsize   & \cellcolor{green!5}Mulher & \cellcolor{green!5}Gender - General & \cellcolor{green!5}1/30 & \cellcolor{green!5}3.333 \\  \hline
  \cellcolor{green!27}\small Essa cadeira tá muito baixa para elas. A carol está corcunda\normalsize   & \cellcolor{green!27}Corcunda & \cellcolor{green!27}Physical Identity - Physical (and Mental) Impairments & \cellcolor{green!27}1/11 & \cellcolor{green!27}9.091 \\  \hline
  \cellcolor{green!5}\small Só um adendo: \textbf{Transgénero} é um termo guarda-chuva criado pela medicina para patologizar as identidades trans com a doença "disforia de genero". \textbf{Transexual} e \textbf{travesti} é um termo utilizado pelo Brasil para atribuir a mulheres trans, o primeiro termo muito utilizado de maneira higienizado e o segundo ligado a marginalização, já que travestis são consideradas a margem da sociedade por viverem em sua maioria na prostituição.  Mas tanto travestis quantos transexuais são as mesmas coisas.\normalsize   & \cellcolor{green!5}Transexual, Transgénero, Travesti & \cellcolor{green!5}Sexual Identity - Transexuality & \cellcolor{green!5}3/75 & \cellcolor{green!5}4.0 \\  \hline
  \cellcolor{green!27}\small Poxa eu REALMENTE não entendo porque tanto preconceito. Porque não ser amiga de um homem trans ou de uma \textbf{mulher} trans?!? Qual a diferença?!? Eu ia amar ter um amigo (a) assim, amizade lealdade respeito são coisas do coração e não de sexualidade ou orgão ou orientação....  Amei a entrevista e adorei a Thiessa ❤️❤️\normalsize   & \cellcolor{green!27}Mulher & \cellcolor{green!27}Gender - General & \cellcolor{green!27}1/55 & \cellcolor{green!27}1.818 \\  \hline
  \cellcolor{green!5}\small Me desculpe falar isso mas se vc \textbf{muda} tanto assim de pensamento e opinião vc é uma vai com as outras, e ser uma vai com as outras só vai te trazer males 😊 tenha sua própria opinião ouça a opinião dos outros mas não se deixe levar sempre por elas. Bjs\normalsize   & \cellcolor{green!5}Muda & \cellcolor{green!5}Physical Identity - Physical (and Mental) Impairments & \cellcolor{green!5}1/52 & \cellcolor{green!5}1.923 \\  \hline
  \cellcolor{green!27}\small \@Diihh Não, muito pelo contrário. Parei para viver minhas próprias experiências e tenho construído minha identidade com base nelas. Maria vai com as outras, com certeza não me enquadro nessa categoria.Mas, sobre eu mudar... Isso é verdade. Hoje eu sou uma pessoa, amanhã já me reinventei.  Concluo este comentário com a seguinte frase: Eu prefiro ser, essa metamorfose ambulante, 
Do que ter aquela \textbf{v\textbf{elha}} opinião formada sobre tudo ♪\normalsize   & \cellcolor{green!27}Velha & \cellcolor{green!27}Age - Over 65s, Gender - Female age and physical appearance & \cellcolor{green!27}2/70 & \cellcolor{green!27}2.857 \\  \hline
  \cellcolor{green!5}\small Fora que toda hora ela interrompe a \textbf{mulher}, irrita kkk\normalsize   & \cellcolor{green!5}Mulher & \cellcolor{green!5}Gender - General & \cellcolor{green!5}1/10 & \cellcolor{green!5}10.0 \\  \hline
  \cellcolor{green!27}\small Satanás é o pai da mentira e pai de todos os mentirosos. Castrado para de tomar hormônios para mostrar ao mundo o que acontece. Não tem coisas diferentes, são todos homossexuais e esse aí é \textbf{homossexual}. Voz de homem. Cara de homem. A verdade sempre prevalecerá é homem. A cirurgia é castração e mutilação. A cirurgia não transforma pênis em vagina eu vi o vídeo dessa cirurgia. Vagina é criação de Deus dentro do DNA da \textbf{mulher}. Você homem castrado usa um aparelho para manter o buraco que você chama de vagina aberto e usa lubrificantes junto. \textbf{Mulher} não é assim. O tempo de vida de uma pessoa que passa por essa cirurgia é de 35 anos devido a problemas hormonais. Obstinação é pecado. Obstinação é fazer tudo oque quer a qualquer custo, sem medir as consequências e os médicos mercenários de plantão estão aguardando os obstinados para encherem o bolso de dinheiro.\normalsize   & \cellcolor{green!27}Homossexual, Mulher & \cellcolor{green!27}Gender - General, Sexual Identity - General & \cellcolor{green!27}3/153 & \cellcolor{green!27}1.961 \\  \hline
  \cellcolor{green!5}\small Amanda Santiago você é \textbf{mulher}? É solteira? O dia que casar colocar um filho no mundo e ver ele se perverter depois de ser aliciado.por trans homossexuais e ver seu filho  prostituir-se e chamar um castrado de mãe aí você volta aqui pra pedir pra eu respeitar esse tipo de gente que está militando para crianças e adolescentes na internet e fazendo eles saírem de casa pra viverem como eles tá querida. Por hora você não perdeu ninguém que você pariu para os homossexuais. Sou inimiga número 2 deles. Vai ver as redes sociais deles e pesquisa bastante pra saber qual a faixa etária que segue eles na internet. Eles fazem um grande estrago na vida das pessoas. Já chorei muito pelo meu até Deus me mostrar por quem eu realmente devo chorar. Eles se fazem de vítima da sociedade em vídeos são uns coitadinhos perseguidos. Meu marido não queria que meu filho tivesse amizade com esse tipo de gente, eu não ligava achado que ele era forte o bastante para não se corromper mas meu marido estava certo.\normalsize   & \cellcolor{green!5}Mulher & \cellcolor{green!5}Gender - General & \cellcolor{green!5}1/179 & \cellcolor{green!5}0.559 \\  \hline
  \cellcolor{green!27}\small Felipe Tapler tem sim seu nentiroso! Vai no Instagram do meu filho ver o estrago que o trans fez nele. Meu filho se prostitui, virou alcoólatra. Posta todas as orgias no Instagram. Já vi ele na cama com 3. Não sai da casa do trans. Foi embora de casa falando que ia estudar, nos enganou, mandávamos dinheiro pra sustentar ele é ele na bagunça, sabe o que é ver seu filho beber 9 garrafas de pinga e depois se prostituir e ainda postar tudo na internet? Não você não sabe. Ódio? Seu mentiroso. O ódio está em você. O trans disse para ele que nós não merecemos ele. Um diAbo desgraçado. Sabe o que é ver seu filho em baladas \textbf{gay} de quinta a domingo se prostituindo não né. Ou você vive assim também é também faz militância né. Descobrimos tudo pela internet. Foram anos de enganos e mentiras. Meu filho não tinha nenhuma tatuagem no corpo, agora tá lotado e piercing , também, nunca falou palavrões e agora cada dez palavras 11 São palavrões. Vocês não prestam.\normalsize   & \cellcolor{green!27}Gay & \cellcolor{green!27}Sexual Identity - General & \cellcolor{green!27}1/178 & \cellcolor{green!27}0.562 \\  \hline
  \cellcolor{green!5}\small Felipe Tapler quanta mãe e pai chorando, passando a mesma situação. Olha esse do vídeo doente mental 28 anos só fala em transar e balada e nunca mais vai ver a pessoa então vamos beijar e transar né. A merda de vida que eles levam está estampado na vida do meu filho todos os dias na internet. Meu marido falou para ele que nem \textbf{prostituta} não torna público a sua vida \textbf{sexual} e não fica convencendo crianças e adolescentes a ser como ele.\normalsize   & \cellcolor{green!5}Prostituta, Sexual & \cellcolor{green!5}Gender - Female sexuality, Gender - General & \cellcolor{green!5}2/83 & \cellcolor{green!5}2.41 \\  \hline
  \cellcolor{green!27}\small Isadora Altoé você é \textbf{prostituta}? As que Jesus salvou se arrependeram de seus pecados e passaram a viver uma vida Santa é irrepreensível como está nas escrituras. Deus é Deus ele não precisa de defensores. Vai parir filho e depois chorar por ele como o Senhor Jesus disse na cruz e depois volta aqui tá querida mas se arrepende primeiro para que o teu choro seja aceito por Deus.\normalsize   & \cellcolor{green!27}Prostituta & \cellcolor{green!27}Gender - Female sexuality & \cellcolor{green!27}1/69 & \cellcolor{green!27}1.449 \\  \hline
  \cellcolor{green!5}\small Marcia Machado Sou \textbf{mulher}, sou solteira e não preciso de nada disso pra saber que preconceito é a pior coisa que o ser humano pode sentir. Primeiro, acho que pelo menos você poderia usar o termo correto, é extremamente feio usar a palavra 'castrado'. Segundo, não é pq seu filho foi 'levado' pra essa vida, que todos LGBT's são assim também, seu erro é generalizar e querer culpar pessoas que nao tem nada a ver com a vida que seu filho escolheu pra ele. Não te acrescenta em absolutamente NADA vir aqui denegrir qualquer que seja a causa ou o movimento, te garanto que seria bem melhor se ao invés de sair na internet destilando ódio, voce fosse procurar formas de ajuda-lo. Terceiro, seja \textbf{homossexual} ou transgênero, ninguém precisa influenciar ninguém, são condições que a pessoa NASCE assim, ela nao opta por ser \textbf{gay}, ser lésbica ou não se identificar com o gênero que nasceu. Quanta ignorância. Pare de querer achar uma causa para culpar as coisas que acontecem na sua casa, na sua familia. Ninguém se faz de vítima nao, ja que voce é tao bem informada, deveria saber que infelizmente a minoria é sempre a que mais sofre com discursos de ódio (como o seu), intolerância, preconceito.. então sim, enquanto orientação \textbf{sexual} for motivo de morte, haverá luta. Obs: Pelos comentários acima, vc deveria procurar ajuda pra vc e seu esposo, pra saber como lidar com essa situação, um profissional iria conseguir orienta-lós melhor, assim vocês conseguem tirar o filho de vcs dessa situaçao, basta querer ajudar tambem, primeiro passo é aceita-lo dentro de casa, te garanto que as vezes o que ele precisa é so do amor e da compreensão de vcs, o resto vem aos poucos. Ser \textbf{gay} nao é sinônimo de promiscuidade, esse pode ter sido unico jeito que seu filho viu de poder viver a vida dele, ja que provavelmente nao seria aceito pelos pais. Pense nisso. 😉\normalsize   & \cellcolor{green!5}Gay, Homossexual, Mulher, Sexual & \cellcolor{green!5}Gender - General, Sexual Identity - General & \cellcolor{green!5}5/324 & \cellcolor{green!5}1.543 \\  \hline
  \cellcolor{green!27}\small A única verdade aqui é  que vc é uma infeliz recalcada com as trans, tá  cada vez mais evidente e numeroso esse tipinho de \textbf{mulher} que não aceita a beleza das trans. Não podemos fazer nada meu amor, eu que não vou me matar ou desistir  da minha identidade por causa de mulheres frustradas  como vc.\normalsize   & \cellcolor{green!27}Mulher & \cellcolor{green!27}Gender - General & \cellcolor{green!27}1/56 & \cellcolor{green!27}1.786 \\  \hline
  \cellcolor{green!5}\small Isadora Altoé o único que é luz é Deus 1 João versos 5 a 10. Não existem espíritos de luz. Deus é luz. Eu não rezo, não sou católica. Não estou vomitando nenhum rancor tudo o que falei é a verdade. Não existe \textbf{mulher} trans. Eu tenho um intercessor e advogado junto de meu pai Jesus Cristo o justo. Mas, realmente deveria ir no canal do YouTube do homem transsexual que foi usado por satanás para mentir e enganar meu filho e fazer militância lá né! Porque aí sim estaria fazendo o certo. Mas meu recado está dado. Quem tem filhos cuide porque os homossexuais militantes estão na internet. Esse do vídeo foi expulso de casa. Eu não expulsei ninguém da minha casa e nem meu marido. Mas meu Deus é poderoso pra devolver o que o diabo nos roubou.\normalsize   & \cellcolor{green!5}Mulher & \cellcolor{green!5}Gender - General & \cellcolor{green!5}1/140 & \cellcolor{green!5}0.714 \\  \hline
  \cellcolor{green!27}\small +Marcia Machado entendo sua dor como mãe por ver seu filho nessa situação!Depois de ver todos os comentários á cima, seus, e das outras pessoas, pude perceber o quanto as pessoas são intolerantes e não respeitam o pensamento dos outros!Como assim? -Deixa eu explicar...Sou professora e minha principal área de pesquisa é no campo comportamental das pessoas.Realmente vivemos em uma Era em que as pessoas são extremamente influenciadas pelas mídias sociais em todos os aspectos, moda, comportamento, estilo de vida, música etc... ( de certa forma compreendo sua revolta contra YouTubers e seus posicionamentos midiaricos). Realmente é triste ver como as pessoas se deixam manipular em todos os aspectos por outras pessoas!Também sou adepta ao cristianismo, fui educada e instruída segundo as estruturas bíblicas. ... Mas sou uma pessoa aberta ao conhecimento. Gosto de me informar e estar por dentro de novos assuntos. Porém não me deixo influenciar por coisas não construtivas e que não me acrescentam como pessoa!!!Depois de analisar seu posicionamento como uma mãe preocupada com o bem estar de seu filho, que se expõe nas redes sociais em condições preocupantes de alcoolismo e prostituição, posso dizer que compreendo claramente sua dor e preocupação. E essa mesma história com certeza fere a muitas famílias... Não só pelo fato de seu filho ser \textbf{homossexual}.... Mas pelo ataque físico que ele se expõe diariamente! 1°- Alcoolismo é extremamente devastador ao organismo humano, mata o corpo aos poucos e agride a mente... E em muitos casos tornasse doença, por conta do vício incessante! ( trazendo também casos de depressão, e doenças físicas como cirrose hepática, problemas no coração entre outros.)2° A prostituição também é uma forma de agressão ao corpo, onde o indivíduo se expõe ao risco de graves doenças  sexualmente transmissíveis MESMO COM O USO DE PRESERVATIVOS!!! E em muitos casos o preservativo nem é usado!Além de que a pessoa também se expõe a uma possível depressão, pelo desvalorização como indivíduo, e muitos acabam cometendo suicídio!E por fim falarei sobre o quesito homossexualidade!!!Pela forma que vi seu posicionamento nos comentários, percebi que você é uma pessoa instruída... Como me posicionei á cima, também sou uma \textbf{mulher} cristã, creio nas palavras da Bíblia e em Cristo! Minha compreensão sobre o pecado é a seguinte:Na Bíblia encontramos instruções claras de como manter a continuidade da vida!É esse o objetivo de Deus em seus mandatos, fazer com que A VIDA SE MULTIPLIQUE E TENHA CONTINUIDADE!!!De forma que o amor ao próximo e a obediência a Deus além de nos conservar vivos, nos trás paz, felicidade, família, nos faz gozar do melhor desta vida... Nos faz ter CONTINUIDADE!!! E a continuidade se da pela reprodução humana. Assim como as plantas, e os outros animais da terra para se multiplicarem precisam da união de dois indivíduos sendo um progenitor de características femininas( com células,  hormônios e órgãos femininos), e outro com características masculinas (com células, hormônios e órgãos masculinos), ou seja fêmea + macho, para assim fundir suas características genéticas e dar origem a outro indivíduo (Continuidade)!!!!Por tanto, quando um indivíduo fere a lei de Deus, que também é a lei da natureza da vida provada pela ciência... Se mutilando e impedindo que a haja continuidade da vida, isso pode ser classificado para nos cristãos como PECADO, já que vai contra os planos de Deus para a CONTINUIDADE DO HOMEM!!! E falando agora de uma forma que qualquer \textbf{imbecil} entenderia... Se todos os seres humanos da terra se multilarem, cortarem seus órgãos e se homonizarem, simplesmente não haverá continuidade, já que uma \textbf{mulher} trans não é capaz de gerar um filho e um homem trans não produz espermatozóides... Ou seja por mais que se mudem por fora, não podem se mudar por dentro, e digo isso referente a estrutura corporal interna do ser humano!!!!Então depois de analisar todas as nomenclaturas inventadas pelo homem para se auto definir (trans, hetero, \textbf{gay}, sis etc...), posso concluir que a questão de gênero não passa de algo da mente do ser humano, que foi influenciado também por questões culturais e religiosas, mas que independente do que digam, é algo inalterável já que não podemos interferir na lei da natureza em relação ao corpo humano e suas funcionalidades!!!Não sou uma pessoa PRECONCEITUOSA, TRANSFOBICA, ou qualquer outra nomenclatura inventada e usada pelas pessoas... Respeito a todo o ser humano, cultura, religião, opção \textbf{sexual}... porém sou uma pessoa sensata, e busco fatos...e contra fatos não há argumentos!!!Então... Respeito a preocupação de uma mãe com a vida de seu filho!!! Respeito também sua fé cristã, que acredita que Cristo é a verdade é a vida, já que o mesmo somente prega o amor!Respeito e também concordo que as mídias sociais influenciam e desestabilizam as pessoas, principalmente os jovens que no auge de tantas mudanças e descobertas, apenas buscam por respostas... Porém em lugares errados!!!Também respeito qualquer pessoa que se sinta em particular ofendido com a forma em que essa mãe se refere às pessoas trans e afins... Já que todos temos o livre harbitrio de sermos quem quisermos ser, afinal cada um é responsável por seus atos!!!Tenho várias pessoas em minha família e até amigos que tem uma orientação \textbf{sexual} diferente da minha, e particularmente admiro muitas dessas pessoas pelo caráter e atitudes como ser humano, e apesar de não partilhar dos mesmos sentimentos e pensamentos referentes a questão de gênero, acredito que todas as pessoas deveriam respeitar mais uns aos outros. E que a educação do nosso país não se preocupasse tanto em querer fortalecer questões de gênero na escola, e sim formar cidadãos de bem, independente da orientação \textbf{sexual} do indivíduo!E friso também a questão da LIBERDADE DE PENSAMENTO...  Onde outras pessoas não respeitam a visão Cristã dessa Mãe, e alguns até zombam dela, a atacando com comentários desagradável, diminuindo ela é o esposo.E usando o triste acontecido de seu filho como um castigo a ela e o esposo por serem maus país!!!Ninguém nasce sabendo ser uma boa mãe ou um bom pai! São as experiência da vida unidas a nosso caráter que nós ensina a moldar uma criança! Acredito fielmente que ela seja uma excelente mãe, já que se preocupa com seu filho... E é mais do que normal que se revolte por ver seu filho como está!Porém apesar de influências externas existem escolhas que foram tomadas por esse filho, o tornando também responsável pelo estado em que se encontra!-Todos nós vivemos conflitos internos!!!-Ninguém é vítima da sociedade!!! Algumas escolhas nos expõe a situações desconfortáveis!-Com certeza os pais devem amar e apoiar seus filhos... Mas existem limites para esse apoio!-Não importa qual é sua orientação \textbf{sexual}! O que importa é que você vive em uma sociedade que independente do que você é eu pensa,  é e sempre será cruel e isso é um fato! -E a perversão ou pecado nada mais é do que atitudes que tomamos que coloca em risco nossa vida, saúde, família, e que aos poucos mata o corpo e a alma pela infelicidade de buscarmos uma realização carnal destruindo o que somos, por não aceitarmos quem somos!!! destruindo muitas vezes nossa continuidade!!!Sejamos quem quisermos ser... Porém sejamos fortes o suficiente para assumir as consequências!!! Não confunda liberdade com libertinagem!NOVAMENTE. .. NÃO EXISTEM VÍTIMAS DA SOCIEDADE!!!Espero que ninguém tenha se sentido ofendido... Porém trouxe aqui uma explicação Onde a CIÊNCIA E A RELIGIÃO se unem e não se contestam!!!  pensem e reflitam sobre isso antes de se julgarem e me julgarem... Ou julgarem novamente essa mãe!Um grande abraço.Boas reflexões a todos!by: Nathália Gimenes.\normalsize   & \cellcolor{green!27}Gay, Homossexual, Imbecil, Mulher, Sexual & \cellcolor{green!27}Gender - General, Physical Identity - Physical (and Mental) Impairments, Sexual Identity - General & \cellcolor{green!27}9/1279 & \cellcolor{green!27}0.704 \\  \hline
  \cellcolor{green!5}\small Tantas trans melhores e mais preparadas que a Thiessa para esse tipo de conversa. Mas decidem chamar logo aquela que esbanja infantilidade em todos os vídeos e que claramente mostra um despreparo, comum para a \textbf{idade}. Mas pelo menos deram certa visibilidade para uma trans, espero que continue assim e chamem outras mais preparadas para discutir assuntos mais "sérios" como esse.\normalsize   & \cellcolor{green!5}Idade & \cellcolor{green!5}Age - General & \cellcolor{green!5}1/61 & \cellcolor{green!5}1.639 \\  \hline
  \cellcolor{green!27}\small Gabriela Zanella cara, eu acho que a intenção era ter um papo bem básico mesmo. Afinal, esse assunto já é tão pouco tratado e o público desse canal não necessariamente já se interessava pelo assunto (já que é um canal de cabelo né), então deve ter gente bem iniciante assistindo. Acho que a Karol sabe mais do que ela deixou transparecer, até pq ela não deu nenhum \textbf{bola} fora né. Ela só fez as perguntar básicas pq eram as de maior interesse pra esse público.\normalsize   & \cellcolor{green!27}Bola & \cellcolor{green!27}Physical Identity - Physical Features & \cellcolor{green!27}1/85 & \cellcolor{green!27}1.176 \\  \hline
  \cellcolor{green!5}\small ~> 1 desculpe mas não a conhecia 2 esse negócio de saber q era H e hj em dia é M, desculpa mas se não falasse p mim vc é \textbf{mulher} e ponto !!!3 lamento mto q no mercado de trabalho ainda a preconceito e isso é vergonhoso, sério, todos são iguais INDEPENDENTE DO GÊNERO, tem sim o direito de trabalhar afinal somos seres humanos e pagamos impostos. Adorei te conhecer e a Karol arrasa nas entrevistas 😘\normalsize   & \cellcolor{green!5}Mulher & \cellcolor{green!5}Gender - General & \cellcolor{green!5}1/79 & \cellcolor{green!5}1.266 \\  \hline
  \cellcolor{green!27}\small Nem ela sabe a diferença entre transgênero e \textbf{transexual} kkkkkkkkkk a pessoa que é transgênero não se identifica com o sexo mais não quer mudar, Uma pessoa que é \textbf{transexual} como você passa por mudanças no corpo hormônios, cabelo, silicone, pra ficar com o corpo que se identifica.\normalsize   & \cellcolor{green!27}Transexual & \cellcolor{green!27}Sexual Identity - Transexuality & \cellcolor{green!27}2/48 & \cellcolor{green!27}4.167 \\  \hline
  \cellcolor{green!5}\small Transgénero é um termo guarda-chuva criado pela medicina para patologizar as identidades trans com a doença "disforia de genero". \textbf{Transexual} e \textbf{travesti} é um termo utilizado pelo Brasil para atribuir a mulheres trans, o primeiro termo muito utilizado de maneira higienizado e o segundo ligado a marginalização, já que travestis são consideradas a margem da sociedade por viverem em sua maioria na prostituição.  Mas tabto travestis quantos transexuais são as mesmas coisas.\normalsize   & \cellcolor{green!5}Transexual, Transgénero, Travesti & \cellcolor{green!5}Sexual Identity - Transexuality & \cellcolor{green!5}3/72 & \cellcolor{green!5}4.167 \\  \hline
  \cellcolor{green!27}\small Existe integração das trans em alguns grupos sim. Outros grupos rad, não. Assim como o feminismo \textbf{negro} não integra mulheres brancas, mesmo que seja da periferias. O que ocorre: a luta não exclui mulheres trans. Porem, não é uma das "prioridades", entende? A explicação é a seguinte: as mulheres trans não sofrem por serem mulheres MAS SIM, em sua maioria, TRANSFOBIA. Elas sofrem agressões por serem trans, não por serem mulheres. Mulheres trans não sofrem opressões desde criança como modo de agir, modo de pensar, influência da maternidade compulsória na vida, ouvir em caso de patologias envolvendo o utero que não é uma \textbf{mulher} de verdade e etc. Por essa razão, as radfem não priorizam trans, porém NENHUMA radfem não apoiaria uma trans. Essa foi a explicação que eu recebi e estou repassando\normalsize   & \cellcolor{green!27}Mulher, Negro & \cellcolor{green!27}Ethnicity - Black, Gender - General & \cellcolor{green!27}2/133 & \cellcolor{green!27}1.504 \\  \hline
  \cellcolor{green!5}\small \@Leoanny Bueno homens se protegem e mulheres protegem e dão prioridade a homens, inclusive muitas "feministas". Por isso que \textbf{mulher} só se ferra.\normalsize   & \cellcolor{green!5}Mulher & \cellcolor{green!5}Gender - General & \cellcolor{green!5}1/23 & \cellcolor{green!5}4.348 \\  \hline
  \cellcolor{green!27}\small Allan G O que um homem que decidiu virar \textbf{mulher} sabe o que é de fato ser uma \textbf{mulher}? Não é porque você se vê uma sereia que você é uma, acorda povo tonto\normalsize   & \cellcolor{green!27}Mulher & \cellcolor{green!27}Gender - General & \cellcolor{green!27}2/34 & \cellcolor{green!27}5.882 \\  \hline
  \cellcolor{green!5}\small \@Joanna Augusta Concordo.  É incrível como  A Femrad defende com tanto afinco a liberação do aborto como direito da \textbf{mulher} mas abortando matará muitas meninas que viriam a nascer! Posso eu tirar o direito de vida de uma  irmã pra ter os meus garantidos?\normalsize   & \cellcolor{green!5}Mulher & \cellcolor{green!5}Gender - General & \cellcolor{green!5}1/44 & \cellcolor{green!5}2.273 \\  \hline
  \cellcolor{green!27}\small \@flower power acho que tá na hora de você dá uma olhada nas pesquisas científicas sobre pessoas trans. Transexualidade vai muito além de uma construção social de gênero. Existem várias evidências científicas de que o cérebro não identica o órgão \textbf{sexual} q foi criado. E mesmo que fosse apenas construção social.  Uma \textbf{mulher} cis é criada com x expectativas e opressões e uma \textbf{mulher} trans mesmo que não sofra as mesmas sofre por todos os sinais de feminilidade que exibe. Dá uma pesquisada que a mão na vai cair.\normalsize   & \cellcolor{green!27}Mulher, Sexual & \cellcolor{green!27}Gender - General & \cellcolor{green!27}3/89 & \cellcolor{green!27}3.371 \\  \hline
  \cellcolor{green!5}\small Seria mais legal se mudassem as poltronas de lugar. A azul no \textbf{a\textbf{marelo}} e a amarela no azul.\normalsize   & \cellcolor{green!5}Amarelo & \cellcolor{green!5}Nationality - Chinese, Nationality - Japanese & \cellcolor{green!5}2/18 & \cellcolor{green!5}11.111 \\  \hline
  \cellcolor{green!27}\small Julia Helena que a azul estaria no \textbf{a\textbf{marelo}} e o \textbf{a\textbf{marelo}} no azul\normalsize   & \cellcolor{green!27}Amarelo & \cellcolor{green!27}Nationality - Chinese, Nationality - Japanese & \cellcolor{green!27}4/13 & \cellcolor{green!27}30.769 \\  \hline
  \cellcolor{green!5}\small Tbm pensei a mesma coisa, iria destacar mais, os copos tbm deveriam ser \textbf{a\textbf{marelo}} e verde e trocados.\normalsize   & \cellcolor{green!5}Amarelo & \cellcolor{green!5}Nationality - Chinese, Nationality - Japanese & \cellcolor{green!5}2/18 & \cellcolor{green!5}11.111 \\  \hline
  \cellcolor{green!27}\small \@Julia Helena A diferença é que esta tudo muito verde e tudo muito \textbf{a\textbf{marelo}}, então não fica legal!\normalsize   & \cellcolor{green!27}Amarelo & \cellcolor{green!27}Nationality - Chinese, Nationality - Japanese & \cellcolor{green!27}2/18 & \cellcolor{green!27}11.111 \\  \hline
  \cellcolor{green!5}\small PMW - Elen Cristhine a comunidade LGBTQI+ é muito marginalizada sim e, quando a Thiessa fala sobre marginalização, acontece mesmo. As travestis, por exemplo, recorrem à prostituição como forma de trabalho e sobrevivência porque, em muitas empresas, a chance delas serem contratadas é quase nula por conta de preconceito. Outro exemplo de marginalização é notar que homens gays (ou bi/pan que se relacionaram com outros homens em 12 meses) não podem doar sangue porque são consideradas 'grupo de risco' por conta do surto de AIDS dos anos 90 (sendo que há diversos estudos mostrando que casais héteros tem recorrência maior de HIV). Esses são dois exemplos de marginalização, mas existem muitos outros. Se tu conhece gays e se da bem com eles, ótimo, mas esses gays podem ser uma exceção à 'regra da marginalização'. Afinal, por mais empoderamento que os \textbf{LGBT} estejam adquirindo, a opressão machista/heterossexual ainda trava muita coisa em âmbito social e faz a marginalização ser 'adaptada', entende?E só mais um ponto: você mesma diz que não conhece nenhuma pessoa trans e só esse pedaço já dá pra te fazer refletir sobre: se tu não conhece trans nos teus círculos sociais (local onde mora, faculdade/escola, trabalho, lugares que frequenta), significa que essas pessoas trans estão sendo impedidas desse acesso, logo, sendo marginalizadas pela sociedade.\normalsize   & \cellcolor{green!5}LGBT & \cellcolor{green!5}Sexual Identity - General & \cellcolor{green!5}1/217 & \cellcolor{green!5}0.461 \\  \hline
  \cellcolor{green!27}\small PMW - Elen Cristhine Amada, nem todo mundo vive na caixinha cor-de-rosa, que aceite respeitar todo mundo. As pessoas trans são marginalizados sim, tem medo de serem discriminado por sua orientação \textbf{sexual}. Isso é terrível mas real\normalsize   & \cellcolor{green!27}Sexual & \cellcolor{green!27}Gender - General & \cellcolor{green!27}1/37 & \cellcolor{green!27}2.703 \\  \hline
  \cellcolor{green!5}\small Elen Cristine acredito realmente q você não tenha preconceito mas em um contexto geral os trans sao marginalizados sim. Pense de um modo geral como seria uma pessoa trans frequentando os mesmos ambientes que você. Como seria a reação das pessoas. Sou hetero, cis, tenho uma mente super aberta e convivo com pessoas preconceituosas dentro da minha própria família. Observo de fora esse preconceito mas acredito q pra quem vive isso deve ser muito mais intenso. E igual a pequenos gestos que eu percebo quando sofro algum tipo de preconceito racial. Muitas vezes se tem uma pessoa branca comigo ela nem percebe. Mas eu sim. Imagino q com trans seja pior ainda porque ao contrário do \textbf{racismo} q na maioria das vezes é velado com os trans, travestis e etcAs pessoas nem disfarçamPor isso para eles é mais fácil transitar nos meios onde sao aceitos. Talvez vc até conheça  uma pessoa trans e nem saiba. Eu não consigo nem imaginar como é o dia a dia de pessoas assim mas acredito q seja uma batalha constante\normalsize   & \cellcolor{green!5}Racismo & \cellcolor{green!5}Ethnicity - General & \cellcolor{green!5}1/178 & \cellcolor{green!5}0.562 \\  \hline
  \cellcolor{green!27}\small Eu amo essa \textbf{mulher}! Thiessa, rainha! Minha inspiração! Que bom que vocês abordaram esse tema! Parabéns! 💋💋💋💋🏳️‍🌈🏳️‍🌈🏳️‍🌈🏳️‍🌈🏳️‍🌈❤️❤️❤️❤️❤️\normalsize   & \cellcolor{green!27}Mulher & \cellcolor{green!27}Gender - General & \cellcolor{green!27}1/17 & \cellcolor{green!27}5.882 \\  \hline
  \cellcolor{green!5}\small Mano na hr qe Vi qe Karol ia entrevista Thiessa vim correndo😍😱 amoo essa mulher\normalsize   & \cellcolor{green!5}Mulher & \cellcolor{green!5}Gender - General & \cellcolor{green!5}1/15 & \cellcolor{green!5}6.667 \\  \hline
  \cellcolor{green!27}\small gostei muito, mas tenho uma observação... pq uma trans pode dizer : Nasci assim, smp me senti assim... E a \textbf{mulher} comum n, ela n nasce \textbf{mulher}, a sociedade que a transforma.... não entendo\normalsize   & \cellcolor{green!27}Mulher & \cellcolor{green!27}Gender - General & \cellcolor{green!27}2/34 & \cellcolor{green!27}5.882 \\  \hline
  \cellcolor{green!5}\small Nunca ouvi essa segunda frase mas... \textbf{mulher} cis também pode falar que nasceu assim e se sente assim :)\normalsize   & \cellcolor{green!5}Mulher & \cellcolor{green!5}Gender - General & \cellcolor{green!5}1/19 & \cellcolor{green!5}5.263 \\  \hline
  \cellcolor{green!27}\small Pensando agora acredito que na frase "não se nasce \textbf{mulher}, a sociedade que transforma" é a questão da sociedade sempre falar o que é "\textbf{mulher} e de homem", meio que uma forma de cupcake kkkk já o "nasci \textbf{mulher} e sempre me senti assim" seria a pessoa concordando com essas questões femininas e se identificando mais do que as masculinas na qual foi dada no começo de sua vida\normalsize   & \cellcolor{green!27}Mulher & \cellcolor{green!27}Gender - General & \cellcolor{green!27}3/69 & \cellcolor{green!27}4.348 \\  \hline
  \cellcolor{green!5}\small Acho que o que a sociedade nos impõe é o que a \textbf{mulher} deve vestir, o que a \textbf{mulher} pode ou não fazer. Desde sempre sabemos que "menina brinca de boneca e de casinha" e aquela coisa toda lá...\normalsize   & \cellcolor{green!5}Mulher & \cellcolor{green!5}Gender - General & \cellcolor{green!5}2/39 & \cellcolor{green!5}5.128 \\  \hline
  \cellcolor{green!27}\small A segunda frase de refere a "papéis de gênero", que é a questão de "trabalho pra homem", "trabalho pra \textbf{mulher}", "meninas vestem rosa", "meninos vestem azul", enfim, outra besteira que a sociedade inventou só pra ser mais preconceituosa.\normalsize   & \cellcolor{green!27}Mulher & \cellcolor{green!27}Gender - General & \cellcolor{green!27}1/38 & \cellcolor{green!27}2.632 \\  \hline
  \cellcolor{green!5}\small Existe uma diferença entre aquilo que a sociedade vê como o que é ser \textbf{mulher} e com o que é ser \textbf{mulher} de fato. Nascer \textbf{mulher} biologicamente é cromossomo XX e pepeka, nascer do gênero feminino tem a ver com o cérebro, por exemplo, a \textbf{mulher} trans vai querer ter um corpo feminino, pq o cérebro dela funciona do mesmo jeito que o cérebro de uma \textbf{mulher} cis funciona, e é claro que ter um corpo feminino pode significar muito mais do que ter "peito, bunda" não precisa ser um corpo padrão, mas precisa ser um corpo de \textbf{mulher}, não tem como você convencer uma \textbf{mulher} cis a querer ter um corpo de homem (lê-se corpo afetado pela testosterona) assim como não dá pra você convencer uma \textbf{mulher} trans a gostar de ter um corpo de homem, isso é nascer do gênero feminino, estou tentando pensar em algo que absolutamente todas as pessoas do gênero feminino tem em comum, mas é difícil, estudos mostram que desde bebê meninas geralmente se interessam mais por bonecas e meninos por carrinhos, existem vários estudos que tentam mostrar as diferenças comportamentais entre homens e mulheres, mas a verdade é que esses comportamentos não determinam se alguém é ou não \textbf{mulher}, é bem complexo na verdade, mas sim, existem algumas diferenças de origem evolutiva entre homens e mulheres, o que não significa que existe "coisa de homem" ou "coisa de \textbf{mulher}" significa só que alguns comportamentos são mais típicos em homens do que em mulheres e vice versa, mas a sociedade ainda tem um grande papel nisso, o que leva a sua pergunta. A frase "torna-se \textbf{mulher}" não diz que vc pode criar uma pessoa para ela ser \textbf{mulher}, mas sim que você pode criar uma pessoa para tornar ela àquilo que a sociedade vê como ser \textbf{mulher} (ser mãe, ficar em casa, ser sempre doce, não gostar de matemática, gostar de rosa, etc)\normalsize   & \cellcolor{green!5}Mulher & \cellcolor{green!5}Gender - General & \cellcolor{green!5}13/319 & \cellcolor{green!5}4.075 \\  \hline
  \cellcolor{green!27}\small Alice Almeida isso vem de uma expressão da Simone Beauvoir que diz 'não se nasce \textbf{mulher}, torna-se \textbf{mulher}'. Essa frase está dentro de um livro dela e tem um contexto, não é pra ser aplicada pra qualquer situação. O contexto da frase é o de que os papéis de gênero são criados e ensinados pela sociedade (por exemplo os costumes de a \textbf{mulher} ser a cuidadora da casa, dos filhos, dos enfermos da família, ser atenciosa e doce enquanto o homem é bruto e provedor) e por isso vc se torna \textbf{mulher}, pq vc vai aprendendo com a sociedade a ser aquilo que esperam que vc seja por causa do seu gênero. Não tem a ver com o que ela está falando aqui, de se identificar com o seu gênero desde que vc nasceu ou não. Uma \textbf{mulher} cis pode sim falar que sempre se identificou com o gênero feminino, pq se trata da vida dela e é verdade...a questão da frase é sobre a influência que a sociedade tem e que vai moldando a gente a termos sempre esses papéis por sermos mulheres.\normalsize   & \cellcolor{green!27}Mulher & \cellcolor{green!27}Gender - General & \cellcolor{green!27}4/184 & \cellcolor{green!27}2.174 \\  \hline
  \cellcolor{green!5}\small O sentido da frase é que a \textbf{mulher} precisa lutar contra a opressão e pela liberdade.  Mas ninguém deveria ser oprimido.  O homem nasce livre enquanto a sociedade transforma a \textbf{mulher} em um ser oprimido, dizendo que mulheres não podem fazer uma série de coisas, deixando a \textbf{mulher} em uma posição inferior.  Mas ninguém nasce melhor do que ninguém, todo ser oprimido foi transformado. Opressão não é natural. A imagem de uma \textbf{mulher} submissa ao homem não é natural, foi uma imposição,  uma transformação imposta.\normalsize   & \cellcolor{green!5}Mulher & \cellcolor{green!5}Gender - General & \cellcolor{green!5}4/85 & \cellcolor{green!5}4.706 \\  \hline
  \cellcolor{green!27}\small \@Amanda Franco então todas as mulheres "cis" se identificam com essa descrição do que uma \textbf{mulher} deve ser? cuidadora da casa, dos filhos, dos enfermos da família, atenciosa e doce? Perceba como a ideologia trans é conservadora, misógina e nada progressista.\normalsize   & \cellcolor{green!27}Mulher & \cellcolor{green!27}Gender - General & \cellcolor{green!27}1/41 & \cellcolor{green!27}2.439 \\  \hline
  \cellcolor{green!5}\small \@Amanda Franco querida, eu entendo perfeitamente o que a simone de Beauvoir disse. Eu estou contestando o conceito de CIS. Uma \textbf{mulher} "cis", segundo a ideologia trans, é uma pessoa do sexo feminino que se identifica com os estereótipos do gênero feminino, não é?\normalsize   & \cellcolor{green!5}Mulher & \cellcolor{green!5}Gender - General & \cellcolor{green!5}1/44 & \cellcolor{green!5}2.273 \\  \hline
  \cellcolor{green!27}\small flower power Bom, eu não sou transgênero pra dizer o que eles acham. Mas vou dar aqui a minha opinião: eu não acho que o conceito de cisgênero (uma pessoa que se identifica com o seu gênero de nascença) seja o problema. De fato eu acho que é possível se identificar com ser \textbf{mulher} ou ser homem. A questão pra mim é que hoje o 'feminino' e o 'masculino' são conceitos muito fechados. O que acontece hoje é que pelo conceito de feminino/masculino ser tão restrito, fica difícil pra maioria das mulheres/homens dizerem que se identificam 100 com aquilo que o seu gênero 'deveria ser', mas isso não é um problema do movimento trans, é um problema de \textbf{machismo} mesmo, bem anterior ao movimento trans. Porém, eu acho que uma vez que esses conceitos forem ampliados, essa 'identificação' com o seu gênero vai deixar de ser um problema pra quem é cis. Já pra quem é trans eu acho que a situação é muuuito diferente da de uma pessoa que só não se identifica com alguns aspectos sociais do seu gênero, acho que envolve questões pessoais, que nós que não somos não podemos julgar/entender. O que eu estou querendo dizer é que, por mais que eu como \textbf{mulher} cis não me identifique com algumas características tidas como femininas (tipo ser doce por exemplo), eu estou longe de não me identificar do jeito que uma pessoa trans não se identifica...uma vez que a sociedade abrir a cabeça pro fato de que nem toda \textbf{mulher} precisa ser doce (ou isso ou aquilo) o meu problema vai deixar de existir, mas o da pessoa trans não iria...por mais que os papéis de gênero sejam desconstruídos, a pessoa trans se vê como uma pessoa do outro gênero (incluindo o órgão \textbf{sexual}), então ela ainda assim seria trans.\normalsize   & \cellcolor{green!27}Machismo, Mulher, Sexual & \cellcolor{green!27}Gender - General & \cellcolor{green!27}5/303 & \cellcolor{green!27}1.65 \\  \hline
  \cellcolor{green!5}\small \@flower power eu acho que consigo explicar de uma forma bem simples. A \textbf{mulher} cis se olha no espelho e se contenta consigo (não necessariamente pq pode não gostar das gordurinhas ou sei lá o que. Mas com seu órgão \textbf{sexual}) já o homem transgênero (disse homem pq concluímos que ainda não tenha feito cirurgia ou mudado fisicamente) se olha no espelho e sente que está olhando outra pessoa. E não a si mesmo. Pq não se identifica com aquele órgão \textbf{sexual}.\normalsize   & \cellcolor{green!5}Mulher, Sexual & \cellcolor{green!5}Gender - General & \cellcolor{green!5}3/82 & \cellcolor{green!5}3.659 \\  \hline
  \cellcolor{green!27}\small \@Gloria Lins eu ia curtir mas quem diabos disse que \textbf{mulher} não podia gostar de matemática? Pelo menos na minha família é papel da \textbf{mulher} cuidar das finanças da casa.\normalsize   & \cellcolor{green!27}Mulher & \cellcolor{green!27}Gender - General & \cellcolor{green!27}2/30 & \cellcolor{green!27}6.667 \\  \hline
  \cellcolor{green!5}\small Que vídeo incrível! Karol do céu! Eu queria tanto que as pessoas fossem como você!Sou evangélica desde que nasci e tenho opiniões muito polêmicas quanto a assuntos que para a maioria dos cristãos é bem "\textbf{preto} no \textbf{branco}". Pra mim nunca foi uma coisa simples apontar o dedo para alguém e dizer que essa pessoas está errada, especialmente quando esse "erro" não foi uma escolha ou quando esse "erro" não é algo que possa ser mudado.Como eu posso dizer que uma pessoa como a Thiessa está errada, se ela não escolheu ser como é? Entende?É cruel. É desumano.Ser trans, ser \textbf{homossexual}... Nunca é um caminho fácil. Nenhum ser humano escolheria o caminho mais difícil por pura diversão, entende?É quem as pessoas são, e não quem elas escolheram ser.Sempre debati esses assuntos deixando claro que as pessoas precisam aprender a se colocar no lugar das outras e tentar entender o que as outras pessoas sentem e porque elas sentem. "Amai ao próximo" não vem com condições para esse amor, é simples, é amor, de graça e sem julgamentos.Muito obrigada por esse vídeo Karol e Thiessa!\normalsize   & \cellcolor{green!5}Branco, Homossexual, Preto & \cellcolor{green!5}Ethnicity - Black, Ethnicity - White, Sexual Identity - General & \cellcolor{green!5}3/191 & \cellcolor{green!5}1.571 \\  \hline
  \cellcolor{green!27}\small Sou evangélica no momento não frequento nenhuma igreja pq mesmo as igrejas consideradas mais liberais rola muita hipocrisia com relação a diversidade de genero. O que eu percebo é que eles acolhem os homossexuais, transgêneros e afins mas os coloca no mesmo patamar de pessoas "doentes" como alcoólatras, dependentes químicos e etc. Num primeiro momento essas pessoas se sentem a vontade tipo " até q enfim encontrei  lugar onde as pessoas me aceitam como eu sou" mas assim que se tornaram membros da igreja e não apenas frequentadores começa a pressão do tipo se vc crê  Deus ele vai transformar a sua vida, pra vc aceitar Jesus vc tem q abrir mão dos seus pecados. Eles oram pelos homossexuais pedindo a libertação, essas pessoas são direcionadas a participarem de seminários de cura e libertação como se fosse uma doença. Citam a bíblia pra dizer q isso é errado. Tipo eu te aceito mas pra Deus te aceitar vc tem q mudar e etc. Pra mim embora seja \textbf{mulher} cis, hétero isso nunca foi um tabu. E não vejo as coisas como \textbf{preto} no \textbf{branco}. Muitas pessoas quando escutam q um trans é como se tivesse nascido no corpo errado dizem mas Deus nunca erra. Aí eu questiono, e as pessoas por exemplo que nasceram com os dois sexos? Muitas vezes são registrados como homens porque é o órgão mais evidente e depois quando crescem aparece todas as características femininas. E todas as pessoas que nasceram com alguma má formação ou síndrome. Acredito q realmente Deus nunca erra mas ele colocou cada ser humano no mundo para q nos aprendecemos a amar incondicionalmente. Acho engraçado porque na bíblia tem tantas coisas que Deus "condena" mas que se tornou perfeitamente aceitável perante a sociedade como o divórcio, a castidade e outras coisas no entanto com relação a esse assunto sempre são categóricos. Por esse tipo de pensamento hj em dia estou afastada. Continuo acreditando  em Deus, me considero evangélica mas não tenho disposição para compactuar com tanta hipocrisia. O meu Deus é amor, o meu Deus nos criou para q amassemos e fossemos felizes não para que vivêssemos  em sofrimento e negação. Esse corpo é só matéria. Temos que ser capazes de amar as pessoas independente de ser homem, \textbf{mulher}, cis ou não. Quem se limita a isso perde a oportunidade de se relacionar com pessoas maravilhosas. Thiessa adoro vc. Aprendo muito com vc a cada dia. Me desculpem se  em alguma de minhas colocações eu tenha me expressado mal. Mas o fato é que não só eu respeito como eu amo e me machuca muito ver o preconceito sob qualquer forma. Tenho vontade de chorar quando vejo aquelas mensagens dos haters. Da vontade de trazer todos vcs pra casa e proteger.   Um beijo na teta. " Sempre quis falar isso pra vc."\normalsize   & \cellcolor{green!27}Branco, Mulher, Preto & \cellcolor{green!27}Ethnicity - Black, Ethnicity - White, Gender - General & \cellcolor{green!27}4/471 & \cellcolor{green!27}0.849 \\  \hline
  \cellcolor{green!5}\small Gente eu tô me identificando demais com esses comentários!!! Tenho 16 anos, sempre fui cristã desde criança, mas por conta das pessoas da igreja eu comecei a me afastar com 14 anos, a maioria era preconceituoso, preconceito mesmo! Falavam que todos os gays não seriam salvos se não deixasse de ser \textbf{gay}. Eu sempre me incomodei com esse tipo de coisa desde criança, nunca gostei de apontar o dedo pra ninguém. Antigamente, quando eu tinha uns 8 ou 9 anos eu sentia uma coisa muito boa quando estava conversando com qualquer cristão! Hoje é difícil achar, a maioria só sabe julgar os outros falam mal da roupa das mulheres na igreja e esquecem de pregar a palavra. Nesses últimos dias tava lendo algumas partes da bíblia e senti uma energia muito boa e estou tentando me aproximar de deus mais uma vez❤\normalsize   & \cellcolor{green!5}Gay & \cellcolor{green!5}Sexual Identity - General & \cellcolor{green!5}1/142 & \cellcolor{green!5}0.704 \\  \hline
  \cellcolor{green!27}\small Quanta \textbf{mulher} maravilhosa em um vídeo só ❤️\normalsize   & \cellcolor{green!27}Mulher & \cellcolor{green!27}Gender - General & \cellcolor{green!27}1/8 & \cellcolor{green!27}12.5 \\  \hline
  \cellcolor{green!5}\small Eu to no chão, \textbf{velho} melhor vídeo de 2019 Thiessa como sempre um hino de pessoa!!!!\normalsize   & \cellcolor{green!5}Velho & \cellcolor{green!5}Age - Over 65s & \cellcolor{green!5}1/16 & \cellcolor{green!5}6.25 \\  \hline
  \cellcolor{green!27}\small Gente, vamos acabar com o preconceito! Ser \textbf{homossexual}, trans, cis é só um detalhe, como a cor dos olhos por exemplo🌈🏳️‍🌈\normalsize   & \cellcolor{green!27}Homossexual & \cellcolor{green!27}Sexual Identity - General & \cellcolor{green!27}1/21 & \cellcolor{green!27}4.762 \\  \hline
  \cellcolor{green!5}\small \@Ana Grahl Como assim não é considerado pelo restante? Se não fosse, não existiria oque eles chamam de preconceito, pois nem td é preconceito. Essas pessoas têm que entender que cada um tem um pensamento diferente, criação diferente, tem pessoas que acreditam na palavra da Bíblia, e isso tbm tem que ser respeitado, afinal de contas, não somos obrigados à acreditar em nda, nem achar nada normal, bonito, maravilhoso etc... Mas sim, respeitar, essa é a palavra chave, temos que respeitar uns aos outros, independente de nossa opinião, de orientação \textbf{sexual}, religião, cor e td mais.\normalsize   & \cellcolor{green!5}Sexual & \cellcolor{green!5}Gender - General & \cellcolor{green!5}1/96 & \cellcolor{green!5}1.042 \\  \hline
  \cellcolor{green!27}\small \@Dani Sá não é questão de querer os outros sigam princípios nós não somos donos do nosso nariz nós não surgimos do Big Bang ou do \textbf{macaco} Somos criação de Deus ou seja temos um dono um pai e temos que ter ordem e desescia até pra viver creio que na sua casa seus pais sempre te orientou sob o que vc pode ou não fazer sempre visando seu bem estar por mais que vc não gostasse/aceitasse a opinião deles com Deus não e diferente eles só nos priva de algo que nos coisa dano ou afeta o nosso próximo Pois ele é O AMOR ou seja a bondade em pessoa.Ao ver o quanto minha geração está sega não posso e nem vou me calar a verdade precisa ser dita porque quando partimos para eternidade aí sim será tarde de mais para se arrepender.Jesus te ama.\normalsize   & \cellcolor{green!27}Macaco & \cellcolor{green!27}Ethnicity - Black & \cellcolor{green!27}1/148 & \cellcolor{green!27}0.676 \\  \hline
  \cellcolor{green!5}\small Aos pés de Jesus. Se ele fosse tão 'amor' assim as escolhas que trás a felicidade de sua criação seria o certo. É incrível como vocês conseguem distorcer as coisas desse jeito sério hehejekekek devemos ter nossos próprios pensamentos e não seguir algo que tal pessoa disse que se tu não seguir tu é um pecador, negado, isolado e vai queimar no inferno porquê tal pessoa disse que não é certo veio desse jeito e vai ser desse jeito. \textbf{Puta} que pariu né não sei se é pura ignorância ou lealdade \textbf{cega}.\normalsize   & \cellcolor{green!5}Cega, Puta & \cellcolor{green!5}Gender - Female sexuality, Physical Identity - Physical (and Mental) Impairments & \cellcolor{green!5}2/92 & \cellcolor{green!5}2.174 \\  \hline
  \cellcolor{green!27}\small rae não necessariamente a pessoa tem que fazer procedimentos estéticos, isso não é regra. Uma \textbf{mulher} trans não é menos \textbf{mulher} por não ter seios, um homem trans não é menos homem por não ter barba. E isso também serve p l órgão genital.\normalsize   & \cellcolor{green!27}Mulher & \cellcolor{green!27}Gender - General & \cellcolor{green!27}2/44 & \cellcolor{green!27}4.545 \\  \hline
  \cellcolor{green!5}\small \@Amanda Vitória não foi isso que eu disse, eu expliquei a diferença entre transgênero e transsexual, uma \textbf{mulher} transgênero é uma \textbf{mulher}, e uma \textbf{mulher} transsexual também é uma \textbf{mulher}. :)\normalsize   & \cellcolor{green!5}Mulher & \cellcolor{green!5}Gender - General & \cellcolor{green!5}4/31 & \cellcolor{green!5}12.903 \\  \hline
  \cellcolor{green!27}\small Transgênero é a classificação que une todas as manifestações de identidades (transgêneras) que não se sentem correspondidas e representadas pelo sexo biológico, a qual foi designado ao  nascimento, e/ou pelo gênero que lhe foi atribuído devido ao sexo. Logo, esta classificação engloba homens trans, mulheres trans, travestis, não-binários, gender fluid...A diferença entre mulheres trans e travestis, na fala popular, deve-se ao fato de se querer fazer ou não a cirurgia de redesignação \textbf{sexual}. As travestis são pessoas transgêneras que possuem identidade de gênero diferente ao seu sexo atribuído, se expressam assim, porém não sentem necessidade de alterar sua genitália. Não possuem disforia quanto à isso. Já as mulheres trans são as que optam por fazer e querem fazer pois sentem uma disforia enorme, muitas vezes, com esta parte de seu corpo.\normalsize   & \cellcolor{green!27}Sexual & \cellcolor{green!27}Gender - General & \cellcolor{green!27}1/132 & \cellcolor{green!27}0.758 \\  \hline
  \cellcolor{green!5}\small na verdade, uma pessoa transgenero é uma pessoa que transcende ou transiciona de um gênero para outro ou para nenhum.uma pessoa \textbf{transexual} é uma pessoa que não se identifica com seu órgão genital biológico, logo não se identifica com o gênero imposto.então, no transgenero, é incluído drag queens, travestis, transexuais, etc.basicamente, todo \textbf{transexual} é transgenero, mas nem todo transgenero é \textbf{transexual}.\normalsize   & \cellcolor{green!5}Transexual & \cellcolor{green!5}Sexual Identity - Transexuality & \cellcolor{green!5}3/64 & \cellcolor{green!5}4.688 \\  \hline
  \cellcolor{green!27}\small não existe "gênero biológico". PQP prestem atenção no que vcs estão falando. não veem que isso não faz sentido? é SEXO BIOLÓGICO. e sexo é IMUTÁVEL. O que é se "identificar" com um sexo ou outro? Se identificar com os estereótipos associados a homens ou mulheres? Vocês acham isso progressista? Um homem vira \textbf{mulher} pq colocou silicone nos seios, deixou o cabelo crescer e começou a andar rebolando? E se isso não é necessário então homem de barba, voz grossa, pinto é \textbf{mulher} porque ele sente no fundo do coração dele que a alma dele é feminina??? ISSO NÃO FAZ SENTIDO.\normalsize   & \cellcolor{green!27}Mulher & \cellcolor{green!27}Gender - General & \cellcolor{green!27}2/101 & \cellcolor{green!27}1.98 \\  \hline
  \cellcolor{green!5}\small \@flower power Identidade de gênero é solidificada em um experiência profundamente íntima e individual do ser perante o meio social. E lógico que a transgeneridade é progressista e revolucionária, pois rompe com o conceito do cis-tema heteronormativo e determinista. Se entende que gênero nada tem a ver com sexo, pois refletem em questões diferentes: pelo contrário, a transgeneridade incita a reflexão de que o gênero é fluido e pode se transitar entre um e outro a depender das minhas experiências. Aceitar o determinismo biológico do sexo enquanto determinante e essencial nas relações de identidade e expressões da personalidade é abster o ser de sua subjetividade, é desqualificar o sujeito para glorificar, em superioridade, o corpo e tipos de cromossomos. Além disso, a transgeneridade rompe com o cis-tema de binarismo de gênero, mostrando que não existem somente "homem" e "\textbf{mulher}", mas pessoas agêneras, gender fluid, não binários; existe um leque de relações com o social, de expressão da identidade. Gênero é social, sexo é biológico.\normalsize   & \cellcolor{green!5}Mulher & \cellcolor{green!5}Gender - General & \cellcolor{green!5}1/164 & \cellcolor{green!5}0.61 \\  \hline
  \cellcolor{green!27}\small \@Bianca Duarte "Identidade de gênero é solidificada em um experiência profundamente íntima e individual do ser perante o meio social." E como você prova a existência disso? O que existem são personalidades diferentes. Cada pessoa tem uma personalidade. Esse monte de nome inventado não serve pra nada. Ninguém tá falando em determinismo biológico pra determinar expressões de personalidade aqui. Cada um se vista e se comporte como quiser, mas isso não \textbf{muda} o sexo de ninguém.\normalsize   & \cellcolor{green!27}Muda & \cellcolor{green!27}Physical Identity - Physical (and Mental) Impairments & \cellcolor{green!27}1/76 & \cellcolor{green!27}1.316 \\  \hline
  \cellcolor{green!5}\small \@flower power Provo empiricamente pela existência de pessoas transexuais. Existem expressões e performatividades de gênero que são construídas socialmente e não pela natureza. Sexo de nada tem a ver com gênero. O que existe de concreto é construções de identidade, processo de autodescobrimento e revelação de quem é, inclusive sendo transgênero. À medida que você fala que um "homem", mesmo performando e se autodeterminando enquanto \textbf{mulher} nunca vai ser uma você está encarando o sexo biológico enquanto determinante nas interações que o ser pode ter perante o universo social.\normalsize   & \cellcolor{green!5}Mulher & \cellcolor{green!5}Gender - General & \cellcolor{green!5}1/89 & \cellcolor{green!5}1.124 \\  \hline
  \cellcolor{green!27}\small \@Bianca Duarte  \textbf{mulher} = pessoa do sexo feminino. simples assim. qualquer outra definição é incorreta. por que que pessoas trans são autorizadas a mudar o sexo em documentos oficiais se sexo e gênero não tem nada a ver?\normalsize   & \cellcolor{green!27}Mulher & \cellcolor{green!27}Gender - General & \cellcolor{green!27}1/38 & \cellcolor{green!27}2.632 \\  \hline
  \cellcolor{green!5}\small \@flower power \textbf{Mulher} é algo muito mais social do que biológico e este seu discurso beira a hipocrisia. Afinal, ao você reconhecer uma \textbf{mulher}, você não vê os pares de cromossomos, tipos e fluxos hormonais, genitália, órgãos característicos, mas sim a performatividade de gênero, os estereótipos que pertencem ao universo feminino: logo se identifica uma \textbf{mulher} socialmente. É algo SOCIAL. E pessoas transgêneras são autorizadas a mudar o sexo SIM, mas isso não torna sexo e gênero semelhantes. Elas podem mudar o sexo simplesmente por não sentirem representadas, correspondidas com o seu sexo, muitas vezes optando pela cirurgia de redesignação \textbf{sexual}. Um exemplo de que sexo e gênero são coisas diferentes se mostra através das travestis, que se sentem confortáveis com sua genitália, porém se identificam enquanto mulheres e optam por não fazer a cirurgia. Os seus conceitos estão bem equivocados.\normalsize   & \cellcolor{green!5}Mulher, Sexual & \cellcolor{green!5}Gender - General & \cellcolor{green!5}4/141 & \cellcolor{green!5}2.837 \\  \hline
  \cellcolor{green!27}\small \@Bianca Duarte Hipocrisia é saber que sexo e gênero são coisas diferentes e mesmo assim insistir que pessoas trans tenham direito de mudar o SEXO em documentos oficiais. Isso se chama FRAUDE. Mulheres não são "performatividade de gênero". Mulheres são cromossomos, fluxos hormonais, genitália, órgãos característicos etc. Mulheres são uma REALIDADE MATERIAL, não vontade de calçar salto alto, vestir saia e usar maquiagem na cara. Espero que você não seja \textbf{mulher}, porque se for está fazendo um papel ridículo.Tome vergonha na cara.\normalsize   & \cellcolor{green!27}Mulher & \cellcolor{green!27}Gender - General & \cellcolor{green!27}1/82 & \cellcolor{green!27}1.22 \\  \hline
  \cellcolor{green!5}\small \@flower power a pessoa \textbf{ridícula} aqui é vc. Eu em\normalsize   & \cellcolor{green!5}Ridícula & \cellcolor{green!5}Gender - Female sexuality & \cellcolor{green!5}1/10 & \cellcolor{green!5}10.0 \\  \hline
  \cellcolor{green!27}\small \@flower power A única sem noção é você que em vez de contrargumentar, ataca a minha pessoa. Isso só mostra o quanto é ignorante. Lógico que o conceito do que é "\textbf{MULHER}" é algo fruto de construções sociais amarradas em universos rígidos de gêneros e seus respectivos papéis sociais. Afinal, ao você reconhecer a \textbf{mulher} você NÃO CONSEGUE ENXERGAR os tipos de cromossomos, fluxos e tipos hormonais, genitália etc. Pelo contrário, você enxerga pelos estereótipos de gênero associados ao que é \textbf{mulher} e homem. Sabia que existem mulheres que possuem níveis altos de testosterona, que não possuem ovários e nem útero, que são hermafroditas...? Daí por isso não são mulheres? É algo relativo à identidade e isto nunca vai mudar. Acorda para vida e para de falar besteira que não sou obrigada.\normalsize   & \cellcolor{green!27}Mulher & \cellcolor{green!27}Gender - General & \cellcolor{green!27}3/132 & \cellcolor{green!27}2.273 \\  \hline
  \cellcolor{green!5}\small \@Bianca Duarte \textbf{Mulher} é uma pessoa do sexo feminino e homem uma pessoa do sexo masculino. Mulheres não tem pênis e homens não tem vagina. Mulheres e homens não são do mesmo sexo. Afinal, se fossem a mesma coisa não existiriam orientações sexuais já que mulheres e homens precisam ser diferentes para que existam heteros, gays e lésbicas. Ninguém vira \textbf{mulher} só porque diz se identificar como uma.\normalsize   & \cellcolor{green!5}Mulher & \cellcolor{green!5}Gender - General & \cellcolor{green!5}2/68 & \cellcolor{green!5}2.941 \\  \hline
  \cellcolor{green!27}\small \@Bianca Duarte Até parece que você não consegue reconhecer uma \textbf{mulher} pela aparência. Vai dizer que homens e mulheres são iguais? Se uma transsexual parece \textbf{mulher} é porque tomou hormônios, pois sem eles teria aparência masculina. Não se faz de sonsa.\normalsize   & \cellcolor{green!27}Mulher & \cellcolor{green!27}Gender - General & \cellcolor{green!27}2/41 & \cellcolor{green!27}4.878 \\  \hline
  \cellcolor{green!5}\small \@Samantha D. Sexo biológico é diferente de gênero. Já falei milhões de vezes. Você se deu ao trabalho de ler ou apenas é \textbf{cega} mesmo?  Homens e mulheres lógico que não são iguais, existem construções e papéis sociais diferentes, bem como performatividades de gênero. A autodeterminação de uma pessoa enquanto \textbf{mulher} é algo válido sim, pois nada mais é do que um sentimento de ser, de pertencer, de se identificar. Existem pessoas hermafroditas que possuem os dois órgãos sexuais e pelo sistema de classificação que funciona em nosso país (binarismo homem ou \textbf{mulher}) tem q se adotar um dos gêneros. Acorda para vida. Você ignorar a subjetividade do ser para exaltar o determinismo biológico só é fonte para muitos discursos de ódio e inclusive de suicídios. Tenha mais responsabilidade com o que fala. E é muita hipocrisia você falar tanto em biologia, afinal a personalidade, que inclui a maneira de sentir, agir e pensar, é algo processado no cérebro, órgão biológico e natural. É preciso ter COERÊNCIA sobre o que se fala. O fato de eu me identificar enquanto \textbf{mulher} ou homem diz respeito à conexões cerebrais, sinapses nervosas formadas a partir de experiências que se tem durante a vida.\normalsize   & \cellcolor{green!5}Cega, Mulher & \cellcolor{green!5}Gender - General, Physical Identity - Physical (and Mental) Impairments & \cellcolor{green!5}4/200 & \cellcolor{green!5}2.0 \\  \hline
  \cellcolor{green!27}\small \@Samantha D. Para que tá feio. Você distorceu tudo que eu falei. Se mulheres são legitimadas por suas genitálias, como você reconhece uma \textbf{mulher} na rua? Por estereótipos associados ao gênero feminino. Não necessariamente uma \textbf{mulher} \textbf{transexual} tenha que tomar hormônios para se parecer como \textbf{mulher}, afinal existem problemas na hipófise de algumas pessoas que resulta em uma inibição da testosterona, responsável pela puberdade dos homens cis; logo ela não desenvolve características secundárias ditas masculinas.\normalsize   & \cellcolor{green!27}Mulher, Transexual & \cellcolor{green!27}Gender - General, Sexual Identity - Transexuality & \cellcolor{green!27}4/75 & \cellcolor{green!27}5.333 \\  \hline
  \cellcolor{green!5}\small Aaaa😻😻 \textbf{velho} cada dia eu lhe amo mais,Obrigado por estar na minha vida,e ser o motivo de defender a causa \textbf{LGBT}! Te amo!\normalsize   & \cellcolor{green!5}LGBT, Velho & \cellcolor{green!5}Age - Over 65s, Sexual Identity - General & \cellcolor{green!5}2/23 & \cellcolor{green!5}8.696 \\  \hline
  \cellcolor{green!27}\small Vocês são "sisters", e não "brothers" (Risos). Para mim, essa moça trans (Thiessita) é completamente \textbf{mulher}. Por que ela não entra na justiça para colocar o nome e o sexo feminino na Certidão de Nascimento e trocar todos os documentos? Assim, não seria mais prejudicada por pessoas preconceituosas. Beijos!\normalsize   & \cellcolor{green!27}Mulher & \cellcolor{green!27}Gender - General & \cellcolor{green!27}1/49 & \cellcolor{green!27}2.041 \\  \hline
  \cellcolor{green!5}\small Tatá Joaninha   Discordo! Documentos tem que ter registros originais para passar como originais. Isso seria quase a mesma coisa que registrar uma mentira para passar como uma verdade. Nos nossos documentos, a nossa \textbf{idade} registrada é a do momento em que a gente nasceu e não aquela que a gente quer ou se identifica posteriormente. O que deveria fazer é deixar bem claro, tipo assim:SEXO:  trans feminino Quanto ao preconceito (que eu não tenho) existe outras formas de acabar, e acho que já está diminuindo\normalsize   & \cellcolor{green!5}Idade & \cellcolor{green!5}Age - General & \cellcolor{green!5}1/86 & \cellcolor{green!5}1.163 \\  \hline
  \cellcolor{green!27}\small Você é um serumano tão lindo,uma \textbf{mulher} linda,uma pessoa evoluída ❤️\normalsize   & \cellcolor{green!27}Mulher & \cellcolor{green!27}Gender - General & \cellcolor{green!27}1/11 & \cellcolor{green!27}9.091 \\  \hline
  \cellcolor{green!5}\small THIESSA MARAVILHOSAA AAAAAAAAA HINO DE \textbf{MULHER} 💖\normalsize   & \cellcolor{green!5}Mulher & \cellcolor{green!5}Gender - General & \cellcolor{green!5}1/7 & \cellcolor{green!5}14.286 \\  \hline
  
\end{longtable}
\end{center}


\centering\textbf{\large \hypertarget{Table 2}{Table 2}: Summary of the results per sociolinguistic variable 
}
\newcolumntype{C}[2]{>{\centering\arraybackslash}p{#1}}
\begin{center}
\setlength\mylength{\dimexpr\textwidth - 1\arrayrulewidth - 40\tabcolsep}
\begin{longtable}{|C{.50\mylength}|C{.30\mylength}|C{.15\mylength}|C{.15\mylength}|C{.15\mylength}|}
\hline
\textbf{Sociolinguistic variables (Hiper - Hipo)} & \textbf{KeyWords} & \textbf{Number of occurrences} & \textbf{Frequency}  & \textbf{Frequency(\%)} \\
\hline\multirow{1}{*}{\cellcolor{red!27}Nationality - Chinese}  & \cellcolor{red!27}Amarelo & \cellcolor{red!27}9 & \cellcolor{red!27}9/34575& \cellcolor{red!27}0.03 \\  \hline
  \multirow{1}{*}{\cellcolor{red!5}Nationality - Japanese}  & \cellcolor{red!5}Amarelo & \cellcolor{red!5}9 & \cellcolor{red!5}9/34575& \cellcolor{red!5}0.03 \\  \hline
  \multirow{1}{*}{\cellcolor{red!27}Gender - General}  & \cellcolor{red!27}Mulher, Machista, Sexual, Misoginia, Misógino, Patriarcado, Machismo & \cellcolor{red!27}327 & \cellcolor{red!27}327/34575& \cellcolor{red!27}0.95 \\  \hline
  \multirow{1}{*}{\cellcolor{red!5}Sexual Identity - Transexuality}  & \cellcolor{red!5}Transexual, Transição, Travesti, Transgénero & \cellcolor{red!5}23 & \cellcolor{red!5}23/34575& \cellcolor{red!5}0.06999999999999999 \\  \hline
  \multirow{1}{*}{\cellcolor{red!27}Physical Identity - Physical (and Mental) Impairments}  & \cellcolor{red!27}Muda, Idiota, Imbecil, Corcunda, Cega & \cellcolor{red!27}18 & \cellcolor{red!27}18/34575& \cellcolor{red!27}0.05 \\  \hline
  \multirow{1}{*}{\cellcolor{red!5}Sexual Identity - General}  & \cellcolor{red!5}LGBT, Gay, homofóbico, Homossexual & \cellcolor{red!5}57 & \cellcolor{red!5}57/34575& \cellcolor{red!5}0.16 \\  \hline
  \multirow{1}{*}{\cellcolor{red!27}Age - General}  & \cellcolor{red!27}Idade & \cellcolor{red!27}8 & \cellcolor{red!27}8/34575& \cellcolor{red!27}0.02 \\  \hline
  \multirow{1}{*}{\cellcolor{red!5}Gender - Female sexuality}  & \cellcolor{red!5}Puta, Prostituta, Ridícula & \cellcolor{red!5}8 & \cellcolor{red!5}8/34575& \cellcolor{red!5}0.02 \\  \hline
  \multirow{1}{*}{\cellcolor{red!27}Sexual Identity - Male homosexuality}  & \cellcolor{red!27}Baitola, Aberração & \cellcolor{red!27}7 & \cellcolor{red!27}7/34575& \cellcolor{red!27}0.02 \\  \hline
  \multirow{1}{*}{\cellcolor{red!5}Age - Over 65s}  & \cellcolor{red!5}Velho, Velha & \cellcolor{red!5}4 & \cellcolor{red!5}4/34575& \cellcolor{red!5}0.01 \\  \hline
  \multirow{1}{*}{\cellcolor{red!27}Ethnicity - Native American}  & \cellcolor{red!27}Vermelho & \cellcolor{red!27}1 & \cellcolor{red!27}1/34575& \cellcolor{red!27}0.0 \\  \hline
  \multirow{1}{*}{\cellcolor{red!5}Ideological and Political Identity - General}  & \cellcolor{red!5}Vermelho & \cellcolor{red!5}1 & \cellcolor{red!5}1/34575& \cellcolor{red!5}0.0 \\  \hline
  \multirow{1}{*}{\cellcolor{red!27}Ethnicity - Black}  & \cellcolor{red!27}Preto, Negro, Negra, Macaco & \cellcolor{red!27}8 & \cellcolor{red!27}8/34575& \cellcolor{red!27}0.02 \\  \hline
  \multirow{1}{*}{\cellcolor{red!5}Ethnicity - White}  & \cellcolor{red!5}Branco & \cellcolor{red!5}3 & \cellcolor{red!5}3/34575& \cellcolor{red!5}0.01 \\  \hline
  \multirow{1}{*}{\cellcolor{red!27}Physical Identity - Physical Features}  & \cellcolor{red!27}Esqueleto, Bola & \cellcolor{red!27}2 & \cellcolor{red!27}2/34575& \cellcolor{red!27}0.01 \\  \hline
  \multirow{1}{*}{\cellcolor{red!5}Gender - Female age and physical appearance}  & \cellcolor{red!5}Velha & \cellcolor{red!5}1 & \cellcolor{red!5}1/34575& \cellcolor{red!5}0.0 \\  \hline
  \multirow{1}{*}{\cellcolor{red!27}Ethnicity - General}  & \cellcolor{red!27}Racismo & \cellcolor{red!27}1 & \cellcolor{red!27}1/34575& \cellcolor{red!27}0.0 \\  \hline
  
\end{longtable}
\end{center}


\textbf{\Large Result analysis:}

\begin{itemize}\item Taking into account the words that were detected, we can reach the conclusion these comments are associated with : : Nationality - Chinese;Nationality - Japanese;Gender - General;Sexual Identity - Transexuality;Physical Identity - Physical (and Mental) Impairments;Sexual Identity - General;Age - General;Gender - Female sexuality;Sexual Identity - Male homosexuality;Age - Over 65s;Ethnicity - Native American;Ideological and Political Identity - General;Ethnicity - Black;Ethnicity - White;Physical Identity - Physical Features;Gender - Female age and physical appearance;Ethnicity - General;%.

\item The percentage of hate speech related words is 1.4085.

\item Considering that the variable \textbf{Gender - General} has the most occurences in the post, we can interpret that this is the predominant hate speech.

\item Overall there were 489/1312 occurences of hate speech related comments.\end{itemize}\end{document}