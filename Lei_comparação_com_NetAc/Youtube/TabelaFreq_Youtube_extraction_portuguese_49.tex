\documentclass[11pt]{article}
\ExplSyntaxOn
\let\tl_length:n\tl_count:n
\ExplSyntaxOff
\usepackage{graphicx}
\usepackage{multirow}
\usepackage{colortbl}
\usepackage{longtable, array}
\usepackage{hyperref}
\usepackage[usenames,dvipsnames,svgnames,table]{xcolor}
\newlength\mylength
\usepackage[legalpaper, landscape, margin=0.8in]{geometry}
\newcommand{\MinNumber}{0}
\begin{document}

\textbf {\huge In Youtube\_extraction\_portuguese\_49.json :}\newline \par\Large\textbf {Title: \large  Ciganos portugueses no Brasil }\newline {\par\large --- Table  1: Summary of the results per comment; }

 {\par\large --- \hyperlink{Table 2}{\textcolor{blue}{\underline{Table 2}}}: Summary of the results per sociolinguistic variable;}\newline \normalsize\newline

\centering\textbf{\large Table  1: Summary of the results per comment 
}
\newcommand{\MaxNumber}{0}%
\newcommand{\ApplyGradient}[1]{%
\pgfmathsetmacro{\PercentColor}{100.0*(#1-\MinNumber)/(\MaxNumber-\MinNumber)}
\xdef\PercentColor{\PercentColor}%
\cellcolor{LightSpringGreen!\PercentColor!LightRed}{#1}
}
\newcolumntype{C}[2]{>{\centering\arraybackslash}p{#1}}
\begin{center}
\setlength\mylength{\dimexpr\textwidth - 1\arrayrulewidth - 50\tabcolsep}
\begin{longtable}{|C{.65\mylength}|C{.30\mylength}|C{.12\mylength}|C{.12\mylength}|C{.12\mylength}|}
\hline
\textbf{Comment} & \textbf{KeyWords} & \textbf{Sociolinguistic variables (Hiper - Hipo)}  & \textbf{Hate Speech Frequency} & \textbf{Hate Speech Frequency(\%)} \\
\hline\cellcolor{green!27}\small Paulo Ribas vai aprender a escrever. \textbf{Analfabeto}.\normalsize   & \cellcolor{green!27}Analfabeto & \cellcolor{green!27}Social Class - Working class & \cellcolor{green!27}1/7 & \cellcolor{green!27}14.286 \\  \hline
  \cellcolor{green!5}\small Manda esses fila da \textbf{puta} vim pra cá Piauí . Nois aqui corta na bala e joga no rio estrangeiros fila da puta\normalsize   & \cellcolor{green!5}Puta & \cellcolor{green!5}Gender - Female sexuality & \cellcolor{green!5}2/23 & \cellcolor{green!5}8.696 \\  \hline
  \cellcolor{green!27}\small So se vor com seu cu fila da puta\normalsize   & \cellcolor{green!27}Puta & \cellcolor{green!27}Gender - Female sexuality & \cellcolor{green!27}1/9 & \cellcolor{green!27}11.111 \\  \hline
  \cellcolor{green!5}\small Só comentários \textbf{racista} somos portugueses com muito gosto à mais de 500 anos que vivemos em Portugal\normalsize   & \cellcolor{green!5}Racista & \cellcolor{green!5}Ethnicity - General & \cellcolor{green!5}1/17 & \cellcolor{green!5}5.882 \\  \hline
  \cellcolor{green!27}\small Samucabonny seu cabrao se te panhase te fazia aus bocadinhos seu filha da \textbf{puta} a qui não a racismo\normalsize   & \cellcolor{green!27}Puta, Racismo & \cellcolor{green!27}Ethnicity - General, Gender - Female sexuality & \cellcolor{green!27}2/19 & \cellcolor{green!27}10.526 \\  \hline
  \cellcolor{green!5}\small D holmes deixa de ser \textbf{idiota} e otario aqui no brasil se o cara nao presta a mae e o pai nao e culpado ainda assim ele nao deixa de ser brasileiro conheço 3 familia \textbf{imigrante} de porto vila real os garotos chegaram quase todos com 5 anos hoje tanto eles e seus filhos sao brasucas e sem sotaqueee presta atencao seu preconceituoso duma figa estes ciganos sao legitimo portugueses mesmo nao prestando\normalsize   & \cellcolor{green!5}Idiota, Imigrante & \cellcolor{green!5}Nationality - General, Physical Identity - Physical (and Mental) Impairments & \cellcolor{green!5}2/73 & \cellcolor{green!5}2.74 \\  \hline
  
\end{longtable}
\end{center}


\centering\textbf{\large \hypertarget{Table 2}{Table 2}: Summary of the results per sociolinguistic variable 
}
\newcolumntype{C}[2]{>{\centering\arraybackslash}p{#1}}
\begin{center}
\setlength\mylength{\dimexpr\textwidth - 1\arrayrulewidth - 40\tabcolsep}
\begin{longtable}{|C{.50\mylength}|C{.30\mylength}|C{.15\mylength}|C{.15\mylength}|C{.15\mylength}|}
\hline
\textbf{Sociolinguistic variables (Hiper - Hipo)} & \textbf{KeyWords} & \textbf{Number of occurrences} & \textbf{Frequency}  & \textbf{Frequency(\%)} \\
\hline\multirow{1}{*}{\cellcolor{red!27}Social Class - Working class}  & \cellcolor{red!27}Analfabeto & \cellcolor{red!27}1 & \cellcolor{red!27}1/491& \cellcolor{red!27}0.2 \\  \hline
  \multirow{1}{*}{\cellcolor{red!5}Gender - Female sexuality}  & \cellcolor{red!5}Puta & \cellcolor{red!5}3 & \cellcolor{red!5}3/491& \cellcolor{red!5}0.61 \\  \hline
  \multirow{1}{*}{\cellcolor{red!27}Ethnicity - General}  & \cellcolor{red!27}Racista, Racismo & \cellcolor{red!27}2 & \cellcolor{red!27}2/491& \cellcolor{red!27}0.41000000000000003 \\  \hline
  \multirow{1}{*}{\cellcolor{red!5}Nationality - General}  & \cellcolor{red!5}Imigrante & \cellcolor{red!5}1 & \cellcolor{red!5}1/491& \cellcolor{red!5}0.2 \\  \hline
  \multirow{1}{*}{\cellcolor{red!27}Physical Identity - Physical (and Mental) Impairments}  & \cellcolor{red!27}Idiota & \cellcolor{red!27}1 & \cellcolor{red!27}1/491& \cellcolor{red!27}0.2 \\  \hline
  
\end{longtable}
\end{center}


\textbf{\Large Result analysis:}

\begin{itemize}\item Taking into account the words that were detected, we can reach the conclusion these comments are associated with : : Social Class - Working class;Gender - Female sexuality;Ethnicity - General;Nationality - General;Physical Identity - Physical (and Mental) Impairments;%.

\item The percentage of hate speech related words is 1.6293.

\item Considering that the variable \textbf{Gender - Female sexuality} has the most occurences in the post, we can interpret that this is the predominant hate speech.

\item Overall there were 9/32 occurences of hate speech related comments.\end{itemize}\end{document}