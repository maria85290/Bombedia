\documentclass[11pt]{article}
\ExplSyntaxOn
\let\tl_length:n\tl_count:n
\ExplSyntaxOff
\usepackage{graphicx}
\usepackage{multirow}
\usepackage{colortbl}
\usepackage{longtable, array}
\usepackage{hyperref}
\usepackage[usenames,dvipsnames,svgnames,table]{xcolor}
\newlength\mylength
\usepackage[legalpaper, landscape, margin=0.8in]{geometry}
\newcommand{\MinNumber}{0}
\begin{document}

\textbf {\huge In youtube\_extraction\_english\_164.json :}\newline \par\Large\textbf {Title: \large  Students Pranked During Teacher's Lesson on Ageism | Field Tripped }\newline {\par\large --- Table  1: Summary of the results per comment; }

 {\par\large --- \hyperlink{Table 2}{\textcolor{blue}{\underline{Table 2}}}: Summary of the results per sociolinguistic variable;}\newline \normalsize\newline

\centering\textbf{\large Table  1: Summary of the results per comment 
}
\newcommand{\MaxNumber}{0}%
\newcommand{\ApplyGradient}[1]{%
\pgfmathsetmacro{\PercentColor}{100.0*(#1-\MinNumber)/(\MaxNumber-\MinNumber)}
\xdef\PercentColor{\PercentColor}%
\cellcolor{LightSpringGreen!\PercentColor!LightRed}{#1}
}
\newcolumntype{C}[2]{>{\centering\arraybackslash}p{#1}}
\begin{center}
\setlength\mylength{\dimexpr\textwidth - 1\arrayrulewidth - 50\tabcolsep}
\begin{longtable}{|C{.65\mylength}|C{.30\mylength}|C{.12\mylength}|C{.12\mylength}|C{.12\mylength}|}
\hline
\textbf{Comment} & \textbf{KeyWords} & \textbf{Sociolinguistic variables (Hiper - Hipo)}  & \textbf{Hate Speech Frequency} & \textbf{Hate Speech Frequency(\%)} \\
\hline\cellcolor{green!27}\small There is no doubt that teacher are targeted with PAR by administration because of their \textbf{age}. Where I live, we can't even get ahold of the PAR data, but we know PAR is \textbf{ageist}. You can see the data if you click here and scroll down: [LINK] \normalsize   & \cellcolor{green!27}Age, Ageist & \cellcolor{green!27}Age - General & \cellcolor{green!27}2/47 & \cellcolor{green!27}4.255 \\  \hline
  \cellcolor{green!5}\small 5:58 well luckily you still look young. Also the \textbf{old} teacher was boring.\normalsize   & \cellcolor{green!5}Old & \cellcolor{green!5}Age - Over 65s & \cellcolor{green!5}1/13 & \cellcolor{green!5}7.692 \\  \hline
  \cellcolor{green!27}\small Moral of the story: "Supposed youth \textbf{ageism} and discrimination is solved with.... skin moisturizer!"\normalsize   & \cellcolor{green!27}Ageism & \cellcolor{green!27}Age - General & \cellcolor{green!27}1/14 & \cellcolor{green!27}7.143 \\  \hline
  \cellcolor{green!5}\small I'm 70 and I can tell you \textbf{ageism} IS a thing.  Once you hit like 40, you are \textbf{OLD}.  I remember I was out shopping when I was 40.  I was looking at kids clothes...my daughter was 4 and my son was 7, but they weren't with me.  A lady shopper near me looked over and said "Are you a grandmother?"  I was totally flabbergasted!  How in the world could she think that?  It then occurred to me that, in fact, I could be a grandmother at 40.  What a wake up call.  I was carded for real until i was 35, so this just threw me for a loop.  At my current \textbf{age}, younger people always seem to assume I'm senile and set in my ways.  Come on!  I have kept myself in shape and I keep up with current clothing trends (although I do go for looks that would suit a 30 year \textbf{old} and not a teen).  I'm on Facebook, Twitter, Pinterest and Instagram.  I have been playing Pokemon GO since it came out, and I used to be in Second Life.  In other words, I think young.  My Great Aunt used to wear orthopedic shoes and house dresses...typical \textbf{old} lady clothes...I swore I would never look like that, and I don't.  I think I relate to younger people.  I have a 5 year \textbf{old} Granddaughter, and I hope when I die, she won't remember me as this \textbf{old} lady...in fact, I hope she will say "WOW, she was too young to die".  I'm afraid \textbf{ageism} isn't going away, but maybe if people will start giving someone a chance before making assumptions, that would be a start.\normalsize   & \cellcolor{green!5}Age, Ageism, Old & \cellcolor{green!5}Age - General, Age - Over 65s & \cellcolor{green!5}8/280 & \cellcolor{green!5}2.857 \\  \hline
  \cellcolor{green!27}\small I also think that students tend to like substitutes that they see as "easy." When my students tell me they loved the substitute they had I always ask them why. Their response is pretty much the same every time: "he/she let us do  ."  Of course, I don't know the students' motivations in this class, but just an observation from my own experience. I also agree with the top commenter, that if it were a real experiment, the teachers would have had the same lesson. Then you really test if \textbf{age} played a factor.\normalsize   & \cellcolor{green!27}Age & \cellcolor{green!27}Age - General & \cellcolor{green!27}1/94 & \cellcolor{green!27}1.064 \\  \hline
  \cellcolor{green!5}\small It's not \textbf{ageism}.  It's because students are not in school to learn, they're there to socialize, they're stupid.  So in turn they want to stupid person 'teaching' them so they don't have to do work.\normalsize   & \cellcolor{green!5}Ageism & \cellcolor{green!5}Age - General & \cellcolor{green!5}1/35 & \cellcolor{green!5}2.857 \\  \hline
  \cellcolor{green!27}\small Ageism is a real thing, this has already been studied 🤔 Love the concept of exploring it within a class context, but do it with accurate questions in mind! It's equally important to teach viewers about setting up an scientifically plausible premise 👍\normalsize   & \cellcolor{green!27}Ageism & \cellcolor{green!27}Age - General & \cellcolor{green!27}1/43 & \cellcolor{green!27}2.326 \\  \hline
  \cellcolor{green!5}\small This experiment doesn't work correctly. Because the younger teacher approached the students with something they could relate with, tv shows, and asked them to interact. Where the older teacher told them to read definitions and watch videos. I find that teachers (no matter their \textbf{age}) who interact with their students, will always get a much better student response.\normalsize   & \cellcolor{green!5}Age & \cellcolor{green!5}Age - General & \cellcolor{green!5}1/58 & \cellcolor{green!5}1.724 \\  \hline
  \cellcolor{green!27}\small I'm in college myself so I understand why they picked Mr. B. Not because of his \textbf{age}, but because he was the "easier" sub. That said, Mr. Paul's plan sounded significantly more interesting to me.\normalsize   & \cellcolor{green!27}Age & \cellcolor{green!27}Age - General & \cellcolor{green!27}1/35 & \cellcolor{green!27}2.857 \\  \hline
  \cellcolor{green!5}\small HORRIBLE experiment. If 'age' is the independent variable everything else needs to be the same. Youre assuming that students are picking based on the best lesson when they actually have other motivatations. \textbf{Dumb}, boring experiment.\normalsize   & \cellcolor{green!5}Dumb & \cellcolor{green!5}Physical Identity - Physical (and Mental) Impairments & \cellcolor{green!5}1/35 & \cellcolor{green!5}2.857 \\  \hline
  \cellcolor{green!27}\small Age and that which goes with it, is the independent variable. The lesson plan was designed to showcase the individual, their \textbf{age} and their strong suits, which is exactly the point of the experiment. But they were both discussing/teaching/talking about sociopaths and psychopaths, which is as close as you possibly can get. You can't have a social experiment where one person is the same as the other based on \textbf{age}, ability and what goes with that, when people are drastically different ages. This experiment cannot be done in a math class where they could have taught the same material, with the same lesson plan, etc, for a result because this is not a math class... he is not a math teacher. Therefore it will not be taught the same way because it will not be a straight forward, this and this will give you this result, instruction of the material.\normalsize   & \cellcolor{green!27}Age & \cellcolor{green!27}Age - General & \cellcolor{green!27}3/150 & \cellcolor{green!27}2.0 \\  \hline
  \cellcolor{green!5}\small I sure am glad white people founded and built a \textbf{nation} for these people to enjoy.All of that white blood was worth it, that way latrineos, blacks and asians can live in a nice country.\normalsize   & \cellcolor{green!5}Nation & \cellcolor{green!5}Nationality - General & \cellcolor{green!5}1/36 & \cellcolor{green!5}2.778 \\  \hline
  \cellcolor{green!27}\small Oh right I forgot, this is the future now so I have to believe that everyone else founded and built the \textbf{nation} now.\normalsize   & \cellcolor{green!27}Nation & \cellcolor{green!27}Nationality - General & \cellcolor{green!27}1/23 & \cellcolor{green!27}4.348 \\  \hline
  \cellcolor{green!5}\small If you don't want to be a minority in your own \textbf{nation} you should just leave.Also that statement that there are plenty of places is simply false, all white nations world wide are being given away just like the USA.\normalsize   & \cellcolor{green!5}Nation & \cellcolor{green!5}Nationality - General & \cellcolor{green!5}1/41 & \cellcolor{green!5}2.439 \\  \hline
  \cellcolor{green!27}\small A. I don't know how this applies to anything in the video, but okay. B. This has never been a "white \textbf{nation}." Did you forget about all the brown people who were already living on this land when boats full of white people docked on its shores?\normalsize   & \cellcolor{green!27}Nation & \cellcolor{green!27}Nationality - General & \cellcolor{green!27}1/47 & \cellcolor{green!27}2.128 \\  \hline
  \cellcolor{green!5}\small This has \textbf{nothing} to do with \textbf{age} tho..\normalsize   & \cellcolor{green!5}Age, Nothing & \cellcolor{green!5}Age - General, Age - Youngsters & \cellcolor{green!5}2/8 & \cellcolor{green!5}25.0 \\  \hline
  \cellcolor{green!27}\small Had \textbf{nothing} to do with \textbf{ageism} in my opinion. It was more about who the students thought was going to be an easy A and who they could get away with stuff. With the older more experienced gentleman, they knew it was going to be a real class with real work. This generation tends to be more lazier than past. I think the experiment question would have been better phrased as, "Who would be the better teacher". That would have been a true test.\normalsize   & \cellcolor{green!27}Ageism, Nothing & \cellcolor{green!27}Age - General, Age - Youngsters & \cellcolor{green!27}2/84 & \cellcolor{green!27}2.381 \\  \hline
  \cellcolor{green!5}\small Of course it's an easy A....and to a degree \textbf{ageism} is a part of it, however the bigger part of it is the top heavy administration issues we're having (in higher ed particular...as a professor for over 18 years, the issue is within admin).  The issue is with how they utilize adjuncts within the classroom (which is what I've been, as full time jobs are hard to get).  Why pay full time wages when you can get people to work for 2k/semester.  Until that's fixed, we won't see changes.\normalsize   & \cellcolor{green!5}Ageism & \cellcolor{green!5}Age - General & \cellcolor{green!5}1/89 & \cellcolor{green!5}1.124 \\  \hline
  \cellcolor{green!27}\small I see where you're coming from, as I too have taught for 20 years... but as far as this video, my opinion still stands. It's not about \textbf{ageism}. This is about lazy students and who they think is going to be an easier teacher. \textbf{Nothing} more, \textbf{nothing} less.\normalsize   & \cellcolor{green!27}Ageism, Nothing & \cellcolor{green!27}Age - General, Age - Youngsters & \cellcolor{green!27}3/48 & \cellcolor{green!27}6.25 \\  \hline
  \cellcolor{green!5}\small I remember how I used to think in highschool. I would have picked the younger teacher not because of his \textbf{age}, but in order to get high marks with much less effort and be able to be laid back in classes and learn less. I don't think \textbf{age} is a key element in this experiment! Try again with making older teacher give them easy tasks in stead!\normalsize   & \cellcolor{green!5}Age & \cellcolor{green!5}Age - General & \cellcolor{green!5}2/67 & \cellcolor{green!5}2.985 \\  \hline
  \cellcolor{green!27}\small That's the point, based on experience and the \textbf{ageism}, their structure of teaching the material will inherently be different because they have developed the tools and ideas to teach more effectively. This comment is the basis, and the proof, of an \textbf{ageism} experiment with this exact same result.\normalsize   & \cellcolor{green!27}Ageism & \cellcolor{green!27}Age - General & \cellcolor{green!27}2/48 & \cellcolor{green!27}4.167 \\  \hline
  
\end{longtable}
\end{center}


\centering\textbf{\large \hypertarget{Table 2}{Table 2}: Summary of the results per sociolinguistic variable 
}
\newcolumntype{C}[2]{>{\centering\arraybackslash}p{#1}}
\begin{center}
\setlength\mylength{\dimexpr\textwidth - 1\arrayrulewidth - 40\tabcolsep}
\begin{longtable}{|C{.50\mylength}|C{.30\mylength}|C{.15\mylength}|C{.15\mylength}|C{.15\mylength}|}
\hline
\textbf{Sociolinguistic variables (Hiper - Hipo)} & \textbf{KeyWords} & \textbf{Number of occurrences} & \textbf{Frequency}  & \textbf{Frequency(\%)} \\
\hline\multirow{1}{*}{\cellcolor{red!27}Age - General}  & \cellcolor{red!27}Age, Ageist, Ageism & \cellcolor{red!27}22 & \cellcolor{red!27}22/2092& \cellcolor{red!27}1.05 \\  \hline
  \multirow{1}{*}{\cellcolor{red!5}Age - Over 65s}  & \cellcolor{red!5}Old & \cellcolor{red!5}6 & \cellcolor{red!5}6/2092& \cellcolor{red!5}0.29 \\  \hline
  \multirow{1}{*}{\cellcolor{red!27}Physical Identity - Physical (and Mental) Impairments}  & \cellcolor{red!27}Dumb & \cellcolor{red!27}1 & \cellcolor{red!27}1/2092& \cellcolor{red!27}0.05 \\  \hline
  \multirow{1}{*}{\cellcolor{red!5}Nationality - General}  & \cellcolor{red!5}Nation & \cellcolor{red!5}4 & \cellcolor{red!5}4/2092& \cellcolor{red!5}0.19 \\  \hline
  \multirow{1}{*}{\cellcolor{red!27}Age - Youngsters}  & \cellcolor{red!27}Nothing & \cellcolor{red!27}4 & \cellcolor{red!27}4/2092& \cellcolor{red!27}0.19 \\  \hline
  
\end{longtable}
\end{center}


\textbf{\Large Result analysis:}

\begin{itemize}\item Taking into account the words that were detected, we can reach the conclusion these comments are associated with : : Age - General;Age - Over 65s;Physical Identity - Physical (and Mental) Impairments;Nationality - General;Age - Youngsters;%.

\item The percentage of hate speech related words is 1.7686.

\item Considering that the variable \textbf{Age - General} has the most occurences in the post, we can interpret that this is the predominant hate speech.

\item Overall there were 37/69 occurences of hate speech related comments.\end{itemize}\end{document}