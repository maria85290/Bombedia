\documentclass[11pt]{article}
\ExplSyntaxOn
\let\tl_length:n\tl_count:n
\ExplSyntaxOff
\usepackage{graphicx}
\usepackage{multirow}
\usepackage{colortbl}
\usepackage{longtable, array}
\usepackage{hyperref}
\usepackage[usenames,dvipsnames,svgnames,table]{xcolor}
\newlength\mylength
\usepackage[legalpaper, landscape, margin=0.8in]{geometry}
\newcommand{\MinNumber}{0}
\begin{document}

\textbf {\huge In youtube\_extraction\_english\_144.json :}\newline \par\Large\textbf {Title: \large  I'm Angry About Ageism! Let's Work Together to Fix it! }\newline {\par\large --- Table  1: Summary of the results per comment; }

 {\par\large --- \hyperlink{Table 2}{\textcolor{blue}{\underline{Table 2}}}: Summary of the results per sociolinguistic variable;}\newline \normalsize\newline

\centering\textbf{\large Table  1: Summary of the results per comment 
}
\newcommand{\MaxNumber}{0}%
\newcommand{\ApplyGradient}[1]{%
\pgfmathsetmacro{\PercentColor}{100.0*(#1-\MinNumber)/(\MaxNumber-\MinNumber)}
\xdef\PercentColor{\PercentColor}%
\cellcolor{LightSpringGreen!\PercentColor!LightRed}{#1}
}
\newcolumntype{C}[2]{>{\centering\arraybackslash}p{#1}}
\begin{center}
\setlength\mylength{\dimexpr\textwidth - 1\arrayrulewidth - 50\tabcolsep}
\begin{longtable}{|C{.65\mylength}|C{.30\mylength}|C{.12\mylength}|C{.12\mylength}|C{.12\mylength}|}
\hline
\textbf{Comment} & \textbf{KeyWords} & \textbf{Sociolinguistic variables (Hiper - Hipo)}  & \textbf{Hate Speech Frequency} & \textbf{Hate Speech Frequency(\%)} \\
\hline\cellcolor{green!27}\small I totally agree with your thoughts and ideas for women making a difference in the workplace. For me personally I believe that many women can create an income using talents and skills they have acquired in the home or by learning new skills in the present. For those who don't like technology there are opportunities to teach classes of all manner of things. A ninety year \textbf{old} friend taught basket making and made good money at it. Many are joining up with companies such as Stampin' Up selling paper crafting supplies, making piles of money, going on trips and having a ball. There are endless opportunities for the creative thinker, an outside of the box thinker. Many are becoming professional organizers for the organized impaired.  If you are blessed with musical talent, most parents are looking for teachers for their children. Tutoring is another option. Getting back to basics such as gardening, canning, weaving, knitting, cooking, sewing, etc. are all the rage right now. These would make great classes.If you can't join the existing companies, make your own!\normalsize   & \cellcolor{green!27}Old & \cellcolor{green!27}Age - Over 65s & \cellcolor{green!27}1/179 & \cellcolor{green!27}0.559 \\  \hline
  \cellcolor{green!5}\small In the United States there used to be an advocacy group called Gray Panthers. They broke the assumptions about older people. In addition to countering stereotypes about \textbf{age} and aging, policies and practices need to catch up to reality. The conversations about people being able to carry benefits (health insurance, retirement savings, and even vacation and sick/health leave) from one job to the next -- even for contract workers is one way that the new reality of work is being recognized. There's also a proposal about a guaranteed minimum income. This is particularly critical in the "gig economy." Also, what would it look like to have \textbf{age}-based "affirmative action?"\normalsize   & \cellcolor{green!5}Age & \cellcolor{green!5}Age - General & \cellcolor{green!5}1/109 & \cellcolor{green!5}0.917 \\  \hline
  \cellcolor{green!27}\small i posted a relevant comment a minute ago regarding this very subject, but somehow it  posted, but disappeared. The main thing I want to say is to never let one's \textbf{age} define one's worth or who you are and never give anyone the right to do so.  I am living proof that if you work hard enough at fifty or sixty or in your seventies it is never too late to start a new chapter in your life.  The internet has opened up a new world of opportunities for us as entrepreneurs, employees or  self-motivated 'over sixty' women, especially.  it is possible to get through life with grace and humor all the while not forgetting, your worth or giving others the right to impose anything lesser. You do have the power.\normalsize   & \cellcolor{green!27}Age & \cellcolor{green!27}Age - General & \cellcolor{green!27}1/131 & \cellcolor{green!27}0.763 \\  \hline
  \cellcolor{green!5}\small Margaret, I'm so grateful that you have broached this topic. it has been a cause I have championed for many years now. In my head I never put my \textbf{age} as an equation that was negotiable. I was fortunate to have had a healthy television and business career until, at fifty, I decided to go for my childhood dream to be an actress. I worked hard to get my foot onto that stage and knocked repeatedly on every door that slammed in my face until it happened. I got work and my coveted SAG-AFTRA AND EQUITY cards. At fifty-two I added author to my resume, and at seventy-three started blogging about the joys and opportunities and new chapters that are possible realistically of the over sixty life. I'm seventy-eight now and recently started  a YouTube channel addressing these same 'agist' issues, especially where older women are concerned.  In my mind the internet will save us. Through your efforts, numerous tools and conversations and opportunities now available to us to champion our worth and motivate others from right in our own homes are genuinely valuable resources.  For those of the over sixty community who are not too much in to being an entrepreneur, corporations are now crying out for help with telemarketing and hiring experienced seniors who have proven to be reliable employees.  I would advise those over sixty out there to never let \textbf{age} define you, or give permission for others to do so. The image you project is what the world sees.  Think young, think ready and willing, and don't let 'too old' to deter you from your dreams in life.  New and exciting chapters are awaiting out there for you if you are positive and believe in yourself.  You still have purpose and are relevant. My mantra always has been to get through life with grace and humor and never forgetting my worth as a human being. Thank you again, Margaret for getting this wheel rolling again.\normalsize   & \cellcolor{green!5}Age & \cellcolor{green!5}Age - General & \cellcolor{green!5}2/329 & \cellcolor{green!5}0.608 \\  \hline
  \cellcolor{green!27}\small So important! I know work places have representatives talk about harassment, communication, health and safety, but what about \textbf{ageism} in the work place?  Actually the older employees at my work place are (generally speaking)  much more conscientious about the quality of their work compared to younger employees. Maybe send emails to CEOs , write articles in the paper,  such as "What's your company doing about \textbf{ageism}?"  We need to share what we've learned, what works, where we went wrong; what a sad society that discards such wealth. Yes, let's get mad!\normalsize   & \cellcolor{green!27}Ageism & \cellcolor{green!27}Age - General & \cellcolor{green!27}2/91 & \cellcolor{green!27}2.198 \\  \hline
  \cellcolor{green!5}\small I landed a job in an office at \textbf{age} 57 in a Town, competing with many "young" applicants. My experience and personality won me the position, but when I was job hunting, many people who interviewed me were very young and the first questions would be coddling me "Do you know how to do computers?" "Do you know how to use eMail?" etc...there is a stereotype of older workers as not being tech savvy. I embraced it back in the 1980s when new technology was introduced to office work. I deal with older customers all the time who resent technology and this fans the flames with younger workers. It is bad to discriminate, but it is also bad to stick heels in the sand refusing to learn new software, computer processes or use online media. We communicate in modern offices using texts, emails, and Outlook calendars, with very little hard copies, which older people can tend to lean on too much. I've been in offices since 1977 and I learned this...Move forward with technology or be left behind.That being said, my Youtube channel has over 8300 subscribers, but if I were 21, I am sure I'd be in the hundred thousands. Our generation is just getting the hang of youtube and other social media, where young viewers have been watching and making videos since their childhood. I agree we need to support each other and exchange ideas and info. Power to the (\textbf{Old}) People! lol Proud to be 59 here. : D\normalsize   & \cellcolor{green!5}Age, Old & \cellcolor{green!5}Age - General, Age - Over 65s & \cellcolor{green!5}2/252 & \cellcolor{green!5}0.794 \\  \hline
  \cellcolor{green!27}\small They are doing great work! I also just found a new site called \textbf{Ageist}  [LINK]  So many people writing about it - I'd love to find a way to activate a movement ! :-)\normalsize   & \cellcolor{green!27}Ageist & \cellcolor{green!27}Age - General & \cellcolor{green!27}1/34 & \cellcolor{green!27}2.941 \\  \hline
  \cellcolor{green!5}\small Retail usually hires women our \textbf{age}. It also pays the lowest out of all the employment so you are not going to really get ahead. I know many women who can't get a job based on their high level skill sets because of their \textbf{age}. It appears that the one of the best ways to get around this is have your own business then you won't be subject to having a boss half your \textbf{age} who "knows it all". However, that calls for a completely different set of skills. Look at the roles actresses are offered after 50 even.The movement to stop \textbf{ageism} would be huge if those actresses got on board. It also comes down to outward appearances. A major shift has occurred in the last 20 years for the push to look eternally young that didn't exist before. That push is now impacting teenagers to have plastic surgery and fillers. That speaks volumes to what our society values. So yes, tech skills are important but outward appearances are just as important if not more so and it won't be fixed with makeup techniques. Spending countless dollars on lotions and potions that don't really do much doesn't help our finances and the more aggressive treatments are all over the board with results and side effects with no real place to get actual fact based knowledge for those who do want to look "ageless" unless you frequent the journals themselves for information provided to skin care \textbf{Doctors}.\normalsize   & \cellcolor{green!5}Age, Ageism, doctors & \cellcolor{green!5}Age - General, Nationality - General & \cellcolor{green!5}5/247 & \cellcolor{green!5}2.024 \\  \hline
  \cellcolor{green!27}\small This is a topic that really annoys me too.  I've noticed there are so few successful YouTubers over 50.  That's really disappointing for someone like me just starting on YouTube.  That's why it's so refreshing to see you doing so well.  I have a BS in Computer Science and an MS in Informatics and I am pretty positive I could not get a job in my field.  I have my own web design business which luckily has shielded me from \textbf{ageism}.  I didn't want to give anyone else control over my ability to earn money.  It's sad that our society doesn't see the value of older people.  We know SO MUCH and have so much to offer.  I'm glad you are bringing up this issue.  I will put some thought into how we can take action.  You're right about sticking together and working as a team.\normalsize   & \cellcolor{green!27}Ageism & \cellcolor{green!27}Age - General & \cellcolor{green!27}1/146 & \cellcolor{green!27}0.685 \\  \hline
  
\end{longtable}
\end{center}


\centering\textbf{\large \hypertarget{Table 2}{Table 2}: Summary of the results per sociolinguistic variable 
}
\newcolumntype{C}[2]{>{\centering\arraybackslash}p{#1}}
\begin{center}
\setlength\mylength{\dimexpr\textwidth - 1\arrayrulewidth - 40\tabcolsep}
\begin{longtable}{|C{.50\mylength}|C{.30\mylength}|C{.15\mylength}|C{.15\mylength}|C{.15\mylength}|}
\hline
\textbf{Sociolinguistic variables (Hiper - Hipo)} & \textbf{KeyWords} & \textbf{Number of occurrences} & \textbf{Frequency}  & \textbf{Frequency(\%)} \\
\hline\multirow{1}{*}{\cellcolor{red!27}Age - Over 65s}  & \cellcolor{red!27}Old & \cellcolor{red!27}2 & \cellcolor{red!27}2/1834& \cellcolor{red!27}0.11 \\  \hline
  \multirow{1}{*}{\cellcolor{red!5}Age - General}  & \cellcolor{red!5}Age, Ageism, Ageist & \cellcolor{red!5}13 & \cellcolor{red!5}13/1834& \cellcolor{red!5}0.7100000000000001 \\  \hline
  \multirow{1}{*}{\cellcolor{red!27}Nationality - General}  & \cellcolor{red!27}doctors & \cellcolor{red!27}1 & \cellcolor{red!27}1/1834& \cellcolor{red!27}0.05 \\  \hline
  
\end{longtable}
\end{center}


\textbf{\Large Result analysis:}

\begin{itemize}\item Taking into account the words that were detected, we can reach the conclusion these comments are associated with : : Age - Over 65s;Age - General;Nationality - General;%.

\item The percentage of hate speech related words is 0.8724.

\item Considering that the variable \textbf{Age - General} has the most occurences in the post, we can interpret that this is the predominant hate speech.

\item Overall there were 16/22 occurences of hate speech related comments.\end{itemize}\end{document}