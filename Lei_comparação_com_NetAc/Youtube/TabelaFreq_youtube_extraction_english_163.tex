\documentclass[11pt]{article}
\ExplSyntaxOn
\let\tl_length:n\tl_count:n
\ExplSyntaxOff
\usepackage{graphicx}
\usepackage{multirow}
\usepackage{colortbl}
\usepackage{longtable, array}
\usepackage{hyperref}
\usepackage[usenames,dvipsnames,svgnames,table]{xcolor}
\newlength\mylength
\usepackage[legalpaper, landscape, margin=0.8in]{geometry}
\newcommand{\MinNumber}{0}
\begin{document}

\textbf {\huge In youtube\_extraction\_english\_163.json :}\newline \par\Large\textbf {Title: \large  6 Warning Signs Of Age Discrimination - What You Should Do }\newline {\par\large --- Table  1: Summary of the results per comment; }

 {\par\large --- \hyperlink{Table 2}{\textcolor{blue}{\underline{Table 2}}}: Summary of the results per sociolinguistic variable;}\newline \normalsize\newline

\centering\textbf{\large Table  1: Summary of the results per comment 
}
\newcommand{\MaxNumber}{0}%
\newcommand{\ApplyGradient}[1]{%
\pgfmathsetmacro{\PercentColor}{100.0*(#1-\MinNumber)/(\MaxNumber-\MinNumber)}
\xdef\PercentColor{\PercentColor}%
\cellcolor{LightSpringGreen!\PercentColor!LightRed}{#1}
}
\newcolumntype{C}[2]{>{\centering\arraybackslash}p{#1}}
\begin{center}
\setlength\mylength{\dimexpr\textwidth - 1\arrayrulewidth - 50\tabcolsep}
\begin{longtable}{|C{.65\mylength}|C{.30\mylength}|C{.12\mylength}|C{.12\mylength}|C{.12\mylength}|}
\hline
\textbf{Comment} & \textbf{KeyWords} & \textbf{Sociolinguistic variables (Hiper - Hipo)}  & \textbf{Hate Speech Frequency} & \textbf{Hate Speech Frequency(\%)} \\
\hline\cellcolor{green!27}\small At 65 in some states its \textbf{elder} abuse depending on the issues at hand..\normalsize   & \cellcolor{green!27}Elder & \cellcolor{green!27}Age - Over 65s & \cellcolor{green!27}1/14 & \cellcolor{green!27}7.143 \\  \hline
  \cellcolor{green!5}\small Slander is a verbal attack and can affect your business due to destroying your reputation...This can be a media person on air using your name and making fun of your \textbf{age} using farting sounds or  name...to humiliate and destroy you..\normalsize   & \cellcolor{green!5}Age & \cellcolor{green!5}Age - General & \cellcolor{green!5}1/40 & \cellcolor{green!5}2.5 \\  \hline
  \cellcolor{green!27}\small just got a job with an interview over the phone. I was there one day and was fired. I was hired with another \textbf{woman} on the same day who was 10 years younger and she was trained immediately on the computer for the work we were hired for. I was given filing and stamping volumes of documents. Of course when I refuse the stamping due to wrist injuries. I was fired.\normalsize   & \cellcolor{green!27}Woman & \cellcolor{green!27}Gender - General & \cellcolor{green!27}1/71 & \cellcolor{green!27}1.408 \\  \hline
  \cellcolor{green!5}\small Great video but the federal law has identified \textbf{age} discrimination being 40 and older not 50.\normalsize   & \cellcolor{green!5}Age & \cellcolor{green!5}Age - General & \cellcolor{green!5}1/16 & \cellcolor{green!5}6.25 \\  \hline
  \cellcolor{green!27}\small I know \textbf{doctors} who work with 83-86 years \textbf{old} operating on the brains and stuff, because they are bored or want to help humanity, working even for free. People needs them then to save their lives, right? What hypocrites some people are!!!! In Europe you are not allowed to even study with 40 years \textbf{old} at the universities. Even studying is a crime? AND when we are young, they scream at us, " But this young dude or girl doesn't know anything! No experience, no \textbf{nothing}! How can they be handled? We don't have this time. Out!" So, people are never satisfied and that's stupid and \textbf{crazy}. Not making any sense!\normalsize   & \cellcolor{green!27}Crazy, Nothing, Old, doctors & \cellcolor{green!27}Age - Over 65s, Age - Youngsters, Nationality - General, Physical Identity - Physical (and Mental) Impairments & \cellcolor{green!27}5/111 & \cellcolor{green!27}4.505 \\  \hline
  \cellcolor{green!5}\small What he says at :50 to :52 is so true . \textbf{Age} discrimination is at an all time high in the United States.\normalsize   & \cellcolor{green!5}Age & \cellcolor{green!5}Age - General & \cellcolor{green!5}1/23 & \cellcolor{green!5}4.348 \\  \hline
  \cellcolor{green!27}\small Age discrimination is so real and the laws are a joke . Its not good to continue to do a great job and be passed over for younger less qualified people . Best save your money and be prepared  because there is \textbf{nothing} you can do . Very sad that we dont have a cohesive national labor union in the USA .\normalsize   & \cellcolor{green!27}Age, Nothing & \cellcolor{green!27}Age - General, Age - Youngsters & \cellcolor{green!27}2/62 & \cellcolor{green!27}3.226 \\  \hline
  \cellcolor{green!5}\small The US Government is the BIGGEST most BRAZEN abuser of \textbf{age} discrimination and FORCED retirement.\normalsize   & \cellcolor{green!5}Age & \cellcolor{green!5}Age - General & \cellcolor{green!5}1/15 & \cellcolor{green!5}6.667 \\  \hline
  \cellcolor{green!27}\small The laws in place protecting people from \textbf{age} discrimination are very clear and are constructed in such a way that makes it near impossible to prosecute perpetrators successfully. The recognizable signs of the acts of \textbf{age} discrimination, both overt and covert, are so well known, and hence avoided by the perpetrators, that the possibility of any legal success in prosecution of the offense is impossible. The usual offenders can either afford to employ a legal department or keep legal counsel on retainer to protect their interests...individuals typically cannot afford this luxury. Employers use instructional videos that all employees are required to view, that address these laws, so employers can 'wash their hands' of any liability, while they themselves practice the abuse of \textbf{age} discrimination without repercussion. There are several reasons why the success rate of offender prosecution is negligible. States that are in the 'Right to Work' category, enjoy bathing in the ability to execute an \textbf{age}-based termination without ethical justification or having to prove why. This burden falls on the victim who can nether prove, much less afford to prove the act of discrimination. The EEOC who is in place to provide protections in this area has proven to be useless in its management of \textbf{age} discrimination claims and does little to \textbf{nothing} to help the victims. There is also no money to be made in the attempt to prove \textbf{age} discrimination, so victims are further victimized by a system designed to protect the perpetrator…not the abused. Employers are 'encouraged' by health organizations to 'find ways' to keep costs low and in so doing \textbf{age} discrimination is employed through the use of higher premiums and/or termination of the older worker. This results in what we currently have today, which are useless laws that are used to identify 'what' are considered valid examples of \textbf{age} discrimination, as defined by current law, used by legal counsel to 'coach' the perpetrators avoidance rather than ethical governance of a fair, just, and non-discriminatory work environment. IF these laws worked, we would see an equitable distribution of ages represented in the workplace, but that is not the case. While there may be a few instances, they are far and few between and few employers demonstrate that equitable \textbf{age} distribution is being practiced in the workplace today. We all suffer from \textbf{age}-based discrimination…some of us sooner than others but one day it will happen to all of us who are lucky(?) enough to last that long.\normalsize   & \cellcolor{green!27}Age, Nothing & \cellcolor{green!27}Age - General, Age - Youngsters & \cellcolor{green!27}9/411 & \cellcolor{green!27}2.19 \\  \hline
  \cellcolor{green!5}\small Not only that, the "\textbf{sex} discrimination" videos slightly address other types of harassment, and actually give excuses or ideas on how to do it!  I've heard a bully boss say "I learned a lot" after watching one of those vids, meaning they learned more tactics.\normalsize   & \cellcolor{green!5}Sex & \cellcolor{green!5}Gender - General & \cellcolor{green!5}1/45 & \cellcolor{green!5}2.222 \\  \hline
  \cellcolor{green!27}\small Are these videos addressing \textbf{age} discrimination working? Nope...........\normalsize   & \cellcolor{green!27}Age & \cellcolor{green!27}Age - General & \cellcolor{green!27}1/8 & \cellcolor{green!27}12.5 \\  \hline
  \cellcolor{green!5}\small 'Despite the fact the United States' \textbf{Age} Discrimination in Employment Act (ADEA) has been in place for over 35 years, \textbf{age} discrimination in employment remains pervasive.'~ Laurie McCann, \textbf{Senior} attorney, AARP Litigation Foundation…'Negative perceptions of older workers were perceived as inflexible, unwilling to adapt technology, lacking an aggressive spirit, resistant to new ways, having some physical limitations, costing more for health insurance and complacent…Typically 90 of \textbf{age} discrimination suits do not make it to trial…One reason is that \textbf{age} discrimination in the employment process is difficult to prove'… [LINK]  OPINION...INTRODUCTIONCONSTANCE JEANNE SAMMARCO pro se, Appellant, uses her first amendmentU.S. constitutional rights to offer her opinion to expose the Incompetencyof The Prince George's County Board of Education (PGCPS BOE) in theirdeliberate deceptiveness, in their deliberate failure to state pertinentfacts, and in their deliberate untrue statements that supported their falseallegations of her unsatisfactory evaluations issued by Principal NakiaNicholson of Fairmont Heights High (FHHS) for two consecutive years thatruined her credibility and caused her job dismissal…The following is theAPPELLANT'S 'feedback'(which is 50  effective of all communication) onThe Maryland State Board of Education's Opinion No. 15-01: [LINK] \normalsize   & \cellcolor{green!5}Age, Senior & \cellcolor{green!5}Age - General, Age - Over 65s & \cellcolor{green!5}5/196 & \cellcolor{green!5}2.551 \\  \hline
  \cellcolor{green!27}\small Typically 90 of \textbf{age} discrimination suits do not make it to trial, and Workers Win Only 1 Of Federal Civil Rights lawsuits At trial. The court system is flawed because it usually comes down to who can afford the best lawyer, and the judges are always on the side of public entities like schools, hospitals, and rarely do big corporations lose cases.\normalsize   & \cellcolor{green!27}Age & \cellcolor{green!27}Age - General & \cellcolor{green!27}1/62 & \cellcolor{green!27}1.613 \\  \hline
  \cellcolor{green!5}\small I'm in my early 30's and I get \textbf{age} discriminated against.\normalsize   & \cellcolor{green!5}Age & \cellcolor{green!5}Age - General & \cellcolor{green!5}1/11 & \cellcolor{green!5}9.091 \\  \hline
  \cellcolor{green!27}\small I'm 52 recently moved to NZ from the Middle East I am British and have huge experience in Business development yet I get \textbf{zero} response from recruitment companies. Usual \textbf{ageist} response and NZ provincial attitudes ie if you have no experience here then we don't want you . I'm going back to Dubai as I miss the dynamic nature of the commercial and living environment which I don't get here. Really disappointing, sorry NO I don't want to drive a bus!!! And I don't want to live here.....\normalsize   & \cellcolor{green!27}Ageist, Zero & \cellcolor{green!27}Age - General, Age - Youngsters & \cellcolor{green!27}2/88 & \cellcolor{green!27}2.273 \\  \hline
  \cellcolor{green!5}\small 1 they do not joke in front of you about your \textbf{age}, 2 yes unexpected and early-ish retirements, 3 they don't advertise an \textbf{age} range or preference, but they do request items that never used to be required, like SOQ's and things like that, that younger people are MUCH better at, having just been out of college,  4 disciplinary action, yes.  5 not sure about that one.\normalsize   & \cellcolor{green!5}Age & \cellcolor{green!5}Age - General & \cellcolor{green!5}2/67 & \cellcolor{green!5}2.985 \\  \hline
  \cellcolor{green!27}\small I'm 57 and CANNOT get hired! I HAVE to have work so I can pay my student loans for when I went back to school to better myself. I'm f'ing fed up with getting the door slammed in my face, and it's because of my \textbf{age}. Has to be!\normalsize   & \cellcolor{green!27}Age & \cellcolor{green!27}Age - General & \cellcolor{green!27}1/49 & \cellcolor{green!27}2.041 \\  \hline
  \cellcolor{green!5}\small Im 30 and i jokingly call myself an \textbf{old} man because i have had heradatry back problems for years and when you have pain in your body it makes you feel \textbf{old}. And generally though I am respecful to all my co workers but sometimes feel because of my "young" \textbf{age} they generally dont treat me respecfully though never as much to call it discrimination. One of the older workers calls me the baby though me and her are really good work friends so it really is just friendly banter.\normalsize   & \cellcolor{green!5}Age, Old & \cellcolor{green!5}Age - General, Age - Over 65s & \cellcolor{green!5}3/90 & \cellcolor{green!5}3.333 \\  \hline
  \cellcolor{green!27}\small Okay interesting that Facebook completely treats profiles differently depending on if you are a male or female and how \textbf{old} you are. If you can try to follow the example that I personally checked and rechecked. I work with 5 close friends and all agreed to submit their Facebook credentials to prove the discrimination against older Facebook males. Two males profiles, one 30's and one 50's. Three female profiles, one 20's one 30's and one 40's. I selected a real profile knowing every detail of the profile to search, a person with an unusual name but not too \textbf{weird} and not too \textbf{common}. Kate Moorehouse (Barbie). When searching this name the younger females could search her name and she appeared in the results even though out of one of the females two of the females had no connection to Kate at all. The younger male could not find her in a seach until he had at least three other \textbf{common} friends with Kate. Then in the search result she appeared only after searching her entire first and last name. The 50 year \textbf{old} male could not search her. Even when he entered her first and last name and her entered nickname the search would not locate Kate Moorehouse (Barbie) not possible because Facebook would not allow an older male to see that profile. When Kate's home town, home city, employment were entered with her first and last name and nick name she didn't appear. The older male had to go to a friend, select her out and when he pushed the friend request Facebook warned only send request to people you know. Even though the older male's account was in good standing. When the \textbf{old} male again pushed continue Facebook would not send the request stating he could only send request to people he knew, and if he didn't understand that to contact Facebook. I have attempted to contact Facebook over a few months about various stories and have never, never ever heard any response from Facebook. Okay so the \textbf{old} guy doesn't talk to the young girl, but what other check roadblocks does Facebook use to determine real life situations. Could this guy go up at the library and talk to this young girl appropriately, sure, but not on Facebook's real life situations?\normalsize   & \cellcolor{green!27}Common, Old, Weird & \cellcolor{green!27}Age - Over 65s, Physical Identity - Physical (and Mental) Impairments, Social Class - Working class & \cellcolor{green!27}7/384 & \cellcolor{green!27}1.823 \\  \hline
  \cellcolor{green!5}\small Even though I have a Bachelor's and a Master's degree, they wanted to see my High School Diploma.  Once they saw I graduated high school in the 1970's they suddenly figured out that I'm just over qualified, under qualified, or some other \textbf{dumb} excuse not to hire me.\normalsize   & \cellcolor{green!5}Dumb & \cellcolor{green!5}Physical Identity - Physical (and Mental) Impairments & \cellcolor{green!5}1/48 & \cellcolor{green!5}2.083 \\  \hline
  
\end{longtable}
\end{center}


\centering\textbf{\large \hypertarget{Table 2}{Table 2}: Summary of the results per sociolinguistic variable 
}
\newcolumntype{C}[2]{>{\centering\arraybackslash}p{#1}}
\begin{center}
\setlength\mylength{\dimexpr\textwidth - 1\arrayrulewidth - 40\tabcolsep}
\begin{longtable}{|C{.50\mylength}|C{.30\mylength}|C{.15\mylength}|C{.15\mylength}|C{.15\mylength}|}
\hline
\textbf{Sociolinguistic variables (Hiper - Hipo)} & \textbf{KeyWords} & \textbf{Number of occurrences} & \textbf{Frequency}  & \textbf{Frequency(\%)} \\
\hline\multirow{1}{*}{\cellcolor{red!27}Age - Over 65s}  & \cellcolor{red!27}Elder, Old, Senior & \cellcolor{red!27}10 & \cellcolor{red!27}10/2352& \cellcolor{red!27}0.43 \\  \hline
  \multirow{1}{*}{\cellcolor{red!5}Age - General}  & \cellcolor{red!5}Age, Ageist & \cellcolor{red!5}25 & \cellcolor{red!5}25/2352& \cellcolor{red!5}1.06 \\  \hline
  \multirow{1}{*}{\cellcolor{red!27}Gender - General}  & \cellcolor{red!27}Woman, Sex & \cellcolor{red!27}2 & \cellcolor{red!27}2/2352& \cellcolor{red!27}0.09 \\  \hline
  \multirow{1}{*}{\cellcolor{red!5}Age - Youngsters}  & \cellcolor{red!5}Nothing, Zero & \cellcolor{red!5}4 & \cellcolor{red!5}4/2352& \cellcolor{red!5}0.16999999999999998 \\  \hline
  \multirow{1}{*}{\cellcolor{red!27}Nationality - General}  & \cellcolor{red!27}doctors & \cellcolor{red!27}1 & \cellcolor{red!27}1/2352& \cellcolor{red!27}0.04 \\  \hline
  \multirow{1}{*}{\cellcolor{red!5}Physical Identity - Physical (and Mental) Impairments}  & \cellcolor{red!5}Crazy, Weird, Dumb & \cellcolor{red!5}3 & \cellcolor{red!5}3/2352& \cellcolor{red!5}0.13 \\  \hline
  \multirow{1}{*}{\cellcolor{red!27}Social Class - Working class}  & \cellcolor{red!27}Common & \cellcolor{red!27}1 & \cellcolor{red!27}1/2352& \cellcolor{red!27}0.04 \\  \hline
  
\end{longtable}
\end{center}


\textbf{\Large Result analysis:}

\begin{itemize}\item Taking into account the words that were detected, we can reach the conclusion these comments are associated with : : Age - Over 65s;Age - General;Gender - General;Age - Youngsters;Nationality - General;Physical Identity - Physical (and Mental) Impairments;Social Class - Working class;%.

\item The percentage of hate speech related words is 1.9558.

\item Considering that the variable \textbf{Age - General} has the most occurences in the post, we can interpret that this is the predominant hate speech.

\item Overall there were 47/38 occurences of hate speech related comments.\end{itemize}\end{document}