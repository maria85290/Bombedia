\documentclass[11pt]{article}
\ExplSyntaxOn
\let\tl_length:n\tl_count:n
\ExplSyntaxOff
\usepackage{graphicx}
\usepackage{multirow}
\usepackage{colortbl}
\usepackage{longtable, array}
\usepackage{hyperref}
\usepackage[usenames,dvipsnames,svgnames,table]{xcolor}
\newlength\mylength
\usepackage[legalpaper, landscape, margin=0.8in]{geometry}
\newcommand{\MinNumber}{0}
\begin{document}

\textbf {\huge In Sol\_extraction\_portuguese\_160.json :}\newline \par\Large\textbf {Title: \large  Vaticano expulsa padre que revelou homossexualidade }\newline {\par\large --- Table  1: Summary of the results per comment; }

 {\par\large --- \hyperlink{Table 2}{\textcolor{blue}{\underline{Table 2}}}: Summary of the results per sociolinguistic variable;}\newline \normalsize\newline

\centering\textbf{\large Table  1: Summary of the results per comment 
}
\newcommand{\MaxNumber}{0}%
\newcommand{\ApplyGradient}[1]{%
\pgfmathsetmacro{\PercentColor}{100.0*(#1-\MinNumber)/(\MaxNumber-\MinNumber)}
\xdef\PercentColor{\PercentColor}%
\cellcolor{LightSpringGreen!\PercentColor!LightRed}{#1}
}
\newcolumntype{C}[2]{>{\centering\arraybackslash}p{#1}}
\begin{center}
\setlength\mylength{\dimexpr\textwidth - 1\arrayrulewidth - 50\tabcolsep}
\begin{longtable}{|C{.65\mylength}|C{.30\mylength}|C{.12\mylength}|C{.12\mylength}|C{.12\mylength}|}
\hline
\textbf{Comment} & \textbf{KeyWords} & \textbf{Sociolinguistic variables (Hiper - Hipo)}  & \textbf{Hate Speech Frequency} & \textbf{Hate Speech Frequency(\%)} \\
\hline\cellcolor{green!27}\small O catolicismo é para quem abdica do casamento, de uma união com outra pessoa. São as regras e só veste o hábito quem quer, ninguém é arrastado, obrigado, ou escravizado para tal . Não estou a ver nesta notícia a questão do homossexualismo e sim o facto de o padre ter um companheiro, desrespeitando assim as regras da Igreja Católica.Também não entendo que um padre que se define como \textbf{homossexual} continue a sê-lo porque certamente quererá ter uma relação, uma relação que tanto poderia ser heterossexual como \textbf{homossexual}, o catolicismo não permite uniões.Vejo neste padre o desejo de enxovalhar a Igreja Católica só por maldade. Tem como alternativa o Anglicanismo. Claro que se for bater à porta do islamismo o caso pode tornar-se num caso de vida ou morte. Melhor, num caso de morte se não conseguir fugir a tempo.Pensemos que quando entramos para uma instituição temos de nos reger pelas suas regras, por isso só entramos se quisermos. Se já lá estivermos e não quisermos mais aceitar tais regras, saímos. A IC não obriga ninguém a aceitar os seus dogmas e regras, só lá está ou a segue quem quer.\normalsize   & \cellcolor{green!27}Homossexual & \cellcolor{green!27}Sexual Identity - General & \cellcolor{green!27}2/194 & \cellcolor{green!27}1.031 \\  \hline
  
\end{longtable}
\end{center}


\centering\textbf{\large \hypertarget{Table 2}{Table 2}: Summary of the results per sociolinguistic variable 
}
\newcolumntype{C}[2]{>{\centering\arraybackslash}p{#1}}
\begin{center}
\setlength\mylength{\dimexpr\textwidth - 1\arrayrulewidth - 40\tabcolsep}
\begin{longtable}{|C{.50\mylength}|C{.30\mylength}|C{.15\mylength}|C{.15\mylength}|C{.15\mylength}|}
\hline
\textbf{Sociolinguistic variables (Hiper - Hipo)} & \textbf{KeyWords} & \textbf{Number of occurrences} & \textbf{Frequency}  & \textbf{Frequency(\%)} \\
\hline\multirow{1}{*}{\cellcolor{red!27}Sexual Identity - General}  & \cellcolor{red!27}Homossexual & \cellcolor{red!27}1 & \cellcolor{red!27}1/392& \cellcolor{red!27}0.26 \\  \hline
  
\end{longtable}
\end{center}


\textbf{\Large Result analysis:}

\begin{itemize}\item Taking into account the words that were detected, we can reach the conclusion these comments are associated with : : Sexual Identity - General;%.

\item The percentage of hate speech related words is 0.2551.

\item Considering that the variable \textbf{Sexual Identity - General} has the most occurences in the post, we can interpret that this is the predominant hate speech.

\item Overall there were 2/8 occurences of hate speech related comments.\end{itemize}\end{document}