\documentclass[11pt]{article}
\ExplSyntaxOn
\let\tl_length:n\tl_count:n
\ExplSyntaxOff
\usepackage{graphicx}
\usepackage{multirow}
\usepackage{colortbl}
\usepackage{longtable, array}
\usepackage{hyperref}
\usepackage[usenames,dvipsnames,svgnames,table]{xcolor}
\newlength\mylength
\usepackage[legalpaper, landscape, margin=0.8in]{geometry}
\newcommand{\MinNumber}{0}
\begin{document}

\textbf {\huge In Sol\_extraction\_portuguese\_198.json :}\newline \par\Large\textbf {Title: \large  Emocionado, Goucha revela em que momento se sentiu magoado com Cristina Ferreira }\newline {\par\large --- Table  1: Summary of the results per comment; }

 {\par\large --- \hyperlink{Table 2}{\textcolor{blue}{\underline{Table 2}}}: Summary of the results per sociolinguistic variable;}\newline \normalsize\newline

\centering\textbf{\large Table  1: Summary of the results per comment 
}
\newcommand{\MaxNumber}{0}%
\newcommand{\ApplyGradient}[1]{%
\pgfmathsetmacro{\PercentColor}{100.0*(#1-\MinNumber)/(\MaxNumber-\MinNumber)}
\xdef\PercentColor{\PercentColor}%
\cellcolor{LightSpringGreen!\PercentColor!LightRed}{#1}
}
\newcolumntype{C}[2]{>{\centering\arraybackslash}p{#1}}
\begin{center}
\setlength\mylength{\dimexpr\textwidth - 1\arrayrulewidth - 50\tabcolsep}
\begin{longtable}{|C{.65\mylength}|C{.30\mylength}|C{.12\mylength}|C{.12\mylength}|C{.12\mylength}|}
\hline
\textbf{Comment} & \textbf{KeyWords} & \textbf{Sociolinguistic variables (Hiper - Hipo)}  & \textbf{Hate Speech Frequency} & \textbf{Hate Speech Frequency(\%)} \\
\hline\cellcolor{green!27}\small Goucha, ainda te vais magoar mais, essa \textbf{bicha} cristina não é de confiança.\normalsize   & \cellcolor{green!27}Bicha & \cellcolor{green!27}Sexual Identity - Male homosexuality & \cellcolor{green!27}1/13 & \cellcolor{green!27}7.692 \\  \hline
  \cellcolor{green!5}\small A \textbf{bichona} fachola, diga-se!\normalsize   & \cellcolor{green!5}Bichona & \cellcolor{green!5}Sexual Identity - Male homosexuality & \cellcolor{green!5}1/4 & \cellcolor{green!5}25.0 \\  \hline
  
\end{longtable}
\end{center}


\centering\textbf{\large \hypertarget{Table 2}{Table 2}: Summary of the results per sociolinguistic variable 
}
\newcolumntype{C}[2]{>{\centering\arraybackslash}p{#1}}
\begin{center}
\setlength\mylength{\dimexpr\textwidth - 1\arrayrulewidth - 40\tabcolsep}
\begin{longtable}{|C{.50\mylength}|C{.30\mylength}|C{.15\mylength}|C{.15\mylength}|C{.15\mylength}|}
\hline
\textbf{Sociolinguistic variables (Hiper - Hipo)} & \textbf{KeyWords} & \textbf{Number of occurrences} & \textbf{Frequency}  & \textbf{Frequency(\%)} \\
\hline\multirow{1}{*}{\cellcolor{red!27}Sexual Identity - Male homosexuality}  & \cellcolor{red!27}Bicha, Bichona & \cellcolor{red!27}2 & \cellcolor{red!27}2/146& \cellcolor{red!27}1.37 \\  \hline
  
\end{longtable}
\end{center}


\textbf{\Large Result analysis:}

\begin{itemize}\item Taking into account the words that were detected, we can reach the conclusion these comments are associated with : : Sexual Identity - Male homosexuality;%.

\item The percentage of hate speech related words is 1.3699.

\item Considering that the variable \textbf{Sexual Identity - Male homosexuality} has the most occurences in the post, we can interpret that this is the predominant hate speech.

\item Overall there were 2/9 occurences of hate speech related comments.\end{itemize}\end{document}