\documentclass[11pt]{article}
\ExplSyntaxOn
\let\tl_length:n\tl_count:n
\ExplSyntaxOff
\usepackage{graphicx}
\usepackage{multirow}
\usepackage{colortbl}
\usepackage{longtable, array}
\usepackage{hyperref}
\usepackage[usenames,dvipsnames,svgnames,table]{xcolor}
\newlength\mylength
\usepackage[legalpaper, landscape, margin=0.8in]{geometry}
\newcommand{\MinNumber}{0}
\begin{document}

\textbf {\huge In Sol\_extraction\_portuguese\_62.json :}\newline \par\Large\textbf {Title: \large  'Ó senhores deixem-se de mariquices ridículas' }\newline {\par\large --- Table  1: Summary of the results per comment; }

 {\par\large --- \hyperlink{Table 2}{\textcolor{blue}{\underline{Table 2}}}: Summary of the results per sociolinguistic variable;}\newline \normalsize\newline

\centering\textbf{\large Table  1: Summary of the results per comment 
}
\newcommand{\MaxNumber}{0}%
\newcommand{\ApplyGradient}[1]{%
\pgfmathsetmacro{\PercentColor}{100.0*(#1-\MinNumber)/(\MaxNumber-\MinNumber)}
\xdef\PercentColor{\PercentColor}%
\cellcolor{LightSpringGreen!\PercentColor!LightRed}{#1}
}
\newcolumntype{C}[2]{>{\centering\arraybackslash}p{#1}}
\begin{center}
\setlength\mylength{\dimexpr\textwidth - 1\arrayrulewidth - 50\tabcolsep}
\begin{longtable}{|C{.65\mylength}|C{.30\mylength}|C{.12\mylength}|C{.12\mylength}|C{.12\mylength}|}
\hline
\textbf{Comment} & \textbf{KeyWords} & \textbf{Sociolinguistic variables (Hiper - Hipo)}  & \textbf{Hate Speech Frequency} & \textbf{Hate Speech Frequency(\%)} \\
\hline\cellcolor{green!27}\small (Vamos lá ver se, entretanto, o censor foi fazer um brunch e a coisa passa, assim como que lubrificada e ninguém nota...)«Eu até podia dizer "mariquices ridículas", um pleonasmo ridículo, é um comentário obviamente \textbf{homofóbico} dito por um \textbf{homossexual} que, diga-se, faz do ridículo (do seu e do alheio) o mais escabroso pleonasmo da vida pública portuguesa.Mas isso era, num interminável quebra-cabeças, alimentar o pleonasmo.Isto porque este \textbf{roto} debochado faz da bandeira colorida uma coutada de pârametros insondáveis. Só ele sabe o que é e não é uma "mariquice \textbf{ridícula}",  porque só ele viveu vinte anos (calma com o depravação..) com um homem e só ele tem esse autêntico às de espadas, qual varinha de condão, que é condenar o outro à cruz da \textbf{homofobia}!Porque levar no traseiro dá essa impunidade, tipo carta da sorte "você está livre da prisão", com o acréscimo de também se poder ser um acusador do Santo Ofício das redes sociais.E é isso que aqui se vê, o gesto aleatório de alguém que dita regras consoante o amuo!O mesmo alguém que milhares de vezes usou as mesmíssimas expressões de condescendência peganhenta que agora repudia e considera ofensivas. O mesmo alguém que não hesita em denunciar como discursos dignos de censura judicial, utiliza os mesmíssimos discursos para se demonstrar moderno e acima da plebe que o alimenta!E faz isso tudo porque leva no traseiro!»\normalsize   & \cellcolor{green!27}Homofobia, Homossexual, Ridícula, Roto, homofóbico & \cellcolor{green!27}Gender - Female sexuality, Sexual Identity - General, Sexual Identity - Male homosexuality & \cellcolor{green!27}5/235 & \cellcolor{green!27}2.128 \\  \hline
  \cellcolor{green!5}\small Faz como o Goucha, não sejas \textbf{maricas}...\normalsize   & \cellcolor{green!5}Maricas & \cellcolor{green!5}Sexual Identity - Male homosexuality & \cellcolor{green!5}1/7 & \cellcolor{green!5}14.286 \\  \hline
  
\end{longtable}
\end{center}


\centering\textbf{\large \hypertarget{Table 2}{Table 2}: Summary of the results per sociolinguistic variable 
}
\newcolumntype{C}[2]{>{\centering\arraybackslash}p{#1}}
\begin{center}
\setlength\mylength{\dimexpr\textwidth - 1\arrayrulewidth - 40\tabcolsep}
\begin{longtable}{|C{.50\mylength}|C{.30\mylength}|C{.15\mylength}|C{.15\mylength}|C{.15\mylength}|}
\hline
\textbf{Sociolinguistic variables (Hiper - Hipo)} & \textbf{KeyWords} & \textbf{Number of occurrences} & \textbf{Frequency}  & \textbf{Frequency(\%)} \\
\hline\multirow{1}{*}{\cellcolor{red!27}Gender - Female sexuality}  & \cellcolor{red!27}Ridícula & \cellcolor{red!27}1 & \cellcolor{red!27}1/298& \cellcolor{red!27}0.33999999999999997 \\  \hline
  \multirow{1}{*}{\cellcolor{red!5}Sexual Identity - General}  & \cellcolor{red!5}Homofobia, homofóbico, Homossexual & \cellcolor{red!5}3 & \cellcolor{red!5}3/298& \cellcolor{red!5}1.01 \\  \hline
  \multirow{1}{*}{\cellcolor{red!27}Sexual Identity - Male homosexuality}  & \cellcolor{red!27}Roto, Maricas & \cellcolor{red!27}2 & \cellcolor{red!27}2/298& \cellcolor{red!27}0.67 \\  \hline
  
\end{longtable}
\end{center}


\textbf{\Large Result analysis:}

\begin{itemize}\item Taking into account the words that were detected, we can reach the conclusion these comments are associated with : : Gender - Female sexuality;Sexual Identity - General;Sexual Identity - Male homosexuality;%.

\item The percentage of hate speech related words is 2.0134.

\item Considering that the variable \textbf{Sexual Identity - General} has the most occurences in the post, we can interpret that this is the predominant hate speech.

\item Overall there were 6/6 occurences of hate speech related comments.\end{itemize}\end{document}