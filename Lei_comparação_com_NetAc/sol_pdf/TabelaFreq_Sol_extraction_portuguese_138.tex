\documentclass[11pt]{article}
\ExplSyntaxOn
\let\tl_length:n\tl_count:n
\ExplSyntaxOff
\usepackage{graphicx}
\usepackage{multirow}
\usepackage{colortbl}
\usepackage{longtable, array}
\usepackage{hyperref}
\usepackage[usenames,dvipsnames,svgnames,table]{xcolor}
\newlength\mylength
\usepackage[legalpaper, landscape, margin=0.8in]{geometry}
\newcommand{\MinNumber}{0}
\begin{document}

\textbf {\huge In Sol\_extraction\_portuguese\_138.json :}\newline \par\Large\textbf {Title: \large  Fernando Rocha testa positivo à covid-19 pela sexta vez }\newline {\par\large --- Table  1: Summary of the results per comment; }

 {\par\large --- \hyperlink{Table 2}{\textcolor{blue}{\underline{Table 2}}}: Summary of the results per sociolinguistic variable;}\newline \normalsize\newline

\centering\textbf{\large Table  1: Summary of the results per comment 
}
\newcommand{\MaxNumber}{0}%
\newcommand{\ApplyGradient}[1]{%
\pgfmathsetmacro{\PercentColor}{100.0*(#1-\MinNumber)/(\MaxNumber-\MinNumber)}
\xdef\PercentColor{\PercentColor}%
\cellcolor{LightSpringGreen!\PercentColor!LightRed}{#1}
}
\newcolumntype{C}[2]{>{\centering\arraybackslash}p{#1}}
\begin{center}
\setlength\mylength{\dimexpr\textwidth - 1\arrayrulewidth - 50\tabcolsep}
\begin{longtable}{|C{.65\mylength}|C{.30\mylength}|C{.12\mylength}|C{.12\mylength}|C{.12\mylength}|}
\hline
\textbf{Comment} & \textbf{KeyWords} & \textbf{Sociolinguistic variables (Hiper - Hipo)}  & \textbf{Hate Speech Frequency} & \textbf{Hate Speech Frequency(\%)} \\
\hline\cellcolor{green!27}\small Este \textbf{labrego} nortenho gasta os testes todos destinados a população portuguesa .\normalsize   & \cellcolor{green!27}Labrego & \cellcolor{green!27}Social Class - Working class & \cellcolor{green!27}1/12 & \cellcolor{green!27}8.333 \\  \hline
  \cellcolor{green!5}\small melhor \textbf{labrego} nortenho k 1 boi como tu...mouro d merda\normalsize   & \cellcolor{green!5}Labrego & \cellcolor{green!5}Social Class - Working class & \cellcolor{green!5}1/10 & \cellcolor{green!5}10.0 \\  \hline
  \cellcolor{green!27}\small Andou foi a comer mulheres há pala da miss piggy sem a \textbf{mulher} saber nos EUA.\normalsize   & \cellcolor{green!27}Mulher & \cellcolor{green!27}Gender - General & \cellcolor{green!27}1/16 & \cellcolor{green!27}6.25 \\  \hline
  
\end{longtable}
\end{center}


\centering\textbf{\large \hypertarget{Table 2}{Table 2}: Summary of the results per sociolinguistic variable 
}
\newcolumntype{C}[2]{>{\centering\arraybackslash}p{#1}}
\begin{center}
\setlength\mylength{\dimexpr\textwidth - 1\arrayrulewidth - 40\tabcolsep}
\begin{longtable}{|C{.50\mylength}|C{.30\mylength}|C{.15\mylength}|C{.15\mylength}|C{.15\mylength}|}
\hline
\textbf{Sociolinguistic variables (Hiper - Hipo)} & \textbf{KeyWords} & \textbf{Number of occurrences} & \textbf{Frequency}  & \textbf{Frequency(\%)} \\
\hline\multirow{1}{*}{\cellcolor{red!27}Social Class - Working class}  & \cellcolor{red!27}Labrego & \cellcolor{red!27}2 & \cellcolor{red!27}2/530& \cellcolor{red!27}0.38 \\  \hline
  \multirow{1}{*}{\cellcolor{red!5}Gender - General}  & \cellcolor{red!5}Mulher & \cellcolor{red!5}1 & \cellcolor{red!5}1/530& \cellcolor{red!5}0.19 \\  \hline
  
\end{longtable}
\end{center}


\textbf{\Large Result analysis:}

\begin{itemize}\item Taking into account the words that were detected, we can reach the conclusion these comments are associated with : : Social Class - Working class;Gender - General;%.

\item The percentage of hate speech related words is 0.566.

\item Considering that the variable \textbf{Social Class - Working class} has the most occurences in the post, we can interpret that this is the predominant hate speech.

\item Overall there were 3/21 occurences of hate speech related comments.\end{itemize}\end{document}