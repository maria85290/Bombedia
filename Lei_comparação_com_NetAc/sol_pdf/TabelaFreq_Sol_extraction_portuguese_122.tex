\documentclass[11pt]{article}
\ExplSyntaxOn
\let\tl_length:n\tl_count:n
\ExplSyntaxOff
\usepackage{graphicx}
\usepackage{multirow}
\usepackage{colortbl}
\usepackage{longtable, array}
\usepackage{hyperref}
\usepackage[usenames,dvipsnames,svgnames,table]{xcolor}
\newlength\mylength
\usepackage[legalpaper, landscape, margin=0.8in]{geometry}
\newcommand{\MinNumber}{0}
\begin{document}

\textbf {\huge In Sol\_extraction\_portuguese\_122.json :}\newline \par\Large\textbf {Title: \large  Futura ministra de Bolsonaro defende que a mulher 'nasceu para ser mãe' }\newline {\par\large --- Table  1: Summary of the results per comment; }

 {\par\large --- \hyperlink{Table 2}{\textcolor{blue}{\underline{Table 2}}}: Summary of the results per sociolinguistic variable;}\newline \normalsize\newline

\centering\textbf{\large Table  1: Summary of the results per comment 
}
\newcommand{\MaxNumber}{0}%
\newcommand{\ApplyGradient}[1]{%
\pgfmathsetmacro{\PercentColor}{100.0*(#1-\MinNumber)/(\MaxNumber-\MinNumber)}
\xdef\PercentColor{\PercentColor}%
\cellcolor{LightSpringGreen!\PercentColor!LightRed}{#1}
}
\newcolumntype{C}[2]{>{\centering\arraybackslash}p{#1}}
\begin{center}
\setlength\mylength{\dimexpr\textwidth - 1\arrayrulewidth - 50\tabcolsep}
\begin{longtable}{|C{.65\mylength}|C{.30\mylength}|C{.12\mylength}|C{.12\mylength}|C{.12\mylength}|}
\hline
\textbf{Comment} & \textbf{KeyWords} & \textbf{Sociolinguistic variables (Hiper - Hipo)}  & \textbf{Hate Speech Frequency} & \textbf{Hate Speech Frequency(\%)} \\
\hline\cellcolor{green!27}\small O projecto de lei (PL) 490, orquestrado pela chamada «bancada ruralista», que defende os interesses do agronegócio, visa alterar o Estatuto do \textbf{Índio}, datado de 1973, e, segundo o Brasil de Fato, pode «inviabilizar o reconhecimento dos territórios» das comunidades indígenas.O PL 490 tramita em conjunto com mais 11 projectos de conteúdo semelhante, tendo como relator o deputado Jerônimo Goergen, em quem o vencedor das presidenciais brasileiras, Jair Bolsonaro, terá mesmo pensado para comandar o Ministério da Agricultura.\normalsize   & \cellcolor{green!27}Índio & \cellcolor{green!27}Ethnicity - Native American & \cellcolor{green!27}1/79 & \cellcolor{green!27}1.266 \\  \hline
  \cellcolor{green!5}\small - Ó + mestre , nasceu a \textbf{mulher} , para ser mãe ?+ TAMBÉM ( i.e. , sim , mas não só) .- Ó + mestre , é a gravidez , um PROBLEMA ?+ Em SI , não .- Então , o que é ?+ Um Estado .- Mas um estado , de quê ?+ De gravidez..- ?..!...Áh Áh ! Pois..+ Pois..- Ó + mestre , é o aborto um problema , que caminha a vida inteira com a \textbf{mulher} ?+ Num número significativo de vezes , o aborto SÓ é PROBLEMA , quando é CONFUNDIDO com a SOLUÇÃO .- E isso quer dizer que...?+ Em discriminação inteligente , de entre outras qualidades amplamente desejáveis , o aborto é encontrado numa de situações três (3) : | Acidental , PREmeditado(I.V.G.) e ( minimamente) COMPREENSÍVEL | .- Mas ó  + mestre , um aborto PODE ser MINIMAMENTe COMPREENSÍVEL ?+ PODE .- É capaz de me dar dois exemplos ?+ Sou : 1º  Quando ( o aborto ) constituir meio único de subtrair risco considerável de morte ou de grave e irreversível lesão para o corpo ou/e saúde física - psíquica da \textbf{mulher} grávida . 2 º Aquando de fetos INviáveis ( INcapazes de sobreviver FORA do ÚTERO ).- Ó + mestre , é verdade QUe , NENHUMA \textbf{mulher} QUER abortar ?+ NÃO : a generalização em causa tem tanto de agradável para quem a faz , como objectivamente ,  de falsa . Lamentavelmente , existem monstrAs autênticas , MESMO no mundo real .- Ó + mestre , peço - lhe um breve comentário , a esta frase de Damares Alves : " Se necessário, estarei nas ruas com as TRAVESTIS, na porta das escolas com as CRIANÇAS que são discriminadas" . "+ No mínimo , PROFUNDAMENTe INFELIz e PRECIPITADA .- E porquê ?+ Não se deve colocar, na mesma frase , as palavras travestis         e          crianças .- E porquê + mestre ?+ São virtualmente INcompativeis .- Aaaaaaa...pois . <br/>               \normalsize   & \cellcolor{green!5}Mulher & \cellcolor{green!5}Gender - General & \cellcolor{green!5}4/354 & \cellcolor{green!5}1.13 \\  \hline
  \cellcolor{green!27}\small Pela cara dela já foi homem, a \textbf{mulher} é híbrida...\normalsize   & \cellcolor{green!27}Mulher & \cellcolor{green!27}Gender - General & \cellcolor{green!27}1/10 & \cellcolor{green!27}10.0 \\  \hline
  \cellcolor{green!5}\small Futura ministra de Bolsonaro defende que a \textbf{mulher} “nasceu para ser mãe”...É só para lembrar aos comentadores aqui em baixo que se as mulheres não nascessem para serem mães, eles nunca estariam aqui a comentar nem a fazer mehrda nenhuma... Ou como já nasceram não querem que nasçam mais...??<br/>Agora o marxismo cultural também quer negar que o futuro e a sustentabilidade da \textbf{raça} humana se deve ao facto das mulheres serem mães...?? Está bonito tá...!!<br/>Outra coisa é compatibilizar a maternidade com a possibilidade de todas as mulheres, todas, poderem ter uma carreira naquilo que quiserem e tiverem capacidade... e isso está a acontecer, mas sem o auxílio necessário dos Estados... E por isso a Europa está a definhar demograficamente como é natural... e a longo prazo com estas teorias as culturas Europeias como as conhecemos desaparecerão...!!!<br/>Claro que a globalização já prevê a importação de "parideiras" do 3º mundo, porque estas teorias não são para todas...<br/>Isto para não relembrar a nossa natureza animal com ciclos biológicos que a cultura emergente pretende ignorar...<br/>A maternidade é a função humana e social mais nobre da \textbf{mulher}...!!\normalsize   & \cellcolor{green!5}Mulher, Raça & \cellcolor{green!5}Ethnicity - General, Gender - General & \cellcolor{green!5}3/183 & \cellcolor{green!5}1.639 \\  \hline
  \cellcolor{green!27}\small Esta \textbf{mulher} traz um novo significado à palavra burro, as mulheres não servem só para ter filhos, algumas nunca tiveram por opção e ninguém tem nada a ver  com isso, as mulheres também servem para ser advogadas, ´medicas , enfermeiras, empresárias...\normalsize   & \cellcolor{green!27}Mulher & \cellcolor{green!27}Gender - General & \cellcolor{green!27}1/41 & \cellcolor{green!27}2.439 \\  \hline
  \cellcolor{green!5}\small coitada da \textbf{mulher} , enganou se , \textbf{mulher} nasceu para ser \textbf{prostituta} ........\normalsize   & \cellcolor{green!5}Mulher, Prostituta & \cellcolor{green!5}Gender - Female sexuality, Gender - General & \cellcolor{green!5}3/13 & \cellcolor{green!5}23.077 \\  \hline
  \cellcolor{green!27}\small Misturas de religião e política só pode dar nisto … ainda não está no poleiro e já mete bojardas da boca para fora, à \textbf{Boche} Naro … a mim me parece … !!! TAL € QUAL !!!\normalsize   & \cellcolor{green!27}Boche & \cellcolor{green!27}Nationality - German & \cellcolor{green!27}1/37 & \cellcolor{green!27}2.703 \\  \hline
  \cellcolor{green!5}\small A \textbf{mulher} nasceu para ser mãe e ser tudo aquilo que quiser , é uma opção em cada uma delas . <br/>Pode ser  mãe e conciliar outras coisas , dependendo dacsua vontade como vontade de ambos os progenitores e vida familiar em sua organização e consentimento de colaboração entre as partes que habitam o lar .A OPÇÃO SER MÃE , UMA DECISÃO DA \textbf{MULHER} MESMO QUE TENHA ESSE PREVILÉGIO , PODENDO ABDICAR DELE !Cada qual saberá as funções quer representar pela vida .Anterior , mãe dos filhos , assim desejo dela .Já outra \textbf{mulher} hoje presente na vida , não quer filhos , mas disponibilidade para os filhos dos outros , uma opção !Portanto os meus filhos são preveligiados têm duas mães ...\normalsize   & \cellcolor{green!5}Mulher & \cellcolor{green!5}Gender - General & \cellcolor{green!5}3/128 & \cellcolor{green!5}2.344 \\  \hline
  \cellcolor{green!27}\small Pois , parece que ser contra o aborto se tornou num defeito pessoal gravíssimo, como ser \textbf{gay}, \textbf{preto}, lgbtyh , chinês, etc, uma virtude para desempenhar um cargo público.<br/>Por cá , os nossos revolucionários contra a discriminação, ainda vão por em letra de lei que os homens também têm o direito a engravidar e serem mães.\normalsize   & \cellcolor{green!27}Gay, Preto & \cellcolor{green!27}Ethnicity - Black, Sexual Identity - General & \cellcolor{green!27}2/56 & \cellcolor{green!27}3.571 \\  \hline
  \cellcolor{green!5}\small Dantes ser \textbf{gay}, \textbf{preto}, lgbtyh , chinês, etc, era um defeito, agora é uma coisa normal,, são só seres-humanos.\normalsize   & \cellcolor{green!5}Gay, Preto & \cellcolor{green!5}Ethnicity - Black, Sexual Identity - General & \cellcolor{green!5}2/19 & \cellcolor{green!5}10.526 \\  \hline
  \cellcolor{green!27}\small Isso já acontece desde que casal \textbf{gay} pide adorar...\normalsize   & \cellcolor{green!27}Gay & \cellcolor{green!27}Sexual Identity - General & \cellcolor{green!27}1/9 & \cellcolor{green!27}11.111 \\  \hline
  
\end{longtable}
\end{center}


\centering\textbf{\large \hypertarget{Table 2}{Table 2}: Summary of the results per sociolinguistic variable 
}
\newcolumntype{C}[2]{>{\centering\arraybackslash}p{#1}}
\begin{center}
\setlength\mylength{\dimexpr\textwidth - 1\arrayrulewidth - 40\tabcolsep}
\begin{longtable}{|C{.50\mylength}|C{.30\mylength}|C{.15\mylength}|C{.15\mylength}|C{.15\mylength}|}
\hline
\textbf{Sociolinguistic variables (Hiper - Hipo)} & \textbf{KeyWords} & \textbf{Number of occurrences} & \textbf{Frequency}  & \textbf{Frequency(\%)} \\
\hline\multirow{1}{*}{\cellcolor{red!27}Ethnicity - Native American}  & \cellcolor{red!27}Índio & \cellcolor{red!27}1 & \cellcolor{red!27}1/1781& \cellcolor{red!27}0.06 \\  \hline
  \multirow{1}{*}{\cellcolor{red!5}Gender - General}  & \cellcolor{red!5}Mulher & \cellcolor{red!5}10 & \cellcolor{red!5}10/1781& \cellcolor{red!5}0.5599999999999999 \\  \hline
  \multirow{1}{*}{\cellcolor{red!27}Ethnicity - General}  & \cellcolor{red!27}Raça & \cellcolor{red!27}1 & \cellcolor{red!27}1/1781& \cellcolor{red!27}0.06 \\  \hline
  \multirow{1}{*}{\cellcolor{red!5}Gender - Female sexuality}  & \cellcolor{red!5}Prostituta & \cellcolor{red!5}1 & \cellcolor{red!5}1/1781& \cellcolor{red!5}0.06 \\  \hline
  \multirow{1}{*}{\cellcolor{red!27}Nationality - German}  & \cellcolor{red!27}Boche & \cellcolor{red!27}1 & \cellcolor{red!27}1/1781& \cellcolor{red!27}0.06 \\  \hline
  \multirow{1}{*}{\cellcolor{red!5}Ethnicity - Black}  & \cellcolor{red!5}Preto & \cellcolor{red!5}2 & \cellcolor{red!5}2/1781& \cellcolor{red!5}0.11 \\  \hline
  \multirow{1}{*}{\cellcolor{red!27}Sexual Identity - General}  & \cellcolor{red!27}Gay & \cellcolor{red!27}3 & \cellcolor{red!27}3/1781& \cellcolor{red!27}0.16999999999999998 \\  \hline
  
\end{longtable}
\end{center}


\textbf{\Large Result analysis:}

\begin{itemize}\item Taking into account the words that were detected, we can reach the conclusion these comments are associated with : : Ethnicity - Native American;Gender - General;Ethnicity - General;Gender - Female sexuality;Nationality - German;Ethnicity - Black;Sexual Identity - General;%.

\item The percentage of hate speech related words is 1.0668.

\item Considering that the variable \textbf{Gender - General} has the most occurences in the post, we can interpret that this is the predominant hate speech.

\item Overall there were 22/39 occurences of hate speech related comments.\end{itemize}\end{document}