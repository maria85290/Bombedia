\documentclass[11pt]{article}
\ExplSyntaxOn
\let\tl_length:n\tl_count:n
\ExplSyntaxOff
\usepackage{graphicx}
\usepackage{multirow}
\usepackage{colortbl}
\usepackage{longtable, array}
\usepackage{hyperref}
\usepackage[usenames,dvipsnames,svgnames,table]{xcolor}
\newlength\mylength
\usepackage[legalpaper, landscape, margin=0.8in]{geometry}
\newcommand{\MinNumber}{0}
\begin{document}

\textbf {\huge In Sol\_extraction\_portuguese\_93.json :}\newline \par\Large\textbf {Title: \large  José Carlos Malato: 'Devia ter morrido no ano passado. Não gosto nada do presente' }\newline {\par\large --- Table  1: Summary of the results per comment; }

 {\par\large --- \hyperlink{Table 2}{\textcolor{blue}{\underline{Table 2}}}: Summary of the results per sociolinguistic variable;}\newline \normalsize\newline

\centering\textbf{\large Table  1: Summary of the results per comment 
}
\newcommand{\MaxNumber}{0}%
\newcommand{\ApplyGradient}[1]{%
\pgfmathsetmacro{\PercentColor}{100.0*(#1-\MinNumber)/(\MaxNumber-\MinNumber)}
\xdef\PercentColor{\PercentColor}%
\cellcolor{LightSpringGreen!\PercentColor!LightRed}{#1}
}
\newcolumntype{C}[2]{>{\centering\arraybackslash}p{#1}}
\begin{center}
\setlength\mylength{\dimexpr\textwidth - 1\arrayrulewidth - 50\tabcolsep}
\begin{longtable}{|C{.65\mylength}|C{.30\mylength}|C{.12\mylength}|C{.12\mylength}|C{.12\mylength}|}
\hline
\textbf{Comment} & \textbf{KeyWords} & \textbf{Sociolinguistic variables (Hiper - Hipo)}  & \textbf{Hate Speech Frequency} & \textbf{Hate Speech Frequency(\%)} \\
\hline\cellcolor{green!27}\small Este malato teve falta de umas malaguetas na lingua quando era pequeno. A dizer estas enormes bacoradas  extremamente ofensivas a quem realmente sofre!!!! Se ele quiser ver bem quem sofre convido-o a ir ao IPO por exemplo ver crianças com doenças muito graves. Devia ter vergonha este \textbf{gajo}!!!\normalsize   & \cellcolor{green!27}Gajo & \cellcolor{green!27}Ethnicity - White & \cellcolor{green!27}1/48 & \cellcolor{green!27}2.083 \\  \hline
  \cellcolor{green!5}\small O vazio existencial é característico deste grupo, que sofre de distúrbio mental. O chamado orgulho \textbf{gay} e o afirmarem-se publicamente, funciona como uma espécie de paliativo para esse vazio existencial, durante um período. A juntar a isto durante o "período dourado", o serem famosos, terem fãs, terem sucesso, funciona também como paliativo, até perceberem que à medida que o tempo passa e saem da redoma ilusória que foi criada e eles próprios criaram à sua volta,  em nada contribuíram para um mundo melhor. Perceberem que afinal não passaram de modas/tendências como um produto para as massas estupidificadas consumirem. Quando percebem que estão a chegar bichas velhas, envoltos em ilusões e recheados de nada, em que é apenas uma questão de tempo até serem colocados num canto, colapsam.\normalsize   & \cellcolor{green!5}Gay & \cellcolor{green!5}Sexual Identity - General & \cellcolor{green!5}1/127 & \cellcolor{green!5}0.787 \\  \hline
  \cellcolor{green!27}\small Já não suporto mais notícias sobre esta \textbf{bichona} !!!<br/>Só podem ser encomendadas pelo próprio !\normalsize   & \cellcolor{green!27}Bichona & \cellcolor{green!27}Sexual Identity - Male homosexuality & \cellcolor{green!27}1/15 & \cellcolor{green!27}6.667 \\  \hline
  \cellcolor{green!5}\small Só não percebo a ambivalência...Se se concordar com a frase é ser-se, no mínimo, \textbf{homofóbico}, devo concluir que o Malato é \textbf{homofóbico} ou é só realmente estúpido?\normalsize   & \cellcolor{green!5}homofóbico & \cellcolor{green!5}Sexual Identity - General & \cellcolor{green!5}2/28 & \cellcolor{green!5}7.143 \\  \hline
  \cellcolor{green!27}\small caluda....que isso na \textbf{tuga} é uma heresia!!!!!\normalsize   & \cellcolor{green!27}Tuga & \cellcolor{green!27}Nationality - Portuguese & \cellcolor{green!27}1/7 & \cellcolor{green!27}14.286 \\  \hline
  \cellcolor{green!5}\small Que pena não ter morrido.<br/>Era menos uma \textbf{aberração} em Portugal.\normalsize   & \cellcolor{green!5}Aberração & \cellcolor{green!5}Sexual Identity - Male homosexuality & \cellcolor{green!5}1/10 & \cellcolor{green!5}10.0 \\  \hline
  \cellcolor{green!27}\small aberração és tu, \textbf{rabeta} envergonhado.\normalsize   & \cellcolor{green!27}Aberração, Rabeta & \cellcolor{green!27}Sexual Identity - Male homosexuality & \cellcolor{green!27}2/5 & \cellcolor{green!27}40.0 \\  \hline
  \cellcolor{green!5}\small Aberração porquê?\normalsize   & \cellcolor{green!5}Aberração & \cellcolor{green!5}Sexual Identity - Male homosexuality & \cellcolor{green!5}1/2 & \cellcolor{green!5}50.0 \\  \hline
  
\end{longtable}
\end{center}


\centering\textbf{\large \hypertarget{Table 2}{Table 2}: Summary of the results per sociolinguistic variable 
}
\newcolumntype{C}[2]{>{\centering\arraybackslash}p{#1}}
\begin{center}
\setlength\mylength{\dimexpr\textwidth - 1\arrayrulewidth - 40\tabcolsep}
\begin{longtable}{|C{.50\mylength}|C{.30\mylength}|C{.15\mylength}|C{.15\mylength}|C{.15\mylength}|}
\hline
\textbf{Sociolinguistic variables (Hiper - Hipo)} & \textbf{KeyWords} & \textbf{Number of occurrences} & \textbf{Frequency}  & \textbf{Frequency(\%)} \\
\hline\multirow{1}{*}{\cellcolor{red!27}Ethnicity - White}  & \cellcolor{red!27}Gajo & \cellcolor{red!27}1 & \cellcolor{red!27}1/590& \cellcolor{red!27}0.16999999999999998 \\  \hline
  \multirow{1}{*}{\cellcolor{red!5}Sexual Identity - General}  & \cellcolor{red!5}Gay, homofóbico & \cellcolor{red!5}3 & \cellcolor{red!5}3/590& \cellcolor{red!5}0.51 \\  \hline
  \multirow{1}{*}{\cellcolor{red!27}Sexual Identity - Male homosexuality}  & \cellcolor{red!27}Bichona, Aberração, Rabeta & \cellcolor{red!27}5 & \cellcolor{red!27}5/590& \cellcolor{red!27}0.8500000000000001 \\  \hline
  \multirow{1}{*}{\cellcolor{red!5}Nationality - Portuguese}  & \cellcolor{red!5}Tuga & \cellcolor{red!5}1 & \cellcolor{red!5}1/590& \cellcolor{red!5}0.16999999999999998 \\  \hline
  
\end{longtable}
\end{center}


\textbf{\Large Result analysis:}

\begin{itemize}\item Taking into account the words that were detected, we can reach the conclusion these comments are associated with : : Ethnicity - White;Sexual Identity - General;Sexual Identity - Male homosexuality;Nationality - Portuguese;%.

\item The percentage of hate speech related words is 1.6949.

\item Considering that the variable \textbf{Sexual Identity - Male homosexuality} has the most occurences in the post, we can interpret that this is the predominant hate speech.

\item Overall there were 10/18 occurences of hate speech related comments.\end{itemize}\end{document}