\documentclass[11pt]{article}
\ExplSyntaxOn
\let\tl_length:n\tl_count:n
\ExplSyntaxOff
\usepackage{graphicx}
\usepackage{multirow}
\usepackage{colortbl}
\usepackage{longtable, array}
\usepackage{hyperref}
\usepackage[usenames,dvipsnames,svgnames,table]{xcolor}
\newlength\mylength
\usepackage[legalpaper, landscape, margin=0.8in]{geometry}
\newcommand{\MinNumber}{0}
\begin{document}

\textbf {\huge In Sol\_extraction\_portuguese\_157.json :}\newline \par\Large\textbf {Title: \large  Padre de Pedrógão diz ser um maroto sem maldade }\newline {\par\large --- Table  1: Summary of the results per comment; }

 {\par\large --- \hyperlink{Table 2}{\textcolor{blue}{\underline{Table 2}}}: Summary of the results per sociolinguistic variable;}\newline \normalsize\newline

\centering\textbf{\large Table  1: Summary of the results per comment 
}
\newcommand{\MaxNumber}{0}%
\newcommand{\ApplyGradient}[1]{%
\pgfmathsetmacro{\PercentColor}{100.0*(#1-\MinNumber)/(\MaxNumber-\MinNumber)}
\xdef\PercentColor{\PercentColor}%
\cellcolor{LightSpringGreen!\PercentColor!LightRed}{#1}
}
\newcolumntype{C}[2]{>{\centering\arraybackslash}p{#1}}
\begin{center}
\setlength\mylength{\dimexpr\textwidth - 1\arrayrulewidth - 50\tabcolsep}
\begin{longtable}{|C{.65\mylength}|C{.30\mylength}|C{.12\mylength}|C{.12\mylength}|C{.12\mylength}|}
\hline
\textbf{Comment} & \textbf{KeyWords} & \textbf{Sociolinguistic variables (Hiper - Hipo)}  & \textbf{Hate Speech Frequency} & \textbf{Hate Speech Frequency(\%)} \\
\hline\cellcolor{green!27}\small O que não ê possível esquecer é o teu Português, \textbf{imbecil}...\normalsize   & \cellcolor{green!27}Imbecil & \cellcolor{green!27}Physical Identity - Physical (and Mental) Impairments & \cellcolor{green!27}1/11 & \cellcolor{green!27}9.091 \\  \hline
  \cellcolor{green!5}\small Isso são apenas eufemismos para quem quer defender o padre! Gostava de saber como reagiriam as pessoas se ele tivesse sexo com a \textbf{mulher} dum paroquiano!\normalsize   & \cellcolor{green!5}Mulher & \cellcolor{green!5}Gender - General & \cellcolor{green!5}1/26 & \cellcolor{green!5}3.846 \\  \hline
  \cellcolor{green!27}\small Todo o empregado que não respeita as regras da empresa é despedido com justa causa.Quem quer ter uma vida \textbf{sexual}, filhos e casar não vai para padre. Agora este ex-padre até pode mostrar o pirilau, não tem que se queixar.\normalsize   & \cellcolor{green!27}Sexual & \cellcolor{green!27}Gender - General & \cellcolor{green!27}1/41 & \cellcolor{green!27}2.439 \\  \hline
  
\end{longtable}
\end{center}


\centering\textbf{\large \hypertarget{Table 2}{Table 2}: Summary of the results per sociolinguistic variable 
}
\newcolumntype{C}[2]{>{\centering\arraybackslash}p{#1}}
\begin{center}
\setlength\mylength{\dimexpr\textwidth - 1\arrayrulewidth - 40\tabcolsep}
\begin{longtable}{|C{.50\mylength}|C{.30\mylength}|C{.15\mylength}|C{.15\mylength}|C{.15\mylength}|}
\hline
\textbf{Sociolinguistic variables (Hiper - Hipo)} & \textbf{KeyWords} & \textbf{Number of occurrences} & \textbf{Frequency}  & \textbf{Frequency(\%)} \\
\hline\multirow{1}{*}{\cellcolor{red!27}Physical Identity - Physical (and Mental) Impairments}  & \cellcolor{red!27}Imbecil & \cellcolor{red!27}1 & \cellcolor{red!27}1/569& \cellcolor{red!27}0.18 \\  \hline
  \multirow{1}{*}{\cellcolor{red!5}Gender - General}  & \cellcolor{red!5}Mulher, Sexual & \cellcolor{red!5}2 & \cellcolor{red!5}2/569& \cellcolor{red!5}0.35000000000000003 \\  \hline
  
\end{longtable}
\end{center}


\textbf{\Large Result analysis:}

\begin{itemize}\item Taking into account the words that were detected, we can reach the conclusion these comments are associated with : : Physical Identity - Physical (and Mental) Impairments;Gender - General;%.

\item The percentage of hate speech related words is 0.5272.

\item Considering that the variable \textbf{Gender - General} has the most occurences in the post, we can interpret that this is the predominant hate speech.

\item Overall there were 3/18 occurences of hate speech related comments.\end{itemize}\end{document}