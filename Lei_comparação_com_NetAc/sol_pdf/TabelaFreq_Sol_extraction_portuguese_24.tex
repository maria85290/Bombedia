\documentclass[11pt]{article}
\ExplSyntaxOn
\let\tl_length:n\tl_count:n
\ExplSyntaxOff
\usepackage{graphicx}
\usepackage{multirow}
\usepackage{colortbl}
\usepackage{longtable, array}
\usepackage{hyperref}
\usepackage[usenames,dvipsnames,svgnames,table]{xcolor}
\newlength\mylength
\usepackage[legalpaper, landscape, margin=0.8in]{geometry}
\newcommand{\MinNumber}{0}
\begin{document}

\textbf {\huge In Sol\_extraction\_portuguese\_24.json :}\newline \par\Large\textbf {Title: \large  Maria Cerqueira Gomes assumiu programa sem Goucha e aproximou-se de Cristina Ferreira }\newline {\par\large --- Table  1: Summary of the results per comment; }

 {\par\large --- \hyperlink{Table 2}{\textcolor{blue}{\underline{Table 2}}}: Summary of the results per sociolinguistic variable;}\newline \normalsize\newline

\centering\textbf{\large Table  1: Summary of the results per comment 
}
\newcommand{\MaxNumber}{0}%
\newcommand{\ApplyGradient}[1]{%
\pgfmathsetmacro{\PercentColor}{100.0*(#1-\MinNumber)/(\MaxNumber-\MinNumber)}
\xdef\PercentColor{\PercentColor}%
\cellcolor{LightSpringGreen!\PercentColor!LightRed}{#1}
}
\newcolumntype{C}[2]{>{\centering\arraybackslash}p{#1}}
\begin{center}
\setlength\mylength{\dimexpr\textwidth - 1\arrayrulewidth - 50\tabcolsep}
\begin{longtable}{|C{.65\mylength}|C{.30\mylength}|C{.12\mylength}|C{.12\mylength}|C{.12\mylength}|}
\hline
\textbf{Comment} & \textbf{KeyWords} & \textbf{Sociolinguistic variables (Hiper - Hipo)}  & \textbf{Hate Speech Frequency} & \textbf{Hate Speech Frequency(\%)} \\
\hline\cellcolor{green!27}\small Ahaha só perceberam agora que personagens como o Goucha já estão gastas. Ninguém vê aquilo por causa do Goucha.<br/>O Goucha andava há anos a reboque da Cristina e agora é um peso morto para a Maria.<br/>Deixem a \textbf{miúda} sozinha que são capazes de ter uma agradável surpresa.<br/>Mandem-no para a sua reforma dourada, já é tempo.\normalsize   & \cellcolor{green!27}Miúda & \cellcolor{green!27}Gender - General & \cellcolor{green!27}1/55 & \cellcolor{green!27}1.818 \\  \hline
  \cellcolor{green!5}\small Parabéns à Maria. Sem gritos, dentes de fora,  o estomago à mostra através do esofago, poses ultrapassadas em fotos, um pouco descaída ( . ), lol, e roupa imprópria mas tendo em conta a loja onde as mulheres mais tontas se empenham, a Maria brilha pela sua natualidade, alegria genuina, beleza, educação e uma figura linda.<br/>Uma fibra que aço para quem começa em terreno tão hóstil. Parabéns \textbf{mulher}.\normalsize   & \cellcolor{green!5}Mulher & \cellcolor{green!5}Gender - General & \cellcolor{green!5}1/68 & \cellcolor{green!5}1.471 \\  \hline
  \cellcolor{green!27}\small Isto quer dizer que a \textbf{paneleiragem} tem os dias contados.\normalsize   & \cellcolor{green!27}Paneleiragem & \cellcolor{green!27}Sexual Identity - Male homosexuality & \cellcolor{green!27}1/10 & \cellcolor{green!27}10.0 \\  \hline
  \cellcolor{green!5}\small o esforço dos comunistas em impor a \textbf{paneleiragem} como norma de vida não resultou desta vez<br/>mas vão continuar a tentar<br/>uma sociedade corrompida é mais fraca e susceptível de ser dominada pelo partido\normalsize   & \cellcolor{green!5}Paneleiragem & \cellcolor{green!5}Sexual Identity - Male homosexuality & \cellcolor{green!5}1/32 & \cellcolor{green!5}3.125 \\  \hline
  
\end{longtable}
\end{center}


\centering\textbf{\large \hypertarget{Table 2}{Table 2}: Summary of the results per sociolinguistic variable 
}
\newcolumntype{C}[2]{>{\centering\arraybackslash}p{#1}}
\begin{center}
\setlength\mylength{\dimexpr\textwidth - 1\arrayrulewidth - 40\tabcolsep}
\begin{longtable}{|C{.50\mylength}|C{.30\mylength}|C{.15\mylength}|C{.15\mylength}|C{.15\mylength}|}
\hline
\textbf{Sociolinguistic variables (Hiper - Hipo)} & \textbf{KeyWords} & \textbf{Number of occurrences} & \textbf{Frequency}  & \textbf{Frequency(\%)} \\
\hline\multirow{1}{*}{\cellcolor{red!27}Gender - General}  & \cellcolor{red!27}Miúda, Mulher & \cellcolor{red!27}2 & \cellcolor{red!27}2/380& \cellcolor{red!27}0.53 \\  \hline
  \multirow{1}{*}{\cellcolor{red!5}Sexual Identity - Male homosexuality}  & \cellcolor{red!5}Paneleiragem & \cellcolor{red!5}2 & \cellcolor{red!5}2/380& \cellcolor{red!5}0.53 \\  \hline
  
\end{longtable}
\end{center}


\textbf{\Large Result analysis:}

\begin{itemize}\item Taking into account the words that were detected, we can reach the conclusion these comments are associated with : : Gender - General;Sexual Identity - Male homosexuality;%.

\item The percentage of hate speech related words is 1.0526.

\item Considering that the variable \textbf{Gender - General} has the most occurences in the post, we can interpret that this is the predominant hate speech.

\item Overall there were 4/21 occurences of hate speech related comments.\end{itemize}\end{document}