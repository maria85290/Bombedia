\documentclass[11pt]{article}
\ExplSyntaxOn
\let\tl_length:n\tl_count:n
\ExplSyntaxOff
\usepackage{graphicx}
\usepackage{multirow}
\usepackage{colortbl}
\usepackage{longtable, array}
\usepackage{hyperref}
\usepackage[usenames,dvipsnames,svgnames,table]{xcolor}
\newlength\mylength
\usepackage[legalpaper, landscape, margin=0.8in]{geometry}
\newcommand{\MinNumber}{0}
\begin{document}

\textbf {\huge In Sol\_extraction\_portuguese\_110.json :}\newline \par\Large\textbf {Title: \large  Escola cristã expulsa aluna por usar camisola com arco-irís nas redes sociais }\newline {\par\large --- Table  1: Summary of the results per comment; }

 {\par\large --- \hyperlink{Table 2}{\textcolor{blue}{\underline{Table 2}}}: Summary of the results per sociolinguistic variable;}\newline \normalsize\newline

\centering\textbf{\large Table  1: Summary of the results per comment 
}
\newcommand{\MaxNumber}{0}%
\newcommand{\ApplyGradient}[1]{%
\pgfmathsetmacro{\PercentColor}{100.0*(#1-\MinNumber)/(\MaxNumber-\MinNumber)}
\xdef\PercentColor{\PercentColor}%
\cellcolor{LightSpringGreen!\PercentColor!LightRed}{#1}
}
\newcolumntype{C}[2]{>{\centering\arraybackslash}p{#1}}
\begin{center}
\setlength\mylength{\dimexpr\textwidth - 1\arrayrulewidth - 50\tabcolsep}
\begin{longtable}{|C{.65\mylength}|C{.30\mylength}|C{.12\mylength}|C{.12\mylength}|C{.12\mylength}|}
\hline
\textbf{Comment} & \textbf{KeyWords} & \textbf{Sociolinguistic variables (Hiper - Hipo)}  & \textbf{Hate Speech Frequency} & \textbf{Hate Speech Frequency(\%)} \\
\hline\cellcolor{green!27}\small " A palavra Deus para mim é nada mais que a expressão é produto da fraqueza humana ..." .Em parte até está certo , quando diz " A palavra Deus ..."A palavra Deus é uma utilização com sentido personificação humana , a necessidade do humano rotular , etiquetar , referenciar , endereçar ...Mas aquilo que intitulamos como Deus , um \textbf{género} de personificação humana , é tudo mais que essa simples palavra e pensamento , é tudo e o todo .É tudo e o todo mesmo aquilo que possamos dizer " nada" ou a palavra " vazio " .Porque lá está o " nada " não existe mais tudo é muito mais que nada , como aquilo que dizemos vazio , também não existe , mas mais tudo , porque tudo mesmo aquilo consideramos vazio , tudo é preenchido , está preenchido .Aquilo que dizemos Deus , é tudo e da forma que possam crer . Mas nem sempre ou seja não existem as palavras para caraterizar tudo daquilo que é existente , tanto que continuam a inventar palavras pelas descobertas daquilo que sempre esteve presente no universo , mas desconhecido anterior do humano .Portanto vou expressar desta forma para quem é céptico , numa linguagem rudimentar : Deus é a fonte de todas as energias conhecidas e desconhecidas a central cujo pode ser tudo é transformar o todo pelos tempos .\normalsize   & \cellcolor{green!27}Género & \cellcolor{green!27}Gender - General & \cellcolor{green!27}1/239 & \cellcolor{green!27}0.418 \\  \hline
  \cellcolor{green!5}\small Gente medieval....a \textbf{miúda} tem sorte em ser "banida" de uma escola que será banida a seu tempo\normalsize   & \cellcolor{green!5}Miúda & \cellcolor{green!5}Gender - General & \cellcolor{green!5}1/17 & \cellcolor{green!5}5.882 \\  \hline
  \cellcolor{green!27}\small "O vinho é coisa Santa,<br/>Que nasce da cêpa torta,<br/>A uns faz perder o tino,<br/>A outros errar a porta."Tasca \textbf{Rasca}, 3º azulejo a seguir à 2º garrafa de bagaceira.\normalsize   & \cellcolor{green!27}Rasca & \cellcolor{green!27}Social Class - Working class & \cellcolor{green!27}1/29 & \cellcolor{green!27}3.448 \\  \hline
  \cellcolor{green!5}\small + Que aí , no mundo real , só se engane quem se quiser enganar..+ Todos os pecados outros , são fora da carne , mas quem comete o pecado da imoralidade \textbf{sexual} , peca contra o corpo próprio ; o corpo vosso é santuário do Espírito Santo, que habita em vós, o qual tendes da parte de |||Deus||| , pelo que não pertenceis a vós mesmos .+ Os injustos , não herdarão o Reino de |||Deus||| , assim , não deveis deixar - vos enganar por imorais,  idolatras , adúlteros, nem pelos que se entregam a práticas homossexuais de espécie qualquer.<br/>             <br/>Termo do comunicado pontual do + mestre esclarecedor de ponto certo e determinado , Para o mundo real.\normalsize   & \cellcolor{green!5}Sexual & \cellcolor{green!5}Gender - General & \cellcolor{green!5}1/121 & \cellcolor{green!5}0.826 \\  \hline
  \cellcolor{green!27}\small Ser \textbf{gay} não é errado. Deus não deixa de amar ninguém por causa de cor de pele, origem, religião, classe social e/ou sexualidade.\normalsize   & \cellcolor{green!27}Gay & \cellcolor{green!27}Sexual Identity - General & \cellcolor{green!27}1/23 & \cellcolor{green!27}4.348 \\  \hline
  \cellcolor{green!5}\small Quem disse que ser \textbf{Gay} é errado ? Se uma escola tem uma filosofia diferente, só deve frequentar essa escola quem está de acordo. O respeito pela liberdade de cada um, só tem um sentido ?\normalsize   & \cellcolor{green!5}Gay & \cellcolor{green!5}Sexual Identity - General & \cellcolor{green!5}1/36 & \cellcolor{green!5}2.778 \\  \hline
  \cellcolor{green!27}\small O arco - íris é um fenómeno natural , que nada , mas rigorosamente nada , tem , teve ou terá a ver , com o ""movimento \textbf{LGBT}"" , este último sim , duma contra-natura ferocíssima , pelo que SE  "o arco-íris é hoje um símbolo reconhecido como sendo de apoio ao movimento \textbf{LGBT}" , pelo menos , não é por mim .Prosseguindo com o meu comentário e como se pode ler na própria notícia , não foi por este facto , isolado , que a aluna em causa foi excluída definitivamente da Escola Cristã que frequentava , mas sim , por uma sequência comportamental em nada coerente com as crenças da Whitefield Academy .Por último , não concordo que a homossexualidade ou outra prática \textbf{sexual} humana à excepção da heterossexualidade , seja uma "orientação" , antes , isso sim E ainda dentro da esfera humana , uma DESorientação (\textbf{sexual}) , pelo que se compreende fácil , expectável E obviamente , não se encontrar em concórdia com a doutrina da Escola em questão .\normalsize   & \cellcolor{green!27}LGBT, Sexual & \cellcolor{green!27}Gender - General, Sexual Identity - General & \cellcolor{green!27}4/176 & \cellcolor{green!27}2.273 \\  \hline
  \cellcolor{green!5}\small Só perguntando a quem veste se sabe se tem ou não a ver com o movimento \textbf{LGBT}.\normalsize   & \cellcolor{green!5}LGBT & \cellcolor{green!5}Sexual Identity - General & \cellcolor{green!5}1/17 & \cellcolor{green!5}5.882 \\  \hline
  \cellcolor{green!27}\small É livre de escolha. Só vai para essa escola quem quer, e ainda bem que é assim. A liberdade tem 2 sentidos. Ou será que a ditadura da \textbf{LGBT} é boa ?\normalsize   & \cellcolor{green!27}LGBT & \cellcolor{green!27}Sexual Identity - General & \cellcolor{green!27}1/32 & \cellcolor{green!27}3.125 \\  \hline
  \cellcolor{green!5}\small Que se diz livre? Quem não gosta, \textbf{muda} de escola. Aqui, quem não gosta, vai à mesma escola! Mas eles é que se calhar não são livres.... Eheheheh\normalsize   & \cellcolor{green!5}Muda & \cellcolor{green!5}Physical Identity - Physical (and Mental) Impairments & \cellcolor{green!5}1/28 & \cellcolor{green!5}3.571 \\  \hline
  \cellcolor{green!27}\small Felizmente vive num país em que o acesso à educação privada está por todo o lado. Não gosta, \textbf{muda} de escola. Em Portugal estavam tramados, a escola pública reina e é virtualmente impossível mudar de escola sempre que não nos agradam os valores e princípios que ensinam aos nossos filhos. Lá, quase todos podem escolher mudar, aqui quase ninguém consegue escolher... E caladinhos!\normalsize   & \cellcolor{green!27}Muda & \cellcolor{green!27}Physical Identity - Physical (and Mental) Impairments & \cellcolor{green!27}1/63 & \cellcolor{green!27}1.587 \\  \hline
  
\end{longtable}
\end{center}


\centering\textbf{\large \hypertarget{Table 2}{Table 2}: Summary of the results per sociolinguistic variable 
}
\newcolumntype{C}[2]{>{\centering\arraybackslash}p{#1}}
\begin{center}
\setlength\mylength{\dimexpr\textwidth - 1\arrayrulewidth - 40\tabcolsep}
\begin{longtable}{|C{.50\mylength}|C{.30\mylength}|C{.15\mylength}|C{.15\mylength}|C{.15\mylength}|}
\hline
\textbf{Sociolinguistic variables (Hiper - Hipo)} & \textbf{KeyWords} & \textbf{Number of occurrences} & \textbf{Frequency}  & \textbf{Frequency(\%)} \\
\hline\multirow{1}{*}{\cellcolor{red!27}Gender - General}  & \cellcolor{red!27}Género, Miúda, Sexual & \cellcolor{red!27}5 & \cellcolor{red!27}5/1494& \cellcolor{red!27}0.33 \\  \hline
  \multirow{1}{*}{\cellcolor{red!5}Social Class - Working class}  & \cellcolor{red!5}Rasca & \cellcolor{red!5}1 & \cellcolor{red!5}1/1494& \cellcolor{red!5}0.06999999999999999 \\  \hline
  \multirow{1}{*}{\cellcolor{red!27}Sexual Identity - General}  & \cellcolor{red!27}Gay, LGBT & \cellcolor{red!27}6 & \cellcolor{red!27}6/1494& \cellcolor{red!27}0.4 \\  \hline
  \multirow{1}{*}{\cellcolor{red!5}Physical Identity - Physical (and Mental) Impairments}  & \cellcolor{red!5}Muda & \cellcolor{red!5}2 & \cellcolor{red!5}2/1494& \cellcolor{red!5}0.13 \\  \hline
  
\end{longtable}
\end{center}


\textbf{\Large Result analysis:}

\begin{itemize}\item Taking into account the words that were detected, we can reach the conclusion these comments are associated with : : Gender - General;Social Class - Working class;Sexual Identity - General;Physical Identity - Physical (and Mental) Impairments;%.

\item The percentage of hate speech related words is 0.9371.

\item Considering that the variable \textbf{Sexual Identity - General} has the most occurences in the post, we can interpret that this is the predominant hate speech.

\item Overall there were 14/38 occurences of hate speech related comments.\end{itemize}\end{document}