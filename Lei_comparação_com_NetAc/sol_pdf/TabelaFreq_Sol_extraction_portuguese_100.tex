\documentclass[11pt]{article}
\ExplSyntaxOn
\let\tl_length:n\tl_count:n
\ExplSyntaxOff
\usepackage{graphicx}
\usepackage{multirow}
\usepackage{colortbl}
\usepackage{longtable, array}
\usepackage{hyperref}
\usepackage[usenames,dvipsnames,svgnames,table]{xcolor}
\newlength\mylength
\usepackage[legalpaper, landscape, margin=0.8in]{geometry}
\newcommand{\MinNumber}{0}
\begin{document}

\textbf {\huge In Sol\_extraction\_portuguese\_100.json :}\newline \par\Large\textbf {Title: \large  'A minha mãe foi proibida de privar comigo porque sou gay' }\newline {\par\large --- Table  1: Summary of the results per comment; }

 {\par\large --- \hyperlink{Table 2}{\textcolor{blue}{\underline{Table 2}}}: Summary of the results per sociolinguistic variable;}\newline \normalsize\newline

\centering\textbf{\large Table  1: Summary of the results per comment 
}
\newcommand{\MaxNumber}{0}%
\newcommand{\ApplyGradient}[1]{%
\pgfmathsetmacro{\PercentColor}{100.0*(#1-\MinNumber)/(\MaxNumber-\MinNumber)}
\xdef\PercentColor{\PercentColor}%
\cellcolor{LightSpringGreen!\PercentColor!LightRed}{#1}
}
\newcolumntype{C}[2]{>{\centering\arraybackslash}p{#1}}
\begin{center}
\setlength\mylength{\dimexpr\textwidth - 1\arrayrulewidth - 50\tabcolsep}
\begin{longtable}{|C{.65\mylength}|C{.30\mylength}|C{.12\mylength}|C{.12\mylength}|C{.12\mylength}|}
\hline
\textbf{Comment} & \textbf{KeyWords} & \textbf{Sociolinguistic variables (Hiper - Hipo)}  & \textbf{Hate Speech Frequency} & \textbf{Hate Speech Frequency(\%)} \\
\hline\cellcolor{green!27}\small Existem problemas maiores e os media persistem nesta agenda \textbf{LGBT} com estes personagens patéticos.<br/>Tem problemas? Resolva-os. Esta paneleirice de vir para as redes sociais carpir os problemas pessoais, é uma paneleirice.<br/>O gayzola procura pais? Já tem pais adoptivos, o "estado" que com o dinheiro de todos nós, que lhe paga ordenado para exibir no canal público, a sua marcha gayzola.\normalsize   & \cellcolor{green!27}LGBT & \cellcolor{green!27}Sexual Identity - General & \cellcolor{green!27}1/60 & \cellcolor{green!27}1.667 \\  \hline
  \cellcolor{green!5}\small "todo o homem precisa de uma mãe" !!!!<br/>ups<br/>como ficamos casais \textbf{gay}?\normalsize   & \cellcolor{green!5}Gay & \cellcolor{green!5}Sexual Identity - General & \cellcolor{green!5}1/11 & \cellcolor{green!5}9.091 \\  \hline
  \cellcolor{green!27}\small E não é assim - expondo os trapos na praça pública - que resolve as coisas.<br/>Eu sei que a primeira tendência é culpabilizar os outros, mas... a irmã também é \textbf{gay}? Ou é só por ter-se desassociado?<br/>Espanta-me um pouco que assim seja porque conheço várias famílias em que um elemento é Testemunha de Jeová e convivem todos em Paz e Harmonia. Aliás tinha uma Tia-Avó que era Testemunha de Jeová e, bem sei que eram outros tempos há 40 anos, viviam todos no mesma casa e só ela era TdJ. Nem sequer o marido, as filhas, os genros, os netos e as irmãs que viviam na mesma casa eram. Ía às suas reuniões no Salão, honrava a disciplinas e parametros da sua religião mas convivia, e amava, quem não era.<br/>Sinto muito que ele, Malato, sinta necessidade de expor a Mãe desta forma. Uma pessoa fica com medo de adoptar uma pessoa assim... é que perfeito só ele, na visão dele\normalsize   & \cellcolor{green!27}Gay & \cellcolor{green!27}Sexual Identity - General & \cellcolor{green!27}1/160 & \cellcolor{green!27}0.625 \\  \hline
  \cellcolor{green!5}\small Coitado, deve ter problemas de auto-estima, está constantemente a dizer que é \textbf{Gay}, partilha fotos \textbf{gay}'s, fala de \textbf{gay}'s, deve sonhar com \textbf{gay}'s e pensar em \textbf{gay}'s.Oh homem, resolva lá isso, consulte um psiquiatra que o seu mal é sono.\normalsize   & \cellcolor{green!5}Gay & \cellcolor{green!5}Sexual Identity - General & \cellcolor{green!5}1/41 & \cellcolor{green!5}2.439 \\  \hline
  \cellcolor{green!27}\small É isso mesmo, é um pobre coitado. Um caso de psiquiatria ou algo do \textbf{género}.\normalsize   & \cellcolor{green!27}Género & \cellcolor{green!27}Gender - General & \cellcolor{green!27}1/15 & \cellcolor{green!27}6.667 \\  \hline
  
\end{longtable}
\end{center}


\centering\textbf{\large \hypertarget{Table 2}{Table 2}: Summary of the results per sociolinguistic variable 
}
\newcolumntype{C}[2]{>{\centering\arraybackslash}p{#1}}
\begin{center}
\setlength\mylength{\dimexpr\textwidth - 1\arrayrulewidth - 40\tabcolsep}
\begin{longtable}{|C{.50\mylength}|C{.30\mylength}|C{.15\mylength}|C{.15\mylength}|C{.15\mylength}|}
\hline
\textbf{Sociolinguistic variables (Hiper - Hipo)} & \textbf{KeyWords} & \textbf{Number of occurrences} & \textbf{Frequency}  & \textbf{Frequency(\%)} \\
\hline\multirow{1}{*}{\cellcolor{red!27}Sexual Identity - General}  & \cellcolor{red!27}LGBT, Gay & \cellcolor{red!27}4 & \cellcolor{red!27}4/1569& \cellcolor{red!27}0.25 \\  \hline
  \multirow{1}{*}{\cellcolor{red!5}Gender - General}  & \cellcolor{red!5}Género & \cellcolor{red!5}1 & \cellcolor{red!5}1/1569& \cellcolor{red!5}0.06 \\  \hline
  
\end{longtable}
\end{center}


\textbf{\Large Result analysis:}

\begin{itemize}\item Taking into account the words that were detected, we can reach the conclusion these comments are associated with : : Sexual Identity - General;Gender - General;%.

\item The percentage of hate speech related words is 0.3187.

\item Considering that the variable \textbf{Sexual Identity - General} has the most occurences in the post, we can interpret that this is the predominant hate speech.

\item Overall there were 5/22 occurences of hate speech related comments.\end{itemize}\end{document}