\documentclass[11pt]{article}
\ExplSyntaxOn
\let\tl_length:n\tl_count:n
\ExplSyntaxOff
\usepackage{graphicx}
\usepackage{multirow}
\usepackage{colortbl}
\usepackage{longtable, array}
\usepackage{hyperref}
\usepackage[usenames,dvipsnames,svgnames,table]{xcolor}
\newlength\mylength
\usepackage[legalpaper, landscape, margin=0.8in]{geometry}
\newcommand{\MinNumber}{0}
\begin{document}

\textbf {\huge In Sol\_extraction\_portuguese\_96.json :}\newline \par\Large\textbf {Title: \large  José Carlos Malato deixa mensagem sobre homossexualidade nas redes sociais | Foto }\newline {\par\large --- Table  1: Summary of the results per comment; }

 {\par\large --- \hyperlink{Table 2}{\textcolor{blue}{\underline{Table 2}}}: Summary of the results per sociolinguistic variable;}\newline \normalsize\newline

\centering\textbf{\large Table  1: Summary of the results per comment 
}
\newcommand{\MaxNumber}{0}%
\newcommand{\ApplyGradient}[1]{%
\pgfmathsetmacro{\PercentColor}{100.0*(#1-\MinNumber)/(\MaxNumber-\MinNumber)}
\xdef\PercentColor{\PercentColor}%
\cellcolor{LightSpringGreen!\PercentColor!LightRed}{#1}
}
\newcolumntype{C}[2]{>{\centering\arraybackslash}p{#1}}
\begin{center}
\setlength\mylength{\dimexpr\textwidth - 1\arrayrulewidth - 50\tabcolsep}
\begin{longtable}{|C{.65\mylength}|C{.30\mylength}|C{.12\mylength}|C{.12\mylength}|C{.12\mylength}|}
\hline
\textbf{Comment} & \textbf{KeyWords} & \textbf{Sociolinguistic variables (Hiper - Hipo)}  & \textbf{Hate Speech Frequency} & \textbf{Hate Speech Frequency(\%)} \\
\hline\cellcolor{green!27}\small Convenhamos que a homossexualidade não é normal, podemos ser tolerantes e respeitar a tendência \textbf{sexual} de cada um. Uma coisa é a diferença a que cada um tem direito a ser, outra coisa é a normalidade. Quando as coisas são normais não é preciso "assumir". Só se assume a diferença, o resto é conversa do politicamente correto. O problema dos homossexuais é quererem convencer os outros que a sua sexualidade é normal, quando não é ! É uma anomalia ! E que sejam felizes com ela !!\normalsize   & \cellcolor{green!27}Sexual & \cellcolor{green!27}Gender - General & \cellcolor{green!27}1/87 & \cellcolor{green!27}1.149 \\  \hline
  \cellcolor{green!5}\small A vida íntima de cada um, a cada um diz respeito, não sendo admissível se a orientação \textbf{sexual} seja tornado público, principalmente quando fazem questão de espalhar aos quatro ventos e em alto em bom som,  que  gostam particularmente  da parte que serve para defecar.<br/>Tanto tentam publicitar tal anormalidade, disfunção, patologia ou pura e simples moda que qualquer dia terá de vir outro cataclismo para repor a ordem a esta "natureza".\normalsize   & \cellcolor{green!5}Sexual & \cellcolor{green!5}Gender - General & \cellcolor{green!5}1/71 & \cellcolor{green!5}1.408 \\  \hline
  \cellcolor{green!27}\small Mas que discurso mais INcoerente e sem nexo . Quando não , reveja - se ..Começa com  "Na minha ( malato ) geração, há coisas que a gente não faz por respeito aos pais." Ora , se se refere nomeadamente à assunção \textbf{homossexual} como depreendo e acho , COMO é que mais á frente , depois , se sai com esta "..  na verdade NUNCa ESCONDEU , mas também não falava publicamente, pelo menos até hoje. " Mas nunca escondeu...de QUEM ? De TODOs MENOS do Pai ??!! Isso não existe...Quer convencer - me , que durante os anos em que o pai foi vivo , o último seria o ÚNICO , que NÃO sabia que tinha um filho \textbf{homossexual} ??! Que NUNCA houve ALGUÉM que lhe TENHA dito ??!! Até pode ser verdade , mas pessoalmente , não consigo acreditar nisso .( ponto final parágrafo ) .Quanto ao te esconderes ou revelares como \textbf{homossexual} , para mim é me total E literalmente INdiferente..desde que não me incomodes com esse teu desvio à sexualidade humana normal ,  claro .\normalsize   & \cellcolor{green!27}Homossexual & \cellcolor{green!27}Sexual Identity - General & \cellcolor{green!27}3/181 & \cellcolor{green!27}1.657 \\  \hline
  \cellcolor{green!5}\small Não teve coragem para dizer ao pai, quando era vivo, que é \textbf{maricas} e agora vem com a treta de que o pai, "onde estiver, compreenderá, finalmente, todas as dimensões da palavra amor". Francamente !\normalsize   & \cellcolor{green!5}Maricas & \cellcolor{green!5}Sexual Identity - Male homosexuality & \cellcolor{green!5}1/35 & \cellcolor{green!5}2.857 \\  \hline
  \cellcolor{green!27}\small (É certo como o destino, onde há anilha há dificuldades de expressão...)Normalmente dou pouca importância às futilidades que saem dos orifícios das nulidades mediáticas (e aqui não será excepção), de qualquer forma duas notas rápidas:Já que o agora assumido paineleiro (e como não vivemos no Egipto ou no Paquistão ou noutro antro da religião da paz, tal assumir nada tem de "corajoso" que ninguém lhe faz mal por chupar na palhinha. Pelo contrário, pelo menos no meio artístico, o pegar de marcha-atrás é uma patente vantagem!) vem matar o pai, que já está morto, em público, julgando que é beijar outro homem que, comparativamente, é que choca. Tenho que lembrar, sobre isso, uma coisa que pelos vistos o Malato faz por esquecer ou não quer saber:O nunca ter contado ao pai que era um \textbf{homossexual} que correu Ceca e Meca (salvo seja) a ser feliz, significa que preferiu um amor falso, preferiu um amor dedicado a outra pessoa que não ele a permitir ao pai ser (ou não ser) o seu real pai, amando-o ou não.Viver a vida como que uma fantasia de lantejoulas, vale o que vale, no mínimo, é uma sobranceria contra os outros, é subestimar a capacidade de cada um de estar ou não à altura dos seus deveres morais. E com toda a certeza, não é um beijo \textbf{gay} que apaga tal tristeza!A segunda coisa que quero sublinhar neste fait diver de qualidade duvidosa é a necessidade que tantos têm de constantemente exibirem o seu intimo, mesmo coisas realmente próximas e, por isso mesmo, quando expostas, grotescas (e não não é o beijo, é a mentira feia!). É um aquário imundo, que fede e que de todo está plausível para exibição e no entanto ninguém os convence e por isso insistem!De dar vómitos...\normalsize   & \cellcolor{green!27}Gay, Homossexual & \cellcolor{green!27}Sexual Identity - General & \cellcolor{green!27}2/303 & \cellcolor{green!27}0.66 \\  \hline
  \cellcolor{green!5}\small Respeito teve sim, enquanto o pai esteve vivo. Também és daquelas que, depois de o marido morrer, passam a vestir de \textbf{negro} e não olham para mais nenhum homem? Tudo bem, é contigo, mas não tens que o impor aos outros nem julgá-los por isso.\normalsize   & \cellcolor{green!5}Negro & \cellcolor{green!5}Ethnicity - Black & \cellcolor{green!5}1/45 & \cellcolor{green!5}2.222 \\  \hline
  \cellcolor{green!27}\small Mas os filhos não são " nossos " são do mundo .São responsabilidade nossa enquanto são menores , após maiores são responsáveis pelas suas vidas e restam confraternizações em momentos após saírem do tecto do lar . Isso que as pessoas terão que se capacitar , ninguém é dono de ninguém após derteminada \textbf{idade} e cada qual tem de viver a vida como lhe aprouver .Sempre existiu essa característica desde os primórdios dos tempos . QUEM SOMOS NÓS ?Para contrariar o ser em cada . <br/>Desde que ninguém seja obrigado a nada contra a sua vontade , o restante fica para dois que comungam de igual . <br/>E, para mim nesse aspecto não dizem nada como a \textbf{mulher} do vizinho é igual , porque vivo e partilha a vida com a minha . O restante é para respeitar como seres humanos semelhantes como qual o respeito de igual mesmo sem o conhecer . <br/>É vida e certas coisas não temos o poder de as modificar , senão então o mundo ainda seria muito pior .Ê assim e pronto aceitar o cada quer ser desde que não haja intromissão pelo nosso viver . É SIMPLES !\normalsize   & \cellcolor{green!27}Idade, Mulher & \cellcolor{green!27}Age - General, Gender - General & \cellcolor{green!27}2/199 & \cellcolor{green!27}1.005 \\  \hline
  \cellcolor{green!5}\small Comentário idiota\normalsize   & \cellcolor{green!5}Idiota & \cellcolor{green!5}Physical Identity - Physical (and Mental) Impairments & \cellcolor{green!5}1/2 & \cellcolor{green!5}50.0 \\  \hline
  \cellcolor{green!27}\small ARROZ ! <br/>Vou aproveitar o seu comentário para avivar outro ver nas coisas pela vida .Dai então que os pais , quando filhos pequenos em crescimento terão também ter o respeito para com eles e agir no conhecimento e convencimento sem a violência física . <br/>Pela razão todos pela \textbf{idade} definhamos e quando grandes esses filhos , nós mais velhos , senão tiverem o tal respeito , podemos levar " coça " \textbf{v\textbf{elha}} pelas agressões a eles , quando novos e crianças .Após por vezes escutamos notícias que são repugnantes filhos agrediram país idosos , mas nem sabemos o tal histórico que possam ter feito em crianças esses pais , para não merecerem o respeito ?Sabe por vezes , muitos pais ainda tem muito sorte , pela razão daquilo que menosprezaram os filhos em pequenos , os considerar como país .Isto , uma variante do  tema , mas aproveitei a senda , para se ter o respeito tem que se saber respeitar também . Bem haja .\normalsize   & \cellcolor{green!27}Idade, Velha & \cellcolor{green!27}Age - General, Age - Over 65s, Gender - Female age and physical appearance & \cellcolor{green!27}3/171 & \cellcolor{green!27}1.754 \\  \hline
  \cellcolor{green!5}\small a comunidade \textbf{lgbt} não quer que a população mude a sua orientação \textbf{sexual}. Quer que aceitem as "diferenças". Se o pai do Carlos se identificar como heterosexual não percebi o objetivo deste comentário. Deve informar-se melhor em vez de usar argumentos, quem nem o são, sem nexo algum. Se a senhora fosse um animal irracional também não estaria aqui a comentar\normalsize   & \cellcolor{green!5}LGBT, Sexual & \cellcolor{green!5}Gender - General, Sexual Identity - General & \cellcolor{green!5}2/61 & \cellcolor{green!5}3.279 \\  \hline
  \cellcolor{green!27}\small Com doação através de barriga de aluguer , o facto de não utilizar \textbf{mulher} tem mãozinha . 😱.\normalsize   & \cellcolor{green!27}Mulher & \cellcolor{green!27}Gender - General & \cellcolor{green!27}1/18 & \cellcolor{green!27}5.556 \\  \hline
  
\end{longtable}
\end{center}


\centering\textbf{\large \hypertarget{Table 2}{Table 2}: Summary of the results per sociolinguistic variable 
}
\newcolumntype{C}[2]{>{\centering\arraybackslash}p{#1}}
\begin{center}
\setlength\mylength{\dimexpr\textwidth - 1\arrayrulewidth - 40\tabcolsep}
\begin{longtable}{|C{.50\mylength}|C{.30\mylength}|C{.15\mylength}|C{.15\mylength}|C{.15\mylength}|}
\hline
\textbf{Sociolinguistic variables (Hiper - Hipo)} & \textbf{KeyWords} & \textbf{Number of occurrences} & \textbf{Frequency}  & \textbf{Frequency(\%)} \\
\hline\multirow{1}{*}{\cellcolor{red!27}Gender - General}  & \cellcolor{red!27}Sexual, Mulher & \cellcolor{red!27}5 & \cellcolor{red!27}5/1822& \cellcolor{red!27}0.27 \\  \hline
  \multirow{1}{*}{\cellcolor{red!5}Sexual Identity - General}  & \cellcolor{red!5}Homossexual, Gay, LGBT & \cellcolor{red!5}4 & \cellcolor{red!5}4/1822& \cellcolor{red!5}0.22 \\  \hline
  \multirow{1}{*}{\cellcolor{red!27}Sexual Identity - Male homosexuality}  & \cellcolor{red!27}Maricas & \cellcolor{red!27}1 & \cellcolor{red!27}1/1822& \cellcolor{red!27}0.05 \\  \hline
  \multirow{1}{*}{\cellcolor{red!5}Ethnicity - Black}  & \cellcolor{red!5}Negro & \cellcolor{red!5}1 & \cellcolor{red!5}1/1822& \cellcolor{red!5}0.05 \\  \hline
  \multirow{1}{*}{\cellcolor{red!27}Age - General}  & \cellcolor{red!27}Idade & \cellcolor{red!27}2 & \cellcolor{red!27}2/1822& \cellcolor{red!27}0.11 \\  \hline
  \multirow{1}{*}{\cellcolor{red!5}Physical Identity - Physical (and Mental) Impairments}  & \cellcolor{red!5}Idiota & \cellcolor{red!5}1 & \cellcolor{red!5}1/1822& \cellcolor{red!5}0.05 \\  \hline
  \multirow{1}{*}{\cellcolor{red!27}Gender - Female age and physical appearance}  & \cellcolor{red!27}Velha & \cellcolor{red!27}1 & \cellcolor{red!27}1/1822& \cellcolor{red!27}0.05 \\  \hline
  \multirow{1}{*}{\cellcolor{red!5}Age - Over 65s}  & \cellcolor{red!5}Velha & \cellcolor{red!5}1 & \cellcolor{red!5}1/1822& \cellcolor{red!5}0.05 \\  \hline
  
\end{longtable}
\end{center}


\textbf{\Large Result analysis:}

\begin{itemize}\item Taking into account the words that were detected, we can reach the conclusion these comments are associated with : : Gender - General;Sexual Identity - General;Sexual Identity - Male homosexuality;Ethnicity - Black;Age - General;Physical Identity - Physical (and Mental) Impairments;Gender - Female age and physical appearance;Age - Over 65s;%.

\item The percentage of hate speech related words is 0.8782.

\item Considering that the variable \textbf{Gender - General} has the most occurences in the post, we can interpret that this is the predominant hate speech.

\item Overall there were 18/31 occurences of hate speech related comments.\end{itemize}\end{document}