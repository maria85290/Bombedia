\documentclass[11pt]{article}
\ExplSyntaxOn
\let\tl_length:n\tl_count:n
\ExplSyntaxOff
\usepackage{graphicx}
\usepackage{multirow}
\usepackage{colortbl}
\usepackage{longtable, array}
\usepackage{hyperref}
\usepackage[usenames,dvipsnames,svgnames,table]{xcolor}
\newlength\mylength
\usepackage[legalpaper, landscape, margin=0.8in]{geometry}
\newcommand{\MinNumber}{0}
\begin{document}

\textbf {\huge In Sol\_extraction\_portuguese\_37.json :}\newline \par\Large\textbf {Title: \large  Manuel Luís Goucha não foi discriminado pelos tribunais }\newline {\par\large --- Table  1: Summary of the results per comment; }

 {\par\large --- \hyperlink{Table 2}{\textcolor{blue}{\underline{Table 2}}}: Summary of the results per sociolinguistic variable;}\newline \normalsize\newline

\centering\textbf{\large Table  1: Summary of the results per comment 
}
\newcommand{\MaxNumber}{0}%
\newcommand{\ApplyGradient}[1]{%
\pgfmathsetmacro{\PercentColor}{100.0*(#1-\MinNumber)/(\MaxNumber-\MinNumber)}
\xdef\PercentColor{\PercentColor}%
\cellcolor{LightSpringGreen!\PercentColor!LightRed}{#1}
}
\newcolumntype{C}[2]{>{\centering\arraybackslash}p{#1}}
\begin{center}
\setlength\mylength{\dimexpr\textwidth - 1\arrayrulewidth - 50\tabcolsep}
\begin{longtable}{|C{.65\mylength}|C{.30\mylength}|C{.12\mylength}|C{.12\mylength}|C{.12\mylength}|}
\hline
\textbf{Comment} & \textbf{KeyWords} & \textbf{Sociolinguistic variables (Hiper - Hipo)}  & \textbf{Hate Speech Frequency} & \textbf{Hate Speech Frequency(\%)} \\
\hline\cellcolor{green!27}\small O tipo quer ser \textbf{mulher} e depois ofende-se quando alguém lhe chama senhora.<br/>Devia estar orgulhosa.\normalsize   & \cellcolor{green!27}Mulher & \cellcolor{green!27}Gender - General & \cellcolor{green!27}1/15 & \cellcolor{green!27}6.667 \\  \hline
  \cellcolor{green!5}\small este  grande \textbf{gay} já era banido da televisão\normalsize   & \cellcolor{green!5}Gay & \cellcolor{green!5}Sexual Identity - General & \cellcolor{green!5}1/8 & \cellcolor{green!5}12.5 \\  \hline
  \cellcolor{green!27}\small Ah! A estafada desculpa: sou perseguido porque sou \textbf{gay}. Sou perseguido porque sou \textbf{preto}. Sou perseguido porque sou \textbf{estrangeiro}. Sou perseguido porque sou ativista. Sou perseguido porque sou isto ou aquilo. Quando não há razões objetivas para justificar o injustificável recorre-se a estas linhas. Tristeza, julgam-se acima da lei.\normalsize   & \cellcolor{green!27}Estrangeiro, Gay, Preto & \cellcolor{green!27}Ethnicity - Black, Nationality - General, Sexual Identity - General & \cellcolor{green!27}3/49 & \cellcolor{green!27}6.122 \\  \hline
  
\end{longtable}
\end{center}


\centering\textbf{\large \hypertarget{Table 2}{Table 2}: Summary of the results per sociolinguistic variable 
}
\newcolumntype{C}[2]{>{\centering\arraybackslash}p{#1}}
\begin{center}
\setlength\mylength{\dimexpr\textwidth - 1\arrayrulewidth - 40\tabcolsep}
\begin{longtable}{|C{.50\mylength}|C{.30\mylength}|C{.15\mylength}|C{.15\mylength}|C{.15\mylength}|}
\hline
\textbf{Sociolinguistic variables (Hiper - Hipo)} & \textbf{KeyWords} & \textbf{Number of occurrences} & \textbf{Frequency}  & \textbf{Frequency(\%)} \\
\hline\multirow{1}{*}{\cellcolor{red!27}Gender - General}  & \cellcolor{red!27}Mulher & \cellcolor{red!27}1 & \cellcolor{red!27}1/226& \cellcolor{red!27}0.44 \\  \hline
  \multirow{1}{*}{\cellcolor{red!5}Sexual Identity - General}  & \cellcolor{red!5}Gay & \cellcolor{red!5}2 & \cellcolor{red!5}2/226& \cellcolor{red!5}0.88 \\  \hline
  \multirow{1}{*}{\cellcolor{red!27}Ethnicity - Black}  & \cellcolor{red!27}Preto & \cellcolor{red!27}1 & \cellcolor{red!27}1/226& \cellcolor{red!27}0.44 \\  \hline
  \multirow{1}{*}{\cellcolor{red!5}Nationality - General}  & \cellcolor{red!5}Estrangeiro & \cellcolor{red!5}1 & \cellcolor{red!5}1/226& \cellcolor{red!5}0.44 \\  \hline
  
\end{longtable}
\end{center}


\textbf{\Large Result analysis:}

\begin{itemize}\item Taking into account the words that were detected, we can reach the conclusion these comments are associated with : : Gender - General;Sexual Identity - General;Ethnicity - Black;Nationality - General;%.

\item The percentage of hate speech related words is 2.2124.

\item Considering that the variable \textbf{Sexual Identity - General} has the most occurences in the post, we can interpret that this is the predominant hate speech.

\item Overall there were 5/9 occurences of hate speech related comments.\end{itemize}\end{document}