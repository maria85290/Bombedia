\documentclass[11pt]{article}
\ExplSyntaxOn
\let\tl_length:n\tl_count:n
\ExplSyntaxOff
\usepackage{graphicx}
\usepackage{multirow}
\usepackage{colortbl}
\usepackage{longtable, array}
\usepackage{hyperref}
\usepackage[usenames,dvipsnames,svgnames,table]{xcolor}
\newlength\mylength
\usepackage[legalpaper, landscape, margin=0.8in]{geometry}
\newcommand{\MinNumber}{0}
\begin{document}

\textbf {\huge In Sol\_extraction\_portuguese\_182.json :}\newline \par\Large\textbf {Title: \large  Bazuca de pólvora seca desespera UE }\newline {\par\large --- Table  1: Summary of the results per comment; }

 {\par\large --- \hyperlink{Table 2}{\textcolor{blue}{\underline{Table 2}}}: Summary of the results per sociolinguistic variable;}\newline \normalsize\newline

\centering\textbf{\large Table  1: Summary of the results per comment 
}
\newcommand{\MaxNumber}{0}%
\newcommand{\ApplyGradient}[1]{%
\pgfmathsetmacro{\PercentColor}{100.0*(#1-\MinNumber)/(\MaxNumber-\MinNumber)}
\xdef\PercentColor{\PercentColor}%
\cellcolor{LightSpringGreen!\PercentColor!LightRed}{#1}
}
\newcolumntype{C}[2]{>{\centering\arraybackslash}p{#1}}
\begin{center}
\setlength\mylength{\dimexpr\textwidth - 1\arrayrulewidth - 50\tabcolsep}
\begin{longtable}{|C{.65\mylength}|C{.30\mylength}|C{.12\mylength}|C{.12\mylength}|C{.12\mylength}|}
\hline
\textbf{Comment} & \textbf{KeyWords} & \textbf{Sociolinguistic variables (Hiper - Hipo)}  & \textbf{Hate Speech Frequency} & \textbf{Hate Speech Frequency(\%)} \\
\hline\cellcolor{green!27}\small Boa tarde<br/>A EU está completamente morta. Profere que o pluralismo, as tradições e culturas existentes na Europa são a riqueza da EU.<br/>Pões bem eu também penso assim, mas não somente na teoria como a EU. Há dirigentes na EU a acusar a Hungria e a Polónia de não respeitar os direitos humanos. O acesso ao orçamento da EU, somente poderá ser feita por estados que cumpram o chamado estado de direito e não violem os direitos humanos.<br/>Sinceramente há aqui alguma coisa que me fugiu!<br/>Os países que pedem ordem para poder integrar a EU, não têm de terá essa premissa? Então deixaram aceder à EU 2 países que não respeitam o estado de direito e os direitos humanos?<br/>Pobre UE, ou será que esta já não quer saber da pluralidade de pensamento democrático dentro da EU e somente exige que todos leiam pela mesma cartilha de conveniência dos grandes? Mas quem pode do ponto de vista político dizer que a Hungria e a Polonia não cumprem 2 regras básica da EU: direitos humanos e estado de direito? Os políticos? Ou terão de ser as leis a definir e julgar os desvios dessas regras?<br/>Pelos vistos as bazucas não são a melhor arma de combate em caso de pandemia, mas tão e somente o humanismo, a justiça, independência e o bom senso que pelos vistos se estão a perder numa EU que até tem na sua liderança uma \textbf{mulher} que eu sempre admirei.<br/>Espero não mudar de ideias<br/>Cumprimentos\normalsize   & \cellcolor{green!27}Mulher & \cellcolor{green!27}Gender - General & \cellcolor{green!27}1/243 & \cellcolor{green!27}0.412 \\  \hline
  
\end{longtable}
\end{center}


\centering\textbf{\large \hypertarget{Table 2}{Table 2}: Summary of the results per sociolinguistic variable 
}
\newcolumntype{C}[2]{>{\centering\arraybackslash}p{#1}}
\begin{center}
\setlength\mylength{\dimexpr\textwidth - 1\arrayrulewidth - 40\tabcolsep}
\begin{longtable}{|C{.50\mylength}|C{.30\mylength}|C{.15\mylength}|C{.15\mylength}|C{.15\mylength}|}
\hline
\textbf{Sociolinguistic variables (Hiper - Hipo)} & \textbf{KeyWords} & \textbf{Number of occurrences} & \textbf{Frequency}  & \textbf{Frequency(\%)} \\
\hline\multirow{1}{*}{\cellcolor{red!27}Gender - General}  & \cellcolor{red!27}Mulher & \cellcolor{red!27}1 & \cellcolor{red!27}1/1478& \cellcolor{red!27}0.06999999999999999 \\  \hline
  
\end{longtable}
\end{center}


\textbf{\Large Result analysis:}

\begin{itemize}\item Taking into account the words that were detected, we can reach the conclusion these comments are associated with : : Gender - General;%.

\item The percentage of hate speech related words is 0.0677.

\item Considering that the variable \textbf{Gender - General} has the most occurences in the post, we can interpret that this is the predominant hate speech.

\item Overall there were 1/32 occurences of hate speech related comments.\end{itemize}\end{document}