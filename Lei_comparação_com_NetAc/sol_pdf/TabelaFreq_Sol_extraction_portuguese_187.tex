\documentclass[11pt]{article}
\ExplSyntaxOn
\let\tl_length:n\tl_count:n
\ExplSyntaxOff
\usepackage{graphicx}
\usepackage{multirow}
\usepackage{colortbl}
\usepackage{longtable, array}
\usepackage{hyperref}
\usepackage[usenames,dvipsnames,svgnames,table]{xcolor}
\newlength\mylength
\usepackage[legalpaper, landscape, margin=0.8in]{geometry}
\newcommand{\MinNumber}{0}
\begin{document}

\textbf {\huge In Sol\_extraction\_portuguese\_187.json :}\newline \par\Large\textbf {Title: \large  António Costa afirma que o país enfrenta gigantesca responsabilidade }\newline {\par\large --- Table  1: Summary of the results per comment; }

 {\par\large --- \hyperlink{Table 2}{\textcolor{blue}{\underline{Table 2}}}: Summary of the results per sociolinguistic variable;}\newline \normalsize\newline

\centering\textbf{\large Table  1: Summary of the results per comment 
}
\newcommand{\MaxNumber}{0}%
\newcommand{\ApplyGradient}[1]{%
\pgfmathsetmacro{\PercentColor}{100.0*(#1-\MinNumber)/(\MaxNumber-\MinNumber)}
\xdef\PercentColor{\PercentColor}%
\cellcolor{LightSpringGreen!\PercentColor!LightRed}{#1}
}
\newcolumntype{C}[2]{>{\centering\arraybackslash}p{#1}}
\begin{center}
\setlength\mylength{\dimexpr\textwidth - 1\arrayrulewidth - 50\tabcolsep}
\begin{longtable}{|C{.65\mylength}|C{.30\mylength}|C{.12\mylength}|C{.12\mylength}|C{.12\mylength}|}
\hline
\textbf{Comment} & \textbf{KeyWords} & \textbf{Sociolinguistic variables (Hiper - Hipo)}  & \textbf{Hate Speech Frequency} & \textbf{Hate Speech Frequency(\%)} \\
\hline\cellcolor{green!27}\small "O objetivo é conciliar a “máxima transparência e o mínimo de burocracia”"Coisa que esta geringonça não sabe fazer... ou seja, este \textbf{macaco} PS só sabe governar-se. isto, enquanto há dinheiro.\normalsize   & \cellcolor{green!27}Macaco & \cellcolor{green!27}Ethnicity - Black & \cellcolor{green!27}1/31 & \cellcolor{green!27}3.226 \\  \hline
  \cellcolor{green!5}\small Até quando tenho de aturar um \textbf{m\textbf{onhé}} como primeiro-ministro?\normalsize   & \cellcolor{green!5}Monhé & \cellcolor{green!5}Ethnicity - Asian (South- India, Pakistan, Bangladesh), Nationality - Indian & \cellcolor{green!5}2/9 & \cellcolor{green!5}22.222 \\  \hline
  
\end{longtable}
\end{center}


\centering\textbf{\large \hypertarget{Table 2}{Table 2}: Summary of the results per sociolinguistic variable 
}
\newcolumntype{C}[2]{>{\centering\arraybackslash}p{#1}}
\begin{center}
\setlength\mylength{\dimexpr\textwidth - 1\arrayrulewidth - 40\tabcolsep}
\begin{longtable}{|C{.50\mylength}|C{.30\mylength}|C{.15\mylength}|C{.15\mylength}|C{.15\mylength}|}
\hline
\textbf{Sociolinguistic variables (Hiper - Hipo)} & \textbf{KeyWords} & \textbf{Number of occurrences} & \textbf{Frequency}  & \textbf{Frequency(\%)} \\
\hline\multirow{1}{*}{\cellcolor{red!27}Ethnicity - Black}  & \cellcolor{red!27}Macaco & \cellcolor{red!27}1 & \cellcolor{red!27}1/1284& \cellcolor{red!27}0.08 \\  \hline
  \multirow{1}{*}{\cellcolor{red!5}Ethnicity - Asian (South- India, Pakistan, Bangladesh)}  & \cellcolor{red!5}Monhé & \cellcolor{red!5}1 & \cellcolor{red!5}1/1284& \cellcolor{red!5}0.08 \\  \hline
  \multirow{1}{*}{\cellcolor{red!27}Nationality - Indian}  & \cellcolor{red!27}Monhé & \cellcolor{red!27}1 & \cellcolor{red!27}1/1284& \cellcolor{red!27}0.08 \\  \hline
  
\end{longtable}
\end{center}


\textbf{\Large Result analysis:}

\begin{itemize}\item Taking into account the words that were detected, we can reach the conclusion these comments are associated with : : Ethnicity - Black;Ethnicity - Asian (South- India, Pakistan, Bangladesh);Nationality - Indian;%.

\item The percentage of hate speech related words is 0.2336.

\item Considering that the variable \textbf{Ethnicity - Black} has the most occurences in the post, we can interpret that this is the predominant hate speech.

\item Overall there were 3/27 occurences of hate speech related comments.\end{itemize}\end{document}