\documentclass[11pt]{article}
\ExplSyntaxOn
\let\tl_length:n\tl_count:n
\ExplSyntaxOff
\usepackage{graphicx}
\usepackage{multirow}
\usepackage{colortbl}
\usepackage{longtable, array}
\usepackage{hyperref}
\usepackage[usenames,dvipsnames,svgnames,table]{xcolor}
\newlength\mylength
\usepackage[legalpaper, landscape, margin=0.8in]{geometry}
\newcommand{\MinNumber}{0}
\begin{document}

\textbf {\huge In Sol\_extraction\_portuguese\_22.json :}\newline \par\Large\textbf {Title: \large  Manuel Luís Goucha conta como assumiu homossexualidade perante a mãe }\newline {\par\large --- Table  1: Summary of the results per comment; }

 {\par\large --- \hyperlink{Table 2}{\textcolor{blue}{\underline{Table 2}}}: Summary of the results per sociolinguistic variable;}\newline \normalsize\newline

\centering\textbf{\large Table  1: Summary of the results per comment 
}
\newcommand{\MaxNumber}{0}%
\newcommand{\ApplyGradient}[1]{%
\pgfmathsetmacro{\PercentColor}{100.0*(#1-\MinNumber)/(\MaxNumber-\MinNumber)}
\xdef\PercentColor{\PercentColor}%
\cellcolor{LightSpringGreen!\PercentColor!LightRed}{#1}
}
\newcolumntype{C}[2]{>{\centering\arraybackslash}p{#1}}
\begin{center}
\setlength\mylength{\dimexpr\textwidth - 1\arrayrulewidth - 50\tabcolsep}
\begin{longtable}{|C{.65\mylength}|C{.30\mylength}|C{.12\mylength}|C{.12\mylength}|C{.12\mylength}|}
\hline
\textbf{Comment} & \textbf{KeyWords} & \textbf{Sociolinguistic variables (Hiper - Hipo)}  & \textbf{Hate Speech Frequency} & \textbf{Hate Speech Frequency(\%)} \\
\hline\cellcolor{green!27}\small Nísia devias procurar ajuda veterinária. Serias melhor pessoa<br/>se fosses inseminada com um "fetus" de porco espinho \textbf{africano}<br/>o Hystrix Cristata.\normalsize   & \cellcolor{green!27}Africano & \cellcolor{green!27}Ethnicity - Black & \cellcolor{green!27}1/19 & \cellcolor{green!27}5.263 \\  \hline
  \cellcolor{green!5}\small - Olha lá +clarificador virtual , QUEM é que NÃO diria , que o manuel luis goucha é \textbf{homossexual} ?+ Os mudos..- ÁH ÁH ÁH ÁH ÁH ÁH ÁHHHHHHH !+ eh eh eh .. pois ..\normalsize   & \cellcolor{green!5}Homossexual & \cellcolor{green!5}Sexual Identity - General & \cellcolor{green!5}1/39 & \cellcolor{green!5}2.564 \\  \hline
  \cellcolor{green!27}\small + Se ( SE ) tivesses uma | Mãe | manuel luis goucha , NÃO te QUERERIA ver MAIS .- E porquê ? A uma | Mãe | , NÃO basta que o filho seja FELIZ ?+ Não . A uma | Mãe | , interessa COMO é que o filho ADQUIRE a sua " felicidade " .- E porquê ?+ Porque , | A Mãe | virtual é CATÓLICA .- E daí ...?+ NÃO aceita a HOMOSSEXUALIDADE.- NEM que SEJA a do PRÓPRIo FILHO ??!!+ Nem essa ! Aliás , MUITO menos ESSA !- Então , no Mundo Virtual da Discriminação Inteligente , Razoável e Equilibrada (MVDIRE) do +mestre , QUEM é que fica muito feliz por o filho ser um \textbf{homossexual} feliz ?+ A |progenitora conivente com o pecado pesado | .- Áaaaa...pois....e é uma | progenitora conivente com o pecado pesado | , considerada uma mãe no MVDIRE ?+ NÃO .- E porquê ?+ Porque aqui (MVDIRE) , existe uma DISCRIMINAÇÃO .- E como tal..?+ Situações DIFERENTES , são TRATADAS de forma DIFERENTE .- Aaaaaaa...pois....completamente diferente daqui da minha realidade , aliás , quase totalmente..\normalsize   & \cellcolor{green!27}Homossexual & \cellcolor{green!27}Sexual Identity - General & \cellcolor{green!27}1/205 & \cellcolor{green!27}0.488 \\  \hline
  \cellcolor{green!5}\small Boa tarde! Nas opiniões que li, li do ridículo ao escabroso. Houve partes em que concordei em absoluto, como o Goucha ter estado mal, na abordagem que fez à sua mãe sobre o assunto, até chamar "\textbf{raça}" a uma condição que com a \textbf{raça}, nada tem a ver, tendo isso sim a ver com a opção \textbf{sexual}, de cada indivíduo independente da sua cor, \textbf{raça} ou crença. Tudo o que fuja disto mostra preconceito e desrespeito por uma opção tão legítima, que não merece ser discutida!\normalsize   & \cellcolor{green!5}Raça, Sexual & \cellcolor{green!5}Ethnicity - General, Gender - General & \cellcolor{green!5}4/86 & \cellcolor{green!5}4.651 \\  \hline
  \cellcolor{green!27}\small Encostar a Mãe à parede desta maneira, é coisa de cobarde! Lamentável.<br/>Se calhar foi uma \textbf{mulher} que criou sozinha o filho, com muitos sacrifícios e quando o malandro, devido a más companhias ou influências assexuadas frustradas, obriga a coitada a dizer qualquer coisa, assim a sangue frio, ou sim ou sopas....<br/>Até para apresentar a primeira namorada aos pais, os rapazes NORMAIS tremem com medo, devido à insegurança típica da juventude.<br/>Todos nós devemos ter respeito pelos nossos pais, foram eles que nos trouxeram ao mundo e aceitar as suas opiniões, mesmo que sejam contrárias. Se não agradarem tentaremos com calma fazer "com tempo e vagar" que mudem de opinião.<br/>Por amor os pais acabam sempre por mudar de opinião, mesmo que tenham vergonha e conformam-se com o azar.<br/>O tipo ri-se de quê, sente orgulho em quê... perdeu uma oportunidade para estar calado.\normalsize   & \cellcolor{green!27}Mulher & \cellcolor{green!27}Gender - General & \cellcolor{green!27}1/140 & \cellcolor{green!27}0.714 \\  \hline
  
\end{longtable}
\end{center}


\centering\textbf{\large \hypertarget{Table 2}{Table 2}: Summary of the results per sociolinguistic variable 
}
\newcolumntype{C}[2]{>{\centering\arraybackslash}p{#1}}
\begin{center}
\setlength\mylength{\dimexpr\textwidth - 1\arrayrulewidth - 40\tabcolsep}
\begin{longtable}{|C{.50\mylength}|C{.30\mylength}|C{.15\mylength}|C{.15\mylength}|C{.15\mylength}|}
\hline
\textbf{Sociolinguistic variables (Hiper - Hipo)} & \textbf{KeyWords} & \textbf{Number of occurrences} & \textbf{Frequency}  & \textbf{Frequency(\%)} \\
\hline\multirow{1}{*}{\cellcolor{red!27}Ethnicity - Black}  & \cellcolor{red!27}Africano & \cellcolor{red!27}1 & \cellcolor{red!27}1/999& \cellcolor{red!27}0.1 \\  \hline
  \multirow{1}{*}{\cellcolor{red!5}Sexual Identity - General}  & \cellcolor{red!5}Homossexual & \cellcolor{red!5}2 & \cellcolor{red!5}2/999& \cellcolor{red!5}0.2 \\  \hline
  \multirow{1}{*}{\cellcolor{red!27}Gender - General}  & \cellcolor{red!27}Sexual, Mulher & \cellcolor{red!27}2 & \cellcolor{red!27}2/999& \cellcolor{red!27}0.2 \\  \hline
  \multirow{1}{*}{\cellcolor{red!5}Ethnicity - General}  & \cellcolor{red!5}Raça & \cellcolor{red!5}1 & \cellcolor{red!5}1/999& \cellcolor{red!5}0.1 \\  \hline
  
\end{longtable}
\end{center}


\textbf{\Large Result analysis:}

\begin{itemize}\item Taking into account the words that were detected, we can reach the conclusion these comments are associated with : : Ethnicity - Black;Sexual Identity - General;Gender - General;Ethnicity - General;%.

\item The percentage of hate speech related words is 0.6006.

\item Considering that the variable \textbf{Sexual Identity - General} has the most occurences in the post, we can interpret that this is the predominant hate speech.

\item Overall there were 8/16 occurences of hate speech related comments.\end{itemize}\end{document}