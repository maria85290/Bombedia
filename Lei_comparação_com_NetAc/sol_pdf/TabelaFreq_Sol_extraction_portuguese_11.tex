\documentclass[11pt]{article}
\ExplSyntaxOn
\let\tl_length:n\tl_count:n
\ExplSyntaxOff
\usepackage{graphicx}
\usepackage{multirow}
\usepackage{colortbl}
\usepackage{longtable, array}
\usepackage{hyperref}
\usepackage[usenames,dvipsnames,svgnames,table]{xcolor}
\newlength\mylength
\usepackage[legalpaper, landscape, margin=0.8in]{geometry}
\newcommand{\MinNumber}{0}
\begin{document}

\textbf {\huge In Sol\_extraction\_portuguese\_11.json :}\newline \par\Large\textbf {Title: \large  Manuel Luís Goucha passa fim de semana na Holanda para assistir a provas de equitação }\newline {\par\large --- Table  1: Summary of the results per comment; }

 {\par\large --- \hyperlink{Table 2}{\textcolor{blue}{\underline{Table 2}}}: Summary of the results per sociolinguistic variable;}\newline \normalsize\newline

\centering\textbf{\large Table  1: Summary of the results per comment 
}
\newcommand{\MaxNumber}{0}%
\newcommand{\ApplyGradient}[1]{%
\pgfmathsetmacro{\PercentColor}{100.0*(#1-\MinNumber)/(\MaxNumber-\MinNumber)}
\xdef\PercentColor{\PercentColor}%
\cellcolor{LightSpringGreen!\PercentColor!LightRed}{#1}
}
\newcolumntype{C}[2]{>{\centering\arraybackslash}p{#1}}
\begin{center}
\setlength\mylength{\dimexpr\textwidth - 1\arrayrulewidth - 50\tabcolsep}
\begin{longtable}{|C{.65\mylength}|C{.30\mylength}|C{.12\mylength}|C{.12\mylength}|C{.12\mylength}|}
\hline
\textbf{Comment} & \textbf{KeyWords} & \textbf{Sociolinguistic variables (Hiper - Hipo)}  & \textbf{Hate Speech Frequency} & \textbf{Hate Speech Frequency(\%)} \\
\hline\cellcolor{green!27}\small A natureza em marcha atrás...\textbf{m\textbf{onhé}} disse: o "\textbf{Cego} do Ó"  só enxerga "banana"!\normalsize   & \cellcolor{green!27}Cego, Monhé & \cellcolor{green!27}Ethnicity - Asian (South- India, Pakistan, Bangladesh), Nationality - Indian, Physical Identity - Physical (and Mental) Impairments & \cellcolor{green!27}3/13 & \cellcolor{green!27}23.077 \\  \hline
  \cellcolor{green!5}\small "Juntou-se ao marido..." diz a peça.<br/>Mas qual "marido"..?!!!  Marido é o cônjuge da \textbf{mulher} ! <br/>O \textbf{gajo} que vive com o Goucha é um companheiro ajuntado a ele...!\normalsize   & \cellcolor{green!5}Gajo, Mulher & \cellcolor{green!5}Ethnicity - White, Gender - General & \cellcolor{green!5}2/28 & \cellcolor{green!5}7.143 \\  \hline
  \cellcolor{green!27}\small Marido - Pessoa do sexo masculino casada com outra.Cada vez mais me convenço que além de maluco o Citharoedus Romano também é \textbf{analfabeto}.\normalsize   & \cellcolor{green!27}Analfabeto & \cellcolor{green!27}Social Class - Working class & \cellcolor{green!27}1/24 & \cellcolor{green!27}4.167 \\  \hline
  \cellcolor{green!5}\small Na verdade essa propaganda que a Tvi faz ao marido do Goucha, como se o Goucha fosse uma \textbf{mulher}, é uma grandessíssima pouca vergonha!\normalsize   & \cellcolor{green!5}Mulher & \cellcolor{green!5}Gender - General & \cellcolor{green!5}1/24 & \cellcolor{green!5}4.167 \\  \hline
  
\end{longtable}
\end{center}


\centering\textbf{\large \hypertarget{Table 2}{Table 2}: Summary of the results per sociolinguistic variable 
}
\newcolumntype{C}[2]{>{\centering\arraybackslash}p{#1}}
\begin{center}
\setlength\mylength{\dimexpr\textwidth - 1\arrayrulewidth - 40\tabcolsep}
\begin{longtable}{|C{.50\mylength}|C{.30\mylength}|C{.15\mylength}|C{.15\mylength}|C{.15\mylength}|}
\hline
\textbf{Sociolinguistic variables (Hiper - Hipo)} & \textbf{KeyWords} & \textbf{Number of occurrences} & \textbf{Frequency}  & \textbf{Frequency(\%)} \\
\hline\multirow{1}{*}{\cellcolor{red!27}Ethnicity - Asian (South- India, Pakistan, Bangladesh)}  & \cellcolor{red!27}Monhé & \cellcolor{red!27}1 & \cellcolor{red!27}1/357& \cellcolor{red!27}0.27999999999999997 \\  \hline
  \multirow{1}{*}{\cellcolor{red!5}Nationality - Indian}  & \cellcolor{red!5}Monhé & \cellcolor{red!5}1 & \cellcolor{red!5}1/357& \cellcolor{red!5}0.27999999999999997 \\  \hline
  \multirow{1}{*}{\cellcolor{red!27}Physical Identity - Physical (and Mental) Impairments}  & \cellcolor{red!27}Cego & \cellcolor{red!27}1 & \cellcolor{red!27}1/357& \cellcolor{red!27}0.27999999999999997 \\  \hline
  \multirow{1}{*}{\cellcolor{red!5}Gender - General}  & \cellcolor{red!5}Mulher & \cellcolor{red!5}2 & \cellcolor{red!5}2/357& \cellcolor{red!5}0.5599999999999999 \\  \hline
  \multirow{1}{*}{\cellcolor{red!27}Ethnicity - White}  & \cellcolor{red!27}Gajo & \cellcolor{red!27}1 & \cellcolor{red!27}1/357& \cellcolor{red!27}0.27999999999999997 \\  \hline
  \multirow{1}{*}{\cellcolor{red!5}Social Class - Working class}  & \cellcolor{red!5}Analfabeto & \cellcolor{red!5}1 & \cellcolor{red!5}1/357& \cellcolor{red!5}0.27999999999999997 \\  \hline
  
\end{longtable}
\end{center}


\textbf{\Large Result analysis:}

\begin{itemize}\item Taking into account the words that were detected, we can reach the conclusion these comments are associated with : : Ethnicity - Asian (South- India, Pakistan, Bangladesh);Nationality - Indian;Physical Identity - Physical (and Mental) Impairments;Gender - General;Ethnicity - White;Social Class - Working class;%.

\item The percentage of hate speech related words is 1.9608.

\item Considering that the variable \textbf{Gender - General} has the most occurences in the post, we can interpret that this is the predominant hate speech.

\item Overall there were 7/16 occurences of hate speech related comments.\end{itemize}\end{document}