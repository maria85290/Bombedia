\documentclass[11pt]{article}
\ExplSyntaxOn
\let\tl_length:n\tl_count:n
\ExplSyntaxOff
\usepackage{graphicx}
\usepackage{multirow}
\usepackage{colortbl}
\usepackage{longtable, array}
\usepackage{hyperref}
\usepackage[usenames,dvipsnames,svgnames,table]{xcolor}
\newlength\mylength
\usepackage[legalpaper, landscape, margin=0.8in]{geometry}
\newcommand{\MinNumber}{0}
\begin{document}

\textbf {\huge In Sol\_extraction\_portuguese\_155.json :}\newline \par\Large\textbf {Title: \large  Fiéis de Caminha manifestam-se contra saída de padre motard e sex symbol}\newline {\par\large --- Table  1: Summary of the results per comment; }

 {\par\large --- \hyperlink{Table 2}{\textcolor{blue}{\underline{Table 2}}}: Summary of the results per sociolinguistic variable;}\newline \normalsize\newline

\centering\textbf{\large Table  1: Summary of the results per comment 
}
\newcommand{\MaxNumber}{0}%
\newcommand{\ApplyGradient}[1]{%
\pgfmathsetmacro{\PercentColor}{100.0*(#1-\MinNumber)/(\MaxNumber-\MinNumber)}
\xdef\PercentColor{\PercentColor}%
\cellcolor{LightSpringGreen!\PercentColor!LightRed}{#1}
}
\newcolumntype{C}[2]{>{\centering\arraybackslash}p{#1}}
\begin{center}
\setlength\mylength{\dimexpr\textwidth - 1\arrayrulewidth - 50\tabcolsep}
\begin{longtable}{|C{.65\mylength}|C{.30\mylength}|C{.12\mylength}|C{.12\mylength}|C{.12\mylength}|}
\hline
\textbf{Comment} & \textbf{KeyWords} & \textbf{Sociolinguistic variables (Hiper - Hipo)}  & \textbf{Hate Speech Frequency} & \textbf{Hate Speech Frequency(\%)} \\
\hline\cellcolor{green!27}\small «Naughty, naughty, naughty!!!»a paroquia deste deve ser ultra concorrida...hahaha mas parece \textbf{velho} demais para o Vaticano.<br/>o que é andar de mota e cuidar do corpo tem a ver com a maneira como exerce a profissão? - N A D A\normalsize   & \cellcolor{green!27}Velho & \cellcolor{green!27}Age - Over 65s & \cellcolor{green!27}1/41 & \cellcolor{green!27}2.439 \\  \hline
  \cellcolor{green!5}\small Pudera não é um \textbf{velho} rabugento haaa\normalsize   & \cellcolor{green!5}Velho & \cellcolor{green!5}Age - Over 65s & \cellcolor{green!5}1/7 & \cellcolor{green!5}14.286 \\  \hline
  \cellcolor{green!27}\small A Igreja tem uma história funesta em relação a comportamentos sexuais porque a maioria dos padres segue o "sacerdócio" porque não têm que trabalhar a sério, nem desbravar "os caminhos do pão" para se sustentarem, e assim conseguem uma vida de autênticos burgueses sem qualquer esforço físico ou sacrifício. Castidade?? Arranja-se forma de programar umas escapadelas sem ninguém saber. Não fora uma escorregadela no facebook, e ninguém saberia que em Portugal até há um padre que pagou para se saciar com uma \textbf{prostituta}. A igreja católica mexe com muito dinheiro, e esse dinheiro é muito atractivo! Veja-se a quantidade de padrecos que desfilam no Vaticano. O que fazem lá? Teorizam o que não é teorizável, entretêm-se com livros e escritos e filosofias, completamente afastados das vidas duras de tanta gente, para se auto-intitularem moralmente superiores aos demais e assim justificarem a existência de uma quantidade de súbditos espirituais que acham que oferecendo dinheiro à igreja limpam os seus pecados ou ficam mais bem vistos aos "olhos de Deus".O que acontece aos padres pedófilos? Eu diria que são "reformados" com uma boa reforma, paga com as esmolas de muitos pobres... Ora é fácil de ver que este padre tem um apego a bens materiais e à ostentação que não se coadunam com a suposta missão espiritual que lhe foi conferida. Mas e os outros padres, será que têm mesmo vocação ou fingem que têm, para manterem a sua legião de "fiéis" pagadores de "esmolas"?Se os padres se podem casar, então que se extinga de vez o que nunca devia ter começado: a instituição da eucaristia como forma ardilosa de sacar dinheiro às pessoas. Na verdade, com toda a hipocrisia que os escândalos da igreja têm revelado, fica cada vez mais claro que as missas têm um único propósito: o peditório. Tudo o resto é encenação.Portugal é um país que se apregoa católico e tem mais de dois milhões de pessoas a viver na pobreza. No entanto, as dioceses ostentam obras luxuosas... mas não era suposto distribuírem pelos pobres? Pois...\normalsize   & \cellcolor{green!27}Prostituta & \cellcolor{green!27}Gender - Female sexuality & \cellcolor{green!27}1/341 & \cellcolor{green!27}0.293 \\  \hline
  \cellcolor{green!5}\small Hummm, ou é \textbf{gay} ou as beatas passam muito tempo na confissão...\normalsize   & \cellcolor{green!5}Gay & \cellcolor{green!5}Sexual Identity - General & \cellcolor{green!5}1/12 & \cellcolor{green!5}8.333 \\  \hline
  
\end{longtable}
\end{center}


\centering\textbf{\large \hypertarget{Table 2}{Table 2}: Summary of the results per sociolinguistic variable 
}
\newcolumntype{C}[2]{>{\centering\arraybackslash}p{#1}}
\begin{center}
\setlength\mylength{\dimexpr\textwidth - 1\arrayrulewidth - 40\tabcolsep}
\begin{longtable}{|C{.50\mylength}|C{.30\mylength}|C{.15\mylength}|C{.15\mylength}|C{.15\mylength}|}
\hline
\textbf{Sociolinguistic variables (Hiper - Hipo)} & \textbf{KeyWords} & \textbf{Number of occurrences} & \textbf{Frequency}  & \textbf{Frequency(\%)} \\
\hline\multirow{1}{*}{\cellcolor{red!27}Age - Over 65s}  & \cellcolor{red!27}Velho & \cellcolor{red!27}2 & \cellcolor{red!27}2/1420& \cellcolor{red!27}0.13999999999999999 \\  \hline
  \multirow{1}{*}{\cellcolor{red!5}Gender - Female sexuality}  & \cellcolor{red!5}Prostituta & \cellcolor{red!5}1 & \cellcolor{red!5}1/1420& \cellcolor{red!5}0.06999999999999999 \\  \hline
  \multirow{1}{*}{\cellcolor{red!27}Sexual Identity - General}  & \cellcolor{red!27}Gay & \cellcolor{red!27}1 & \cellcolor{red!27}1/1420& \cellcolor{red!27}0.06999999999999999 \\  \hline
  
\end{longtable}
\end{center}


\textbf{\Large Result analysis:}

\begin{itemize}\item Taking into account the words that were detected, we can reach the conclusion these comments are associated with : : Age - Over 65s;Gender - Female sexuality;Sexual Identity - General;%.

\item The percentage of hate speech related words is 0.2817.

\item Considering that the variable \textbf{Age - Over 65s} has the most occurences in the post, we can interpret that this is the predominant hate speech.

\item Overall there were 4/38 occurences of hate speech related comments.\end{itemize}\end{document}